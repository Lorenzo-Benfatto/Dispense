\documentclass[twoside]{article}
\usepackage[utf8]{inputenc}

\title{\textbf{Physĭca est omnis divisa in partes tres}}
\author{Alto Concilio}
\date{Dicembre 2020}

\usepackage[italian]{babel}
\usepackage{tikz}
\tikzstyle{mybox} = [draw=black, very thick, rectangle, rounded corners, inner ysep=5pt, inner xsep=5pt]
\usepackage[italian]{babel}
\usepackage{natbib}
\usepackage{graphicx}
\usepackage{physics}
\usepackage{amsmath,amssymb,mathrsfs}
\usepackage{dsfont}
\usepackage{tensor}
\usepackage{latexsym}
\usepackage{verbatim}
\newcommand\hmmax{0} % default 3
\newcommand\bmmax{0} % default 4
\usepackage{bm}
\usepackage{url,amsfonts,epsfig}
\usepackage{amsthm}
\usepackage{tikz}
\usepackage{geometry}
\geometry{a4paper, top=2.5cm, bottom=2.75cm, left=2.5cm, right=2.5cm, heightrounded, bindingoffset=5mm}
\usepackage{hyperref}
\usepackage{chngcntr}
\usepackage{epigraph}
\usepackage{fancyhdr}
\usepackage{marginnote}
\usepackage{titlesec}
\usepackage{float}
\usepackage{fourier}
\usepackage{wrapfig}
\usepackage{xcolor}
\usepackage{booktabs}
\usepackage{array}
\definecolor{mygray}{gray}{0.6}
\definecolor{gold(metallic)}{rgb}{0.83, 0.69, 0.22}
\definecolor{slategray}{rgb}{0.44, 0.5, 0.56}
\definecolor{darkscarlet}{rgb}{0.34, 0.01, 0.1}
\definecolor{test}{rgb}{0.52, 0.0, 0.0}
\definecolor{bronze}{rgb}{0.8, 0.5, 0.2}
\definecolor{burgundy}{rgb}{0.5, 0.0, 0.13}
\hypersetup{
    colorlinks=true,
    linkcolor=black,
    filecolor=magenta,      
    urlcolor=cyan,
}

\newcommand {\paral} {\mathrel{/\!/}}
\newcommand{\angstrom}{\textup{\AA}}

\newtheorem{definition}{Definizione}[section]
\newtheorem{proposition}{Proposizione}[section]
\newtheorem{theorem}{Teorema}[section]
\newtheorem{corollary}{Corollario}[theorem]
\newtheorem{lemma}[theorem]{Lemma}

\renewcommand\qedsymbol{$\blacksquare$}

\pagestyle{fancy}
\fancyhf{}
\fancyhead[LE,RO]{\textcolor{gold(metallic)}{B}\textcolor{burgundy}{C}}
\fancyhead[RE,LO]{\textcolor{slategray}{A}\textcolor{bronze}{B}}
\fancyhead[CE,CO]{\includegraphics*[scale=0.15]{logo.jpg}}
\fancyfoot[CE,CO]{\textcolor{mygray}{WE APOLOGISE FOR THE INCONVENIENCE}}
\fancyfoot[LE,RO]{\thepage}
\fancyfoot[RE,LO]{\includegraphics*[scale=0.15]{LB(good).jpg}}

\setlength{\headheight}{47.3pt}% ...at least 51.60004pt

\counterwithin{equation}{section}
%\counterwithin{equation}{subsection}
\setlength\parindent{0pt}


\begin{document}


\maketitle

\titleformat{\section}[display]
{\sffamily\bfseries\filleft}
{\color{mygray}\fontsize{26}{32}\selectfont\thesection}
{0pt}
{\huge\raggedleft}
[{\titlerule[0.7pt]}]


\noindent\rule{\textwidth}{0.7pt}

\vspace{1cm}

\begin{figure}[h]
\centering
\includegraphics[scale=0.4]{cern-atlas-higgs-event.jpg}
\label{fig:cern}
\end{figure}

\vfill

\begin{figure}[h]
\centering
\includegraphics[scale=0.25]{logo.JPG}
\label{fig:logo}
\end{figure}

\epigraph{Se mi ricordassi i nomi di tutte queste particelle sarei un botanico.}{\textit{Enrico Fermi, \\ interrogato sui nomi delle particelle scoperte}}

\newpage
\section*{Introduzione}
''Preparare 300 domande della checklist è impossibile''
\\
''No, è necessario.''
\\
\url{https://www.youtube.com/watch?v=6ixvpLCdqkA}


\newpage

\tableofcontents

  \newpage 
  
\section{Prerequisiti}
\epigraph{Many were increasingly of the opinion that they'd all made a big mistake in coming down from the trees in the first place. And some said that even the trees had been a bad move, and that no one should ever have left the oceans.}{}


\paragraph{1.a.1} \textbf{Definire le quantità $\beta$ e $\gamma$ per le trasformazioni di Lorentz}\\
\\
Siano due sistemi $\mathnormal{S}$ e $\mathnormal{S'}$, con $\mathnormal{S'}$ in moto rispetto a $\mathnormal{S}$ con velocità $\mathnormal{v}$, allora nelle trasformazioni di Lorentz si ha:
\begin{align*}
    \beta=\frac{\mathnormal{v}}{c}\ \ \ \ \ \ \gamma=\frac{1}{\sqrt{1-\beta^2}}
\end{align*}

\paragraph{1.a.2} \textbf{Dare la definizione di 4-vettore covariante e controvariante}\\
\\
Un 4-vettore controvariante è una quaterna $x^{\mu}=(x^0,x^1,x^2,x^3)$ che trasforma da un sistema di riferimento all'altro secondo le trasformazioni di Lorentz nel seguente modo:
\begin{equation*}
    {x'}^{\mu}=\tensor{\Lambda}{^{\mu}_{\nu}}x^{\nu}
\end{equation*}
dove $\tensor{\Lambda}{^{\mu}_{\nu}}$ è una matrice del gruppo di Lorentz. 
\[
\tensor{\Lambda}{^{\mu}_{\nu}}=\begin{pmatrix}
\gamma & -\gamma\beta & 0 & 0 \\
-\gamma\beta & \gamma & 0 & 0 \\
0 & 0 & 1 & 0 \\
0 & 0 & 0 & 1 \\
\end{pmatrix}
\]
Considerando la stessa trasformazione, un 4-vettore covariante $x_{\mu}$ è una quaterna che trasforma come le derivate di una funzione scalare come 
\begin{equation*}
    {x'}_{\mu}=\tensor{\Lambda}{_{\mu}^{\nu}}x_{\nu}
\end{equation*}
dove
\begin{equation*}
    \tensor{\Lambda}{_{\mu}^{\nu}}=g_{\mu\alpha}\tensor{\Lambda}{^{\alpha}_{\beta}}g^{\beta\nu}
\end{equation*}
con 
\[
g_{\mu\nu}=g^{\mu\nu}=\begin{pmatrix}
1 & 0 & 0 & 0 \\
0 & -1 & 0 & 0 \\
0 & 0 & -1 & 0 \\
0 & 0 & 0 & -1 \\
\end{pmatrix}
\]
\paragraph{1.a.3} \textbf{Definire il prodotto scalare di due 4-vettori}\\
\\
Dati due 4-vettori $x$ e $y$ si definisce il loro prodotto scalare come:
\[
\mathbf{x}\cdot \mathbf{y}=\begin{pmatrix}
x_0 &x_1 &x_2&x_3\\
\end{pmatrix}\begin{pmatrix}
1&0&0&0\\
0&-1&0&0\\
0&0&-1&0\\
0&0&0&-1\\
\end{pmatrix}\begin{pmatrix}
y_0\\
y_1\\
y_2\\
y_3\\
\end{pmatrix}=x_0y_0-x_1y_1-x_2y_2-x_3y_3
\]
utilizzando la notazione covariante si ha
\begin{equation*}
    x\cdot y=g_{\mu\nu}x^{\nu}y^{\mu}=x_{\mu}y^{\mu}
\end{equation*}
\paragraph{1.a.4} \textbf{Definire il modulo di un 4-vettore}\\
\\
Il modulo quadro di un quadrivettore è definito come il prodotto scalare del vettore con se stesso come
\begin{equation*}
    |A|^2=A_{\mu}A^{\mu}=A_0A^0-A_1A^1-A_2A^2-A_3A^3
\end{equation*}
da cui
\begin{equation*}
    |A|^2=(A^0)^2-|\mathbf{A}|^2
\end{equation*}
\paragraph{1.a.5}\textbf{ Scrivere le trasformazioni di Lorentz per il boost lungo un asse (asse x)}\\
\\
Siano $O$ e $O'$ due sistemi di riferimento inerziali, con $O'$ in moto rispetto a $O$ con velocità $\mathbf{V}=(v,0,0)$. Sia $A^{\mu}$ un quadrivettore controvariante in $O$, in $O'$ si ha:
\begin{equation*}
    {A'}^{\mu}=\tensor{\Lambda}{^{\mu}_{\nu}}A^{\nu}
\end{equation*}
Dove $\tensor{\Lambda}{^{\mu}_{\nu}}$ è  
\[
\tensor{\Lambda}{^{\mu}_{\nu}}=\begin{pmatrix}
\gamma & -\gamma\beta & 0 & 0 \\
-\gamma\beta & \gamma & 0 & 0 \\
0 & 0 & 1 & 0 \\
0 & 0 & 0 & 1 \\
\end{pmatrix}
\]
Esplicitando il calcolo
\begin{equation*}
    \begin{cases}
        A^{0'} = \gamma (A^{0} - \beta A^{1})   \\
        A^{1'} = \gamma (A^{1} - \beta A^{0}) 	\\
        A^{2'} = A^{2}	\\
        A^{3'} = A^{3} \, .
    \end{cases}
\end{equation*}
La trasformazione inversa è data da
\begin{equation}
A^\mu=\tensor{(\Lambda^{-1})}{^\mu_\nu} A'^\nu \, ,
\end{equation}
dove
\begin{equation}
\tensor{(\Lambda^{-1})}{^\mu_\nu}=
\begin{pmatrix}
	\gamma		& \beta\gamma 	& 0 & 0 \\
\beta\gamma 	&  \gamma		& 0 & 0 \\
 		0		& 		0		& 1 & 0 \\
 		0		&		0		& 0 & 1 \\
\end{pmatrix} \, ,
\end{equation}
ovvero
\begin{equation}
\begin{cases}
A^{0} = \gamma (A^{0'} + \beta A^{1'})  \\
A^{1} = \gamma (A^{1'} + \beta A^{0'}) 	\\
A^{2} = A^{2'}	\\
A^{3} = A^{3'} \, .
\end{cases}
\end{equation}
\paragraph{1.a.6}\textbf{Definire un tensore covariante di rango 2 e la sua traccia}\\
\\
Un tensore 4-dimensionale covariante di rango 2 è un oggetto a 2 indici che trasforma come prodotto di componenti di di due quadrivettori. Il numero di componenti del tensore è dato da $n^r$ dove $n$ è il numero di dimensioni (4) e $r$ il rango, quindi in questo caso sono necessarie 16 componenti, e può essere espresso sotto forma di matrice:
\begin{equation*}
    A_{\mu\nu}=
    \begin{pmatrix}
        A_{00} &A_{01} &A_{02} &A_{03} \\
        A_{10} &A_{11} &A_{12} &A_{13} \\
        A_{20} &A_{21} &A_{22} &A_{23} \\
        A_{30} &A_{31} &A_{32} &A_{33} \\
    \end{pmatrix}
\end{equation*}
La traccia di un tensore covariante di rango 2 è definita come 
\begin{equation*}
    Tr(A)=\sum_{\mu} \tensor{A}{^{\mu}_{\mu}}=\tensor{A}{^0_0}+\tensor{A}{^1_1}+\tensor{A}{^2_2}+\tensor{A}{^3_3}
\end{equation*}
\paragraph{1.a.7}\textbf{Definire le derivate in 4-dimensioni, la quadridivergenza, il differenziale di uno
scalare di Lorentz, l’operatore di D’Alembert}\\
\\
Le derivate in 4 dimensioni hanno una versione covariante e controvariante e sono definite come:\\
\textit{Covariante}
\begin{equation*}
    \partial_{\mu}=\frac{\partial}{\partial x^{\mu}}=\biggl(\frac{1}{c}\frac{\partial}{\partial t}, \mathbf{\nabla}\biggr)
\end{equation*}
\textit{Controvariante}
\begin{equation*}
    \partial^{\mu}=\frac{\partial}{\partial x_{\mu}}=\biggl(\frac{1}{c}\frac{\partial}{\partial t}, -\mathbf{\nabla}\biggr)
\end{equation*}
La quadridivergenza è definita come:
\begin{equation*}
    \partial_{\mu}A^{\mu}=\frac{\partial}{\partial x^{\mu}}A^{\mu}=\frac{1}{c} \frac{\partial}{\partial t}A^{0}+ \mathbf{\nabla} \cdot \mathbf{A}
\end{equation*}
La quadridivergenza è equivalente se calcolata in forma covariante o controvariante $\partial_{\mu}A^{\mu}=\partial^{\mu}A_{\mu}$.\\
Il differenziale di uno scalare di Lorentz $\phi(x^\mu)$ è:
\begin{equation}
d\phi=\frac{d\phi}{dx^{\mu}} dx^{\mu} \, .
\end{equation}
L'operatore di D'Alembert è:
\begin{equation}
\partial_{\mu} \partial^{\mu}= \Box = \frac{1}{c^{2}} \frac{\partial^2}{\partial t^{2}} - \textbf{$\nabla$} ^{2} \, .
\end{equation}
\paragraph{1.a.8} \textbf{Definire il tempo proprio e dare la relazione (differenziale) fra tempo proprio e tempo nel sistema in cui si osserva il moto}\\
\\
Il tempo proprio di un corpo è il tempo misurato da un orologio solidale con il corpo stesso.\\
Consideriamo $O$ e $O'$ due sistemi inerziali con $O'$ in moto rispetto a $O$ con velocità $\mathbf{v}=d\mathbf{x}/dt$. Poiché l'intervallo spazio temporale tra due eventi è costante, prendendo un orologio solidale a $O'$ abbiamo che:
\begin{equation*}
    c^2d\tau^2=c^2dt^2-d\mathbf{x}=c^2dt^2\biggl(1-\frac{d\mathbf{x}^2}{c^2dt^2}\biggr)=c^2dt^2(1-\beta^2)=\frac{c^2dt^2}{\gamma^2}
\end{equation*}
da cui 
\begin{equation*}
    dt=\gamma d\tau
\end{equation*}
\paragraph{1.a.9} \textbf{Dare la definizione di invariante di Lorentz}\\
\\
Un'invariante di Lorentz è una grandezza scalare invariante sotto trasformazioni di Lorentz per le coordinate.
\paragraph{1.a.10} \textbf{Definire la 4-velocità ed il 4-impulso di un punto materiale di massa m, esprimere le loro unità di misura nei sistemi MKS e $\hbar=\mathnormal{c}=1$ , dimostrare che il loro modulo è costante}\\
\\
Dato un punto materiale di massa $m$ la sua 4-velocità definita come la derivata rispetto al tempo proprio della sua 4-posizione:
\begin{equation*}
    u^{\mu}=\frac{dx^{\mu}}{d\tau}=\gamma\frac{d}{dt}(ct, \mathbf{x})=(\gamma c,\gamma\mathbf{v})
\end{equation*}
Il 4-impulso del punto materiale è il prodotto della sua 4-velocità per la massa:
\begin{equation*}
    p^{\mu}=mu^{\mu}=(m\gamma c,m\gamma\mathbf{v})
\end{equation*}
sapendo che $E=\gamma m c^2$ si può riscrivere come
\begin{equation*}
    p^{\mu}=\biggl(\frac{E}{c},\mathbf{p}\biggr)
\end{equation*}
Nel sistema MKS le unità di misura sono
\begin{equation*}
    [u^{\mu}]=\frac{m}{s}\ \ \ \ \ \ \ \ \ \ \ \ [p^{\mu}]=\frac{kg\cdot m}{s}
\end{equation*}
Nel sistema $\hbar=c=1$ le unità di misura sono
\begin{equation*}
    [u^{\mu}]=adimensionale\ \ \ \ \ \ \ \ \ \ \ \ [p^{\mu}]=eV
\end{equation*}
Il modulo quadro della 4-velocità è pari a 
\begin{equation*}
    u^2=u_{\mu}u^{\mu}=\gamma^2 c^2-\gamma^2 v^2=\gamma^2 c^2(1-\beta^2)=c^2
\end{equation*}
ed è costante poiché $c$ è costante in ogni sistema di riferimento.\\
Il modulo quadro del 4-impulso è 
\begin{equation*}
    p^2=p_{\mu}p^{\mu}=m^2\gamma^2c^2-m^2\gamma^2v^2=m^2\gamma^2c^2(1-\beta^2)=m^2c^2=
\end{equation*}
che è costante perché $m$ è la massa a riposo. Da quest'ultima relazione si può ricavare la relazione tra energia e impulso:
\begin{equation*}
    \frac{E^2}{c^2}-|\mathbf{p}|^2=m^2c^2\,.
\end{equation*}
\paragraph{1.a.11} \textbf{Enunciare la legge di conservazione del 4-impulso} \\ \\
Per un sistema isolato, cioè non sottoposto a forze esterne,
il 4-impulso totale si conserva nel tempo.
\[
P_{tot}=\sum_{i=1}^n P_i \ \ \ \ \ \text{con} \ \ \ \ P_i= \begin{pmatrix}
E_i \\
\vec{p}_i
\end{pmatrix}
\]

\paragraph{1.a.12} \textbf{Definire il tensore metrico $g_{\mu \nu}$}
\\ \\
Il tensore metrico definisce il prodotto scalare nello spazio-tempo di Minkowski, che può avere o segnatura (1,3) o (3,1):
\[
g_{\mu \nu}=\begin{pmatrix}
1 & 0 & 0 & 0 \\
0 & -1 & 0 & 0 \\
0 & 0 & -1 & 0 \\
0 & 0 & 0 & -1 \\
\end{pmatrix} \ \ \ \ \ \ \begin{pmatrix}
-1 & 0 & 0 & 0 \\
0 & 1 & 0 & 0 \\
0 & 0 & 1 & 0 \\
0 & 0 & 0 & 1 \\
\end{pmatrix}
\]


\paragraph{1.a.13} \textbf{Dare la definizione di tensore antisimmetrico di rango 2 ed indicare quali dei
suoi elementi siano le componenti di un vettore polare e quali quelle di un vettore
assiale tridimensionale}\\ \\

Un tensore $F^{\mu \nu}$ è antisimmettrico se cambia segno per scambio degli indici
\begin{equation*}
    F^{\mu \nu}=-F^{\nu \mu}
\end{equation*}
Qualora il tensore abbia la seguente forma:
\begin{equation*}
    F^{\mu \nu}=\begin{pmatrix}
    0 & -v_x & -v_y & - v_z \\
    v_x & 0 & -w_z & w_y \\
    v_y & w_z & 0 & -w_x \\
    v_z & -w_y & w_x & 0
    \end{pmatrix}
\end{equation*}
Si possono identificare un vettore polare e un vettore assiale, rispettivamente, come:
\begin{equation*}
    \vec{v}=(v_x , v_y , v_z) \ \ \ \vec{w}=(w_x , w_y , w_z)
\end{equation*}

\paragraph{1.a.14} \textbf{Definire quando una certa legge fisica è scritta in forma "relativisticamente
covariante"} \\ \\
Un'equazione si dice covariante di Lorentz se può essere scritta in termini di quantità covarianti di Lorentz. La proprietà fondamentale di tali equazioni è che forniscono lo stesso risultato in qualsiasi sistema di riferimento inerziale.

% non approvo il dark scarlet
%trovamene uno migliore
% ma sembra violaccio
%la blanca lo vuole rosso scuro, trovane uno purchè non sia nero che almeno si capisce cosa è cliccabile e cosa no
\paragraph{1.a.15} \textbf{Dimostrare la legge relativistica di composizione delle velocità 
} \\ \\
Considerando un sistema inerziale \textbf{$S$} e passando ad un altro sistema inerziale \textbf{$S^*$}, con velocità relativa $\vec{v}=v_x >0$ la trasformazione del quadrivettore posizione è:
\begin{equation*}
    \left\{ 
    \begin{matrix}
    t'=\gamma \left( t-\frac{v}{c^2}x \right) \\
    x'=\gamma(x-vt) \\
    y'=y \\
    z'=z
    \end{matrix}
    \right. \ \ \ \ \ \text{differenziando} \ \ \ \ \ \left\{ 
    \begin{matrix}
    dt'=\gamma \left( dt-\frac{v}{c^2}dx \right) \\
    dx'=\gamma(x-vdt) \\
    dy'=dy \\
    dz'=dz
    \end{matrix}
    \right.
\end{equation*}
Considerando la definizione di velocità ($v=dx/dt$) si trova:
\begin{equation*}
    \left\{ 
    \begin{matrix}
    v_x '=\frac{dx'}{dt'} =\frac{\gamma(x-vdt)}{\gamma \left( dt-\frac{v}{c^2}dx \right)}=\frac{v_x - v}{1-\frac{v_x v}{c^2}}
    \\
    v_y '=\frac{dy'}{dt'}=\frac{dy}{\gamma \left( dt-\frac{v}{c^2}dx \right)}=\frac{v_y \sqrt{1-\frac{v^2}{c^2}}}{1-\frac{v v_x}{c^2}} \\
    v_z ' =\frac{dz'}{dt'}=\frac{dz}{\gamma \left( dt-\frac{v}{c^2}dx \right)}=\frac{v_z \sqrt{1-\frac{v^2}{c^2}}}{1-\frac{v v_x}{c^2}} 
    \end{matrix}
    \right.
\end{equation*}



\paragraph{1.a.16} \textbf{Dimostrare che il modulo di un 4-vettore ed il prodotto di due 4-vettori sono
invarianti di Lorentz}\\ \\
Dal momento che 
\begin{equation*}
    |p|\equiv\sqrt{p^{\mu}p_{\mu}}
\end{equation*}
basta dimostrare che il prodotto di due 4-vettori è invariante.
\\
Dal momento che il Gruppo di Lorentz è definito come il gruppo delle trasformazioni che lasciano invariata la metrica di Minkowski
\begin{equation*}
    \tensor{\Lambda}{^{\mu}_{\alpha}}\tensor{g}{_{\mu}_{\beta}}\tensor{\Lambda}{^{\beta}_{\gamma}}=\tensor{g}{_{\alpha}_{\gamma}}
\end{equation*}
Esplicitando il prodotto scalare in un sistema di riferimento boostato si trova:
\begin{equation*}
    {x'} ^{\mu}y' _{\mu}=\tensor{\Lambda}{^{\mu}_{\alpha}}\tensor{g}{_{\mu}_{\beta}}\tensor{\Lambda}{^{\beta}_{\gamma}}x^{\alpha}y^{\gamma}=\tensor{g}{_{\alpha}_{\gamma}}x^{\alpha}y^{\gamma}=x^{\mu}y_{\mu}
\end{equation*}


\paragraph{1.a.17} \textbf{Spiegare il “paradosso dei gemelli”}\\ \\
Dei due gemelli identici, i Dioscuri Castore e Polluce, il primo parte insieme agli argonauti spaziali per un viaggio interplanetario alla Interstellar che tra l'altro ho visto proprio ieri sera a una velocità costante che è una frazione considerevole della velocità della luce, mentre a Polluce pesa il culo e resta sulla Terra.
\\
Dal punto di vista di Polluce, Castore sta viaggiando alla velocità sopracitata, conseguentemente afferma che il tempo del fratello è dilatato e scorre più lentamente per lui.
\\
Dal punto di vista di Castore, lui è fermo e Polluce sta viaggiando alla velocità sopracitata ma in direzione opposta. Perciò afferma che il tempo di Polluce scorre più lentamente del suo.
\\
Dopo un po' di tempo, Castore e gli argonauti arrivano ad un piccolo pianeta vicino alla stella Betelgeuse dove conquistano il vello d'oro e subito ripartono verso la Terra alla stessa velocità costante.
\\
Al ritorno sulla Terra, Castore risulta più giovane di Polluce.
\\
Eppure, per il principio di relatività, durante il viaggio di Castore ognuno dei due dioscuri poteva affermare che il tempo scorresse più lentamente per il fratello.
\\
La soluzione del paradosso sta nel fatto che Castore si è fermato ed è ripartito in direzione opposta. Conseguentemente il suo sistema di riferimento proprio non è inerziale per tutta la durata del viaggio, dal momento che ha dovuto decelerare per fermarsi ed accelerare per ripartire.
\\
Risulta chiaro se si osserva il seguente diagramma di Minkowski \textbf{Figura (\ref{gemelli})}:
\begin{figure}[H]
    \centering
    \includegraphics[scale=0.5]{gemelli.png}
    \label{gemelli}
\end{figure}

\paragraph{1.a.18} \textbf{Dimostrare che l’operatore di D’Alembert è un invariante di Lorentz}\\ \\
L'operatore di D'Alembert è definito come:
\begin{equation*}
    \Box u=\partial ^{\mu} \partial _{\nu}=g^{\mu \nu} \partial _{\mu }\partial_{\nu}
\end{equation*}
Essendo definito come un prodotto scalare e dal momento che gli operatori $\partial_{\mu}$ trasformano come vettori covarianti, risulta che il Dalembertiano è invariante sotto trasformazioni di Lorentz.


\paragraph{1.a.19} \textbf{Dimostrare che 4-accelerazione e 4-velocità sono perpendicolari
} \\ \\
Se si indica la 4-velocità con $u^{\mu}$ e il tempo proprio con $d\tau=dt/\gamma$, la 4-accelerazione risulta:
\begin{equation*}
    a^{\mu}\equiv\frac{du^{\mu}}{d\tau}
\end{equation*}
Risulta, con calcolo esplicito:
\begin{equation*}
    u^{\mu}u_{\mu}=c^2 \ \ \ \Rightarrow \ \ \ 0=\frac{d}{d\tau}( u^{\mu}u_{\mu}))=2u_{\mu}\frac{du^{\mu}}{d\tau}=2u_{\mu}a^{\mu}
\end{equation*}
Conseguentemente il prodotto scalare tra 4-velocità e 4-accelerazione è nullo, per cui i due 4-vettori sono ortogonali.
\paragraph{1.a.20} \textbf{Quanto valgono, in unità MKSA e negli altri sistemi di misura comunemente
utilizzati, le costanti: $c$, $\varepsilon_0$, $\mu_0$, e
$e^2/4\pi$, $\hbar$ ?}\\
\begin{center}
    \begin{tabular}{c|c|c}
         Grandezza & MKSA & cgs  \\
         \hline
         $c$ & $3 \ 10^8 \ m/s$ & $3 \ 10^{10} \ cm/s$\\
         $\varepsilon_0$ & 8.854 \ $10^{-12}$ \ $\frac{C^2}{N \cdot  m^2}$ & $\frac{1}{4\pi}$ \\
         $\mu_0$ & $4\pi \ 10^{-7} \ H/m$ & $\frac{4\pi}{c^2}$ \\
         $e^2/4\pi$ & $2.04 \ 10^{-39} \ C^2$ & $1.83 \ 10^{-20} \ esu^2$ \\
         $\hbar$ & $1.05 \ 10^{-34} \ J\cdot s$ & $1.05 \ 10^{-27} \ erg\cdot s$
    \end{tabular} 
\end{center}
Con $1 \ esu=3,3356 \ 10^{-10} \ C$, $1 \ erg=10^{-7} \ J$.



\paragraph{1.a.21}  \textbf{Quanto vale, entro il 5\%, la costante $\hbar$c in eV-nm e in MeV-fm ?}\\
\\
La costante $\hbar$c vale:
\[
\hbar c= 197.32 \;\text{eV nm}= 197.32 \;\text{MeV fm}
\]
RICORDA: 1 fm = $10^{-15}$ m

\paragraph{1.a.22} \textbf{Spiegare la differenza fra le seguenti categorie di fotoni: infrarossi – visibili –
ultravioletti – raggi X – raggi $\gamma$.}\\
\\
Le diverse categorie di fotoni si distinguono in base alla lunghezza d'onda, o frequenza, o in base all'energia(tutte grandezze collegate tra loro).
\begin{table}[H]
    \centering
    \begin{tabular}{c|c|c|c}
    Fotoni & $\lambda$ [m] & $\nu$ [Hz] & E [eV]\\
    \hline
    Infrarossi  & $10^{-3}$-0.78 $10^{-6}$&  5 $10^{11}$-4 $10^{14}$& 2 $10^{-3}$-1.5\\
    Visibili    & 0.78 $10^{-6}$-0.38 $10^{-6}$&  4 $10^{14}$-8 $10^{14}$&  1.5-3\\
    Ultravioletti   & 0.38 $10^{-6}$-0.6 $10^{-9}$ & 8 $10^{14}$-3 $10^{17}$& 3-$10^{3}$\\
    Raggi X & 5 $10^{-10}$-5 $10^{-12}$ & 3 $10^{17}$- 5 $10^{19}$ & $10^{3}$-2 $10^{5}$\\
    Raggi $\gamma$  & $\leq$ 5 $10^{-12}$ & $\geq$5 $10^{19}$  &$\geq$2 $10^{5}$\\
    \hline
    \end{tabular}
\end{table}
\[
E= h \nu= \frac{h c}{\lambda}
\]

\paragraph{1.a.23} \textbf{Quanto vale la massa del fotone?}\\
\\
Il fotone ha massa nulla.

\paragraph{1.a.24} \textbf{Quanto valgono, entro il 5\%, la carica elettrica dell’elettrone e del protone (in
MKSA)?}\\
\[
e^-=-1.602 \cdot 10^{-19} \;\text{C}
\]
\[
p^+=1.602 \cdot 10^{-19} \;\text{C}
\]

\paragraph{1.a.25}\textbf{Quanto vale, entro il 5\%, la costante di struttura fine ($\alpha$)?}\\
\[
\alpha=\frac{e^2}{4\pi \epsilon_0 \hbar c}=7.29 \cdot 10^{-3} \simeq \frac{1}{137}
\]

\paragraph{1.a.26}\textbf{Quanto valgono, entro il 10\%, la massa dell’elettrone e del protone (in MKSA e in MeV/c2)?}\\
\[
m_e=0.511 \;\text{Mev/$c^2$} = 9.11 \cdot 10^{-31} \; \text{kg}
\]
\[
m_p= 938.27 \;\text{Mev/$c^2$} = 1.67 \cdot 10^{-27} \; \text{kg}
\]

\paragraph{1.a.27}\textbf{Dire se la differenza fra la massa del neutrone e la somma della massa del protone e dell’elettrone sia: i) circa 1 MeV; ii) circa 10 MeV; iii) circa 100 MeV; iv) negativa}\\
\\
\[
m_n-(m_p+m_n)=[939.56-(938.27+0.51)] \; \text{MeV}/c^2=0. \; \text{MeV}/c^2 \sim 1 \text{MeV}/c^2
\]

\paragraph{1.a.28}\textbf{Quanto è l’ordine di grandezza dell’energia media di legame di un elettrone all’interno di un atomo?}\\
\\
L’ordine di grandezza dell'energia media di legame di un elettrone all'interno di un atomo è tra 1 e 100 eV. . Infatti le energie rilasciate o assorbite dalle transizioni atomiche nel visibile sono dell'ordine delle decine di eV.

\paragraph{1.a.29}\textbf{Spiegare la differenza fra ottica fisica ed ottica geometrica}\\
\\
L’ottica fisica è la branca dell'ottica che studia i fenomeni di interferenza, diffrazione, polarizzazione della luce, ovvero tutti quei fenomeni per i quali `e necessario ricorrere alla descrizione del carattere ondulatorio
della luce come radiazione elettromagnetica, applicando quindi le equazioni di Maxwell. L’ottica geometrica è un’approssimazione dell'ottica fisica, che permette di studiare il comportamento della luce quando incontra discontinuità nello spazio in cui si propaga nei casi in cui la lunghezza d’onda della
radiazione visibile è molto minore delle dimensioni lineari degli ostacoli che
incontra. Tale approssimazione ($\lambda \rightarrow 0$) permette di considerare trascurabili i fenomeni di diffrazione.


\paragraph{1.a.30}\textbf{Esprimere tutte le relazioni fra campo elettrico, magnetico e direzione di propagazione di un’onda e.m. piana}\\
\\
In un'onda elettromagnetica il campo elettrico \textbf{E} ed il campo magnetico \textbf{B} sono ortogonali tra loro e ortogonali alla direzione di propagazione $\widehat{\textbf{n}}$. I campi e il vettore d'onda (\textbf{k}) formano una terna ortogonale.
\[
\widehat{\textbf{n}} \cdot \textbf{E} = 0 \;\;\;\;\;\; \widehat{\textbf{n}} \cdot \textbf{B} = 0
\]
Per un'onda piana valgono le seguenti relazioni:
\[
\textbf{E}=\textbf{B} \times \textbf{v} \;\;\;\;\;\; |\textbf{E}|=|\textbf{B}|v
\]
\[
\widehat{\textbf{n}} \times \textbf{E}=v\textbf{B}
 \;\;\;\;\;\; \widehat{\textbf{n}} \times \textbf{B}=-\frac{1}{v}\textbf{E}
\]
In CGS:
\[
\textbf{B}=\textbf{k} \times \textbf{E}
\]
\[
\textbf{E}=  \textbf{B} \times \textbf{k}
\]

\paragraph{1.a.31} \textbf{Dare la definizione di onda piana e.m. monocromatica e delle seguenti quantità:
ampiezza, frequenza angolare, vettore d’onda, frequenza, periodo, lunghezza
d’onda. Scrivere le relazioni esistenti fra le grandezze sopra definite.}\\
\\
Un'onda a frequenza costante i cui fronti d'onda sono infiniti piani paralleli perpendicolari alla direzione di propagazione, e la cui distanza picco-picco è costante.\\
L'onda piana rappresenta un'astrazione matematica che non corrisponde ad alcun fenomeno fisico equivalente in senso stretto, poiché a partire da una descrizione analitica esatta si ottiene un'onda che per essere generata necessita di una sorgente di lunghezza infinita. L'onda piana è tuttavia utilizzata per approssimare il caso in cui la sorgente dell'onda è posta a distanza infinita dal punto di osservazione del fronte d'onda considerato, che viene quindi assunto localmente piano.\\
Una caratteristica che la differenzia da altri tipi di propagazione ondosa, come l'onda sferica o circolare, è l'assenza di attenuazione isotropica nello spazio, in virtù della direzionalità dell'emissione e della propagazione di energia associata all'onda.\\
Si rappresenta in termini di campo EM come:
\begin{equation}
    \begin{cases}
    \Vec{E}(\Vec{r},t)=\Vec{E}_0 e^{i(\Vec{k}\cdot\Vec{r}-\omega t)}= \hat{e}E_0 e^{i(\Vec{k}\cdot\Vec{r}-\omega t)}\\
    \Vec{B}(\Vec{r},t)=\Vec{B}_0 e^{i(\Vec{k}\cdot\Vec{r}-\omega t)}=\hat{e}\times\vec{k} B_0 e^{i(\Vec{k}\cdot\Vec{r}-\omega t)}
    \end{cases}
\end{equation}
\begin{itemize}
    \item ampiezza A: massimo valore della grandezza rappresentativa dell'onda in un periodo;
    \item frequenza angolare $\omega$: grandezza che misura la velocità nel moto armonico, vale inoltre $\omega=2\pi f$;
    \item vettore d'onda $\Vec{k}$: vettore con che punta lungo la direzione di propagazione dell'onda e ha modulo k, detto numero d'onda con $k=\frac{2\pi}{\lambda}$
    \item frequenza f: numero di oscillazioni al secondo;
    \item periodo T: durata in secondi di una oscillazione;
    \item lunghezza d'onda $\lambda$: distanza tra due creste della sua forma d'onda. Detta v la velocità di propagazione dell'onda si ha $\lambda=\frac{v}{f}$.
\end{itemize}
La relazione fra periodo e frequenza è ovviamente $T=\frac{1}{f}$.\\
\paragraph{1.a.32} \textbf{Definire la relazione di dispersione, la velocità di fase e la velocità di gruppo per
un’onda e.m. e spiegarne il loro significato fisico}\\
\\
La dispersione è il fenomeno per cui, in un dato mezzo, la velocità di propagazione di un'onda dipende dalla sua lunghezza d'onda. La particolare legge che, di caso in caso, descrive la dipendenza della velocità dalla lunghezza d'onda si chiama legge di dispersione..\\
Nel caso delle onde elettromagnetiche nel vuoto si ha:
\begin{equation}
    c=\frac{\lambda}{T}=\frac{\omega}{k}
\end{equation}
Mentre nel caso di un mezzo con indice di rifrazione n($\omega$):
\begin{equation}
    \frac{c}{n}=\frac{\omega}{k}
\end{equation}
In caso di mezzo dispersivo (n non costante per variazioni di $\omega$), differenti armoniche viaggiano a diverse velocità, provocando distorsione della forma d'onda iniziale.\\
Si definiscono:
\begin{itemize}
    \item velocità di fase: velocità con cui si propaga ciascuna componente armonica è detta velocità di fase;
    \item velocità di gruppo: velocità con cui si propaga il battimento tra le due (o, più in generale, l'inviluppo dell'onda risultante).
\end{itemize}
In un mezzo non dispersivo queste risultano essere uguali.\\

\paragraph{1.a.33} \textbf{Definire la polarizzazione di un’onda e.m.}\\
\\
E' una caratteristica delle onde elettromagnetiche ed indica la direzione dell'oscillazione del vettore campo elettrico durante la propagazione dell'onda nello spazio-tempo.\\
Riprendendo le rappresentazioni dei campi nel dominio delle frequenze:
\begin{equation}
    \begin{cases}
    \Vec{E}(\Vec{r},t)=\Vec{E}_0 e^{i(\Vec{k}\cdot\Vec{r}-\omega t)}= \hat{e}E_0 e^{i(\Vec{k}\cdot\Vec{r}-\omega t)}\\
    \Vec{B}(\Vec{r},t)=\Vec{B}_0 e^{i(\Vec{k}\cdot\Vec{r}-\omega t)}=\hat{e}\times\vec{k} B_0 e^{i(\Vec{k}\cdot\Vec{r}-\omega t)}
    \end{cases}
\end{equation}
Possono esserci diversi tipi di polarizzazione:
\begin{itemize}
    \item polarizzazione lineare: $\hat{e}$ ha tutte le componenti reali, la direzione rimane costante nel tempo;
    \item polarizzazione circolare: si può vedere come sovrapposizioni di 2 onde piane polarizzate linearmente e sfasate di $\frac{\pi}{2}$, ottenendo 
    \begin{equation}\label{ 1 pol circ}
       \hat{e}=\frac{\hat{e}_1\pm i\hat{e}_2}{\sqrt{2}}
    \end{equation}
    La polarizzazione risultante ruota con frequenza angolare $\omega$ nel piano individuato dai vettori $\hat{e}_1$ ed $\hat{e}_2$ in senso orario se nella (\ref{ 1 pol circ}) è presente il segno +, in senso antiorario con il - (osservando dalla punta del vettore d'onda).
    \item polarizzazione ellitica: si ottiene sommando 2 polarizzazioni lineari sfasate di una fase qualunque. Può esere scomposta anche come somma di 2 polarizzazioni circolari diverse.\\
\end{itemize}

\paragraph{1.a.34} \textbf{In un sistema Oxyz scrivere l‘espressione del campo elettrico e del campo
magnetico di un’onda e.m. piana monocromatica, polarizzata linearmente lungo y
e che si propaga lungo x, sia utilizzando il formalismo reale, sia utilizzando il
formalismo complesso}\\
\\
\begin{equation}
    \begin{cases}
    \vec{E}=(0,E_0e^{i(kx-\omega t)},0)\\
    \vec{B}=(0,0,B_0e^{i(kx-\omega t)})\\
    \end{cases} \Longleftrightarrow \begin{cases}
    \vec{E}=(0,E_0cos(kx-\omega t),0)\\
    \vec{B}=(0,0,B_0cos(kx-\omega t))\\
    \end{cases}
\end{equation}
Con $B_0=\frac{k}{\omega}E_0$.\\

\paragraph{1.a.35} \textbf{Enunciare e spiegare il principio di Huygens}\\
\\
Ogni elemento d$\Sigma$ di un fronte d'onda $\Sigma$  si può considerare formalmente come una sorgente secondaria di onde sferiche in fase con la primaria e di ampiezza proporzionale a quella dell'onda primaria e all'area d$\Sigma$. La perturbazione prodotta in un punto dello spazio si può sempre ottenere come sovrapposizione di tutte le onde sferiche secondarie che raggiungono quel punto.\\
Si immagini, ad esempio, di avere uno schermo opaco con una singola fenditura, su cui incide un'onda piana monocromatica. Secondo il principio di Huygens, il fronte d'onda che si propaga oltre tale fenditura è dovuto all'interferenza creata esclusivamente dai punti nella fenditura, sorgenti secondarie di onde sferiche. Questo fenomeno è interpretabile anche come segue. L'onda incidente sullo schermo opaco è assorbita dagli elettroni presenti sullo schermo. Essi entrano in oscillazione e, di conseguenza, emettono onde elettromagnetiche principalmente alla stessa frequenza con cui oscillano. Si trova che l'effetto dovuto alla somma di tutti i contributi di tali onde coincide con quello ottenuto considerando come sorgenti unicamente i punti della fenditura.\\

\paragraph{1.a.36}\textbf{Definire e calcolare l'impedenza del vuoto, e chiarire il suo significato fisico}\\
\\
\begin{equation}
    Z_0=\sqrt{\frac{\mu_0}{\epsilon_0}}=376.7303113461 \quad \text{in MKSA}
\end{equation}
In MKSA il vettore di Poynting è definito come:
\begin{equation}
    \vec{S}=\frac{|\vec{E}_0|^2}{Z_0}cos(\omega t-\vec{k}\cdot \vec{r})\hat{k}
\end{equation}
l'impedenza del vuoto è legata alla resistività
superficiale di un materiale che assorbe senza riflettere le onde elettromagnetiche piane, tipicamente nella regione delle microonde
~3GHz < f < ~300GHz. Bati dixit, io non ho capito.\\FA

\paragraph{1.a.37} \textbf{Definire - in CGS e in MKSA - per un sistema di cariche e correnti elettriche: i)
momento di dipolo elettrico; ii) momento di quadrupolo elettrico; iii) momento
di dipolo magnetico}\\
\begin{itemize}
    \item momento di dipolo elettrico per una distribuzione di carica $\phi$ :
    \begin{equation}
        \Vec{p}=\int\phi(x',y',z')\vec{r}'d\tau'
    \end{equation}
    \item momento di quadripolo elettrico per la configurazione precedente:
    \begin{equation}
        Q(\hat{r})=\int \phi(x',y',z')\left[\frac{3}{2} (\vec{r}'\cdot \hat{r})^2-\frac{(r')^2}{2}\right]d\tau'
    \end{equation}
    \item momento di dipolo magnetico:
    \begin{equation}
        \vec{m}=\frac{1}{2}\int \vec{r}' \times \vec{J}(x',y',z') d\tau'
    \end{equation}
\end{itemize}
    
\paragraph{1.a.38} \textbf{Calcolare, a partire dalle EDM, la velocità delle onde elettromagnetiche in un
mezzo omogeneo, lineare ed isotropo.}\\
\\
Le equazioni di Maxwell in un mezzo omogeneo ed isotropo sono:
\begin{equation}
    \begin{cases}
    \nabla \cdot \vec{E}\\
    \nabla \cdot \Vec{B}\\
    \nabla \times \vec{E}=-\frac{\partial \vec{B}}{\partial t}\\
    \nabla \cdot \vec{B}=\epsilon\mu \frac{\partial \vec{E}}{\partial t}\\
    \end{cases}
\end{equation}
Si applichi ora il rotore alla III equazione:
\begin{equation}
    \nabla\times\nabla\times \vec{E}=-\nabla^2E+\nabla(\nabla\cdot \vec{E}) \Longrightarrow \nabla^2 \vec{E}=\frac{\partial }{\partial t} (\nabla \times \vec{B})=0
\end{equation}
Da cui infine, applicando analogo procedimento alla IV equazione:
\begin{equation}
    \nabla^2\vec{E}-\epsilon_0\mu\frac{\partial^2\vec{E}}{\partial t^2}=0 \qquad \nabla^2\vec{B}-\epsilon_0\mu\frac{\partial^2\vec{B}}{\partial t^2}=0
\end{equation}
Queste sono quindi equazioni delle onde elettromagnetiche, la cui soluzione è rappresentata da onde che si propaga con velocità $v=\frac{1}{\sqrt{\epsilon\mu}}$.
\paragraph{1.a.39} \textbf{Esprimere la densità di energia di un’onda e.m. piana in funzione dei campi elettrico e/o magnetico}.\\
\\
Nel vuoto
\begin{equation}
    u=\epsilon \frac{E^2}{2}+\epsilon c^2\frac{B^2}{2}  \quad \text{in MKSA} u= \frac{E^2}{8\pi}+\frac{B^2}{8\pi} \quad \text{in CGS}
\end{equation}

\paragraph{1.a.40}\textbf{ Dare la definizione ed esprimere il vettore di Poynting di un’onda e.m. piana in
funzione del campo elettrico e/o magnetico}\\
\begin{equation}
    \vec{S}=\epsilon_0 c^2\vec{E}\times \vec{B}\quad \text{in MKSA} \qquad \vec{S}=\frac{c}{4\pi}\vec{E}\times \vec{B} \quad \text{in CGS}
    \end{equation}
Dove si è supposto che di trovarsi nel vuoto.\\
Esso risulta definito a meno del rotore di un vettore.\\
\paragraph{1.a.41} \textbf{Esprimere la pressione (di radiazione) che un campo e.m. esercita su una superficie piana.}\\
In MKSA:
\begin{equation}
    P=\frac{2I}{c} \qquad \text{con } I=<|\vec{S}|>=\frac{\epsilon_0 c}{2}E_0^2
\end{equation}
Dove si è supposto che la superfice sia perfettamente riflettente.\\ 
\paragraph{1.a.42} \textbf{Dare la definizione di interferenza e diffrazione; di interferenza costruttiva e
distruttiva.}\\
\\
\textbf{L'interferenza} è un fenomeno dovuto alla sovrapposizione, in un punto dello spazio, di due o più onde. Quello che si osserva è che l'intensità (o ampiezza) dell'onda risultante in quel punto può essere diversa rispetto alla somma delle intensità associate ad ogni singola onda di partenza; in particolare, essa può variare tra un minimo, in corrispondenza del quale non si osserva alcun fenomeno ondulatorio, ed un massimo che in generale non coincide con la somma delle intensità.\\
\textbf{La diffrazione}, nella fisica, è un fenomeno associato alla deviazione della traiettoria di propagazione delle onde quando queste incontrano un ostacolo sul loro cammino. È tipica di ogni genere di onda, come il suono, le onde sulla superficie dell'acqua o le onde elettromagnetiche come la luce o le onde radio; il fenomeno si verifica anche nelle particolari situazioni in cui la materia mostra proprietà ondulatorie, in accordo con il dualismo onda-particella.\\
Gli effetti di diffrazione sono rilevanti quando la lunghezza d'onda è comparabile con la dimensione dell'ostacolo: in particolare per la luce visibile (lunghezza d'onda attorno a 0,5 µm) si hanno fenomeni di diffrazione quando essa interagisce con oggetti di dimensione sub-millimetrica.




%-----------------------------------------------------
\newpage
\section{Riepilogo e completamento della formulazione classica dell'interazione elettromagnetica}
\epigraph{The story so far:\\ In the beginning the Universe was created.
This has made a lot of people very angry and been widely regarded as a bad move.}{}
\textit{Domande a}\rule{13.7 cm}{0.4pt}

\paragraph{2.a.1}\textbf{Dare la definizione di quadri-corrente e di quadri-potenziale del campo
elettromagnetico.}\\ 
\begin{equation*}
    j^{\mu}=(c\rho,\vec{j}) \ \ \ \ ; \ \ \ \ A^{\mu}=(\phi,\vec{A})
\end{equation*}

\paragraph{2.a.2}\textbf{Dare la definizione del tensore del campo elettromagnetico e scriverne le
componenti.}\\ 
\begin{equation*}
    F^{\mu \nu}=\partial^{\mu}A^{\nu}-\partial^{\nu}A^{\mu}=\begin{pmatrix}
    0 & -E_x & -E_y & - E_z \\
    E_x & 0 & -B_z & B_y \\
    E_y & B_z & 0 & -B_x \\
    E_z & -B_y & B_x & 0
    \end{pmatrix}
\end{equation*}
\textbf{N.B.:}Si ricordi che $\partial_\mu=\frac{1}{c}\partial_t+\nabla$ e $\partial^\mu=\frac{1}{c}\partial_t-\nabla$.\\
E' un tensore antisimmetrico di tipo (2,0).
\paragraph{2.a.3}\textbf{Dare la definizione della "densità di energia" del campo elettromagnetico, del
"vettore di Poynting" e del "tensore degli sforzi di Maxwell".
}\\ \\
Sia nel sistema CGS che nell'MKSA la densità di energia del campo elettromagnetico è definita secondo:
\begin{equation*}
    u=u_e+u_m=\frac{1}{2}\vec{E}\cdot\vec{D}+\frac{1}{2}\vec{H}\cdot \vec{B} 
\end{equation*}
\\
Nei 2 sistemi di riferimento si ottiene:
\begin{equation}
    u=\frac{1}{2}\epsilon_0 |\vec{E}|^2+\frac{1}{2}\frac{|\vec{B}|^2}{\mu_0}
    \quad \text{MKSA} \qquad u=\frac{1}{8\pi}|\vec{E}|^2+\frac{1}{8\pi}|\vec{B}|^2 \quad \text{CGS}
\end{equation}
Nel sistema CGS il Vettore di Poynting è definito come:
\begin{equation*}
    \vec{S}=\frac{c}{4\pi}\vec{E}\wedge \vec{B}
\end{equation*}

Mentre nell'MKSA:
\begin{equation*}
    \vec{S}=\frac{\vec{E}\wedge\vec{B}}{\mu}
\end{equation*}

Nel sistema CGS il Tensore degli sforzi di Maxwell è definito come:
\begin{equation*}
    T_{ij}=\frac{1}{4\pi}\left( E_i E_j + B_i B_j - \frac{1}{2}(E^2+B^2)\delta_{ij} \right)
\end{equation*}
Mentre nell'MKSA:
\begin{equation*}
    T_{ij}=\varepsilon_0E_i E_j+\frac{1}{\mu_0}B_iB_j-\frac{1}{2}\left(\varepsilon_0E^2+\frac{1}{\mu_0}B^2\right)\delta_{ij}
\end{equation*}
\textbf{NB:} I tensori scritti sopra hanno entrambi gli indici covarianti. Con entrambi gl indici controvarianti comparirebbe un segno meno globale nell'espressione. \\
\textbf{N.B.2:} facendo correre l'indice i da 1 a 3 si ottiene il tensore degli sforzi vero è proprio (3D) noto come $\sigma_j$, se l'indice parte da 0 fino a 3 si ottiene il tensore eneriga-impulso.


\paragraph{2.a.4}\textbf{Scrivere le equazioni di Maxwell (sia quelle non omogenee che quelle omogenee)
in forma covariante.}\\ \\
Introducendo il Tensore duale del tensore di campo 
\begin{equation}
    \Tilde{F}^{\mu \nu}=\frac{1}{2}\varepsilon^{\mu \nu \rho \sigma}F_{\rho \sigma}
\end{equation}
Le equazioni non omogenee possono essere scritte in forma covariante come:
\begin{equation}\label{non omogenea}
    \partial_{\mu}F^{\mu \nu}=\frac{4 \pi }{c}j^{\nu}
\end{equation}
E quelle omogenee come:
\begin{equation}
    \partial_{\mu}\Tilde{F}^{\mu \nu}=0
\end{equation}

\paragraph{2.a.5}\textbf{Scrivere l'equazione di continuità per la quadri-corrente in forma covariante e
verificarne la consistenza con le equazioni di Maxwell.}\\ \\
L'equazione di continuità, facendo uso della 4-corrente e della derivata covariante (quadridivergenza) si scrive come:
\begin{equation}
    \partial_{\mu}j^{\mu}=\frac{\partial \rho}{\partial t}+\vec{\nabla}\cdot \vec{j}=0
\end{equation}
Ma, in base a quanto visto sopra, applicando la 4-divergenza a dx e a sx della (\ref{non omogenea}) :
\begin{equation}
    \frac{4\pi}{c}\partial_{\nu}j^{\nu}=\partial_{\nu}\partial_{\mu}F^{\mu \nu}=0
\end{equation}
In quanto il tensore dei campi è antisimmetrico, mentre l'operatore $\partial_{\mu}\partial_{\nu}$ è simmetrico.

\paragraph{2.a.6}\textbf{Dare la definizione di "Gauge di Lorenz" e "Gauge di Coulomb"}\\
\\
Nelle equazioni di Maxwell, su i potenziali $\mathbf{A}$ e $\phi$ subiscono le seguenti trasformazioni le espressioni dei campi rimangono invariate:
\begin{align}
    \vec{A}'\longrightarrow\vec{A}+\nabla\lambda\\
    \phi'\longrightarrow\phi-\frac{1}{c}\frac{\partial\lambda}{\partial t}
\end{align}
dove $\lambda$ è una funzione scalare. Questa è una trasformazione di gauge e lascia i campi invariati, infatti per $A\longrightarrow A+\nabla\lambda$:
\begin{align*}
    B&=\nabla\times (\vec{A}+\nabla\lambda)=\nabla\times \vec{A}\\
    E&=-\nabla\left(\phi-\frac{1}{c}\frac{\partial\lambda}{\partial t}\right)-\frac{1}{c}\frac{\partial \mathbf{A}}{\partial t}-\nabla\frac{\partial\lambda}{\partial t}=-\nabla\phi-\frac{1}{c}\frac{\partial \vec{A}}{\partial t}
\end{align*}
Riassumendo:
\begin{equation}
    A^\mu\to A^\mu-\partial^\mu \lambda
\end{equation}
Da un altro punto di vista risulta che il tensore dei campi $F^{\mu\nu}$ è invariante per la trasformazione:
\begin{equation}
    F^{\mu\nu} \ \to \ F^{\mu \nu}+\partial_\nu \lambda
\end{equation} Ossia è invariante per somma di una quadrigradiente di una funzione ($\lambda$) scalare.
\\
Definiamo a questo punto le gauge di Lorenz e gauge di Coulomb come quei potenziali che soddisfano le condizioni:
\\
\textit{Gauge di Lorenz}\\
\begin{equation*}
    \tensor{\partial}{_{\mu}}\tensor{A}{^{\mu}}=\nabla\cdot\mathbf{A}+\frac{1}{c}\frac{\partial\phi}{\partial t}=0
\end{equation*}
e la condizione è invariante per $\lambda$ che soddisfa l'equazione
\begin{equation*}
    \tensor{\partial}{_{\mu}}\tensor{\partial}{^{\mu}}\lambda=\tensor{\partial}{_{\nu}}\tensor{A}{^{\nu}}
\end{equation*}
\textit{Gauge di Coulomb}\\
\begin{equation*}
    \nabla\cdot\mathbf{A}=0
\end{equation*}
e $\lambda$ deve soddisfare 
\begin{equation*}
    \nabla^2\lambda=-\nabla\cdot\mathbf{A}
\end{equation*}

\paragraph{2.a.7}\textbf{Scrivere la legge di trasformazione di Lorentz del campo elettrico e del campo magnetico (distinguendo fra componenti parallele e componenti perpendicolare al "boost")}\\ \\
La possibilità di scrivere i campi elettrico e magnetico attraverso un tensore ci permette di ottenere facilmente le espressioni per trasformazioni di Lorentz attraverso
\begin{equation*}
    {F'}^{\mu\nu}=\tensor{\Lambda}{^{\mu}_{\rho}}\tensor{\Lambda}{^{\nu}_{\sigma}}F^{\rho\sigma}
\end{equation*}
utilizzando un boost verso l'asse x 
\begin{equation*}
    \tensor{\Lambda}{^{\mu}_{\nu}}=
    \begin{pmatrix}
        \gamma & -\gamma\beta & 0 & 0 \\
        -\gamma\beta & \gamma & 0 & 0 \\
        0 & 0 & 1 & 0 \\
        0 & 0 & 0 & 1 \\
    \end{pmatrix}
\end{equation*}
otteniamo
\begin{alignat*}{2}
    {E'}_x&=E_x,&\qquad{B'}_x&=B_x\\
    {E'}_y&=\gamma(E_y-\beta B_z),&\qquad{B'}_y&=\gamma(B_y+\beta E_z)\\
    {E'}_z&=\gamma(E_z+\beta B_y),&\qquad{B'}_z&=\gamma(B_z-\beta E_y)\\
\end{alignat*}
Possiamo eliminare la dipendenza del boost dall'asse x utilizzando le componenti parallele e perpendicolari:
\begin{alignat}{2}
    {\vec{E}'}_{\|}&=\vec{E}_{\|},\qquad{\vec{E}'}_{\perp}&=\gamma(\vec{E}_{\perp}+\vec{\beta}\times\vec{B})\\
    {\vec{B}'}_{\|}&=\vec{B}_{\|},\qquad{\vec{B}'}_{\perp}&=\gamma(\vec{B}_{\perp}-\vec{\beta}\times\vec{E})
\end{alignat}
\paragraph{2.a.8}\textbf{Dare la definizione del 4-vettore "Densità di forza di Lorentz"}\\ \\
La densità di forza di Lorentz è definita come
\begin{equation*}
    \tensor{G}{^{\mu}}=\frac{\text{d}p^{\mu}}{\text{d}t\text{d}V}=\biggl(\frac{1}{c}\frac{\text{d}\mathcal{E}}{\text{d}t\text{d}V},\frac{\text{d}\vec{p}}{\text{d}t\text{d}V}\biggr)=\frac{1}{c}F^{\mu \nu}J_\nu
\end{equation*}
La componente G$^0$ è data dalla variazione dell'energia cinetica per unità di tempo e volume, ovvero il lavoro fatto dal campo per unità di tempo e volume.\\
Ricordando che $d\varepsilon=\vec{v}\cdot\vec{p}$:
\begin{equation*}
    \tensor{G}{^{\mu}}=\frac{\text{d}p^{\mu}}{\text{d}t\text{d}V}=\biggl(\frac{1}{c}\frac{\text{d}\mathcal{E}}{\text{d}t\text{d}V},\frac{\text{d}\vec{p}}{\text{d}t\text{d}V}\biggr)=\biggl(\frac{1}{c}\vec{E}\cdot \vec{j},\rho \vec{E}+\frac{1}{c}\vec{j}\times\vec{B}\biggr)=\frac{1}{c}F^{\mu \nu}J_\nu
\end{equation*}
\paragraph{2.a.9}\textbf{Ricavare le espressioni dell'effetto Doppler relativistico(calcolo della frequenza e del'angolo misurati dal rivelatore nel caso di moto relativo fra sorgente e rivelatore stesso).}\\ \\
Consideriamo una sorgente S di onde elettromagnetiche posta nell'origine di un sistema $O'$ che si muove con $\mathbf{v}=(v,0,0)$ rispetto a un sistema $O$ dove è posto un rilevatore. S emette onde elettromagnetiche con un angolo $\theta'$ rispetto all'asse $x'$ con una frequenza propria $\omega'$. Il 4-vettore d'onda $\tensor{{k'}}{^{\mu}}$ in $O'$ è:
\begin{equation*}
    \tensor{{k'}}{^{\mu}}=(\omega',\omega'\cos\theta',\omega'\sin\theta',0)
\end{equation*}
dove abbiamo posto $c=1$, dato che $k=\omega/c$ possiamo porre $k=\omega$. Nel sistema $O$ il 4-vettore d'onda è 
\begin{equation*}
    \tensor{k}{^{\mu}}=(\omega,\omega\,\cos\theta,\omega\,\sin\theta,0)
\end{equation*}
Per ottenere la frequenza $\omega$ ricevuta dal rilevatore effettuiamo una trasformazione di Lorentz inversa sulla componente ${k'}^0$:
\begin{equation*}
    \omega=\gamma\omega'(1+\beta\text{cos}\theta')=\omega'\frac{1+\beta\cos\theta'}{\sqrt{1-\beta^2}}\, ,
\end{equation*}
analogamente per trovare l'angolo effettuiamo una trasformazione inversa sulle componenti ${k'}_x$ e ${k'}_y$
\begin{align*}
\omega\cos\theta&=\gamma\omega'(\cos\theta'+\beta) \\ 
\omega\sin\theta&=\omega'\sin\theta' \, .
\end{align*}
Dividendo la seconda equazione per la prima otteniamo:
\begin{equation*}
    \tan\theta=\frac{\sin\theta'}{\gamma(\cos\theta'+\beta)}
\end{equation*}
\paragraph{2.a.10}\textbf{Scrivere l’espressione per i potenziali ritardati ($\phi$ ed $\vec{A}$) per una qualunque
distribuzione di cariche ($\rho$) e correnti ($\vec{j}$ ).}\\
\[
\phi(\vec{r},t)=\int\frac{\rho(\vec{r}', t- |\vec{r}-\vec{r}'|/c)}{|\vec{r}-\vec{r'}|} d^3r'
\]
\[
\vec{A}(\vec{r},t)=\frac{1}{c}\int \frac{\vec{j}(\vec{r}', t- |\vec{r}-\vec{r}'|/c)}{|\vec{r}-\vec{r'}|} d^3r'
\]


\paragraph{2.a.11}\textbf{Spiegare tutti i termini dell'espressione
\[
\bar{E}=\left[\frac{q}{R^2}\frac{\hat{n}-\vec{\beta}}{\gamma^2(1-\hat{n}\cdot \vec{\beta})^3}+ \frac{q}{R c}\frac{\hat{n}\times[(\hat{n}-\vec{\beta})\times \dot{\vec{\beta}}]}{(1-\hat{n}\cdot \vec{\beta})^3}\right]_{t'=t-R/c} 
\]
per il campo elettrico generato da una carica puntiforme in moto arbitrario
}
\begin{itemize}
    \item q:carica della particella
    \item R=$|\vec{R}|$=$|\vec{r}-\vec{r}'|$: distanza tra la particella e il punto in cui si vuole calcolare il campo elettrico $\vec{E}$;
    \item $\hat{n}$: versore parallelo al vettore $\vec{R}$;
    \item $\vec{\beta}$= $\frac{\vec{v}}{c}$;
    \item $\gamma=\frac{1}{\sqrt{1-\beta^2}}$.
\end{itemize}
In particolare si riconoscono due contributi dovuti alla velocità e all'accelerazione della carica in moto.\\
Il primo termine è detto \textit{campo di prossimità}, è il campo relativo alla velocità ed è proporzionale a $\frac{1}{R^2}$ per cui decresce velocemente all'aumentare della distanza.\\
Il secondo termine è detto \textit{campo di radiazione}, è il campo relativo all'accelerazione ed è proporzionale a $\frac{1}{R}$, a grandi distanze possiamo approssimare il campo elettrico totale con solo questo termine, è inoltre il principale responsabile della perdita di energia.


\paragraph{2.a.12}\textbf{Dare la definizione di “solido di radiazione” e di “diagramma di radiazione” per una carica accelerata}\\
\\
Il "solido di radiazione" è un solido costruito riportando, con origine nel punto dove si trova la carica, nelle diverse direzioni segmenti di lunghezza proporzionale all'intensità del campo elettrico nelle direzioni stesse, misurate su una superficie sferica di raggio sufficientemente grande. Nella \textbf{Figura (\ref{Solid}) } di sinistra le frecce indicano il modulo del campo elettrico ed il verso la direzione di emissione della radiazione, la figura di destra è il solido di radiazione di un'antenna Yagi-Uda. \\
\begin{figure}[H]
    \centering
    \includegraphics[scale=0.5]{solidorad.PNG}
    \label{Solid}
\end{figure}
Il "diagramma di radiazione" sono le curve che si ottengono sezionando il solido di radiazione con opportuni piani.
\begin{figure}[H]
    \centering
    \includegraphics[scale=0.5]{diagrammarad.PNG}
\end{figure}

\paragraph{2.a.13}\textbf{Quanto vale il campo magnetico generato da una carica puntiforme in moto arbitrario se è noto il campo elettrico?}\\
\[
\textbf{B}(\textbf{r},t)=\hat{n} \times \textbf{E}
\] 
dove \textbf{E} è l'espressione riportate nel paragrafo \textbf{2.a.11} sempre calcolata al tempo ritardato.
\[
\vec{E}=\left[\frac{q}{R^2}\frac{\hat{n}-\vec{\beta}}{\gamma^2(1-\hat{n}\cdot \vec{\beta})^3}+ \frac{q}{R c}\frac{\hat{n}\times[(\hat{n}-\vec{\beta})\times \dot{\vec{\beta}}]}{(1-\hat{n}\cdot \vec{\beta})^3}\right]_{t'=t-R/c}
\]
I campi ottenuti sono sono i campi associati ai potenziali di Lienard-Wiechert (pag.51-52 Bonati).



\paragraph{2.a.14}\textbf{Spiega tutti i termini della espressione
\[
 \frac{dI_\omega}{d\Omega}=\frac{q^2}{4\pi^2c}\left|\int \frac{\hat{n}\times[(\hat{n}-\vec{\beta})\times \dot{\vec{\beta}}]}{(1-\hat{n}\cdot \vec{\beta})^2} e^{i\omega(t'-\frac{\vec{r}'\cdot\hat{n}}{c})} dt'\right|^2
\]
}\\
\\
L'espressione rappresenta l'energia totale irraggiata per unità di angolo solido e di frequenza da una carica in moto. I termini a secondo membro rappresentano:
\begin{itemize}
    \item q: carica della particella;
    \item $\hat{n}$: versore del vettore che collega carica ed osservatore, essendo molto lontani dalla sorgente $\hat{n}\simeq \frac{\vec{r}}{|\vec{r}|}$, con r vettore posizione dell'osservatore;
    \item c: non c'è bisogno di dirlo;
    \item $\vec{\beta}= \frac{\vec{v}}{c}$;
    \item $\omega$: frequenza angolare;
    \item $t'$:  tempo impiegato dal segnale che parte dalla particella per arrivare al punto di osservazione;
    \item $r'$: vettore posizione della particella.
\end{itemize}
Dentro l'integrale si trova l'espressione del campo elettrico di radiazione.

DIMOSTRAZIONE\\
Supponendo di trovarsi a grandi distanze il campo di radiazione:
\begin{equation}
    \vec{E}=\left[\frac{q}{R c}\frac{\hat{n}\times[(\hat{n}-\vec{\beta})\times \dot{\vec{\beta}}]}{(1-\hat{n}\cdot \vec{\beta})^3}\right]_{t'=t-R/c} \qquad \vec{B}=\hat{n}\times \vec{E}
\end{equation}
L'energia irraggiata per unità di angolo solido è data da:
\begin{equation}
    \frac{d\epsilon}{d\Omega}=\biggl[\int_{-\infty}^\infty|\vec{S}|R^2dt\biggr]_{t'}=\biggl[\frac{c}{4\pi}\int_{-\infty}^\infty |\vec{E}|^2R^2dt\biggr]_{t'}=\biggl[\frac{q^2}{4\pi c}\int_{-\infty}^\infty \biggl|\frac{\hat{n}\times[(\hat{n}-\vec{\beta})\times \dot{\vec{\beta}}]}{(1-\hat{n}\cdot \vec{\beta})^3}\biggr|^2dt\biggr]_{t'}
\end{equation}
Usando il teorema di Parseval si può scrivere:
\begin{equation}
    \frac{d\epsilon}{d\Omega}=\frac{q^2}{c}\int_0^\infty |\hat{f}(\omega)|^2d\omega
\end{equation}
Dove $\hat{f}$ è la trasformata di Fourier della funzione integranda e l'integrale è esteso alle sole funzioni reali perchè la funzione di partenza è reale: $\hat{f}(-\omega)=\hat{f}*(\omega)$.\\
E' possibile interpretare allora $\frac{q^2}{c}|\hat{f}(\omega)|^2$ come l'energia irraggiata per unità di frequenza e di angolo solido.\\
Si calcola ora $\hat{f}(\omega)$ facendo il cambio di variabili t$\to t'$ con:
\begin{equation}
    t=t'+\frac{|\vec{R}-r'(t')|}{c}\xrightarrow{|\vec{R}|\gg|\vec{r}'|}\ t'+\frac{|\vec{R}|}{c}-\frac{\hat{n}\cdot\vec{r'}}{c} \Longrightarrow \frac{dt}{dt'}=1-\hat{n}\cdot\beta
\end{equation}
Da cui:
\begin{equation}
   \frac{dI_\omega}{d\Omega}=\frac{q^2}{4\pi^2c}\left|\int \frac{\hat{n}\times[(\hat{n}-\vec{\beta})\times \dot{\vec{\beta}}]}{(1-\hat{n}\cdot \vec{\beta})^2} e^{i\omega(t'-\frac{\vec{r}'\cdot\hat{n}}{c})} dt'\right|^2
\end{equation}
\paragraph{2.a.15}\textbf{Una carica elettrica Q si muove con velocità costante (relativistica) di modulo V su una retta, a distanza b da tale retta si trova un osservatore che misura il campo elettrico e magnetico generato dalla carica. Quanto è l'ordine di grandezza del tempo in cui l'osservatore misura un campo elettrico che sia almeno la metà del campo elettrico massimo misurato?}\\
\\
Sia la carica in moto lungo l'asse x e l'osservatore posto in (0,0,b).\\
Mi pongo nel sistema di riferimento S' in cui la carica è a riposo. In tale sistema di riferimento il campo è:
\begin{equation}
    \vec{E'}=\frac{1}{4\pi\epsilon_0}\frac{q\vec{R'}}{|\vec{R'}|^3} \qquad \vec{B'}=0
\end{equation}
Ritorno al sistema di riferimento S sapendo che:
\begin{equation}
    E_x=E_{\paral}'=\frac{q}{4\pi\epsilon_0}\frac{x'}{|\vec{R'}|^3} \qquad E_y= \gamma\frac{q}{4\pi\epsilon_0}\frac{y'}{|\vec{R'}|^3} \qquad E_z= \gamma\frac{q}{4\pi\epsilon_0}\frac{z'}{|\vec{R'}|^3}
\end{equation}
Riscrivo ora $\vec{R'}$ in funzione delle coordinate di S:
\begin{equation}
    |\vec{R}'|=\sqrt{\gamma^2(x-vt)^2+y^2+z^2}
\end{equation}
Dato che nel sistema S l'osservatore si trova in (0,0,b):
\begin{equation}
    |\vec{R}'|=\sqrt{\gamma^2v^2t^2+b^2}\equiv R
\end{equation}
Si ottiene dunque:
\begin{equation}
\begin{cases}
    E_x=-\frac{q}{4\pi\epsilon 0}\frac{\gamma vt}{R^3} \\ E_y=0 \\ E_z= \gamma\frac{q}{4\pi\epsilon 0}\frac{b}{R^3}
\end{cases}
\end{equation}
Se ne calcola il modulo:
\begin{equation}\label{trasc}
    |\vec{E}|^2=Q^2k_0^2\gamma^2\frac{v^2t^2+b^2}{(\gamma^2v^2t^2+b^2)^3}
\end{equation}
Il massimo risulta essere per t=0:
\begin{equation}
    E_{max}=\frac{Qk_0\gamma}{b^2}
\end{equation}
Se la velocità della particella è relativistica il termine con il parametro di impatto b risulta essere trascurabile nell'equazione \ref{trasc}.\\
Per cui si ottiene: $|\vec{E}|=\frac{qk_0}{\gamma^2v^2t^2}$.\\
Imponendo che il campo sia la metà del valore massimo si ottiene:
\begin{equation}\label{koko}
    t_{1/2}=\frac{\sqrt{2}b}{v\gamma^{3/2}}
\end{equation}

Se $\beta\sim 0.8$, sostituendo $|\vec{E}|_{t=0}/2=\frac{e\gamma}{2 b^2}$ nella \ref{koko} si ottiene un equazione in t, risolvendo si ha $t_{\frac{1}{2}}\sim\frac{b}{v\gamma}$.\\

\paragraph{2.a.16}\textbf{Enunciare il principio di Babinet}\\
\\
\underline{Principio di Babinet}: La figura di diffrazione prodotta da un corpo opaco è identica a quella prodotta da un’apertura complementare con la stessa forma e dimensione.

\paragraph{2.a.17}\textbf{Definire il fattore di forma per un'onda che incide su su sistema}\\
\\
Il fattore di forma è la trasformata di Fourier della funzione caratteristica del sistema $\Sigma$:
\begin{equation}
    R(x',y',z')=\begin{cases}
        1 & (x',y',z')\in \Sigma\\
        0 & (x',y',z')\notin \Sigma
    \end{cases}
\end{equation}
Per cui:
\begin{equation}
    \vec{F}(\vec{k})=\int_{-\infty}^\infty\int_{-\infty}^\infty\int_{-\infty}^\infty R(x',y',z')e^{-ik_x x'}e^{-ik_y y'}e^{-ik_z z'}dx'dy'dz'
\end{equation}
Nel caso si tratti di una distribuzione di carica $\phi(\vec{r})$, il fattore di forma elettromagnetico si definisce come:
\begin{equation}
    F(\vec{q})=\frac{\int \phi(\vec{r})e^{-i\vec{r}\cdot \vec{q}}d^3r}{\int \phi(\vec{r})d^3r}
\end{equation}
Dove $\vec{q}$ è l'impulso trasferito, definito come $\vec{q}=\vec{k}-\vec{k}_0$, con $\vec{k}_0$ impulso entrante e $\vec{k}$ impulso uscente.
\paragraph{2.a.18}\textbf{Effettuare un disegno, qualitativo, del solido di radiazione per una carica che acceleri linearmente o in moto circolare uniforme.}\\
\\
Per capire come disegnare il solido di rotazione si considera la potenza irraggiata per unità di angolo solido, poiché essa è proporzionale a $|\vec{E}^2|$.\\
Nel caso di accelerazione lineare si ha:
\begin{equation}
    \frac{dP}{d\theta} \propto \frac{sin^2\theta}{(1-\beta cos\theta)^5}
\end{equation}

Nel caso non realtivistico si ritrova la dipendenza da $sin^2\theta$ e la maggior parte della radiazione si ha per $\theta=\pi/2$. \\
Nel limite ultrarelativistico si ha aberrazione angolare
\begin{equation}
    tan\theta=\frac{1}{\gamma}\frac{sin\theta_0}{\beta+cos\theta_0}
\end{equation}
e quasi tutta la radiazione viene emessa a $\theta \simeq 1/\gamma$, dal momento che $\theta_0=\pi/2 \to \theta\sim 1/\gamma$ nel limite $\gamma \gg 1$.
\\
\begin{figure}[H]
    \centering
    \includegraphics[width=0.6\linewidth, height=6cm]{photo5803239822310094361.jpg}
\end{figure}

Nel caso di accelerazione in moto circolare uniforme si ha:
\begin{equation}
 \frac{dP}{d\Omega} \propto \frac{1}{(1-\beta cos \theta)^3}\left(1-\frac{sin^2\theta sin^2\phi}{\gamma^2 (1-\beta cos\theta)^2}\right)
\end{equation}
Come nel caso di accelerazione lineare la maggior parte della radiazione viene emessa nel limite ultrarelativistico ad un angolo $\bar{\theta}\sim 1/\gamma$.

\textit{Domande b}\rule{13.7cm}{0.4pt}
    

\paragraph{2.b.1}\textbf{Dato il campo elettrico di un'onda progressiva monocromatica nel vuoto $\mathbf{E= ( E_0 \cos{(kz-\omega t)}, 0, 0 )} $, trovare una espressione possibile per il 4-potenziale $\tensor{A}{_\mu}$ e specificare il gauge utilizzato. Riformulare e risolvere il problema in un sistema di riferimento in moto relativo lungo l'asse z.}\\   
\\
Utilizzando la gauge di Lorenz $\partial_\mu A^\mu =0$. L'ulteriore libertà sulle componenti del quadripotenziale permette di fissare $A^0=\phi=0$.
Dunque:
\begin{align*}
    \mathbf{E}=-\frac{1}{c}\frac{\partial \mathbf{A}}{\partial t}\ \ \Rightarrow\ \  E_0 \cos{(kz-\omega t)}=-\frac{1}{c}\frac{\partial A_x}{\partial t}
\end{align*}
e quindi una possibile forma per il potenziale è
\begin{align*}
    \phi=0,\ \ \mathbf{A}=\biggl(-\frac{cE_0}{\omega}\sin{(\mathbf{kz}-\omega t)},0,0\biggr)
\end{align*}
Si considera un sistema di riferimento S' in moto lungo l'asse z con velocità v rispetto al sistema di partenza.\\
Da ora si pone c=1.\\
Risulta conveniente trasformare il quadripotenziale:
\begin{equation}
    \begin{cases}
    A'_0=\gamma(A_0-\beta A_z)=0\\
    A'_x=-\frac{E_0}{\omega}\sin{(\mathbf{kz}-\omega t)}\\
    A'_y=0\\
    A'_z=\gamma(A_z-\beta A_0)=0
    \end{cases}
\end{equation}
Occorre tuttavia trasformare le coordinate (imponiamo c=1):
\begin{equation}
    t=\gamma(t'+\beta z') \qquad z=\gamma(z'+\beta t') \qquad k=\gamma(k'+\beta \omega') \qquad \omega=\gamma(\omega'+\beta k')
\end{equation}
Si ottiene:
\begin{align}
    {A'}^\mu&=(0,-\frac{E_0}{\omega}\sin{\biggl\{\gamma^2\biggl[(z'+\beta t')(k'+\beta \omega')-(t'+\beta z)(\omega'+\beta k')\biggr]\biggr\}},0,0)=\\
    &=(0,-\frac{E_0}{\omega}\sin{(kz'-\omega't')},0,0)
\end{align}
\paragraph{2.b.2}\textbf{Sull'asse z di un sistema di coordinate dato, si trova una carica lineare uniforme $\mathbf{\lambda}$ che si muove con velocità V lungo z. Trovare l'espressione della 4-corrente $\tensor{j}{_{\mu}}$ ,
ed una espressione possibile per il 4-potenziale $\tensor{A}{_{\mu}}$ , specificando il gauge utilizzato. Riformulare e risolvere il problema in un sistema di riferimento in
moto relativo lungo l'asse z.}\\ 
Consideriamo il 4-vettore nel LAB:
\begin{equation}
    j^{\mu}=(c \rho, \vec{j})
\end{equation}
Se chiamiamo S la sezione del filo: $\lambda=\rho S$, $I=|\vec{j}|S$. Dato che $j^{\mu}$ è un 4-vettore e S è un invariante per trasformazione di Lorentz con $\vec{\beta} // $ filo si ottiene:
\begin{equation}
    S\cdot(c\rho,\vec{j})=(\lambda c, 0, 0 ,I)
\end{equation}
trasforma come un 4-vettore.\\
Quindi in definitiva:
\begin{equation}
    j^{\mu}=\left(c\frac{\lambda}{S}, 0,0,\frac{I}{S}  \right)
\end{equation}
con $I=\lambda v$.\\
Il 4-potenziale è invece:
\begin{equation}
    A^{\mu}=(A^0, \vec{A})
\end{equation}
per calcolarlo mi pongo in un SDR' in cui le cariche del filo sono ferme in modo da avere solo potenziale elettrostatico:
\begin{equation}
    A^{'\mu}=(-2\lambda' log r',0,0,0)
\end{equation}
sarebbe $log(r/r_0)$ ma $r_0$ è costante arbitraria e pongo uguale a 1 ($V(r_0=1)=0$).\\
Per ottenere il 4-potenziale nel LAB si usa una trasformazione di Lorentz (boost lungo $-\vec{v}$):
\begin{equation}
    A^{\mu}=(-2\lambda'\gamma log r',0,0,-2\lambda'\beta\gamma log r')
\end{equation}
con $r'=\sqrt{x'^2+y'^2}=r$. (Dato che $\vec{\nabla} \cdot \vec{A}=0$ e dato che$\frac{1}{c}\partial_tA^0=0$ $\Rightarrow\partial_{\mu}A^{\mu}=0$: gauge di Lorentz.)\\
Per trovare $\lambda'$ in funzione di $\lambda$ considero le trasformazioni di Lorentz inverse:
\begin{equation}
    c \lambda=\gamma(c\lambda'+\beta I')
\end{equation}
ma in SDR' le cariche del filo sono ferme e quindi $I'=0\Rightarrow\lambda=\gamma \lambda'$.
In definitiva:
\begin{equation}
 A^{\mu}=(-2\lambda log\sqrt{x^2+y^2},0,0,-2\lambda \beta log\sqrt{x^2+y^2} )
\end{equation}
Per risolvere il problema in in qualsiasi SDR in moto con $\vec{\beta''}$ rispetto al LAB basta trasforma il 4-potenziale usando un boost di Lorentz lungo z e si usa il fatto che  anche  $(\lambda c,0,0,I)$ trasforma come un 4-vettore.

\paragraph{2.b.3}\textbf{Dimostrare l’espressione $\mathbf{\frac{\text{d}t}{\text{d}t'}}=1-\hat{n}\cdot\bm{\beta}$, dove $t$ e $t'$ sono il tempo di osservazione ed il tempo ‘ritardato’, rispettivamente.}\\ \\
Per una particella che si muove con velocità $\mathbf{v}$ per calcolare il campo in un punto $\mathbf{r}$ ci serve la derivata del tempo ritardato, per cui partendo da:
\begin{equation}
    t'=t-\frac{|\mathbf{r}-\mathbf{r}_s(t')|}{c}\, ,
\end{equation}
dove $r$ è il punto ti osservazione, $r_s(t')$ è la posizione della sorgente al tempo ritardato e chiamiamo $\vec{R}=\vec{r}-\vec{r}_s $.\\
Differenziando rispetto a $t'$ si ha:
\begin{align*}
    1&=\frac{d t}{d t'}-\frac{1}{c}\frac{d|\mathbf{r}-\mathbf{r}_s(t')|}{dt'}\\
    1&=\frac{d t}{d t'}+\frac{\hat{n}\cdot\mathbf{v}}{c}
\end{align*}
si è considerato il problema a grandi distanze in cui si può approssimare $\hat{n}= \frac{\vec{R}}{R}\simeq \frac{\vec{r}}{r}$ da cui $R\simeq r-\hat{n}\cdot \vec{r_s}$ ottenendo: 
\begin{equation*}
    \frac{dt}{dt'}=1-\hat{n}\cdot\bm{\beta}
\end{equation*}

\paragraph{2.b.4}\textbf{Date le definizioni "standard" delle variabili $\mathbf{\hat{n}}$, $\bm{\beta}$, $\mathbf{R}$, $\mathbf{r}$, $\mathbf{r'}$, $t$ e $t'$, dimostrare le seguenti relazioni: $\frac{d\vec{R}}{dt'}=-\vec{\beta}c$, $\frac{dR}{dt'}=-\hat{n}\cdot\vec{\beta}c$, $\frac{d(\vec{R}\cdot\vec{\beta})}{dt'}=-\beta^2c+\vec{R}\cdot\vec{\dot{\beta}}$, $\mathbf{\nabla}R=\frac{\hat{n}}{1-\hat{n}\cdot\vec{\beta}}$, $\mathbf{\nabla}t'=-\frac{\hat{n}/c}{1-\hat{n}\cdot\vec{\beta}}$.}\\ 
\\
\textit{Definizioni}
\begin{itemize}
    \item $\bm{\beta}$: $\mathbf{v}/c$
    \item $\mathbf{r}$: punto di osservazione
    \item $\mathbf{r}'$: posizione della sorgente al tempo ritardato
    \item $\mathbf{R}=\mathbf{r}-\mathbf{r}'$
    \item $\mathbf{\hat{n}}$: versore della direzione di $\vec{R}=\vec{r}-\vec{r}'$
    \item $\mathbf{t}$: tempo
    \item $\mathbf{t}'$: tempo ritardato
\end{itemize}
\textit{Dimostrazioni}
\begin{itemize}
\item Essendo $\mathbf{R}=\mathbf{r}-\mathbf{r'}$, abbiamo:
\begin{equation*}
\frac{d\mathbf{R}}{dt'}=-\frac{d\mathbf{r'}}{dt'}=-\mathbf{v}=-\bm{\beta}c \, .
\end{equation*}
\item Da $R=\sqrt{r^2+r'^2-2\mathbf{r}\cdot\mathbf{r'}}$, abbiamo:
\begin{equation*}
\frac{dR}{dt'}=\frac{\mathbf{r'}\cdot \frac{d\mathbf{r'}}{dt'}-\mathbf{r}\cdot\frac{d\mathbf{r'}}{dt'}}{R}=\frac{\mathbf{r'}\cdot\mathbf{v}-\mathbf{r}\cdot\mathbf{v}}{R}=\frac{(\mathbf{r'}-\mathbf{r})\cdot\mathbf{v}}{R}=-\frac{\mathbf{R}\cdot\mathbf{v}}{R}=-\mathbf{\hat{n}}\cdot\bm{\beta}c \, .
\end{equation*} 
\item Avendo appena dimostrato che $\frac{d\mathbf{R}}{dt'}=-\bm{\beta}c$, abbiamo:
\begin{equation*}
\frac{d(\mathbf{R}\cdot\bm{\beta})}{dt'}=\bm{\beta}\cdot\frac{d\mathbf{R}}{dt'}+\mathbf{R}\cdot\frac{d\bm{\beta}}{dt'}=-\beta^2c+\mathbf{R}\cdot\bm{\dot{\beta}} \, .
\end{equation*}
\item Calcoliamo prima:
\begin{equation*}
\frac{dR}{dr_i}=\frac{\partial R}{\partial r_i}+\frac{\partial R}{\partial t'}\frac{\partial t'}{\partial r_i}=\frac{r_i-r'_i}{R}-\mathbf{\hat{n}}\cdot\bm{\beta}c\frac{\partial t'}{\partial r_i}=\frac{R_i}{R}-\mathbf{\hat{n}}\cdot\bm{\beta}c\frac{\partial t'}{\partial r_i} \, ,
\end{equation*}
da cui generalizzando a tutte le componenti si ottiene:
\begin{equation*}
\mathbf{\nabla}R=\mathbf{\hat{n}}-\mathbf{\hat{n}}\cdot\bm{\beta}c\mathbf{\nabla}t'=\mathbf{\hat{n}}+\mathbf{\hat{n}}\cdot\bm{\beta}\mathbf{\nabla}R \Rightarrow (1-\mathbf{\hat{n}}\cdot\bm{\beta})\mathbf{\nabla}R=\mathbf{\hat{n}} \, ,
\end{equation*}
essendo $c\mathbf{\nabla}t'=-\mathbf{\nabla}R$ poiché dalla definizione di $t'$ si ha $t'= t- \frac{|\vec{r}-\vec{r'}|}{c}$, ovvero:
\begin{equation*}
\mathbf{\nabla}R=\frac{\mathbf{\hat{n}}}{1-\mathbf{\hat{n}}\cdot\bm{\beta}} \, .
\end{equation*}
\item Dall'equazione $\mathbf{\nabla}R=\mathbf{\hat{n}}-\mathbf{\hat{n}}\cdot\bm{\beta}c\mathbf{\nabla}t'$, abbiamo:
\begin{equation*}
-c\mathbf{\nabla}t'=\mathbf{\hat{n}}-\mathbf{\hat{n}}\cdot\bm{\beta}c\mathbf{\nabla}t' \Rightarrow \mathbf{\nabla}t'=-\frac{\mathbf{\hat{n}}}{c}+\mathbf{\hat{n}}\cdot\bm{\beta}\mathbf{\nabla}t' \Rightarrow (1-\mathbf{\hat{n}}\cdot\bm{\beta})\mathbf{\nabla}t'=-\frac{\mathbf{\hat{n}}}{c} \, ,
\end{equation*}
ovvero:
\begin{equation*}
\mathbf{\nabla}t'=-\frac{\mathbf{\hat{n}}/c}{1-\mathbf{\hat{n}}\cdot\bm{\beta}} \, .
\end{equation*}
\end{itemize}

\paragraph{2.b.5}\textbf{Calcolare la distribuzione in potenza in funzione dell’angolo di emissione per una carica accelerata in moto non relativistico.}\\ \\
Il campo elettrico di radiazione nel limite non relativistico è ($\beta \sim 0$):
\begin{equation*}
    \mathbf{E}=\frac{q}{cr}\frac{\hat{n}\times[(\hat{n}-\beta)\times\dot{\beta}]}{(1-\beta\cdot\hat{n} )^3}\simeq\frac{q}{cr}\hat{n}\times(\hat{n}\times\dot{\beta})
\end{equation*}
si ha inoltre $\mathbf{B}=\hat{n}\times\mathbf{E}$, quindi la potenza irraggiata è:
\begin{equation*}
    \frac{\text{d}P}{\text{d}\Omega}=|\mathbf{S}|r^2=\frac{c}{4\pi}(\vec{E}\times \vec{B}) r^2= \frac{c}{4\pi}r^2 |\vec{E}\times\hat{n}\times\vec{E}|=\frac{c}{4\pi}r^2|\vec{E}|^2=\frac{q^2|\hat{n}\times\dot{\beta}|^2}{4\pi c}
\end{equation*}
\paragraph{2.b.6}\textbf{Ricavare esplicitamente le leggi di trasformazione di Lorentz del campo elettrico e del campo magnetico. Discutere, in particolare, il caso in cui, in un certo sistema di riferimento inerziale, il campo magnetico è nullo e il caso in cui il campo elettrico è nullo.}\\ \\
Partiamo da vedere come trasforma il tensore dei campi:
\begin{equation*}
    \tensor{F}{^{\mu\nu}}\longrightarrow\tensor{{F'}}{^{\mu\nu}}=\tensor{\Lambda}{^{\mu}_{\alpha}}\tensor{\Lambda}{^{\nu}_{\beta}}\tensor{F}{^{\alpha\beta}}
\end{equation*}
Sotto forma di matrice si ha:
\begin{equation*}
    F'=\Lambda F\Lambda^t
\end{equation*}
Effettuiamo un boost lungo l'asse x:
\begin{align*}
    F'&=
    \begin{pmatrix}
    	\gamma		& -\beta\gamma 	& 0 & 0 \\
        -\beta\gamma 	&  \gamma		& 0 & 0 \\
         		0		& 		0		& 1 & 0 \\
          		0		&		0		& 0 & 1 \\
    \end{pmatrix}
    \begin{pmatrix}
        0	& -E_x 	& -E_y 	& -E_z \\
        E_x & 0		& -B_z 	& B_y  \\
        E_y	& B_z	& 0 	& -B_x \\
        E_z	& -B_y	& B_x 	& 0    \\
    \end{pmatrix}
    \begin{pmatrix}
    	\gamma		& -\beta\gamma 	& 0 & 0 \\
        -\beta\gamma 	&  \gamma		& 0 & 0 \\
         	0		& 		0		& 1 & 0 \\
     		0		&		0		& 0 & 1 \\
    \end{pmatrix} \\
    &=
    \begin{pmatrix}
    	\gamma		& -\beta\gamma 	& 0 & 0 \\
        -\beta\gamma 	&  \gamma		& 0 & 0 \\
     		0		& 		0		& 1 & 0 \\
     		0		&		0		& 0 & 1 \\
    \end{pmatrix}
    \begin{pmatrix}
        \beta \gamma E_x				& -\gamma E_x					& -E_y 	& -E_z \\
        \gamma E_x						& -\beta \gamma E_x				& -B_z	& B_y  \\
        \gamma E_y - \beta \gamma B_z	& -\beta \gamma E_y+\gamma B_z	& 0		& -B_x \\
        \gamma E_z + \beta \gamma B_y 	& -\beta \gamma E_z-\gamma B_y	& B_x 	& 0    \\
    \end{pmatrix} \\
    &=
    \begin{pmatrix}
        0						& -E_x 					& -\gamma (E_y-\beta B_z)	& -\gamma (E_z+\beta B_y) \\
        E_x						& 0 					& -\gamma (B_z-\beta E_y)	& \gamma (B_y+\beta E_z)  \\
        \gamma (E_y-\beta B_z)	&\gamma (B_z-\beta E_y)	& 0							& -B_x                    \\
        \gamma (E_z+\beta B_y)	&-\gamma (B_y+\beta E_z)& B_x						& 0                       \\
    \end{pmatrix} \\
    &\overset{!}{=}
    \begin{pmatrix}
        0		& -E'_x & -E'_y & -E'_z \\
        E'_x 	& 0		& -B'_z & B'_y  \\
        E'_y	& B'_z	& 0 	& -B'_x \\
        E'_z	& -B'_y	& B'_x 	& 0    \\
    \end{pmatrix} \, .
\end{align*}
Da cui si trova:
\begin{equation}
\begin{cases}
E'_x=E_x \\
E'_y=\gamma(E_y-\beta B_z) \\
E'_z=\gamma(E_z+\beta B_y)
\end{cases}
\begin{cases}
B'_x=B_x \\
B'_y=\gamma(B_y+\beta E_z) \\
B'_z=\gamma(B_z-\beta E_y) 
\end{cases} \, .
\end{equation}
Nel caso in cui, in un certo sistema di riferimento inerziale, il campo magnetico è nullo, ovvero $\mathbf{B'}=\mathbf{0}$, abbiamo:
\begin{equation*}
\begin{cases}
0=B_x \\
0=\gamma(B_y+\beta E_z) \\
0=\gamma(B_z-\beta E_y) 
\end{cases} 
\Rightarrow
\begin{cases}
B_x=0 \\
B_y=-\beta E_z \\
B_z=\beta E_y
\end{cases} 
\Rightarrow \mathbf{B}=\bm{\beta}\times\mathbf{E} \, .
\end{equation*}
Nel caso in cui invece, in un certo sistema di riferimento inerziale, il campo elettrico è nullo, ovvero $\mathbf{E'}=\mathbf{0}$, abbiamo:
\begin{equation*}
\begin{cases}
0=E_x \\
0=\gamma(E_y-\beta B_z) \\
0=\gamma(E_z+\beta B_y)
\end{cases}
\Rightarrow
\begin{cases}
E_x=0 \\
E_y=\beta B_z \\
E_z=-\beta B_y
\end{cases}
\Rightarrow \mathbf{E}=-\bm{\beta}\times\mathbf{B} \, .
\end{equation*}
\\ \\

\paragraph{2.b.7}\textbf{Dire quali sono gli "invarianti di Lorentz" che si possono costruire con il tensore
del campo elettromagnetico e ricavarne le espressioni esplicite in termini dei
campi elettrico e magnetico. Ridiscutere, usando gli invarianti, il caso discusso
nel punto precedente e discutere il caso in cui gli invarianti sono nulli.}\\ \\
A partire dal tensore elettromagnetico e dal suo duale, è possibile costruire espressioni scalari contraendo tutti gli indici e si trovano due diversi scalari (invarianti di Lorentz):
\begin{equation*}
    I_1=F^{\mu \nu}F_{\mu \nu} \ \ \ \ \ \ \ \ I_2=\Tilde{F}^{\mu \nu}F_{\mu \nu}
\end{equation*}
Si svolge il calcolo per due generici vettori antisimmetrici $A^{\mu \nu}=(\vec{p},\vec{a})$ e $B^{\mu \nu}=(\vec{q},\vec{b})$:
\begin{equation}
    A^{\mu \nu}B_{\mu \nu}=A^{0i}B_{0i}+A^{i0}B_{i0}+A^{ij}B_{ij}=2(\vec{a}\cdot\vec{b}-\vec{p}\cdot\vec{q})
\end{equation}
Sostituendo $F^{\mu \nu}=(-\vec{E},\vec{B})$, $F_{\mu \nu}=(\vec{E},\vec{B})$, $\Tilde{F}^{\mu \nu}=(-\vec{B,-\vec{E}})$ si trova:
\begin{equation}
    I_1=2(\vec{B}^2-\vec{E}^2) \ \ \ \ \ \ \ \ I_2=-4\vec{B}\cdot \vec{E}
\end{equation}
Il caso del punto precedente è il seguente: \\
Se in un certo sistema di riferimento inerziale $\mathbf{E'}=\mathbf{0}$ oppure $\mathbf{B'}=\mathbf{0}$ allora:
\begin{equation*}
\mathbf{E}\cdot\mathbf{B}=\mathbf{E'}\cdot\mathbf{B'}=0 \, ,
\end{equation*}
ovvero campi ortogonali vanno in campi ortogonali.
\\ \\
Se in un sistema di riferimento $\vec{E}$ o $\vec{B}$ sono ortogonali, $I_2=0$, perciò in ogni sistema di riferimento si ha che il campo elettrico e il campo magnetico sono ortogonali.
\\
Il caso delle onde elettromagnetiche ha $I_1=I_2=0$ e in ogni sistema di riferimento campo elettrico e magnetico sono ortogonali e hanno la stessa intensità.







\paragraph{2.b.8}\textbf{Una carica elettrica Q si muove con velocità costante su una retta con velocità
costante: x=Vt, y=b, z=0. Calcolare in funzione del tempo il campo elettrico ed il
campo magnetico generato dalla carica nel punto O e produrre il grafico di
ognuna delle 6 componenti trovate in funzione del tempo t.}\\ \\

Ponendosi nel sistema di riferimento solidale alla particella $S'$ i campi sono:
\begin{equation}
    \vec{E}'=Q\frac{\vec{R}'}{R'^3} \ \ \ \ \ \ \ \ \vec{B}'=0
\end{equation}
Mentre nel sistema del laboratorio $S$ sono:
\begin{equation}\label{campi}
    \vec{E}_{\paral}=\vec{E}'_{\paral} \ \ \ \ \ \ \ \  \vec{E}_{\perp}=\gamma \vec{E}'_{\perp} \ \ \ \ \ \ \ \ \vec{B}=\frac{\vec{v}}{c}\wedge\vec{E}
\end{equation}
con $\beta=v/c$ e $\gamma=1/\sqrt{1-\beta^2}$. \\
Le coordinate nei due sistemi sono legate dalle solite trasformazioni di Lorentz:
\begin{equation}
    \begin{array}{lr}
         x'=\gamma(x-\beta t)  \\
         y'=y \\
         z'=z
    \end{array}
\end{equation}
Da cui si trova:
\begin{equation}
    R'=\sqrt{x'^2+y'^2+z'^2}=\gamma\sqrt{(x-\beta t)^2+(1-\beta ^2)(y^2+z^2)}\equiv \gamma R_{*}
\end{equation}
I campi, in base alla (\ref{campi})
sono:
\begin{equation}
    \begin{array}{lr}
         E_x=E_x '=\frac{Q(x-\beta t)}{\gamma^2 R_* ^3}  \\
         E_y=\gamma E_y '=\frac{Qb}{\gamma^2 R_* ^3} \\
         E_z=0
    \end{array} \ \ \ \ \ \ \vec{B}=\vec{\beta}\wedge \vec{E}=(0,0,\beta E_y) \ \ \ \ \text{con} \ \ \ \vec{\beta}=(\beta,0,0) \ \ \ \Rightarrow \ \ \begin{array}{lr}
         \vec{E}=(E_x,E_y,0) \\
         \vec{B}=(0,0,\beta E_y)
    \end{array}
\end{equation}

I grafici delle uniche componenti non nulle dei campi sono mostrate in Figura (\ref{carica moto}):


\begin{figure}[H]
    \centering
    \includegraphics[scale=0.5]{campi caric in moto.JPG}
    \label{carica moto}
    \caption{Andamento delle componenti $E_x$ ed $E_y$ del campo elettrico generato da una particella in
moto rettilineo uniforme come viste da un osservatore avente parametro di impatto $b$ per $\beta = 0.7$
(curva continua nera) e $\beta = 0.9$ (curva tratteggiata rossa) per $-2b/c \le t \le 2b/c$.}
\end{figure}



\paragraph{2.b.9} 
\textbf{Dimostrare che $\vec{F}_{rad}=\frac{2}{3}\frac{q^2}{c^3}\dot{\vec{a}}=z^2m_e \tau_e \dot{\vec{a}}$, con $\tau_e=\frac{2}{3}\frac{r_e}{c}=6.2\times 10^{-24} \ s$, è la forza di reazione
radiativa ed indicare il campo di applicazione di questa formula.}\\ \\
Qualora una carica elettrica sia soggetta a una forza esterna, subirà un'accelerazione e conseguentemente accelererà e irraggerà, perdendo energia.
\\
L'equazione $\dot{\vec{p}}=\vec{F}_{ext}$ non può descrivere correttamente questa situazione dal momento che non si ha conservazione dell'energia.
\\
Per tenere conto della radiazione si può inserire nell'equazione cardinale di cui sopra un termine di forza $\vec{F}_{rad}$, che è la reazione di radiazione.
\\
Se si considera il caso non relativistico (che è il campo di applicazione della formula \textit{ndr}), si può porre il lavoro compiuto dalla $\vec{F}_{rad}$ uguale all'energia persa per irraggiamento secondo la formula di Larmor (\ref{larmor1}) $\left(P=\frac{2}{3}\frac{e^2}{c^3}|\dot{\vec{v}}|^2 \right)$ integrata nel tempo:
\begin{equation}
    \int_{\gamma}\vec{F}_{rad}\cdot d\vec{s}=\int_{t_1}^{t_2}\vec{F}_{rad}\cdot \frac{d\vec{s}}{dt}dt=\int_{t_1}^{t_2}\vec{F}_{rad}\cdot\vec{v}\ dt\overset{!}{=}-\frac{2}{3}\frac{e^2}{c^3}\int_{t_1}^{t_2}|\dot{\vec{v}}|^2 \ dt
\end{equation}
Si sostituisce $\dot{\vec{v}}^2=\frac{d}{dt}(\vec{v}\cdot \dot{\vec{v}})-\vec{v}\cdot \Ddot{\vec{v}}$:
\begin{equation}
    \int_{t_1}^{t_2}\vec{F}_{rad}\cdot\vec{v} \ dt=-\frac{2}{3}\frac{e^2}{c^3}\vec{v}\cdot \dot{\vec{v}}\left. \right|_{t_1}^{t_2}+\frac{2}{3}\frac{e^2}{c^3}\int_{t_1}^{t_2}\vec{v}\cdot \Ddot{\vec{v}}\ dt
\end{equation}
Cerchiamo l'espressione della reazione di radiazione nel caso di moto periodico per poi dar buono il risultato a moto qualsiasi. In questo caso, dato che stiamo integrando sul moto totale di una particella, $t_1$ e $t_2$ corrispondono allo stesso punto sulla curva in quanto $t_2-t_1$ corrisponde a un periodo, quindi il primo membro a destra dell'uguale si annulla $\vec{v}\cdot \dot{\vec{v}}\left. \right|_{t_1}^{t_2}=0$, per cui:
\begin{equation}
    \int_{t_1}^{t_2}\left( \vec{F}_{rad}-\frac{2}{3}\frac{e^2}{c^3}\Ddot{\vec{v}} \right)\cdot \vec{v} \ dt=0
\end{equation}
Per cui si può prendere come possibile candidato:
\begin{equation}
    \vec{F}_{rad}=\frac{2}{3}\frac{e^2}{c^3}\Ddot{\vec{v}}=\frac{2}{3}\frac{e^2}{c^3}\dot{\vec{a}}
\end{equation}
E facendo due conti:

\begin{equation}
\vec{F}_{rad}=\frac{2}{3}\frac{q^2}{c^3} \dot{\vec{a}}=\frac{2}{3}\frac{z^2 e^2}{c^3}\dot{\vec{a}}=z^2 m_e \frac{2}{3}\frac{1}{c}\frac{e^2}{m_e c^2} \dot{\vec{a}}=z^2 m_e \frac{2}{3}\frac{r_e}{c} \dot{\vec{a}}=z^2 m_e \tau_e \dot{\vec{a}}
\end{equation}

\paragraph{2.b.10}\textbf{Dare la definizione del "tensore energia-impulso" del campo elettromagnetico e
scrivere la sua relazione con la "densità di forza di Lorentz".}\\ \\
Il tensore energia-impulso del campo elettromagnetico è definito come:
\begin{equation}
    \tensor{T}{^\mu^\rho}=\frac{1}{4\pi} \biggl( -\tensor{F}{^\mu^\nu}\tensor{F}{^\rho_\nu}+\frac{1}{4}\tensor{g}{^\mu^\rho}\tensor{F}{^\alpha^\beta}\tensor{F}{_\alpha_\beta} \biggr)
\end{equation}
La densità di forza di Lorentz è definita come la forza di Lorentz per unità di volume.
\\
Affinchè la si possa estendere come quadrivettore si esprime la forza adoperando la seconda equazione cardinale, ossia:
\begin{equation}
    \frac{d\vec{p}}{dtdV}=\rho \vec{E}+\frac{1}{c}\vec{j}\wedge \vec{B}
\end{equation}
Dal momento che $dtdV$ è uno scalare di Lorentz si può introdurre il quadrivettore controvariante:
\begin{equation}\label{densità di forza}
    G^{\mu}=\frac{dp^{\mu}}{dtdV}=\left( \frac{1}{c}\frac{d\mathcal{E}}{dtdV},\frac{d\vec{p}}{dtdV} \right)=\biggl(\frac{1}{c}\vec{E}\cdot \vec{j},\rho \vec{E}+\frac{1}{c}\vec{j}\times\vec{B}\biggr)=\frac{1}{c}F^{\mu \nu}J_\nu
\end{equation}


Ricaviamo dalle equazioni di Maxwell non omogenee la quadri-corrente $J^\nu$:
\begin{equation*}
\partial_\mu \tensor{F}{^\mu^\nu}=\frac{4\pi}{c}J^\nu \Rightarrow J^\nu=\frac{c}{4\pi} \partial_\mu \tensor{F}{^\mu^\nu}
\end{equation*}
e sostituiamola all'interno della densità di forza di Lorentz:
\begin{align*}
G^\mu&=\frac{1}{c}\tensor{F}{^\mu^\nu}J_\nu \\
&=\frac{1}{c}\tensor{F}{^\mu^\nu}\frac{c}{4\pi} \partial^\rho \tensor{F}{_\rho_\nu} \\
&=\frac{1}{4\pi}\tensor{F}{^\mu^\nu} \partial^\rho \tensor{F}{_\rho_\nu} \\
&=\frac{1}{4\pi}[\partial^\rho(\tensor{F}{^\mu^\nu}\tensor{F}{_\rho_\nu})-(\partial^\rho\tensor{F}{^\mu^\nu})\tensor{F}{_\rho_\nu}] \\
&=\frac{1}{4\pi} \biggl[ \partial^\rho(\tensor{F}{^\mu^\nu}\tensor{F}{_\rho_\nu})-\frac{1}{2}(\partial^\rho \tensor{F}{^\mu^\nu}-\partial^\nu \tensor{F}{^\mu^\rho})\tensor{F}{_\rho_\nu} \biggr] \\
&=\frac{1}{4\pi} \biggl[ \partial^\rho(\tensor{F}{^\mu^\nu}\tensor{F}{_\rho_\nu})-\frac{1}{2}(\partial^\rho \tensor{F}{^\mu^\nu}+\partial^\nu \tensor{F}{^\rho^\mu})\tensor{F}{_\rho_\nu} \biggr] \\
&=\frac{1}{4\pi} \biggl[ \partial^\rho(\tensor{F}{^\mu^\nu}\tensor{F}{_\rho_\nu})+\frac{1}{2}(\partial^\mu \tensor{F}{^\nu^\rho})\tensor{F}{_\rho_\nu} \biggr] \\
&=\frac{1}{4\pi} \biggl[ \partial^\rho(\tensor{F}{^\mu^\nu}\tensor{F}{_\rho_\nu})-\frac{1}{2}(\partial^\mu \tensor{F}{^\nu^\rho})\tensor{F}{_\nu_\rho} \biggr] \\
&=\frac{1}{4\pi} \biggl[ \partial^\rho(\tensor{F}{^\mu^\nu}\tensor{F}{_\rho_\nu})-\frac{1}{4}\partial^\mu(\tensor{F}{^\nu^\rho}\tensor{F}{_\nu_\rho}) \biggr] \\
&=\frac{1}{4\pi}\partial^\rho \biggl[ \tensor{F}{^\mu^\nu}\tensor{F}{_\rho_\nu}-\frac{1}{4}\delta{^\mu_\rho}\tensor{F}{^\alpha^\beta}\tensor{F}{_\alpha_\beta} \biggr] \\
&=-\partial{^\rho}\tensor{T}{^\mu_\rho} \\
&=-\partial{_\rho}\tensor{T}{^\mu^\rho}
\end{align*}


\paragraph{2.b.11}\textbf{Dire come si generalizzano i teoremi di conservazione dell'energia e dell'impulso
a situazioni in cui sia presente un campo elettromagnetico.}\\ \\
In base alla (\ref{densità di forza}) e a quanto trovato sopra si può scrivere:
\begin{equation}
    \frac{dp^\mu}{dtdV}=G^{\mu}=-\partial_{\rho}T^{\mu \rho}=\frac{1}{c}F^{\mu \nu} J_{\nu} \ \ \ \Rightarrow \ \ \ c\partial_{\rho}T^{\mu \rho}+F^{\mu \nu}J_{\nu}=0
\end{equation}
Sapendo che:
\begin{equation}
    T^{\mu\rho=}\begin{pmatrix}
    W & \frac{\vec{S}}{c}\\  \frac{\vec{S}}{c} & \sigma_{ij}
    \end{pmatrix}
\end{equation}
Si può scrivere esplicitamente come:
\begin{equation}
    \frac{\partial W}{\partial t}+\vec{\nabla}\cdot \vec{S}+\frac{d\mathcal{E}}{dtdV}=0
\end{equation}
\begin{equation}
    \frac{1}{c^2}\frac{\partial S_i}{\partial t}+\partial_j\sigma_{ij}+\frac{dp_i}{dtdV}=0
\end{equation} 



\paragraph{2.b.12}\textbf{Scrivere il tensore degli sforzi per un’onda e.m. piana che si propaga in una
direzione $\hat{n}$ con densità di energia $u_{EM}$.}\\ \\
Il tensore degli sforzi di Maxwell è un tensore tridmensionale definito come:
\begin{equation}\label{sforzo}
    \sigma_{ij}=\frac{1}{4\pi}\left[ -E_iE_j - B_i B_j +\frac{1}{2}\delta_{ij} (|\vec{E}|^2+|\vec{B}|^2) \right]
\end{equation}
Con una rotazione porto $\hat{n}$ lungo l'asse $\hat{x}$ e $\vec{B}$ lungo $\hat{y}$. Conseguentemente trovo $\vec{E}$ lungo $\hat{z}$.
\\
In equazioni, trovo:
\begin{equation}\label{compo}
    \hat{n}=(1,0,0) \ \ \ \ \ \ \ \ \vec{B}=(0,B,0) \ \ \ \ \ \ \ \ \vec{E}=(0,0,E) \ \ \ \ \text{con} \ \ \ E=B=|\vec{E}|=|\vec{B}|
\end{equation}
Riscrivo la (\ref{sforzo}) in funzione di $u_{EM}$ come:
\begin{equation}
    \sigma_{ij}=\frac{1}{4\pi}\left[ -E_iE_j-B_iB_j+4\pi \delta_{ij}u_{EM} \right]
\end{equation}

E la riscrivo esplicitamente considerando la \ref{compo} sotto forma di matrice:
\begin{equation}
    \sigma=\begin{pmatrix}
        u_{EM} & 0 & 0 \\
        0 & -\frac{B^2}{4\pi}+u_{EM}
& 0 \\
0 & 0 & -\frac{E^2}{4\pi}+u_{EM}

\end{pmatrix}
\end{equation}



\paragraph{2.b.13}\textbf{Scrivere esplicitamente il 4-tensore impulso-energia per un’onda e.m. piana monocromatica che si propaga lungo l’asse x con densità di energia uem.}\\ \\
Si sceglie il sistema CGS.\\
Un onda EM monocromatica che si propaga lungo x:
\begin{equation}
    \vec{E}=(E_z\hat{z}+E_y\hat{y})e^{i(kx-\omega t)} \qquad \vec{B}=(B_z\hat{z}+B_y\hat{y})e^{i(kx-\omega t)}
\end{equation}
Con $E_zB_z+E_yB_y=0$ poiché il campo elettrico e il campo magnetico sono ortogonali tra loro, $E_z=-B_y$ e $E_y=B_z$ poichè $\vec{B}=\hat{n}\wedge\vec{E}$ (in MKSA ci sarebbe un fattore v).\\
La densità di energia:
\begin{equation}
    u_{em}=\frac{1}{8\pi}(|\vec{E}|^2+|\vec{B}|^2)=\frac{1}{4\pi}|\vec{E}|^2
\end{equation}
Presenta un vettore di Poynting:
\begin{equation}
   \vec{S}=\frac{c}{4\pi}\vec{E}\times\vec{B}=c\frac{E_yB_z-E_zB_y}{4\pi}\hat{x}=c\frac{E_y^2+E_z^2}{4\pi}\hat{x}=\frac{c}{4\pi}|\vec{E}|^2\hat{x}
\end{equation}
Il tensore degli sforzi risulta:
\begin{equation}
        \sigma_{ij}=\frac{1}{4\pi}[-E_iE_j-B_iB_j+\frac{1}{2}\delta_{ij}(|\vec{E}|^2+|\vec{B}|^2)]=\frac{1}{4\pi}\begin{pmatrix}
      E_y^2+E_z^2 & 0 & 0\\ 0 & 0 & 0 \\ 0 & 0 & 0
    \end{pmatrix}
\end{equation}
Da cui si ricava il tensore energia impulso come:
\begin{equation}
    T^{\mu\nu}=\begin{pmatrix} u_{em} & \frac{\vec{S}}{c}\\\frac{\vec{S}}{c}& \sigma_{ij}\end{pmatrix}=\frac{1}{4\pi}\begin{pmatrix} |\vec{E}|^2 & |\vec{E}|^2 & 0 & 0 \\
    |\vec{E}|^2 & |\vec{E}|^2 & 0 & 0 \\ 0&0&0&0\\0&0&0&0\\
    \end{pmatrix}
\end{equation}

\paragraph{2.b.14}\textbf{Calcolare la potenza totale irraggiata da una carica accelerata in moto non relativistico. Esprimere i risultati in MKSA e nelle unità “naturali”.}\\ \\
Per calcolare l'energia irraggiata da una particella accelerata nell'unità di tempo è necessario innanzitutto calcolare il vettore di Poynting \textbf{S}. Poichè $\vec{B}_{rad}=\vec{\hat{n}} \times \vec{E}_{rad}$ si ha (CGS):
\[
\vec{S}=\frac{c}{4 \pi}\vec{E}_{rad}\times \vec{B}_{rad} = \frac{c}{4 \pi} E^2_{rad}\vec{\hat{n}}
\]
dove tutta l'espressione deve essere calcolata al tempo ritardato $t_r$. Per calcolare la potenza irraggiata è necessario calcolare il flusso del vettore di Poynting su una superficie chiusa.  Se una carica è accelerata ma viene osservata in un sistema di riferimento in cui la sua velocità è piccola a confronto della velocità della
luce ($\beta \rightarrow 0$). Il campo elettrico diventa:
\[
\vec{E}_{rad}=\left| \frac{q}{R c}\frac{\hat{\vec{n}}\times[(\hat{\vec{n}}-\vec{\beta})\times \dot{\vec{\beta}}]}{(1-\hat{\vec{n}}\cdot \vec{\beta})^3}\right|_{t_r}
\]
\[
\vec{E}_{rad}^{(nr)}=\frac{e}{cR}\hat{\vec{n}} \times (\hat{\vec{n}}\times \dot{\vec{\beta}})
\]
Indichiamo con $\theta$ l'angolo tra i vettori $\dot{\vec{\beta}}$ e $\vec{n}$, si ha:
\[
|\vec{E}_{rad}^{(nr)}|=\frac{ea sin\theta}{c^2 R}
\]
dove $a=|\vec{a}|$ accelerazione della carica.\\
La quantità di energia irraggiata per unità di tempo è data dal flusso del vettore di Poynting attraverso una sfera di raggio R.
\begin{equation}\label{2b.14 Larmor}
    P=\frac{e^2a^2}{4\pi c^3}2\pi\int_0^{\pi}sin^3\theta d\theta=\frac{e^2a^2}{2c^3}\int_{-1}^{1}(1-cos^2\theta) dcos\theta=\frac{2}{3}\frac{e^2a^2}{c^3}
\end{equation}
Che è nota come \textbf{Formula di Larmor}, per la sua generalizzazione relativistica guarda paragrafo 2.b.15.\\
RICORDA\\
\[
\frac{dP}{d\Omega}=|\vec{S}| R^2 =\frac{q^2}{4 \pi c^3}|\dot{\vec{v}}|sin^2\theta \;\;\;\; \text{Potenza irraggiata per unità di angolo solido}
\]
\[
P=\int \frac{dP}{d\Omega} d\Omega \;\;\;\; \text{Potenza irraggiata}
\]
\[
d\Omega = sin\theta d\theta d\phi \;\;\;\; \text{Angolo solido}
\]
con $\theta \in (0, \pi)$ e $\phi \in (0, 2\pi)$.\\
In MKSA basta ricordare che $4 \pi= 1/\epsilon_0$ :
\[
P=\frac{q^2a^2}{6 \pi \epsilon_0 c^3}
\]



\paragraph{2.b.15}\textbf{Ricavare la formula di Larmor relativistica  $P=\frac{2}{3}\frac{q^2}{c^3} \gamma^6 (|\mathbf{a}|^2-|\mathbf{a}\times\bm{\beta}|^2)$ a partire dalla
formula non relativistica ed utilizzando argomenti di invarianza relativistica.}\\
\\
\textbf{Modo Bonati}:\\
Nel sistema istantaneamente in quiete della particella:
\begin{equation}
d\varepsilon=\frac{2}{3}\frac{e^2a^2}{c^3}dt \qquad   d\vec{p}=0
\end{equation}
Dove tali quantità fanno riferimento all'energia e all'impulso irraggiati nell'intervallo di tempo dt.\\
La generalizzazione covariante risulta essere:
\begin{equation}
    dP^\mu=-\frac{2}{3}\frac{e^2}{c^4}\frac{du^\nu}{d\tau}\frac{du_\nu}{d\tau}u^\mu d\tau
\end{equation}
Difatti in tale sistema $u^\mu=(c,0), \frac{du_\mu}{d\tau}=(0,\vec{a})$.\\
Se 2 quadrivettori sono uguali in un sistema di riferimento, dato che trasformano allo stesso modo, saranno uguali in tutti i sistemi di riferimento.\\
Considero dunque la componente 0 dell'equazione precedente:
\begin{equation}
    \frac{d\varepsilon}{dt}=-\frac{2}{3}\frac{e^2}{c^3}\frac{du^\nu}{d\tau}\frac{du_\nu}{d\tau}
\end{equation}
Utilizzando la formula esplicita per l'accelerazione:
\begin{equation}
    a^\mu=\left(\gamma^4\frac{\vec{v}\cdot\vec{a}}{c},\gamma^4\frac{\vec{v}\cdot\vec{a}}{c^2}\vec{v}+\gamma^2\vec{a}\right)\Longrightarrow \frac{du^\nu}{d\tau}\frac{du_\nu}{d\tau}=\gamma^8\frac{(\vec{v}\cdot\vec{a})^2}{c^2}-\gamma^8\frac{(\vec{v}\cdot\vec{a})^2}{c^4}|\vec{v}|^2-\gamma^4\vec{a}^2=-\frac{\gamma^6}{c^2}(\vec{v}\cdot \vec{a})^2-\gamma^4|\vec{a}|^2
\end{equation}
Utilizzando la relazione $(\vec{v}\times\vec{a})^2=|\vec{v}|^2|\vec{a}|^2-(\vec{v}\cdot\vec{a})^2$:
\begin{equation}
    \frac{du^\mu}{d\tau}\frac{du_m}{d\tau}=-\gamma^6|\vec{a}|^2+\frac{\gamma^6}{c^2}(\vec{v}\times\vec{a})^2
\end{equation}
Sostituendo si ricava la relazione desiderata:
\begin{equation}
     \frac{d\varepsilon}{dt}=\frac{2}{3}\frac{q^2}{c^3} \gamma^6 (|\mathbf{a}|^2-|\mathbf{a}\times\bm{\beta}|^2)
\end{equation}

\textbf{Modo Primordiale:}\\
Riscriviamo la formula di Larmor (\ref{2b.14 Larmor}) nel seguente modo:
\begin{equation} \label{larmor1}
P=\frac{2}{3} \frac{q^2}{c^3} |\mathbf{\dot{v}}|^2=\frac{2}{3} \frac{q^2}{m^2 c^3} \biggl( \frac{d\mathbf{p}}{dt} \cdot \frac{d\mathbf{p}}{dt} \biggr)
\end{equation}
dove $m$ è la massa della particella carica e $\mathbf{p}=m\mathbf{v}$ il suo impulso. La generalizzazione invariante di Lorentz è:
\begin{equation} \label{generale}
P=-\frac{2}{3} \frac{q^2}{m^2 c^3} \biggl( \frac{dP_\mu}{d\tau} \frac{dP^\mu}{d\tau} \biggr)
\end{equation}
dove $P^\mu$ è il quadrimpulso della particella e $d\tau=\frac{dt}{\gamma}$ l'elemento del tempo proprio. Verifichiamo che l'equazione \ref{generale} si riduce alla \ref{larmor1} per $\beta\rightarrow 0$ (ovvero $\gamma \rightarrow 1$):
\begin{equation}
-\frac{dP_\mu}{d\tau} \frac{dP^\mu}{d\tau}=-\frac{1}{c^2} \biggl( \frac{dE}{d\tau} \biggr)^2+\biggl( \frac{d\mathbf{p}}{d\tau}\biggr)^2=-\beta^2 \biggl(\frac{dp}{d\tau}\biggr)^2+\biggl( \frac{d\mathbf{p}}{d\tau}\biggr)^2.
\end{equation}
Per scrivere il primo termine dell'ultima uguaglianza siamo partiti dalla formula:
\begin{equation*}
\frac{E}{c}=\sqrt{p^2+m^2c^2}
\end{equation*}
Differenziando:
\begin{equation*}
\frac{dE}{c}=\frac{p}{\sqrt{p^2+m^2c^2}} dp
\end{equation*}
ed elevando al quadrato otteniamo:
\begin{align*}
\frac{(dE)^2}{c^2}&=\frac{p^2}{p^2+m^2c^2}(dp)^2=\frac{m^2\gamma^2v^2}{m^2\gamma^2v^2+m^2c^2}(dp)^2 \\
&=\frac{\gamma^2v^2}{\gamma^2v^2+c^2}(dp)^2 =\frac{\gamma^2\beta^2}{\gamma^2\beta^2+1}(dp)^2 \\
&=\frac{\gamma^2-1}{\gamma^2-1+1}(dp)^2 =\frac{\gamma^2-1}{\gamma^2}(dp)^2=\beta^2(dp)^2
\end{align*}
Per esprimere l'equazione \ref{generale} in funzione della velocità e dell'accelerazione tridimensionali scriviamo:
\begin{equation*}
\frac{dP^\mu}{d\tau}=\biggl( m\gamma^4 \frac{\mathbf{v}\cdot\mathbf{a}}{c}, m\gamma^4 \frac{\mathbf{v}\cdot\mathbf{a}}{c^2} \mathbf{v}+m \gamma^2 \mathbf{a}\biggr) \, ,
\end{equation*}
da cui:
\begin{align*}
P&=-\frac{2}{3} \frac{q^2}{m^2 c^3} \biggl( m^2 \gamma^8 \frac{|\mathbf{v}\cdot\mathbf{a}|^2}{c^2}-m^2\gamma^8 \frac{|\mathbf{v}\cdot\mathbf{a}|^2}{c^4}|\mathbf{v}|^2-m^2\gamma^4|\mathbf{a}|^2-2m^2\gamma^6 \frac{|\mathbf{v}\cdot\mathbf{a}|^2}{c^2} \biggr) \\
&=-\frac{2}{3} \frac{q^2}{c^3} \biggl(\gamma^8 \frac{|\mathbf{v}\cdot\mathbf{a}|^2}{c^2}-\gamma^8 \frac{|\mathbf{v}\cdot\mathbf{a}|^2}{c^4}|\mathbf{v}|^2-\gamma^4|\mathbf{a}|^2-2\gamma^6 \frac{|\mathbf{v}\cdot\mathbf{a}|^2}{c^2} \biggr) \\
&=\frac{2}{3} \frac{q^2}{c} \gamma^6 \biggl(-\gamma^2 \frac{|\mathbf{v}\cdot\mathbf{a}|^2}{c^4}+\gamma^2 \frac{|\mathbf{v}\cdot\mathbf{a}|^2}{c^4} \frac{|\mathbf{v}|^2}{c^2}+\gamma^{-2} \frac{|\mathbf{a}|^2}{c^2}+2\frac{|\mathbf{v}\cdot\mathbf{a}|^2}{c^4} \biggr) \\
&=\frac{2}{3} \frac{q^2}{c} \gamma^6 \biggl(-\gamma^2 |\bm{\beta}\cdot\bm{\dot{\beta}}|^2+\beta^2\gamma^2|\bm{\beta}\cdot\bm{\dot{\beta}}|^2+(1-\beta^2)\dot{\beta}^2+2|\bm{\beta}\cdot\bm{\dot{\beta}}|^2 \biggr) \\
&=\frac{2}{3} \frac{q^2}{c} \gamma^6 \biggl(-\gamma^2 |\bm{\beta}\cdot\bm{\dot{\beta}}|^2+(\gamma^2-1)|\bm{\beta}\cdot\bm{\dot{\beta}}|^2+(1-\beta^2)\dot{\beta}^2+2|\bm{\beta}\cdot\bm{\dot{\beta}}|^2 \biggr) \\
&=\frac{2}{3} \frac{q^2}{c} \gamma^6 \biggl( |\bm{\beta}\cdot\bm{\dot{\beta}}|^2+(1-\beta^2)\dot{\beta}^2 \biggr) \\
&=\frac{2}{3} \frac{q^2}{c} \gamma^6 \biggl( \beta^2\dot{\beta}^2\cos^2\theta+\dot{\beta}^2-\beta^2\dot{\beta}^2 \biggr) \\
&=\frac{2}{3} \frac{q^2}{c} \gamma^6 \biggl( \dot{\beta}^2+\beta^2\dot{\beta}^2(\cos^2\theta-1) \biggr) \\
&=\frac{2}{3} \frac{q^2}{c} \gamma^6 \biggl( \dot{\beta}^2-\beta^2\dot{\beta}^2\sin^2\theta \biggr) \\
&=\frac{2}{3} \frac{q^2}{c} \gamma^6 \biggl( \dot{\beta}^2-(\bm{\beta}\times\bm{\dot{\beta}})^2 \biggr) \\
&=\frac{2}{3}\frac{q^2}{c^3} \gamma^6 (|\mathbf{a}|^2-|\mathbf{a}\times\bm{\beta}|^2) \, .
\end{align*}


\paragraph{2.b.16}\textbf{Calcolare il fattore di forma elettromagnetico per una sfera uniformemente carica di raggio a.}\\ \\
Si suppone che la sfera uniformemente carica abbia carica Q e raggio a, da cui  $\rho(\vec{r})=\frac{Q}{\frac{4 \pi a^3}{3}} \theta(a-r)$.
\begin{align}
     F(\vec{q})&=\int \rho(\vec{r}')e^{-i\vec{q}\cdot\vec{r}'}dV'=\int_0^\infty\int_{-1}^1\int_0^{2\pi}\frac{3Q}{4 \pi a^3} \theta(a-r) e^{-i\vec{q}\cdot\vec{r}'}r'^2dr'dcos(\beta)d\alpha=\\
     &= \frac{2 \pi 3Q}{4 \pi a^3}\int_0^a\int_{-1}^1 e^{-iqr'cos\beta}(r')^2 dr' dcos\beta=\frac{3 Q}{2 a^3}\int_0^{a} \frac{r'}{i q} (e^{iqr'}-e^{-iqr'}) dr' =\\&=\frac{3 Q}{2 a^3} \left( -\frac{e^{iqa}}{q^2}a +\frac{e^{iqa}}{iq^3} -\frac{e^{-iqa}}{q^2}a -\frac{e^{-iqa}}{iq^3} \right)=\frac{3 Q}{2 a^3} \left( -\frac{a}{q^2}2cos(qa)+\frac{1}{iq^3} 2i sen(qa) \right)=\\
     &=\frac{3Q}{a^3 q^2}\left( \frac{sen(qa)}{q} - a cos(qa)\right)
\end{align}
dove $\vec{q}=\vec{k}-\vec{k_0}$ con $k_0$ impulso entrante e $k$ impulso uscente.

\paragraph{2.b.17}\textbf{Calcolare il fattore di forma per una superficie sferica uniformemente carica di raggio a. [nota: l'interno è vuoto]}\\
Si suppone che la superficie sferica abbia carica Q e raggio a, da cui $\rho(\vec{r})=\frac{Q}{4\pi a^2}\delta(|\vec{r}|-a)$.
\begin{align}
     F(\vec{q})&=\int \rho(\vec{r}')e^{-i\vec{q}\cdot\vec{r}'}dV'=\int_0^\infty\int_{-1}^1\int_0^{2\pi}\frac{Q}{4\pi a^2}\delta(|\vec{r}|-a)e^{-i\vec{q}\cdot\vec{r}'}r'^2dr'dcos(\beta)d\alpha=\\
     &=\frac{Q}{4\pi a^2}2\pi\int_{-1}^1e^{-iqacos(\beta)}a^2cos(\beta)=Q\frac{sin(qa)}{qa}  
\end{align}
Con q=$2k_0sin(\frac{\theta}{2})$: per $\theta=0$ si ha una completa coerenza poichè q=0, mentre gli zeri sono in qa=n$\pi$ con n$\ne 0, n\in \mathds{Z}$.

\paragraph{2.b.18}\textbf{A partire dai campi ritardati, dimostrare che
\[
\frac{dP}{d\Omega}= \frac{q^2 |\vec{a}|^2}{16 \pi^2\epsilon_0 c^3}\frac{sen^2(\theta)}{(1-\beta cos(\theta))^5}
\]
è la potenza (MKSA) irraggiata da una carica accelerata in un moto rettilineo. 
}\\ \\
I campi ritardati nella zona di radiazione sono:
\[
\vec{E}=\left[\frac{q}{R c}\frac{\hat{n}\times[(\hat{n}-\vec{\beta})\times \dot{\vec{\beta}}]}{(1-\hat{n}\cdot \vec{\beta})^3}\right]_{t'=t-R/c}
\]
\[
\vec{B}=[\hat{n}\times \vec{E}]_{t'=t-R/c}
\]
dove $\vec{R}=\vec{r}-\vec{s}(t)$, $s(t)$ è la legge oraria della particella carica, $R=|\vec{R}|$, $\hat{n}=\vec{R}/R$ ed il tempo ritardato $t'$ è definito dalla relazione $t-t'=|\vec{r}-\vec{s}(t')|/c$.
Il vettore di Poynting in CGS è:
\[
\vec{S}=\frac{c}{4 \pi}(\vec{E}\times \vec{B})
\]
Se la particella è in un moto rettilineo $\vec{\beta}\times \dot{\vec{\beta}}=0$ e la componente radiale del vettore di Poynting:
\[
\vec{S}\cdot\hat{n}= \frac{c}{4 \pi} |\vec{E}|^2=\frac{q^2}{4 \pi c} \left[\frac{1}{R^2}\frac{|\hat{n} \times \dot{\vec{\beta}}|^2} {(1-\hat{n}\cdot\vec{\beta})^6}\right]_{t'=t-R/c}
\]
Questa è l’energia per unità di superficie per unità di tempo nella direzione $\hat{n}$ rilevata in un punto al tempo $t$ della radiazione emessa dalla carica al tempo $t'$. Se si vuole calcolare l'energia irraggiata per unità di angolo solido della particella durante un intervallo di tempo $T_1<t'<T_2 $ si deve passare all'integrale e fare un cambio di variabile.
\[
\mathcal{E}=\int_{t=T_1+ |\vec{r}-\vec{s}(T_1)|/c}^{t=T_2+ |\vec{r}-\vec{s}(T_2)|/c} [R^2 \vec{S}\cdot \hat{n}]_{t'=t-R/c} dt= \int_{t'=T_1}^{t'=T_2}(R^2 \vec{S}\cdot \hat{n})\frac{dt}{dt'}dt'
\]
in particolare $(\vec{S}\cdot \hat{\vec{n}})\frac{dt}{dt'}$ è la potenza irraggiata per unità di superficie in termini del tempo proprio della carica.\\
Usando il fatto che $dt/dt'=1-\vec{\beta} \cdot \hat{n}$ e chiamando $\theta$ l'angolo tra il punto di osservazione e la direzione della carica, la potenza irraggiata per unità di angolo solido è:
\[
\frac{dP}{d\Omega}=R^2 (\vec{S}\cdot \hat{n})\frac{dt}{dt'}= R^2\vec{S}\cdot \hat{n}(1-\hat{n}\cdot\vec{\beta})= \frac{q^2}{4 \pi c} \frac{|\hat{n}\times \dot{\vec{\beta}}|^2}{(1-\hat{n}\cdot \vec{\beta})^5}=\frac{q^2 |\dot{\vec{\beta}}|^2}{4 \pi c} \frac{sin^2\theta}{(1-\beta cos \theta)^5}
\]
In definitiva:
\[
\frac{dP}{d\Omega}=\frac{q^2|\vec{a}|^2}{4 \pi c^3}\frac{sin^2\theta}{(1-\beta cos\theta)^5}
\]
Dividendo per $4 \pi \epsilon_0$ si trova il risultato in MKSA:
\[
\frac{dP}{d\Omega}=\frac{q^2|\vec{a}|^2}{16 \pi^2 \epsilon_0 c^3}\frac{sin^2\theta}{(1-\beta cos\theta)^5}
\]
\paragraph{2.b.19}\textbf{Calcolare l’energia persa in una rivoluzione per una carica in moto uniforme su una circonferenza (acceleratore circolare). Calcolare la frazione di energia persa in un giro rispetto alla sua energia cinetica, effettuando una valutazione numerica, nel caso di elettroni a LEP (energia 50 GeV, raggio ~4km) o protoni ad LHC (energia 7 TeV, raggio ~4km). \\Nota: utilizzare la formula di Larmor in GCS:
$P = \frac{2q^2}{3c^3}\gamma^6(|\vec{a}|^2-|\vec{a}\times \vec{\beta}|^2)$}\\ \\
Durante il moto si ha $\vec{a}\perp\vec{\beta}$, per cui $|\vec{a}\times\vec{\beta}|^2=|\vec{a}|^2|\vec{\beta}|^2$.\\
Per cui si ha:
\begin{equation}
    P=\frac{2q^2}{3c^3}\gamma^6(|\vec{a}|^2(1-|\vec{\beta}|^2))=\frac{2q^2}{3c^3}\gamma^4|\vec{a}|^2=\frac{2q^2}{3c^3}\gamma^4 \frac{v^4}{R^2}
\end{equation}
Supponendo che la perdita di energia sia abbastanza piccola da considerare $\beta$ costante si può integrare la potenza irraggiata lungo un giro:
\begin{equation}
    \Delta E=\int_0^{\frac{2\pi R}{v}}\frac{2q^2}{3c^3}\gamma^4\frac{v^4}{R^2}dt=\frac{4\pi q^2}{3R}\gamma^4\beta^3
\end{equation}
(l'integrando non dipende dal tempo)
Utilizzando le note relazioni ($E=m\gamma c^2$, $p=m\gamma v$, $\beta=v/c$):
\begin{equation}
    v=\frac{pc^2}{E} \qquad p=\sqrt{\frac{E^2}{c^2}-m^2c^2} \qquad \gamma=\frac{E}{mc^2}
\end{equation}
Si ottiene($q=Z e$, $r_e^{(CGS)}= e^2/(mc^2)$):
\begin{align}
    \Delta E&=\frac{4\pi q^2}{3c^3 R}\frac{E^4}{m^4c^8}\frac{p^3c^6}{E^3}=\frac{4\pi q^2}{3R}\frac{Ep^3}{m^4c^5}=\frac{4\pi q^2}{3R}\frac{E}{m^4c^5}\biggl(\frac{E^2}{c^2}-m^2c^2\biggr)^{\frac{3}{2}}=\\
    &=\frac{4\pi Z^2r_e^{(CGS)}}{3R}\frac{E}{m^3 c^3}\biggl(\frac{E^2}{c^2}-m^2c^2\biggr)^{\frac{3}{2}}=\frac{4\pi Z^2r_e^{(CGS)}}{3R}E\biggl(\biggl(\frac{E}{mc^2}\biggr)^2-1\biggr)^{\frac{3}{2}}
\end{align}
Da cui
\begin{equation}
    \frac{\Delta E}{E}=\frac{4\pi Z^2r_e^{(CGS)}}{3R}\biggl(\biggl(\frac{E}{mc^2}\biggr)^2-1\biggr)^{\frac{3}{2}}
\end{equation}
Moltiplicando tutto per il fattore $\frac{1}{4\pi\epsilon_0}$ si passa al sistema MKSA, per cui basta considerare il raggio classico dell'elettrone come $r_e=\frac{1}{4\pi\epsilon_0}\frac{e^2}{mc^2}\sim 2.8 \cdot 10^{-15} \ m$ .\\
\begin{equation}
    \text{MKSA:} \;\;\; \frac{\Delta E}{E}=\frac{4\pi Z^2r_e}{3R}\biggl(\biggl(\frac{E}{mc^2}\biggr)^2-1\biggr)^{\frac{3}{2}}
\end{equation}
Di conseguenza nei casi richiesti:
\begin{itemize}
    \item E=50 GeV, R=4 km, Z=-1, m$_e$=0.511 MeV/c$^2$
    \begin{equation}
        \frac{\Delta E}{E}\sim 3\cdot 10^{-8}
    \end{equation}
    \item E=7 TeV, R=4 km, Z=1, m$_p$=938,27 MeV/c$^2$
    \begin{equation}
        \frac{\Delta E}{E}\sim 2\cdot 10^{-10}
    \end{equation}
\end{itemize}
\paragraph{2.b.20}\textbf{Calcolare la potenza emessa in funzione dell’angolo per una carica oscillante armonicamente in linea retta (termine di dipolo elettrico)}\\ \\
CGS\\
Partiamo da un piccolo dipolo oscillante. 
\[
\vec{p}=\vec{p_0}e^{- i\omega t} \;\;\; k=\omega/c
\]
Nella regione di radiazione i campi sono:
\begin{equation}
    \vec{E}=\left|\frac{\hat{n} \wedge(\hat{n} \wedge \ddot{\vec{p}})}{r c^2}\right|_{t-r/c}=\cdots =\frac{k^2}{r}\vec{p}_{0\perp}e^{-i (\omega t- kr)}
\end{equation}
\begin{equation}
    \vec{B}=\hat{n} \times \vec{E}
\end{equation}
con
\[
\vec{p}_{0\perp}=\vec{p}-(\hat{n}\cdot \vec{p})\hat{n} 
\]
\begin{figure}[H]
    \centering
    \includegraphics[scale=0.4]{2.b.20.PNG}
\end{figure}

\begin{align*}
    \vec{S}&=\frac{c}{4 \pi}(Re\vec{E})\times(Re\vec{B})=\\
    &=\frac{c}{4 \pi}\frac{k^4}{r^2}[\vec{p}_{0\perp}\times(\hat{n}\times\vec{p}_{0\perp})]\hat{n}cos^2(\omega t-kr) =\\
    &=\frac{ck^4}{4\pi r^2}|\vec{p}_{0 \perp}|^2 cos^2(\omega t- kr)\hat{n}=\\
    &=\frac{ck^4}{ 4 \pi r^2 }|\vec{p}_0|^2sin^2\theta cos^2(\omega t-k r)\hat{n}\, 
\end{align*}

Per la Potenza totale irraggiata, integriamo su una superficie di raggio r:
\begin{align*}
    P_{TOT}(t)&=\int \vec{S}\cdot \hat{n} dA=\\
    &=\int \frac{ck^4}{ 4 \pi r^2 }|\vec{p}_0|^2sin^2\theta cos^2(\omega t-k r) r^2 dcos\theta d\phi=\\
    &=\frac{ck^4}{ 4 \pi}cos^2(\omega t-k r)\int_{-1}^1\int_{0}^{2\pi} sin^2\theta dcos\theta d\phi=\\
    &=\frac{ck^4|\vec{p}_0^2|}{4\pi}cos^2(\omega t-kr)\frac{8\pi}{3}=\\
    &=\frac{2}{3}ck^4|\vec{p}_0|^2cos^2(\omega t-k r)
\end{align*}
Se si vuole la potenza mediata nel tempo:
\begin{equation}
    <P_{TOT}(t)> =\frac{1}{3}ck^4|\vec{p}_0|^2
\end{equation}

\paragraph{2.b.21}\textbf{Enunciare il teorema ottico e spiegarne il significato fisico nel caso di radiazione elettromagnetica su un ostacolo opaco.}\\ \\
Il teorema ottico mostra come non sia possibile avere assorbimento di un onda senza avere diffusione elastica. Nel caso radiazione elettromagnetica su un ostacolo opaco si ha:
\begin{equation}
    \sigma_{tot}=\frac{\frac{4\pi}{k_0}Im\{\vec{E}_0^*\cdot\vec{f}(0)\}}{|\vec{E}_0|^2}
\end{equation}
Sfruttando il principio di Babinet per un ostacolo opaco lineare di area efficace A si ha difatti:
\begin{equation}
    P_{abs}=|\vec{S}_{in}|A \qquad P_{diff}=P_{abs}=|\vec{S}_{in}|A \qquad \sigma_{tot}=\frac{<P_{tot}^{ost}>}{<|\vec{S}_{in}|>}=2A
\end{equation}
Non è possibile avere assorbimento senza diffusione elastica poiché si trova che sezione d'urto è uguale a $2A$,come predetto dal (2.102), cosa impossibile se non si considerano entrambi i contributi. \\
E l'ampiezza di scattering in avanti si ha difatti: $\vec{f}(\vec{k}=\vec{k}_0)= \vec{f}(\vec{0})=\frac{ik_0\vec{E}_0A}{2\pi} \propto A_{oggetto}$
\paragraph{2.b.22}\textbf{Calcolare la figura di diffrazione per un'onda e.m. di lunghezza d'onda $\lambda$ che incide su una fenditura rettilinea di larghezza L.}\\ \\
Facciamo le seguenti ipotesi: fenditura infinitamente estesa in y e di larghezza L in x, $\vec{E}_{in}$ polarizzato lungo y (parallelamente alle fenditure).
\begin{figure}[H]
    \centering
      \includegraphics[scale=0.2]{2.b.22_1.jpg}
\end{figure}
Calcoliamo il fattore di forma (2D):
\begin{align*}
    F(\vec{k})&=\int_{\Sigma'}e^{-i \vec{k} \vec{r}'} d\Sigma'=\\
    &=\int_{-L/2}^{L/2} \int_{-\infty}^{+\infty}e^{-i k_x x'}e^{-i k_y y'} dx'dy'=\\
    &=\left(\int_{-L/2}^{L/2} e^{-i k_x x'} dx' \right)\left(\int_{-\infty}^{+\infty}e^{-i k_y y'} dy'\right)\;
\end{align*}
Considerando che la Delta di Dirac è definita come  $\delta(t)=\int_{-\infty}^{+\infty} e^{-izt}dz$ il secondo integrale è  $\delta(k_y)$. " Si annulla per $k_y\ne 0$, fisicamente ciò significa che non c'è diffrazione lungo y". Se $k_y=0$ ed il vettore $\vec{k}$ è nel piano xz, $\hat{n}\cdot \vec{E}_0=0$ e $\vec{E}_{0\perp}=\vec{E}_0$.
\begin{equation*}
    F(k_x)=\int_{-L/2}^{L/2} e^{-i k_x x'} dx'= \left[ \frac{e^{-ik_x x'}}{-i k_x}  \right]^{L/2}_{-L/2}= a\frac{sen(\frac{k_x a}{2})}{\frac{k_x a}{2}}
\end{equation*}
Gli zeri si trovano per $\frac{k_x a}{2}=n\pi$ con $n \not= 0$.
Riportiamo in seguito il disegno del fattore di forma trovato.
\begin{figure}[H]
    \centering
      \includegraphics[scale=0.2]{2.b.22_2.jpg}
\end{figure}
Per calcolare la figura di diffrazione si deve antitrasformare $F(k_x)$.
\begin{equation*}
   \hat{F}(k_x)= \frac{1}{2 \pi}\int_{-\infty}^{+\infty} 
   a\frac{sen(\frac{k_x a}{2})}{\frac{k_x a}{2}}
   e^{ik_xx'}dk_x
\end{equation*}
Si trova quindi:
\begin{equation*}
    \hat{F}(x')=\theta\left(x'+\frac{L}{2}\right)-\theta\left(x'-\frac{L}{2}\right)
\end{equation*}
con $\theta=0$ se $t<0$ e  $\theta=1$ se $t>0$
Si trova quindi la seguente figura di diffrazione:
\begin{figure}[H]
    \centering
      \includegraphics[scale=0.2]{2.b.22_3.jpg}
\end{figure}
ATTENZIONE:\\
Verrebbe da pensare che se riesco a misurare il fattore di scattering (ad esempio mandando luce) riesco a trovare la forma dell'oggetto facendo l'antitrasformata. IN REALTA' NO. \\
Problema 1: $\lambda<< a sen\theta$\\
Problema 2: Quello che misuro sperimentalmente sono le intensità, quindi il modulo quadro del fattore di forma, può essere lo stesso per tanti tipi di fattori di forma differenti. Mi occorre conoscere anche la fase.

\paragraph{2.b.23}\textbf{Come si ricava la sezione d'urto differenziale Rayleigh a partire dalla sezione d'urto Thomson?}\\ \\
Lo scattering Thomson è dato da radiazione EM incidente su un singolo elettrone atomico legato.\\
La sezione d'urto differenziale per questo processo è:
\begin{equation}\label{2.b.23 1}
    \frac{d\sigma_{el}}{d\Omega}=\frac{r_e^2\omega^4}{(\omega_0^2-\omega^2)^2+\omega^2\Gamma^2_{tot}(\omega)}sin^2\alpha
\end{equation}
Con $\alpha$ angolo tra la direzione di polarizzazione del campo e la direzione di osservazione.
\\
Diversamente nello scattering Rayleigh la radiazione EM incide sull'intero atomo. E' dunque necessario considerare sia la potenza irraggiata da ogni singolo elettrone e la mutua interferenza tra essi.\\
Supponendo lo scattering elastico ($|\vec{k}|=|\vec{k'}|$) dati 2 centri diffusori a distanza $\vec{r}$ la differenza di cammino ottico è data da:
\begin{equation}
    \Delta l=\vec{r}\cdot\frac{\vec{k'}}{|\vec{k}|}-\vec{r}\cdot\frac{\vec{k}}{|\vec{k}|}=\frac{\vec{r}\cdot(\vec{k}'-\vec{k})}{|\vec{k}|}\equiv \frac{\vec{r}\cdot\vec{q}}{|\vec{k}|}
\end{equation}
\begin{figure}[H]
    \centering
    \includegraphics[scale=0.6]{2.b.23.PNG}
    \label{2.b.23}
\end{figure}
Assumendo che i 2 centri diffusori siano fisicamente indistinguibili si ottiene $\vec{E}=\vec{E}_0(e^{-i\phi_1}+e^{-i\phi_2})$ con $\phi_2-\phi_1=|\vec{k}|\Delta l=(\vec{r}_2-\vec{r}_1)\cdot \vec{q}$.\\
Generalizzando ad N diffusori
\begin{equation}
    \vec{E}=\vec{E}_0\sum_{i=0}^N e^{-i\vec{r}_i\cdot\vec{q}}
\end{equation}
Utilizzando la definizione di fattore di forma elettromagnetico per tenere conto della distribuzione totale:
\begin{equation}
    F(\vec{q})=\frac{\int \rho(\vec{r})e^{-i\vec{r}\cdot \vec{q}}d^3r}{\int \rho(\vec{r})d^3r}
\end{equation}
Si può riscrivere la sezione d'urto differenziale come:
\begin{equation}\label{2.b.23 2}
    \frac{d\sigma_{el}}{d\Omega}=\frac{d\sigma_{el}}{d\Omega}\biggl|_e|ZF(\vec{q})|^2
\end{equation}
Dove con $\frac{d\sigma_{el}}{d\Omega}\biggl|_e$ si fa riferimento alla sezione d'urto differenziale per scattering su singolo elettrone in equazione (\ref{2.b.23 2}).\\
SI osservi che $|\vec{q}|=2|\vec{k}|sin\frac{\theta}{2}$, per cui vi è una dipendenza sia dall'angolo di scattering che dalla frequenza.\\
%------------------------------------
\newpage
\section{Metodi di indagine tramite processi di urto o tramite decadimenti spontanenei; introduzione alle interazioni forti e deboli ed alle strutture sub-atomiche e sub-nucleari}
\epigraph{For a moment or two the old man didn't reply. He was staring at the instruments with the air of one who is trying to convert Fahrenheit to centigrade in his head while his house is burning down}{}
\textit{Domande a}\rule{13.7 cm}{0.4pt}

\paragraph{3.a.1}\textbf{Descrivere qualitativamente il fenomeno dell'assorbimento, il fenomeno della
diffusione elastica ed il fenomeno della diffusione inelastica di un'onda e.m. su
un sistema}\\ \\
Quando un'onda elettromagnetica incide su un sistema:
\begin{itemize}
\item parte della potenza incidente viene assorbita dal bersaglio: fenomeno dell'assorbimento.
\item parte viene diffusa con la stessa frequenza dell'onda incidente: Scattering elastico.
\\
La diffusione avviene perchè il sistema, stimolato dalla radiazione
incidente, contiene cariche elettriche accelerate o correnti variabili nel
tempo, che irraggiano onde e.m. con la stessa frequenza dell'onda
incidente.
\item parte viene diffusa con frequenze differenti rispetto all'onda incidente: Scattering Inelastico. \\
La diffusione inelastica avviene solo se il sistema, stimolato dalla
radiazione incidente, ha moti accelerati di cariche o correnti variabili nel
tempo, che irraggiano onde e.m. con frequenze differenti rispetto alla
frequenza dell'onda incidente.
Avviene in sistemi non lineari.
\end{itemize}


\paragraph{3.a.2}\textbf{Per un’onda e.m. monocromatica che incide su un bersaglio (per esempio un
circuito o un atomo) definire le sezioni d’urto: i) di assorbimento, ii) elastica
differenziale, iii) totale elastica; iv) inelastica differenziale; v) inelastica totale;
vi) totale.}\\ \\
Per un'onda elettromagnetica piana incidente su un bersaglio, si definisce:
\begin{itemize}
    \item \textit{Sezione d'urto di assorbimento} il rapporto tra la potenza mediata nel tempo assorbita dal sistema e il modulo del vettore di Poynting mediato nel tempo dell'onda incidente
    \begin{equation}
        \sigma_{abs}\equiv \frac{<P_{abs}>}{<|\vec{S}_{in}|>}
    \end{equation}
    
    \item \textit{Sezione d'urto differenziale elastica} il rapporto fra la potenza mediata nel tempo
diffusa dal sistema alla stessa frequenza dell'onda incidente ed il modulo
del vettore di Poynting mediato nel tempo dell'onda incidente.
\\
A grande distanza il campo elettrico di radiazione è una funzione di $\vartheta, \varphi$ e varia come $1/R$ quindi:
\begin{equation}
    \frac{d\sigma_{el}}{d\Omega}\equiv R^2\frac{<|\vec{S}_{el}(\vartheta,\varphi)|>}{<|\vec{S}_{in}|>}
\end{equation}
    
    \item \textit{Sezione d'urto totale elastica} l'integrale della sezione d'urto differenziale elastica sull'angolo solido $d\Omega=dcos\vartheta d\varphi$
    \begin{equation}
        \sigma_{el}=\int \frac{d\sigma_{el}}{d\Omega}\ d\Omega=\int \frac{<|\vec{S}_{el}(\vartheta,\varphi)|>}{<|\vec{S}_{in}|>}\ R^2d\Omega=\frac{<P_{el}>}{<|\vec{S}_{in}|>}
    \end{equation}
    
    \item \textit{Sezione d'urto inelastica differenziale} ad una frequenza $\omega_i$ il rapporto fra la potenza mediata nel tempo
    diffusa dal sistema alla frequenza $\omega_i$ ed il modulo del vettore di Poynting
    mediato nel tempo dell'onda incidente.
    \\
    Il campo di radiazione varia come $1/R$ ed è funzione di $\vartheta, \varphi$.
    \\
    Per ogni frequenza $\omega_i$ a cui avviene la diffusione si definisce:
    \begin{equation}
        \frac{d\sigma_{\omega_i}}{d\Omega}\equiv R^2\frac{<|\vec{S}_{\omega_i} (\vartheta, \varphi)|>}{<|\vec{S}_{in}|>}
    \end{equation}
    
    \item \textit{Sezione d'urto totale inelastica} l'integrale della sezione d'urto differenziale inelastica sull'angolo solido
    \begin{equation}
        \sigma_{\omega_i}=\int \frac{d\sigma_{\omega_i}}{d\Omega} \ d\Omega=\int \frac{<|\vec{S}_{\omega_i}(\vartheta, \varphi)|>}{<|\vec{S}_{in}|>} \ R^2 d\Omega=\frac{<P_{\omega_i}>}{<|\vec{S}_{in}|>}
    \end{equation}
    
    \item \textit{Sezione d'urto totale} la somma delle sezioni d'urto di tutti i
possibili stati finali del sistema
\begin{equation}
    \sigma_{tot}=\sigma_{abs}+\sigma_{el}+\sigma_{\omega_i}+...+\sigma_{\omega_N}=\sum_{f}\sigma_f=\frac{\sum_f <P_f>}{<|\vec{S}_{in}|>}
\end{equation}
  Se lo scattering inelastico anzichè avvenire con uno spettro discreto avvenisse con uno continuo, sarebbe necessario sostituire alla sommatoria un integrale sulle frequenze.  
  
\end{itemize}

\paragraph{3.a.3}\textbf{Definire la ampiezza di scattering per un’onda e.m. monocromatica che incide su
un bersaglio fisso (per esempio un circuito o un atomo). }\\ \\
Per un dato stato finale in cui un'onda diffusa viene generata nel bersaglio, a
grandi distanze il campo elettrico si può scrivere come il prodotto di un'onda sferica uscente e di un temine che racchiude tutta la dinamica del
processo, l'ampiezza di scattering $\vec{f}$: 

\begin{equation}\label{scat}
    \vec{E}_f=\vec{f}(\vartheta, \varphi)\frac{e^{-i(\omega_f t -k_f R +\phi)}}{R}
\end{equation}

Dove $\omega_f$ è la frequenza dell'onda uscente, $\vec{k}_f$ è il vettore d'onda uscente e $\phi$ è la fase.
\\
Invertendo la (\ref{scat}) si trova:
\begin{equation}
    \vec{f}(\vartheta, \varphi)=R\vec{E}_f e^{i(\omega_f t -k_f R +\phi)}
\end{equation}


\paragraph{3.a.4}\textbf{Descrivere la situazione in cui la legge
\begin{equation*}
    P=\frac{2}{3c^3}\Ddot{\vec{p}}_{el}^2 +\frac{1}{180 c^5}\dddot{Q}_{ij}^2+\frac{2}{3c^3}\Ddot{\vec{p}}_m^2
\end{equation*}
(che è espressa in
CGS) è applicabile e spiegare il significato e l'unità di misura di ogni grandezza
fisica ivi indicata; trascrivere poi l'espressione in MKSA.}\\ \\
Questa equazione è applicabile in formulazione non relativistica ($|\vec{V}|<<c$) e quando un sistema ha, nel proprio sviluppo in multipoli, i termini di dipolo elettrico, quadrupolo elettrico e dipolo magnetico non nulli.\\

$P$ è la potenza irraggiata di dipolo elettrico, quadrupolo elettrico e dipolo magnetico da un certo sistema. \\
\begin{equation}
\vec{p}_e=\sum_{cariche} q\vec{r} \ \ \ \ \ \ [D(Debye)=10^{-18} \ statC\times cm]
\end{equation} 
è il termine di dipolo elettrico,
\begin{equation}\begin{split}
\vec{p}_m=\frac{1}{2}\sum_{cariche}q\vec{r}\wedge \vec{V} \ \ \ \ \  \text{(MKSA)} \ \ \ \ [m^2\times A] \\ \vec{p}_m=\frac{1}{2c}\sum_{cariche}q\vec{r}\wedge \vec{V} \ \ \ \ \  \text{(CGS)} \ \ \ \ [EMU=1 \ erg/G]
\end{split}\end{equation}
è il termine di dipolo magnetico, mentre 
\begin{equation}
Q_{ij}=\sum_{cariche}q(3x_ix_j-\delta_{ij}r^2) \ \ \ \ [C\times m^2] \ \text{(MKSA)} \ \ \ [statC\times cm^2] \ \text{(CGS)}
\end{equation}
è il termine di quadrupolo elettrico.
\\
In MKSA l'espressione diventa:
\begin{equation}
    P=\frac{1}{6\pi \varepsilon_0 c^3}\Ddot{\vec{p}}^2_e + \frac{1}{6\pi \varepsilon_0 c^5}\Ddot{\vec{p}}^2_m+\frac{1}{720 \pi \epsilon_0 c^5}\dddot{Q}_{ij}^2
\end{equation}


\paragraph{3.a.5}\textbf{Scrivere la distribuzione angolare della radiazione di dipolo elettrico e di dipolo
magnetico nel caso non relativistico.}\\ \\
Un dipolo magnetico variabile nel tempo emette radiazione con una distribuzione angolare:
\begin{equation}
    \frac{dP_m}{d\Omega}=\frac{dP_m}{dcos\alpha \ d\beta}=\frac{1}{16 \pi^2 \varepsilon_0 c^5}\Ddot{\vec{p}}_m^2 sin^2 \alpha
\end{equation}
Mentre un dipolo elettrico emette con distribuzione angolare:
\begin{equation}
    \frac{dP_e}{d\Omega}=\frac{dP_e}{dcos\alpha \ d\beta}=\frac{1}{16 \pi^2 \varepsilon_0 c^3}\Ddot{\vec{p}}_e^2 sin^2 \alpha
\end{equation}
e si può verificare che si ritrovano le note formule per la potenza diffusa dai dipoli integrando le espressioni di cui sopra sull'angolo solido.

\paragraph{3.a.6}\textbf{Definire la "resistenza di irraggiamento" di un circuito elettrico a una maglia e
fornire un esempio.}\\ \\
Nota la corrente che scorre in un circuito, la resistenza di irraggiamento è la resistenza che dovrebbe avere per dissipare una potenza pari alla potenza irraggiata:
\begin{equation}
    R_{irr}=\frac{P_{irr}}{I^2}
\end{equation}
Se si prende come esempio un circuito planare, per il principio di equivalenza di Ampere, questo possiede un momento magnetico:
\begin{equation}
    \vec{p}_m=IS\hat{n}
\end{equation}
Da cui, conoscendo la potenza irraggiata da un dipolo magnetico:
\begin{equation}
    P=\frac{\Ddot{\vec{p}}_m^2}{6\pi \varepsilon_0 c^5}
\end{equation}
si trova:
\begin{equation}
    R_{irr}=\frac{2}{3}\frac{S^2\omega^4}{c^5}
\end{equation}
\paragraph{3.a.7}\textbf{Definire urto elastico ed urto inelastico fra due particelle; fornire poi almeno un esempio di reazione elastica ed una inelastica fra: i) un fotone ed un atomo; ii) due particelle cariche; iii) un protone ed un nucleo.
}\\ \\
Un urto elastico è un urto che non cambia la natura dei corpi che interagiscono, ossia:
\begin{equation}
    a+b \ \rightarrow \ a+b
\end{equation}

Un urto si dice invece inelastico se cambia la natura dei corpi che interagiscono, ossia:
\begin{equation}
    a+b \ \rightarrow \ P_1+P_2+...P_n
\end{equation}
Gli urti inelastici si definiscono \textit{esotermici} se:
\begin{equation}
    Q\equiv \sum m_{in}c^2-\sum m_f c^2 > 0
\end{equation}
Dove Q è il Q-valore dell'urto.\\
Si definiscono \textit{endotermici} se:
\begin{equation}
    Q<0
\end{equation}
Esempi di reazione elastica tra:
\begin{itemize}
    \item \textit{un fotone ed un atomo} \\
    Scattering Rayleigh
    \begin{equation}
        \gamma + A \ \rightarrow \ \gamma + A
    \end{equation}
    \item\textit{due particelle cariche}
    \begin{equation}
        p + p \ \rightarrow \ p + p
    \end{equation}
    \item\textit{un protone ed un nucleo}\\
    Un qualsiasi urto protone su nucleo in cui il nucleo non si rompe nell'urto
    \begin{equation}
        p + A \ \to \ p+A
    \end{equation}
\end{itemize}

Esempi di reazione inelastica tra:
\begin{itemize}
    \item \textit{un fotone ed un atomo} \\
    Effetto fotoelettrico 
    \begin{equation}
        \gamma + A \ \rightarrow \ e^- + A^+
    \end{equation}
    \item\textit{due particelle cariche}
    \begin{equation}
        p+p \ \rightarrow \ p+p+p+\bar{p}
    \end{equation}
    \item\textit{un protone ed un nucleo} \\
    Reazione degli alchimisti
    \begin{equation}
        p+ ^{198}_{80}Hg_{118} \ \rightarrow \ p+p+^{197}_{79}Au^- _{118}
    \end{equation}
\end{itemize}

\paragraph{3.a.8}\textbf{Cercare nel sito del PDG del CERN o di LBL le seguenti particelle : $n,\pi^0,\pi^+,K^+,\mu^+,\tau^+$ e prendere nota delle loro seguenti quantità: massa, larghezze di
decadimento parziali, vita media, spin, modulo del momento magnetico.}\\ \\
massa: MASS\\
larghezza di decadimento parziale:DACAY MODES-BRANCHING RATIOS\\
vita media: MEAN LIFE  \\
spin: $J$, $J^{P}$ (in alto alla pagina)\\
modulo di momento magnetico: MAGNETIC MOMENT\\




$n$: \url{https://pdg.lbl.gov/2020/listings/rpp2020-list-n.pdf}\\
$\pi^0$: \url{https://pdg.lbl.gov/2020/listings/rpp2020-list-pi-zero.pdf}\\
$\pi^+$: \url{https://pdg.lbl.gov/2020/listings/rpp2020-list-pi-plus-minus.pdf}\\
$K^+$: \url{https://pdg.lbl.gov/2020/listings/rpp2020-list-K-plus-minus.pdf}\\
$\mu^+$: \url{https://pdg.lbl.gov/2020/listings/rpp2020-list-muon.pdf}\\
$\tau^+$: \url{https://pdg.lbl.gov/2020/listings/rpp2020-list-tau.pdf}\\

\begin{tabular}{c|c|c|c|c|c}
    Particella & massa [MeV] & lar dec par & vita media [s] & spin & |mom mag| \\
    \toprule
    $n$ & 939.5654133 $\pm$ 0.0000058  & &879.4 $\pm$ 0.6 & 1/2 & 1.91304273$\pm$0.00000045 \\
    $\pi^0$ & 134.9768$\pm$0.0005 & & (8.52$\pm$0.18)$\times 10^{-17}$& 1 & non pervenuto \\
    $\pi^+$ & 139.57039 $\pm$ 0.00017& &(2.6033 $\pm$0.0005)$\times 10^{-8}$ &  1 & non pervenuto\\
    $K^+$ &  493.677 $\pm$ 0.005 & & (1.2379$\pm$0.0021)$\times 10^{-8}$ & 0 & non pervenuto \\
    $\mu^+$ & 105.6583745$\pm$ 0.0000024  & & (2.1969811 $\pm$ 0.0000022)$\times 10^{-6}$ & 1/2 & anomalo \\ 
    $\tau^+$ & 1776.86 $\pm$ 0.12 & & (290.3 $\pm$ 0.5)$\times10^{-15}$ & 1/2 & anomalo \\ 
    \bottomrule
\end{tabular}
\\
\newline
\newline
\newline
\textbf{RIASSUNTINO}\\
\begin{center}
    \begin{tabular}{|c|c|c|}
    \hline
    classificazione & interazioni & esempi  \\
    \hline
       Leptoni  & EM, debole & $e^{\pm}$, $\mu^{\pm}$, $\tau^{\pm}$, neutrini \\
       \hline
       Adroni & forte, debole, EM & barioni e mesoni \\
       \hline
       Bosoni & Mediatori delle interazioni & Bosoni di Higgs e di Gauge \\
       \hline
    \end{tabular}
\end{center}

\begin{itemize}
    \item \textbf{Bosoni :} fotone, gluone, W, Z, gravitone (?), Higgs, ...
    \item \textbf{Mesoni :} $\pi^0$, $\pi^{\pm}$, $K^0$, $K^{\pm}$, ...
    \item \textbf{Barioni :} $p$, $n$, ...
    \item \textbf{Numero barionico :} vale 1 per i barioni, -1 per gli antibarioni, 0 per le altre particelle.
    \item \textbf{Numero leptonico elettronico :} vale 1 per elettroni e neutrini elettronici, -1 per positroni e antineutrini elettronici, 0 per le altre particelle.
    \item \textbf{Numero leptonico muonico :} vale 1 per muoni e neutrini muonici, -1 per antimuoni e antineutrini muonici, 0 per le altre particelle.
\end{itemize}

\paragraph{3.a.9}\textbf{Dire quali fra le seguenti grandezze si conservano sempre nei processi di urto in
cui avvengano interazioni elettromagnetiche e/o forti, ma non deboli: i) carica
elettrica; ii) numero barionico; iii) numero leptonico elettronico; iv) numero
leptonico muonico; v) numero di elettroni; vi) differenza fra il numero di elettroni
ed il numero di positroni; vii) numero di protoni; viii) differenza fra il numero di
protoni ed il numero di antiprotoni. Fornire almeno un esempio per ogni
situazione in cui vi sia una grandezza non conservata.}\\
\begin{itemize}
    \item \textbf{Carica elettrica:} Sì.
    \item \textbf{Numero barionico:} Sì.
    \item \textbf{Numero leptonico elettronico:}  Sì.
    \item \textbf{Numero leptonico muonico:} Sì.
    \item \textbf{Numero di elettroni:} No. Ad esempio:
    \begin{equation}
        \gamma + Atomo \ \to \ e^+ + e^- + Atomo
    \end{equation}
    \item \textbf{Differenza tra numero di elettroni e di positroni:} Si.
    \item \textbf{Numero di protoni:} No. Ad esempio:
    \begin{equation}
        p+p \ \rightarrow \ p+p+p+\bar{p}
    \end{equation}
    \item \textbf{Differenza tra numero di protoni e di antiprotoni:}
\end{itemize}



\paragraph{3.a.10}\textbf{Dire quali fra le seguenti grandezze si conservano sempre nei processi di urto in
cui avvengano esclusivamente interazioni forti: i) carica elettrica; ii) numero
barionico; iii) numero leptonico elettronico; iv) numero leptonico muonico; v)
numero di elettroni; vi) differenza fra il numero di elettroni ed il numero di
positroni; vii) numero di protoni; viii) differenza fra il numero di protoni ed il
numero di antiprotoni. Fornire almeno un esempio per ogni situazione in cui vi
sia una grandezza non conservata.}\\ \\
\begin{itemize}
    \item \textbf{Carica elettrica:} Sì.
    \item \textbf{Numero barionico:} Sì.
    \item \textbf{Numero leptonico elettronico:} Sì.
    \item \textbf{Numero leptonico muonico:} Sì.
    \item \textbf{Numero di elettroni:} Sì
    \item \textbf{Differenza tra numero di elettroni e di positroni:} Sì.
    \item \textbf{Numero di protoni:} No. Ad esempio:
    \begin{equation}
        \Delta^0 \rightarrow p+\pi^-
    \end{equation}
    \item \textbf{Differenza tra numero di protoni e di antiprotoni:} No.
\end{itemize}


\paragraph{3.a.11}\textbf{Dire quali fra le seguenti grandezze si conservano sempre nei processi di urto in
cui avvengano interazioni deboli: i) carica elettrica; ii) numero barionico; iii)
numero leptonico elettronico; iv) numero leptonico muonico; v) numero di
elettroni; vi) differenza fra il numero di elettroni ed il numero di positroni; vii)
numero di protoni; viii) differenza fra il numero di protoni ed il numero di
antiprotoni. Fornire almeno un esempio per ogni situazione in cui vi sia una
grandezza non conservata.}\\ \\
\begin{itemize}
    \item \textbf{Carica elettrica:} Sì.
    \item \textbf{Numero barionico:} Sì.
    \item \textbf{Numero leptonico elettronico:} No.
    \item \textbf{Numero leptonico muonico:} No.\\
   L'oscillazione dei neutrini è un fenomeno dovuto alla massa non nulla dei neutrini stessi, per cui un neutrino di un sapore può trasformarsi in un neutrino delle altre due, violando così il numero leptonico.
    \item \textbf{Numero di elettroni:} No.
    \begin{equation}
        e^- + e^+ \ \rightarrow \ \text{adroni}
    \end{equation}
    \item \textbf{Differenza tra numero di elettroni e di positroni:} No.
        \begin{equation}
            n \rightarrow p+ e^-+ \bar{\nu_e}
        \end{equation}
    \item \textbf{Numero di protoni:} No.
    \item \textbf{Differenza tra numero di protoni e di antiprotoni:} No.
\end{itemize}
\paragraph{3.a.12}\textbf{Calcolare l'energia che dovrebbe avere un protone che incide su un protone fermo per ottenere un'energia nel centro di massa pari a quella di LHC (14 TeV)}\\
\\
Siano $P_1^\mu$ e $P_2^\mu$ rispettivamente il quadrimpulso del protone incidente e del protone fermo nel sistema del laboratorio. Il quadrimpulso totale prima dell'urto (ponendo $c=1$) è:
\begin{equation*}
P^{\mu}_{tot}=P_1^\mu+P_2^\mu=
\begin{pmatrix}
E_1 \\
p_1 \\
0 \\
0 \\
\end{pmatrix}
+
\begin{pmatrix}
E_2 \\
0 \\
0 \\
0 \\
\end{pmatrix}
=
\begin{pmatrix}
E_1+E_2 \\
p_1 \\
0 \\
0 \\
\end{pmatrix}
=
\begin{pmatrix}
E_1+m_2 \\
p_1 \\
0 \\
0 \\
\end{pmatrix}
\end{equation*}
dato che il secondo protone è fermo. Calcoliamo ora il quadrato della massa invariante del sistema che corrisponde al quadrato dell'energia del centro di massa (il 2° è fermo):
\begin{align*}
(M_{inv})^2&=(E_1+m_2)^2-p_1^2 \\
&=E_1^2+2E_1m_2+m_2^2-p_1^2 \\
&=m_1^2+2E_1m_2+m_2^2 \\
&=2m_p^2+2E_1m_p \, ,
\end{align*}
essendo $m_1=m_2=m_p$. Ponendo l'energia del centro di massa uguale a 14 TeV otteniamo:
\begin{align*}
2m_p^2+2E_1m_p&=14^2 \\
2m_p^2+2E_1m_p&=196 \\
m_p^2+E_1m_p&=98 \\
E_1&=\frac{98-m_p^2}{m_p} \\
E_1&=\frac{98-(0.000938)^2}{0.000938} \\
&\simeq 104477 \quad \text{TeV} \, .
\end{align*}

\paragraph{3.a.13}\textbf{Calcolare l'energia degli elettroni/positroni per innescare la reazione $e^+ +e^-\longrightarrow p+\bar{p}$ in cui i due leptoni collidono con 3-impulsi opposti, ma di moduli diversi.}\\
\\
Se le energie dei due leptoni sono $E_1$ e $ E_2$ allora
\begin{equation}
    P^{\mu}_{tot}=\left(\begin{array}{lr}
         E_1  \\
         p_1
    \end{array}\right)+\left(\begin{array}{lr}
         E_2  \\
         -p_2 
    \end{array}\right)=\left(\begin{array}{lr}
         E_1+E_2  \\
         p_1-p_2 
    \end{array}\right)
\end{equation}
La condizione per l'avvenire del processo è attraverso la massa invariante:
\begin{align}
    M_{inv}^2\geq4m_p^2
\end{align}
da cui
\begin{align}
    M_{inv}^2&=(E_1+E_2)^2-(p_1-p_2)^2=\\
    &=(E_1+E_2)^2+\biggl(\sqrt{E_1^2-m_{e^-}^2}-\sqrt{E_2^2-m_{e^+}^2}\biggr)^2\ge4m_p^2
\end{align}
ma l'energia minima per cui il processo è soddisfatto si ha quando $E_1=E_2$ quindi
\begin{equation}
    E_{min}=m_{p}
\end{equation}
\textbf{E se}\\
Considero il sistema del centro di massa per il processo di urto
\begin{equation}
    \left(\begin{array}{lr}
         \bar{E}  \\
         \bar{p} 
    \end{array} \right)+\left(\begin{array}{lr}
         \bar{E}  \\
         -\bar{p} 
    \end{array} \right)=\left(\begin{array}{lr}
         2\bar{E}  \\
         0 
    \end{array} \right)
\end{equation}
Da cui la condizione di soglia sulla massa invariante risulta
\begin{equation}
    4\bar{E}^2\ge 4m_p^2 \ \Rightarrow \ \bar{E}\ge m_p
\end{equation}
Perciò in questo sistema $\bar{E}_{min}=m_p$. A questo punto trasformo per tornare nel sistema del laboratorio
\begin{equation}
    E_{min}^{lab}=\gamma \bar{E}_{min}=\gamma m_p
\end{equation}
Con $\gamma$ relativo alla trasformazione per passare al centro di massa
\begin{equation}
    \gamma=\frac{1}{\sqrt{1-\beta^2}}
\end{equation}
con $\beta=V_{CM}/c$ e $V_{CM}=\frac{m_e v_1-m_e v_2}{2m_e}=\frac{v_1-v_2}{2}$



\paragraph{3.a.14}\textbf{Definire i processi esclusivi ed inclusivi, il Q-valore di un processo e i processi esotermici e endotermici}
\begin{itemize}
    \item Un processo esclusivo è un processo in cui sono osservate e misurate tutte le grandezze fisiche di tutte le particelle nello stato finale;
    \item Un processo inclusivo è un processo in cui vengono misurate e osservate solo alcune grandezze fisiche di alcune particelle nello stato finale;
    \item Il Q valore di un processo è definito come la differenza tra l' energia di tutte le particelle nello stato iniziale e l'energia di tutte le particelle nello stato finale come:
    \begin{equation}
        Q=\sum_i m_{i,in}c^2-\sum_jm_{j,fin}c^2\,;
    \end{equation}
    \item Un processo esotermico è un processo il cui Q-valore è maggiore di $0$ e quindi viene prodotta energia cinetica;
    \item Un processo endotermico è un processo il cui Q-valore e minore di $0$ e richiede un'energia di soglia per avvenire.
\end{itemize}


\paragraph{3.a.15}\textbf{Dare la definizione operativa della grandezza fisica denominata "\textbf{\textit{densità di corrente (di particelle)}}" oppure "\textbf{\textit{rateo di fluenza}}" oppure "\textbf{\textit{densità di flusso (di particelle)}}".}\\
\\
In un processo in cui su un unico bersaglio incidono dei proiettili con velocità $\vec{V}_p$ e densità volumica $n_p$ allora si definisce come densità di corrente di particelle:
\begin{equation}
    \vec{j}_p=n_p\vec{V}_p
\end{equation}


\paragraph{3.a.16}\textbf{Definire la sezione d’urto nei seguenti tre casi, e dimostrare come da ognuno di essi si possano dedurre gli altri due:
\begin{itemize}
    \item Particelle incidenti su un unico bersaglio [dati: densità di corrente di particelle incidenti; frequenza di eventi osservati]
    \item Sottile fascio di particelle incidenti su una lastra contenente i bersagli [dati: flusso di particelle incidenti, densità superficiale dei bersagli, frequenza di eventi osservati]
    \item Urti nel volume fra particelle di due specie diverse e differenti concentrazioni [dati: numero di eventi osservati per unità di tempo e per unità di volume, concentrazione delle particelle interagenti, velocità relativa (si ipotizza che la tutte le particelle di una specie abbiano la stessa velocità)]
\end{itemize}}
\textit{Per particelle incidenti su un unico bersaglio}\\
\\
La sezione d'urto per particelle incidenti su un unico bersaglio è
\begin{equation}
    \sigma=\frac{f_e}{|\vec{j}_p|}
\end{equation}
dove $f_e$ è la frequenza di eventi osservati e $|\vec{j}_p|$ è la densità di corrente di particelle.\\
\\
\textit{Sottile fascio di particelle incidenti su una lastra contenente i bersagli}\\
\\
Per particelle incidenti su una lastra contenente i bersagli la sezione d'urto è 
\begin{equation}
    \sigma=\frac{f_e}{\Phi_p n_s}
\end{equation}
dove $f_e$ è la frequenza di eventi osservati, $\Phi_p$ è il flusso di particelle, $n_s$ è la densità superficiale dei bersagli\\
\\
\textit{Urti nel volume fra particelle di due specie diverse e differenti concentrazioni}\\
\\
Per urti nel volume tra specie diverse di particelle la sezione d'urto è:
\begin{equation}
    \sigma=\frac{f_e}{n_a n_b|\vec{v}_{rel}|}
\end{equation}
dove $f_e$ è il numero di eventi osservati per unità di tempo e volume, $n_a$ e $n_b$ sono le densità delle due specie e $|\vec{v}_{rel}|$ è il modulo della velocità relativa.


\paragraph{3.a.17}\textbf{Per gli urti fra due particelle definire le sezioni d'urto: elastica, inelastica, inclusiva, esclusiva, totale.}\\
Per un processo 
\begin{equation}
    a+b\longrightarrow p_1+p_2+\dots+p_n
\end{equation}
Sappiamo che:
\begin{equation}
    d\sigma=f_{urto}(P_1,...,P_n)dL_p
\end{equation}
sia $f_i(E_i)$ la distribuzione di probabilità dell'energia del prodotto $i-$esimo. 
\\
Dal momento che il processo è inclusivo, l'unica cosa che è possibile stimare è il range energetico dei prodotti della reazione, per cui la sezione d'urto sarà data da un integrale in quel range energetico della distribuzione di probabilità dell'energia dei prodotti.
\\
Allora la sezione d'urto inclusiva del processo è:
\begin{equation}
    \sigma_i=\int_{E_{i,min}}^{E_{i,max}}f(E_i)dE_i
\end{equation}
Se invece $f(P_1,\dots,P_n)$ è la distribuzione di tutti i 4-impulsi di tutte le particelle finali allora la sezione d'urto esclusiva è:
\begin{equation}
    \sigma_e=\int f(P_1,\dots, P_n)\prod_{i=0}^nd^4P_i
\end{equation}
Dal momento che, conoscendo tutti i prodotti finali, sono noti tutti gli impulsi ed è possibile fare l'integrale nello spazio delle fasi della funzione di distribuzione degli impulsi su tutti gli impulsi.
\\
La sezione d'urto elastica è la sezione d'urto relativa a un urto elastico, e analogamente la sezione d'urto inelastica è relativa a un urto inelastico. La sezione d'urto totale è la somma delle sezioni d'urto dei possibili stati finali.\\
-\\
La sezione d'urto $\sigma_f$ del processo $a+b\rightarrow f$ si definisce come l'area sul piano $\pi$ che deve attraversare il proiettile per produrre lo stato finale $f$.\\
\begin{equation}
    \sigma_f=\int b db d\phi
\end{equation}

\paragraph{3.a.18}\textbf{Calcolare la probabilità di interazione per una particella che incide su una lamina sottile [dati: sezione d'urto del processo, numero di bersagli per unità superficie]. Che significato avrebbe una probabilità maggiore di uno? Quest'ultima risposta dipende dalle tipologie degli urti?}\\ 
\\
Data la sezione d'urto del processo $\sigma_{tot}$ e la densità superficiale di bersagli $n_s$ la probabilità che la particella interagisca è data dalla formula
\begin{equation}\label{3.45}
    P_{int}=n_s\sigma_{tot}=n_v\Delta x\sigma_{tot}=\frac{\rho N_{Av}}{M_{molare}}\Delta x\sigma_{tot}=\frac{\Delta x}{l}
\end{equation}
Dove $n_c$ è la densità dei bersagli nella lamina, $\Delta x$ lo spessore, $\rho$ la densità, $N_{Av}$ il numero di Avogadro, $M_{molare}$ la massa molare della lamina e $l$ la lunghezza di attenuazione. \\
Se la probabilità è maggiore di uno significa che non siamo nel caso di lamina sottile e l'equazione \ref{3.45} non vale più, occore dividere la lastra in tante "lastrine" di spessore $dx$, quindi si ha:
\begin{equation}
    dP=\frac{dx}{l}
\end{equation}
Questo non dipende dalla tipologia di urto.


\paragraph{3.a.19}\textbf{Indicare le condizioni per cui la forza di reazione radiativa per una particella (di massa $m$ e carica unitaria) $m\tau\dot{\vec{a}}$ è da considerarsi valida ed utilizzabile.}\\
\\
Come noto, la forza di reazione radiativa ha dei problemi intrinsechi (vedi ad esempio le soluzioni di fuga) dovuti al fatto che trasforma le equazioni del moto da equazioni del secondo ordine a equazioni del terzo ordine. Si può utilizzare un approccio perturbativo per limitare questi problemi:
possiamo risolvere le equazioni del moto ponendo $\vec{F}_{rad} = 0$ e poi trovare dei termini correttivi dovuti alla forza di reazione. Tutto ciò ha senso se $|\vec{F}_{rad}|<< |\vec{F}_{ext}|$. Se supponiamo che le forze esterne siano di natura elettromagnetica, si ha
\begin{equation}
    m\dot{\vec{v}}=q\vec{E}+q\frac{\vec{v}}{c}\wedge\vec{B}
\end{equation}
e supponiamo che li moto non sia relativistico:
\begin{align}
    \vec{F}_{rad}&\simeq q\tau\biggl(\dot{\vec{E}}+\frac{\dot{\vec{v}}}{c}\wedge\vec{B}+\frac{\vec{v}}{c}\wedge\dot{\vec{B}}\biggr)\simeq\\
    &\simeq q\tau\dot{\vec{E}}+\frac{q^2\tau}{mc}\vec{E}\wedge\vec{B}\,.
\end{align}
Si richiede quindi che 
\begin{align}
    &\tau|\dot{\vec{E}}|<<|\vec{E}|\\
    &\frac{q\tau}{mc}|\vec{E}\wedge\vec{B}|<<|\vec{E}|
\end{align}
Supponendo i campi esterni monocromatici e con lunghezza d'onda $\lambda$ allora
\begin{align}
    \frac{q^2}{mc^2}=r_e<<\lambda\\
    B<<\frac{m^2c^4}{q^3}
\end{align}
In realtà, compaiono degli effetti quantistici rilevanti ben prima di raggiungere tali limiti (tipicamente, si deve
richiedere $\lambda>>137r_e$, e si ha una condizione analoga per $B$).
\paragraph{3.a.20}\textbf{Spiegare il significato e indicare le unità di misura di ogni grandezza fisica delle seguenti leggi:}
\begin{itemize}
    \item {\large$\frac{d\sigma_{el}}{d\Omega}=r_e^2\frac{\omega^4}{({\omega_0}^2-\omega^2)^2+\omega^2\Gamma^2_{tot}}\sin^2(\alpha)$}\\
    \item {\large$\frac{d\sigma_{el}}{d\Omega}=r_e^2\frac{\omega^4}{({\omega_0}^2-\omega^2)^2+\omega^2\Gamma^2_{tot}}\frac{1+\cos^2\vartheta}{2}$}\\
    \item {\large$\sigma_{el}=\sigma_{Th}\frac{\omega^4}{({\omega_0}^2-\omega^2)^2+\omega^2\Gamma^2_{tot}}$}\\
    \item {\large$\sigma_{tot}=4\pi r_ec\frac{\omega^2\Gamma_{tot}}{({\omega_0}^2-\omega^2)^2+\omega^2\Gamma^2_{tot}}$}
\end{itemize}
\textbf{dove $\sigma_{th}=8/3\pi r_e=0.66\ barn$.}\\
\\
Le equazioni citate rappresentano in ordine le seguenti grandezze:
\begin{itemize}
    \item \textit{Sezione d'urto differenziale polarizzata linearmente}: derivante dallo scattering Thompson
    \item \textit{Sezione d'urto differenziale non polarizzata}: derivante dallo scattering Thompson
    \item \textit{Formula di Breit-Wigner risonante}: sezione d'urto totale elastica derivante dall'integrazione sull'angolo solido della sezione d'urto differenziale polarizzata linearmente
    \item \textit{Sezione d'urto totale}: La sezione d'urto totale è la somma tra la sezione d'urto di assorbimento e la sezione d'urto elastica
\end{itemize}
Nelle precedenti formule si trovano le seguenti grandezze dovute a un'onda elettromagnetica che incide su un atomo legato elasticamente:
\begin{itemize}
    \item $r_e$ = raggio elettronico $[r_e]=cm$; \\
    \item $\omega_0$ = frequenza caratteristica della particella $[\omega_0]=rad/s$ ($\omega_0\sim 10^{14}-10^{15 s^{-1}}$);
    \item $\omega$ = frequenza angolare dell'onda EM incidente $[\omega]=rad/s$
    \item $\Gamma_{tot}=\Gamma'+\tau\omega^2=\Gamma'+\frac{\omega^2}{\omega_0^2}\Gamma$ : $\Gamma'\sim 10^{10}s^{-1}$ è legato alla forza viscosa (fenomeni dissipativi) e quindi è correlato all'assorbimento, $\Gamma$ è legato  all'irraggiamento quindi è correlato alla diffusione elastica ($\tau\sim 10^24 s^{-1})$;  $[\Gamma_{tot}]=s^-1$ 
    \item $\alpha$ = angolo tra la direzione di polarizzazione e la direzione d'osservazione
    \item $\vartheta$ = angolo tra la direzione dell'onda incidente e la direzione di osservazione
    \item $\sigma_{Th}= 8/3\pi r_e=0.66$ sezione d'urto Thomson $[\sigma_{Th}]=barn$
    \item $c$ = velocità della luce $[c]=cm/s$ ($\sim 10^{10}$)
\end{itemize}


\paragraph{3.a.21}\textbf{Discutere qualitativamente le osservazioni sperimentali dello scattering di Rutherford.}\\
\\
L'esperimento di Rutherford era nato per provare il modello atomico di Thomson, anche detto "modello a panettone", secondo cui l'atomo è costituito da una distribuzione di carica positiva all'interno della quale sono inserite le cariche negative, rendendo l'atomo nel complesso neutro. Rutherford fece lanciare un fascio di particelle alfa, ovvero nuclei di ${}^{4}_2$He$_2$ generati dal decadimento del polonio, contro una sottile lamina d'oro nella quale erano presenti poche migliaia di atomi. Attorno alla lamina d'oro era posizionato uno schermo fluorescente per segnare l'arrivo di ogni particella alfa diffusa, così da poter ricostruire la traccia percorsa. Secondo il modello di Thomson, le particelle alfa avrebbero dovuto attraversare la lamina d'oro, venendo deflesse al più di pochi gradi. Tuttavia Rutherford osservò che alcune particelle (circa 1 ogni 20'000) venivano deflesse addirittura all'indietro, facendo così rigettare il modello di Thomson. Ciò era compatibile con l'idea di un atomo al cui centro vi è un nucleo carico positivamente e massiccio.

\begin{quote}
\textit{It was almost as incredible as if you fired a 15-inch shell at a piece of tissue paper and it came back and hit you. - E. Rutherford}
\end{quote}

\paragraph{3.a.22}\textbf{Spiegare la differenza tra lo scattering Rutherford e lo scattering Mott.}\\
\\
Le particelle $\alpha$ (nuclei di ${}_2^4$He$_2$) sono cariche positivamente, quindi risentono di un potenziale repulsivo, a differenza degli elettroni che, carichi negativamente, risentono di un potenziale attrattivo quando vengono lanciati contro un bersaglio carico positivamente. Inoltre mentre per le particelle $\alpha$, molto massive, possiamo trascurare i termini relativistici, per gli elettroni, la cui massa è tale che $m_e \simeq \frac{1}{2000}$ $m_p$, dobbiamo tenere conto dei termini relativistici. E' necessario quindi utilizzare due diverse relazioni per le sezioni d'urto, in particolare per le particelle $\alpha$, essendo trascurabili i termini relativistici possiamo utilizzare la sezione d'urto di Rutherford, mentre nel caso degli elettroni (relativistici) dobbiamo utilizzare la sezione d'urto di Mott, che tiene conto degli effetti relativistici. 

\paragraph{3.a.23}\textbf{Spiegare il significato di tutti i termini delle seguenti espressioni delle sezioni d’urto differenziali Rutherford e Mott:}
\begin{itemize}
    \item $\frac{d\sigma}{d\Omega}\biggr|_{Rutherford}=\biggl(\frac{zZe^2}{4\pi\epsilon_0}\biggr)^2\biggl(\frac{1}{4T}\biggr)^2\frac{1}{sin^4(\frac{\vartheta}{2})}$\\
    \item $\frac{d\sigma}{d\Omega}\biggr|_{Mott}=\frac{d\sigma}{d\Omega}\biggr|_{Ruth}(1-\beta^2sin^2(\frac{\vartheta}{2}))$  in cui $T\longrightarrow\frac{pV}{2}$
\end{itemize}
Nelle 2 equazioni si fa riferimento a:
\begin{itemize}
    \item z carica della particella incidente, mentre Z carica del bersaglio;
    \item T energia cinetica della particella incidente;
    \item $\beta=\frac{v}{c}$;
    \item $\theta$ angolo di scattering (ovvero angolo fra velocità inziale e finale).
\end{itemize}
N.B.: la Mott nel caso sopracitato è valida se il bersaglio ha massa M$\to\infty$, diversamente si avrebbe:
\begin{equation}
    \frac{d\sigma}{d\Omega}\biggr|_{Mott}=\frac{d\sigma}{d\Omega}\biggr|_{Ruth}\frac{(1-\beta^2sin^2(\frac{\vartheta}{2}))}{1+\frac{2E}{M}sin^2\theta}
\end{equation}
N.B.2: Batigol scrive "in cui $T\longrightarrow\frac{pV}{2}$" intendendo dire che, sostituendo $T\longrightarrow\frac{pV}{2}$ come si è soliti fare nel caso relativistico, la formula risulta generalizzabile al caso relativistico ponendo $p=mv\gamma$.\\
\paragraph{3.a.24}\textbf{Dare la definizione operativa di raggio nucleare mediante lo scattering di Rutherford}\\
\\
\textbf{Modo 1:}\\
Nel caso in cui il parametro di impatto $b$ è nullo, abbiamo che la particella si ferma istantaneamente ad una distanza $d$ dal centro del nucleo. Per la conservazione dell'energia abbiamo:
\begin{equation} \label{cazzata.}
\frac{1}{2}m_{\alpha} |\mathbf{v}_0|^2=\frac{z Z e^2}{4\pi \epsilon_0 d} \Rightarrow d=\frac{z Z e^2}{2\pi \epsilon_0 m_{\alpha} |\mathbf{v}_0|^2} \, .
\end{equation}
Dunque, la $d$ minima trovata sperimentalmente rappresenta il raggio nucleare.\\
VALUTAZIONI NUMERICHE\\
\begin{equation*}
    d=\frac{z Z e^2}{4\pi \epsilon_0 T}\frac{\hbar c}{\hbar c}= \frac{zZ\alpha\hbar c}{T}=zZ\frac{1.44 \;\; \text{MeV fm}}{T}
\end{equation*}
con T energia cinetica limite per cui valga la \ref{cazzata.}, $\alpha$ costante di struttura fine(1/137), $\hbar c=197$ MeV fm.
\textbf{Modo 2:}\\
Ad un dato angolo di scattering $\theta$, si osserva un valore limite di T per cui il modello dello scattering rutherford smette di spiegare le osservazioni sperimentali: aumentando il valore di T, i nuclei vengono a contatto e l'interazione forte prevale.\\
Sostituendo il valore limite T si trova il corrispondente d. Dunque si hanno tutti gli ingredienti per trovare la distanza di minimo avvicinamento x:
\begin{equation}
    x=\frac{d}{2}\left(1+\frac{1}{sin\frac{\theta}{2}}\right)
\end{equation}
Tale valore coincide con il raggio nucleare.
\paragraph{3.a.25}\textbf{Definire le quantità che in un nucleo usualmente si indicano con A, Z, N
(simbologia $^A_ZX_N$ ). Dare la definizione di nuclei isotopi, isobari, isotoni, stabili,
instabili.}\\ \\
Con Z si indica il numero di protoni dell'atomo, con N quello dei neutroni, mentre A fa riferimento alla loro somma.\\
Si definiscono:
\begin{itemize}
    \item ISOBARI: stesso A ma diversi Z e N;
    \item ISOTOPI: stesso Z ma diverso N;
    \item ISOTONI: stesso N ma diverso Z;
    \item STABILI: hanno vita media molto alta;
    \item INSTABILI: hanno vita media bassa rispetto ai nuclei stabili.
\end{itemize}

\paragraph{3.a.26}\textbf{Dopo avere definito l’unità di massa atomica e avere dato il suo valore in MeV/c2, definire l’energia di legame (B) di un atomo ed il “difetto di massa” ($\Delta$) di un atomo}\\
L'unità di massa atomica è definita come la dodicesima parte della massa di un atomo di carbonio-12:
\begin{equation}
m_u=\text{1 u.m.a}\equiv \frac{1}{12}m_{at}({}_{6}^{12}\text{C}_6) \, ,
\end{equation}
e vale 931.49 Mev/c$^2$.\\
L'energia di legame di un atomo è l'energia necessaria per spezzare il nucleo nei singoli nucleoni. L'energia di legame è data da:
\begin{equation}
B(A,Z)=[\underbrace{Zm_p}_\text{p} + \underbrace{(A-Z)m_n}_\text{n} -\underbrace{M(A,Z)}_\text{tot}]c^2 \, ,
\end{equation}
dove $Z$ è il numero di protoni, $A$ il numero di massa, $m_p$ la massa del protone, $m_n$ la massa del neutrone, ed $M(A,Z)$ la massa del nucleo. Osserviamo che l'energia di legame è positiva in quanto la massa del nucleo è leggermente inferiore della somma delle masse dei nucleoni che lo compongono:
\begin{equation}
M(A,Z)<Zm_p+(A-Z)m_n \, .
\end{equation}
Trascurando l'energia di legame degli elettroni, $M(A,Z)=m_{at}({}^AX)-Zm_e$, ovvero la massa nucleare è data dalla differenza tra la massa dell'intero atomo e la massa degli elettroni. L'energia di legame può essere così riscritta come:
\begin{align}
B(A,Z)&=[Zm_p + (A-Z)m_n -m_{at}({}^AX)+Zm_e]c^2 \nonumber \\
&=[Zm_{at}({}^1H)+(A-Z)m_n -m_{at}({}^AX)]c^2 \, .
\end{align}
La grandezza $B_{A,Z}/A$ è detta "energia media di legame per nucleone" è tanto maggiore quanto più il nucleo è stabile, ha una massimo in corrispondenza di A=26 (ferro). Gli altri nuclei possono quindi avere un processo di fusione o di fissione.\\
Il difetto di massa è definito come la differenza fra la massa dell'atomo e la massa che esso avrebbe se l'energia di legame media per nucleone fosse quella del ${}^{12}C$.
\begin{equation}
\Delta=[m_{at}({}^AX)-Am_u]c^2 \, 
\end{equation}
Il difetto di massa è quindi positivo se il nucleo ha energia media di legame minore di quella del ${}^{12}C$, negativo se viceversa.

\paragraph{3.a.27}\textbf{Enunciare la formula semiempirica B = B(A,Z) ed indicare i suoi termini che
sono spiegati dal modello a goccia. Spiegare le ipotesi su cui tale modello è
basato e fornire l'ordine di grandezza dell’energia media di legame di un
nucleone all’interno di un nucleo}\\ \\
La formula semiempirica è:
\begin{equation}
B(A,Z)= a_v A-a_s A^{\frac{2}{3}}-a_c \frac{Z^2}{A^{\frac{1}{3}}} -a_{sym} \frac{(A-2Z)^2}{A}+\delta \, .
\end{equation}\label{3.a.27 semiemp}
\begin{itemize}
    \item $a_v A$: è detto termine di volume, in quanto il volume del nucleo è proporzionale ad $A$. E' dovuto all'interazione nucleare forte, e dato che questa agisce tra tutti i nucleoni (protoni e neutroni), il termine è indipendente da $Z$. Poiché il numero di coppie che si possono formare a partire da $A$ particelle è $\frac{A(A-1)}{2}$, ci si aspetterebbe un termine proporzionale ad $A^2$. Tuttavia non è così. La proporzionalità lineare tra $B$ ed $A$ per nuclei pesanti è dovuta al fatto che l'interazione nucleare forte ha un piccolo raggio d'azione, ovvero ogni nucleone interagisce significativamente soltanto con i nucleoni vicini. Di conseguenza, il numero di coppie di particelle che interagiscono veramente tra loro è pari ad $A$. Il valore misurato di $a_v$ è 15.7 MeV.
    \item $a_s A^{\frac{2}{3}}$: è detto termine di superficie. Anch'esso è dovuto all'interazione nucleare forte e rappresenta una correzione al termine di volume. Tale correzione è dovuta al fatto che i nucleoni che si trovano sulla superficie del nucleo hanno un numero minore di nucleoni con cui interagire rispetto a quelli che si trovano in profondità. Ciò provoca un effetto simile alla tensione superficiale nei liquidi. Se il volume del nucleo è proporzionale ad $A$, allora il suo raggio è proporzionale ad $A^{\frac{1}{3}}$ e la sua superficie ad $A^{\frac{2}{3}}$. Questo spiega perché il termine di superficie è proporzionale ad  $A^{\frac{2}{3}}$.\\
    Inoltre il termine di energia dovuto all'energia potenziale è dato da $E_S\sim4\pi R^2\sigma$, sostituendo ad R=$r_0A^{\frac{1}{3}}$ si trova al giusta dipendenza.\\
    Il valore misurato di $a_s$ è 17.8 MeV.
    \item $a_c \frac{Z^2}{A^{\frac{1}{3}}}$: è detto termine coulombiano, e deriva dall'interazione elettromagnetica repulsiva che si instaura tra i protoni, dato che possiedono la stessa carica. Il segno negativo davanti al termine implica una diminuzione dell'energia di legame. Per questo motivo per nuclei pesanti sono favoriti gli eccessi di neutroni.
    Approssimando il nucleo come una sfera omogenea:
    \begin{equation}
        E=\frac{3}{5}\frac{Z^2e^2}{R}\frac{1}{4\pi\epsilon_0}\sim\frac{3}{5}\frac{Z^2e^2}{r_0A^{\frac{1}{3}}}\frac{1}{4\pi\epsilon_0}=\frac{3\alpha \hbar c}{5r_0}\frac{Z^2}{A^{\frac{1}{3}}}
    \end{equation}
    Da cui $a_C=\frac{3\alpha \hbar c}{5r_0}=$0.69 MeV, mentre dal fit il valore misurato di $a_c$ è 0.71 MeV.
    \item  $a_{sym}\frac{(A-2Z)^2}{A}$: è detto termine di simmetria. Questo termine favorisce i nuclei in cui $Z \simeq A/2$. Questo è dovuto al fatto che i nucleoni dal punto di vista quantistico sono fermioni, facendo riferimento all'isospin in una stessa regione dello spazio devono esserci particelle con isospin opposto. Per i nuclei in cui il numero di protoni è diverso dal numero di neutroni la stabilità è compromessa e l'energia di legame è minore. Il valore misurato di $a_{sym}$ è 23 MeV.
    \item $\delta$: è il termine di pairing o di accoppiamento. Serve per sfavorire i nuclei con un numero di protoni e di neutroni dispari. Esso può assumere tre valori differenti a seconda dei casi: 
\begin{equation}
\delta=
\begin{cases}
+a_p A^{-\frac{3}{4}} \quad \text{se $Z$ ed $N$ sono pari} \\
0 \quad \text{se $A$ è dispari} \\
-a_p A^{-\frac{3}{4}} \quad \text{se $Z$ ed $N$ sono dispari} \, . \\
\end{cases}
\end{equation}
Il valore misurato di $a_p$ è 34 MeV.
\end{itemize}
Gli ultimi 2 termini hanno origine di carattere quantistico.\\
Il modello a goccia modellizza il nucleo come una goccia incomprimibile di fluido resa sferica dalla tensione superficiale, l’ordine di
grandezza dell’energia media di legame di un nucleone varia da -2,2245 MeV per l'idrogeno molecolare (2 nucleoni) a -8,7945 MeV per il nichel-62 (62 nucleoni)
\paragraph{3.a.28}\textbf{Definire i decadimenti $\alpha, \beta$, $\gamma$ e il decadimento tramite cattura elettronica in un nucleo. Calcolare il Q-valore per il decadimento $\beta_+$, $\beta_-$, e per la cattura elettronica a partire dal difetto di massa delle specie coinvolte.}\\
\\
\underline{Decadimento $\alpha$}. Un nucleo atomico emette una particella $\alpha$, ovvero un nucleo di elio ${}_2^4$He$_2$ formato da due protoni e due neutroni, e decade in un atomo con un numero di massa ridotto di quattro rispetto al primo e con due protoni (e quindi due neutroni) in meno:
\begin{equation}
{}_{Z}^{A}X_N \longrightarrow {}_{Z-2}^{A-4}Y_{N-2}^{--}+ \alpha \, 
\end{equation}
$\alpha$ è un nucleo particolarmente stabile (shell chiusa).\\
\\
\underline{Decadimento $\beta$}. Un nucleo atomico può correggere l'eccesso di protoni o di neutroni convertendo un neutrone in un protone ed emettendo un elettrone, oppure convertendo un protone in un neutrone ed  emettendo un positrone. Abbiamo così due tipi di decadimento $\beta$. \\
\textit{\underline{Decadimento $\beta ^-$}}:
\begin{equation}
{}_{Z}^{A}X_N \longrightarrow {}_{Z+1}^{A}Y_{N-1}^+ + e^- + \bar{\nu_e}\, 
\end{equation}

Da un punto di vista microscopico:
\[
n \rightarrow p+e^-+\bar{\nu_e}
\]
Q-valore $\beta^-$:
\begin{equation}
    Q=\Delta_{A,Z}-\Delta_{A,Z+1}
\end{equation}
poiché la somma della massa dell'atomo ionizzato "$Y^+$" con la massa dell'elettrone è molto ben approssimabile (se si trascura l'energia di legame dell'elettrone) con la massa dell'atomo neutro"$Y$".\\
\textit{\underline{Decadimento $\beta ^+$}}:
\begin{equation}
{}_{Z}^{A}X_N \longrightarrow {}_{Z-1}^{A}Y_{N+1}^-+ e^+ + \nu_e \, 
\end{equation}
Da un punto di vista microscopico:
\[
p \rightarrow n+e^++\nu_e
\]
\warning Il decadimento $\beta_+$ NON esiste libero, può avvenire solo dentro un nucleo perchè la massa del neutrone è leggermente più grossa di quella del protone (Q<0: non può avvenire in modo spontaneo).\\
Q-valore $\beta^+$:
\begin{equation}
    Q=\Delta_{A,Z}-\Delta_{A, Z-1}-2m_e
\end{equation}
poiché la massa dello ione "$Y^-$" è molto ben approssimabile con la somma della massa dell'atomo neutro "$Y$" con la massa dell'elettrone.\\
\\
\underline{Decadimento $\gamma$}. Un nucleo atomico in uno stato eccitato passa ad un livello energetico più stabile emettendo un fotone $\gamma$ con energia pari alla differenza di energia tra i due livelli:
\begin{equation}
{}_{Z}^{A}X_N^*\longrightarrow {}_{Z}^{A}X_N + \gamma \, 
\end{equation} 
\\
\underline{Decadimento tramite cattura elettronica in un nucleo}. Un nucleo assorbe uno dei suoi elettroni orbitanti ed un protone del nucleo si trasforma in un neutrone. Abbiamo quindi un nuovo nucleo con un protone in meno ed un neutrone in più rispetto al primo:
\begin{equation}
{}_{Z}^{A}X_N  \longrightarrow {}_{Z-1}^{A}Y_{N+1} + \nu_e \, .
\end{equation}
Da un punto di vista microscopico:
\[
p+e^- \rightarrow n+\nu_e
\]
Q-valore EC:
\begin{equation}
    Q=\Delta_{A,Z}-\Delta _{A,Z-1}
\end{equation}
\\
Qualche esempio:
\begin{itemize}
\item \textbf{$\alpha$:} $^{240}_{94} Pu \longrightarrow ^{236}_{92} U + \alpha$ 
\item \textbf{$\beta^-$:} $^8_3 Li \longrightarrow ^8_4 Be + e^- + \bar{\nu_e}$
\item \textbf{$\beta^+$:} $^{23}_{12} Mg \longrightarrow ^{23}_{11}Na + e^+ + \nu_e$
\item \textbf{$\gamma$:} $^{60}_{28} Ni^* \longrightarrow ^{60}_{28} Ni + \gamma$
\item \textbf{$e^-$:} $^{26}_{13}Al + e^- \longrightarrow ^{26}_{12} Mg + \nu_e$
\end{itemize}

\paragraph{3.a.29}\textbf{Dimostrare che in un tipico decadimento $\alpha$, la particella $\alpha$ emerge con circa il
98$\%$ dell'energia disponibile.}\\\\
Si considera il decadimento $\alpha$:
\begin{equation}
    {}_{Z}^{A}X_N \longrightarrow {}_{Z-2}^{A-4}Y_{N-2}^{--}+ \alpha \, 
\end{equation}
Dato che si stanno trattando nuclei pesanti possiamo considerare l'approssimazione non relativistica:
\begin{equation}
    m_xc^2=m_yc^2+T_y+m_\alpha c^2+T_\alpha\Longrightarrow Q=T_y+T_\alpha \qquad T_i=\frac{p_i^2}{2m_i} \qquad p_y=p_\alpha
\end{equation}
Da cui:
\begin{equation}
    Q=\frac{p_\alpha^2}{2}\left(\frac{m_\alpha+m_y}{m_\alpha m_y}\right)\Longrightarrow T_\alpha=\frac{Q}{1+\frac{m_\alpha}{m_y}}
\end{equation}
In genere sono i nuclei pesanti (A$>$210, ad eccezione di $^8Be$ e $^9B$) a fare decadimento $\alpha$, per cui solitamente $\frac{m_\alpha}{m_y}\sim 0.02$.
Volendo essere più precisi si ha $m_\alpha\sim 4\ uma$, $m_y\sim A-4\ uma$, da cui:
\begin{equation}
     T_\alpha=\frac{Q}{1+\frac{4}{A-4}}=Q\frac{A-4}{A}
\end{equation}

\paragraph{3.a.30}\textbf{Come si è arrivati alla conclusione che nel decadimento beta deve essere emessa una particella neutra non rivelata?}\\
\\
Il positrone ha uno spettro di energia che non può essere spiegato da un decadimento a due corpi, in cui tutto è fissato.\\
Consideriamo il decadimento di un neutrone in un protone ed un elettrone (errato!):
\begin{equation*}
n\longrightarrow p+e \, 
\end{equation*} 
In un sistema di riferimento in cui il neutrone è fermo, ipotizzando che si tratti di un decadimento in due corpi, ci aspettiamo:
\begin{equation*}
\mathbf{p}_p+\mathbf{p}_e=\mathbf{0} \Rightarrow \mathbf{p}_p=-\mathbf{p}_e
\end{equation*}
ovvero l'impulso del protone è uguale ed opposto a quello dell'elettrone. Tuttavia vengono rilevati impulsi diversi in direzioni diverse. Ciò che succede in realtà è che viene emessa anche una terza particella, non carica e quindi non rilevabile, chiamata neutrino:
\begin{equation*}
n\longrightarrow p+e+ \nu_e 
\end{equation*}
con impulso $\mathbf{p}_\nu$ tale che:
\begin{equation*}
\mathbf{p}_p+\mathbf{p}_e+\mathbf{p}_\nu=\mathbf{0} \, 
\end{equation*}
E tramite il metodo della massa invariante è possibile risalire alla massa del neutrino (più propriamente si può imporre un limite superiore).

\paragraph{3.a.31}\textbf{Dimostrare che in un decadimento $\beta$ la somma delle energie dell'elettrone e dell'antineutrino emessi é praticamente uguale al Q-valore della reazione}\\
Schematizzo un decadimento $\beta^-: _Z^AX_N\to^A_{Z+1}Y_{N-1}^++e^-+\bar{\nu_e}$.\\
Le energie cinetiche delle singole particelle devono essere minori di Q.\\
Le quantità di moto di elettrone e neutrine (massa trascurabile), sono limitate:
\begin{equation}
    |\vec{p}_e|=\sqrt{E_e^2-m_e^2}=\sqrt{(T_e+m_e)^2-m_e^2}<\sqrt{Q^2+2m_eQ} \qquad |\vec{p}_{\bar{\nu}}|<Q
\end{equation}
Inoltre la quantità di moto del nucleo deve verificare:
\begin{equation}
    |\vec{p}_Y|<|\vec{p}_e|+|\vec{p}_{\bar{\nu}}|<Q+\sqrt{Q^2+2m_eQ}
\end{equation}
Si ha Q$\sim$ MeV, per cui circa 3 ordini di grandezza inferiore alla massa del nucleo di Y. Per cui l'energia traaferita ad Y ha un limite superiore pari a:
\begin{equation}
    T_Y=\frac{|\vec{p}_Y|^2}{2M_Y}<\frac{(Q+\sqrt{Q^2+2m_eQ})^2}{M_Y}\sim< \frac{Q}{1000}
\end{equation}
Per cui in un decadimento $\beta^-$, e analogamente in un $\beta^+$ si ha:
\begin{equation}
    T_e+T_{\bar{\nu}}=Q
\end{equation}

\paragraph{3.a.32}\textbf{Spiegare perché, sebbene il neutrone libero sia instabile, esso non possa decadere quando è all'interno di taluni nuclei.}\\
\\
Il neutrone è instabile, ha infatti una vita media di $\sim 15$ min. Se si considera la reazione $n\rightarrow p+e^-+\bar{\nu}_e$ si trova un Q valore positivo, cioè questa reazione avviene spontaneamente. Tuttavia se si considera il processo dentro il nucleo esso può essere inibito dal fatto che lo spazio delle fasi è a tre corpi, e dal fatto che il neutrone non è libero ma è legato alle altre particelle dalla forza nucleare la quale può abbassare il Q-valore.
\\

\paragraph{3.a.33}\textbf{Quali particelle incidenti e di quale energia si utilizzano per misurare i fattori di
forma nucleari?}\\ \\
Per misurare i fattori di forma nucleari vengono utilizzati elettroni con energia di circa 500 MeV. 
Ad esempio nel caso dell'esperimento di Hofstadter (1950) si utilizzò un elettrone di energia 188 MeV per misurare il fattore di forma del protone.\\
-
\\
Per misurare il fattore di forma si può partire dalla posizione degli "zeri" dell'intensità quindi occorre $\lambda < a  sen\theta \iff \hbar k_0>\frac{2\pi \hbar}{a sen\theta} \iff |\vec{p_0}|>\frac{h}{a sen\theta}$, dove $a$ è la grandezza dell'oggetto da misurare. I nuclei hanno $a \sim 1$fm ,quindi per vederli serve $|\vec{p_0}|>197 MeV/c$ se $sen\theta<1$.Gli atomi hanno $a \sim 1 A^{\circ}$ , quindi per vederli serve $|\vec{p_0}|>12.4 KeV/c$ se $sen\theta<1$. Queste valutazioni sono legate alla natura ondulatoria del fotone, quindi sono valide anche per "vedere" atomi e nuclei con particelle, per esempio con elettroni. 



\paragraph{3.a.34}\textbf{Dare le definizioni di: larghezza, vita media, emivita (o tempo di dimezzamento), rapporto di decadimento (“Branching fraction” o “Branching ratio”) per il decadimento di una particella}\\
\\
\underline{Larghezza} $\Gamma$: probabilità di decadimento nell'unità di tempo.\\
$[\Gamma]$=1/s, si può esprimere come una energia (tipicamente in eV) moltiplicando per $\hbar$. Solitamente si esprime sempre in eV perché una particella che decade è interpretabile come una risonanza in cui il picco è pari alla massa e la larghezza è proprio $\Gamma$.\\
\\
\underline{Vita media} $\tau$: inverso di probabilità di decadimento nell'unità di tempo.\\
$[\tau]$= s\\
\begin{equation}
\Gamma=\frac{\hbar}{\tau} 
\end{equation}
\\
Se al tempo t=0 esiste una particella metastabile, indicando con P(t) la probabilità che tale particella non sia ancora decaduta al tempo t$>$0, possiamo scrivere (per definizione di larghezza) $P(t+dt)=P(t)(1-\Gamma dt)$. Ne deriva la legge del decadimento esponenziale:
\begin{equation}
    P(t)=e^{-\Gamma t}= e^{-t/\tau}
\end{equation}
Si evince che la \underline{vita media} è il tempo in cui la probabilità di "sopravvivenza" della particella si riduce di un fattore e.\\
Nel caso in cui al tempo t=0 si abbiano $N_0$ particelle metastabili identiche si può definire:
\begin{equation}
    N(t)=N_0 e^{-t/\tau} =N_0 2^{-t/T_{1/2}}
\end{equation}
\underline{Emivita (tempo di dimezzamento)} $T_{1/2}$:
\begin{equation}
    T_{1/2}=\tau ln2= 0.693 \tau
\end{equation}
$[T_{1/2}]$=s\\
\\
Nel caso in cui una particella possa decadere in più stati finali diversi, di definisce:
\begin{itemize}
    \item $\Gamma_f$ "larghezza parziale": probabilità per unità di tempo di decadere nello stato finale f,
    \item $\Gamma$ "larghezza totale": somma delle larghezze parziali di tutti i possibili stati finali $\Gamma=\sum_f \Gamma_f $
\end{itemize}
\warning NON si definiscono vite medie parziali, la vita media è unica\\
\underline{Rapporto di decadimento ("Branching fraction")} $B_f$: probabilità di decadere nello stato finale f.
\begin{equation}
    B_f=\frac{\Gamma_f}{\Gamma}
\end{equation}
\warning  $\sum_f B_f=1$
\paragraph{3.a.35}\textbf{Quali sono gli ordini di grandezza tipici delle sezioni d’urto delle interazioni forti
e delle interazioni deboli?}\\ \\
Le sezioni d’urto delle interazioni forti sono dell’ordine di 10-100
mb (10$^{-26}$-10$^{-25}$ cm$^2$) mentre le sezioni d’urto delle interazioni deboli sono dell'ordine di 1 fb (10$^{-39}$ cm$^2$).\\
Ricordiamo che: 1 b = 10$^{-24}$ cm$^2$
.
\paragraph{3.a.36}\textbf{Quali sono, approssimativamente, gli ordini di grandezza delle vite medie dovute ad interazioni deboli, elettromagnetiche, forti?}\\
\\
Interazioni deboli: $10^{-6}-10^{-13}$
s\\
Interazioni elettromagnetiche: $10^{-16}-10^{-20}$\\
Interazioni forte: $10^{-23}$\\
\\
In un modello "a palline" si definisce una regione di interazione per una particella in moto che incide su una particella ferma.
L'ampiezza della regione di interazione viene definita dall'interazione stessa ed è il suo range (1 fm per la forte ad esempio).
Il tempo di interazione è l'ampiezza della regione di interazione fratto la velocità della particella. Questo è il motivo per cui la durata delle interazioni forti dipende marcatamente dall'energia.
\\
Nel caso di interazioni EM / gravitazionali (range infinito) risulta che la forza ortogonale sentita dalla particella fa un grafico a campana centrato sulla particella ferma. Si considera allora la larghezza a metà altezza (quando la forza è la metà del massimo) come il tempo medio dell'interazione.
\\
Bisogna notare però che è raro dover considerare quest'ultimo caso elettromagnetico poichè di solito si considerano sempre atomi neutri / nuclei ed è sempre presente un effetto di schermaggio.

\paragraph{3.a.37}\textbf{Spiegare la cinematica di un decadimento $\gamma$ nucleare e spiegare qualitativamente
l'effetto Mossbauer.}\\\\
Un decadimento $\gamma$ nucleare prevede che uno stato eccitato  $^A_ZX_N^*$ decada emettendo un fotone $^A_ZX_N^*\to^A_ZX_N+\gamma$ di energia 10 KeV-10 MeV.\\
Lo stato di partenza può essere anche uno stato eccitato un numero maggiore di volte, che può decadere in diverse modalità:
\begin{equation}
    ^A_ZX^{**}\to\begin{cases}
    ^A_ZX_N^*+\gamma\\ ^A_ZX_N+\gamma
\end{cases}
\end{equation}
Un esempio celebre è quello del $^{57}Fe^{**}$ che può decadere in:
\begin{equation}
    ^{57}Fe^{**}\to\begin{cases}
    ^{57}Fe+\gamma(136.4 keV) & B=15\%\\ ^{57}Fe^*+\gamma(122.0 keV) &B=85\%
\end{cases}
\end{equation}
A sua volta lo stato eccitato $^{57}Fe^*$ potrà effettuare un ulteriore decadimento $\gamma$:
\begin{equation}
    ^{57}Fe^*\to ^{57}Fe+\gamma(14.4 KeV)
\end{equation}
Gli stati eccitati sono stati risonanti e sono in genere caratterizzati da larghezze molto piccole, ad esempio nel caso del $^{57}Fe^*$ si ha $\Gamma=4.6 neV$.\\
Allo stesso modo se diamo all'atomo non eccitato una quantità di energia corrispondente al salto di livello si deve ottenere lo stato eccitato.\\
In prima istanza si potrebbe ipotizzare che il fotone emesso nel decadimento possa eccitare un'altro atomo.\\
Si analizza ora la cinematica del decadimento.\\
Indicando con M la massa dello stato fondamentale, con $M^*=M+E_0$ la massa dello stato eccitato e sapendo che $E_\gamma=P_{CM}$ si ottiene nel sistema del nucleo a riposo (centro di massa):
\begin{equation}
    M^*=M+E_0=\sqrt{M^2+E^2_\gamma}+E_\gamma \Longrightarrow E_\gamma=\frac{E_0(E_0+2M)}{2(E_0+M)}\sim E_0-\frac{E_0^2}{2M}
\end{equation}
Per cui l'energia persa nel sistema del laboratorio a causa del rinculo del nucleo risulta essere $\Delta_\gamma\sim-\frac{E_0^2}{2M}$ da cui $\biggl|\frac{\Delta E_\gamma}{E_\gamma}\biggr|\sim1.3 \cdot 10^{-7}$ molto maggiore si $\frac{\Gamma}{E_0}=3.2\cdot10^{-13}$.\\
Per cui non si dovrebbe ottenere risonanza.
Tuttavia in alcuni materiali cristallini il nucleo che effettua il decadimento può non essere considerato isolato ma connesso agli altri nuclei del reticolo. Per cui la quantità di moto del fotone emesso viene assorbita da tutto il cristallo, per cui $\Delta E_\gamma\sim-\frac{E_0}{2M}\to 0$, poichè bisogna considerare la massa di tutti i nuclei. Per cui si ha $\Delta E_\gamma\ll \Gamma$ e quando il fotone incide su un altro nucleo di ferro lo eccita.

%

\paragraph{3.a.38}\textbf{Quante sono le variabili indipendenti nello stato finale di una reazione in cui due particelle collidono ed N particelle sono prodotte?}\\
\begin{equation}
a+b \longrightarrow p_1+p_2+...+p_N \, 
\end{equation}
Il numero di osservabili indipendenti nello stato finale (i.e il numero di grandezze indipendenti che posso misurare) è pari a $3n-4$, in quanto vi sono:
\begin{itemize}
    \item $n$ quadrimplusi $\Rightarrow$ $4n$ variabili
    \item $n$ vincoli dovuti alla massa di ogni singola particella: $m_i^2=P_{0i}^2-\vec{P}_i^2$ per la particella i-esima, 
    avendo definito $P_i=(P_{0i}, \vec{P}_i)$
    \item $4$ vincoli per la conservazione impulso-energia $P_{in}=\sum P_i$
\end{itemize}
$\Rightarrow$ $4n-n-4$=$3n-4$
\paragraph{3.a.39}\textbf{Quante sono le variabili indipendenti nello stato finale di una reazione in cui una particella decade in due particelle? Quali implicazioni avremmo se la particella che decade avesse un momento angolare nullo?}\\\\
In una reazione del tipo:
\begin{equation}
a \longrightarrow p_1+p_2 \, ,
\end{equation}
in cui una particella decade in due particelle, nello stato finale abbiamo 2 quadrimpulsi, e quindi 8 variabili. Inoltre ci sono 2 vincoli dovuti alla massa di ogni singola particella finale ($m_i^2=P_{0i}^2-|\mathbf{p}_i|^2, i=1,2$) e 4 vincoli dovuti alla conservazione energia-impulso ($P_a=P_1+P_2$). Perciò in tutto abbiamo $8-2-4=2$ variabili indipendenti. Il fatto che la particella che decade abbia un momento angolare non nullo è rilevante per la funzione di decadimento, in quanto c'è una direzione privilegiata. Se il momento angolare fosse nullo, la funzione di decadimento non dipenderebbe dagli angoli, e il decadimento sarebbe isotropo.
Difatti nel caso di un decadimento a 2 corpi possiamo scrivere la larghezza di decadimento legata all'elemento infitesimo dello spazio delle fasi come:
\begin{equation}
    d\Gamma=f_{dec}(\Omega)\frac{P_{CM}}{4M}d\Omega
\end{equation}
Se lo spin è nullo allora la $f_{dec}$ non dipende da $\Omega$, per cui è possibile integrare ottenendo:
\begin{equation}
    \Gamma=\frac{\pi f_{dec}}{2}\beta
\end{equation}
\paragraph{3.a.40}\textbf{Definire le variabili utilizzate nel "Dalitz plot"}\\
\\
Il Dalitz plot è un grafico utilizzato nel caso di decadimento a tre corpi, sull'asse delle ascisse si trova $s_{23}$ e sull'asse delle ordinate $s_{12}$, dive $s$ è il quadrato della massa invariante del sistema composto da due particelle. In questo tipo di decadimento, nello stato finale, abbiamo 3 quadrimpulsi, e quindi 12 variabili. Inoltre ci sono 3 vincoli dovuti alla massa di ogni singola particella finale ($m_i^2=P_{0i}^2-|\mathbf{p}_i|^2 , i=1,2,3$) e 4 vincoli dovuti alla conservazione energia-impulso ($P_a=P_1+P_2+P_3$). Perciò in tutto abbiamo $12-3-4=5$ variabili indipendenti. Nel sistema del centro di massa gli impulsi delle tre particelle giacciono tutti su un piano (detto piano di decadimento). Possiamo dare l'orientazione del piano attraverso tre angoli (gli angoli di Eulero) che possono essere utilizzati come 3 delle 5 variabili indipendenti. Le ultime due variabili che completano la definizione dello stato finale del decadimento a tre corpi e che sono utilizzate nel Dalitz plot, sono i quadrati delle masse invarianti di due coppie di particelle, per esempio $m_{12}^2$ e $m_{23}^2$, sottoposte ai seguenti limiti:
\begin{align}
(m_1+m_2)^2 \leq m_{12}^2 &\leq (M-m_3)^2  \\
(m_2+m_3)^2 \leq m_{23}^2 &\leq (M-m_1)^2
\end{align}
dove $M$ è la massa invariante dell'intero sistema. Considerando la Figura \ref{fig:dalitz} abbiamo definito $p_{ij}=p_i+p_j$, $m_{ij}^2=p_{ij}^2$, $m_{12}^2+m_{23}^2+m_{13}^2 = M^2+m_1^2+m_2^2+m_3^2$, $m_{12}^2=\left( P-p_3 \right)^2 = M^2 +m_3^2 -2ME_3$, con $E_3$ energia della particella 3 nel frame di riposo di $M$. Altri due vincoli che otteniamo sono $\left(m_{23}^2\right)_{max}= \left(E_2^* + E_3^* \right)^2-\left(\sqrt{E_{2}^{*2}-m_2^2} - \sqrt{E_3^{*2}-m_3^2}\right)^2$ e $\left(m_{23}^2\right)_{max}= \left(E_2^* + E_3^* \right)^2-\left(\sqrt{E_{2}^{*2}-m_2^2} + \sqrt{E_3^{*2}-m_3^2}\right)^2$;\\ con $E_2^* = \left(m_{12}^2-m_1^2+m_2^2\right)/2m_{12}$ e $E_3^* = \left(M^2-m_{12}^2-m_3^2\right)/2m_{12}$ energie delle particelle 2 e 3 nel sistema di riferimento di $m_{12}$. Lo scatter plot in $m_{12}^2$ e $m_{23}^2$ è il Dalitz plot. Se il modulo quadro della massa inziale $|\bar{M}|^2$ è costante, il plot è uniformemente popolato. Se invece appaiono delle bande allora abbiamo subito delle informazioni riguardo a $|M|^2$ (per esempio se ci sono delle risonanze).
\begin{figure}[H]
    \centering
    \includegraphics[scale=0.3]{dalitz.png}
    \label{fig:dalitz}
\end{figure}
\paragraph{3.a.41}\textbf{Quante sono le variabili indipendenti nello stato finale di una reazione in cui una particella decade in tre particelle? Quali implicazioni avremmo se la particella che decade avesse un momento angolare nullo?}\\\\
In una reazione del tipo:
\begin{equation}
a \longrightarrow p_1+p_2+p_3 \, ,
\end{equation}
in cui una particella decade in tre particelle, nello stato finale abbiamo 3 quadrimpulsi, e quindi 12 variabili. Inoltre ci sono 3 vincoli dovuti alla massa di ogni singola particella finale ($m_i^2=P_{0i}^2-|\mathbf{p}_i|^2 , i=1,2,3$) e 4 vincoli dovuti alla conservazione energia-impulso ($P_a=P_1+P_2+P_3$). Perciò in tutto abbiamo $12-3-4=5$ variabili indipendenti. Il fatto che la particella che decade abbia un momento angolare non nullo è rilevante per la funzione di decadimento, in quanto c'è una direzione privilegiata. Se il momento angolare fosse nullo, la funzione di decadimento non dipenderebbe dagli angoli, e il decadimento sarebbe isotropo.\\
In particolar modo nel caso di decadimento a 3 corpi si trova una larghezza di decadimento nel caso di decadimento isotropo
\begin{equation}
    \Gamma=\int \frac{f_{dec}(s_{12},s_{23},\alpha,\beta,\gamma)}{32s}ds_{12} ds_{23} d\alpha d\beta d\gamma \Longrightarrow \Gamma=\frac{\pi^2}{4s}\int f_{dec}(s_{12},s_{23})ds_{12}ds_{23}
\end{equation}
\paragraph{3.a.42}\textbf{Definire la funzione di distribuzione esclusiva dei 4-impulsi delle particelle emergenti dopo la collisione di due particelle (oppure dopo il decadimento di una particella).}\\
\\
La funzione di distribuzione esclusiva dei quadrimpulsi delle particelle emergenti dopo la collisione di due particelle (oppure dopo il decadimento di una particella) è una funzione delle energie e degli impulsi(quadrimpulsi) delle particelle emergenti, in cui riesco a misurare separatamente tutti gli impulsi delle singole particelle. Se non è possibile misurare un certo numero di grandezze nello stato finale, o non siamo interessati, si parla di distribuzione inclusiva.\\
La probabilità di misurare una larghezza di decadimento è:
\begin{equation}
    d\Gamma= f_{dec}( P_1 \cdots P_N)\frac{d^3\vec{P}_1}{2 E_1} \cdots \frac{d^3\vec{P}_n}{2 E_n} \delta^4(P_{in}-\sum P_i)
\end{equation}
La probabilità di misurare una sezione d'urto è:
\begin{equation}
    d\sigma= f_{urto}( P_1 \cdots P_N)\frac{d^3\vec{P}_1}{2 E_1} \cdots \frac{d^3\vec{P}_n}{2 E_n} \delta^4(P_{in}-\sum P_i)
\end{equation}
$f_{dec}$ e $f_{urto}$ è la probabilità di misurare una larghezza $d\Gamma$ in un intorno $P_1 \cdots P_N$ , contengono tutte le informazioni sulla dinamica del processo ed è calcolata tramite le teorie delle interazioni fra particelle.

\paragraph{3.a.43}\textbf{Spiegare il metodo della ‘massa invariante’ per identificare una particella instabile e misurarne la sua massa}\\ \\
Il metodo della massa invariante consiste nel riconoscere una particella risonante, che dopo poco tempo decade in due particelle, dal grafico del Daliz Plot quando esso non è uniformemente popolato.


\paragraph{3.a.44}\textbf{Dire quali fra le seguenti particelle sono soggette ad interazioni forti:$p$,$\overline{p}$,$\pi^+$,$\pi^-$, $\mu^+$, $\mu^-$,$e^+$, $e^-$,$\alpha$, nucleo di Azoto, $\nu$,$\bar{\nu}$}\\
\\
Le particelle soggette ad interazioni forti sono: $p$, $\overline{p}$, $\pi^+$, $\pi^-$, $\alpha$, nucleo di Azoto. 
\\
Le altre particelle elencate sono sottoposte:
\begin{itemize}
\item $\mu^+, \; \mu^-, \; e^+, \; e^-$ (gravità, debole, elettromagnetica)
\item $\nu, \; \bar{\nu}$ (debole, gravità)
\end{itemize}


\textit{Domande b}\rule{13.7 cm}{0.4pt}
\paragraph{3.b.1}\textbf{Calcolare la "resistenza di irraggiamento" di un circuito elettrico quadrato di lato $l$, piccolo rispetto alla lunghezza d'onda $\lambda$ della radiazione monocromatica incidente, se il circuito è puramente resistivo con resistenza $R$. Calcolare anche la sezione d'urto di assorbimento e la sezione d'urto elastica se l'onda incidente ha campo magnetico perpendicolare al piano del circuito e di modulo massimo $B_0$.}\\
\\
Considerando il circuito giacente nel piano $xz$ e un onda con campi e vettore di Poynting in modo che:
\begin{align}
    \vec{E}=E_0\cos(\omega t-kz)\hat{x}\\
    \vec{B}=\frac{E_0}{c}\cos(\omega t-kz)\hat{y}\\
    \vec{S}=\frac{E_0^2}{Z_0}\cos^2(\omega t-kz)\hat{z}
\end{align}
la corrente $I(t)$ che scorre nel circuito è uniforme dall'ipotesi di \begin{equation*}
    l<<\lambda
\end{equation*}
quindi potendo trascurare capacità e induttanze parassite possiamo dire che il circuito ha momento di dipolo elettrico nullo, mentre invece il momento di dipolo magnetico vale:
\begin{equation}
    \vec{p}_m=I(t)l^2\hat{y}
\end{equation}
Passiamo quindi a calcolare la forza elettromotrice agente sul circuito con una semplice applicazione della legge di Faraday:
\begin{equation}
    \epsilon(t)=-\frac{d\Phi(\vec{B})}{dt}=\frac{d}{dt}\Biggl[\int_{-l/2}^{l/2}B_0\cos(\omega t-kz)ldz\Biggr]=\omega l^2B_0\sin(\omega t)\frac{\sin(kl/2)}{kl/2}
\end{equation}
essendo $\frac{kl}{2}=\frac{\pi l}{\lambda}<<1$ l'ultima frazione tende a uno e 
\begin{equation}
    \epsilon(t)\simeq\epsilon_0\sin(\omega t)\ \ \ \text{con}\ \  \epsilon_0=B_0\omega l^2
\end{equation}
La corrente $I$ sarà una funzione armonica con la stessa frequenza della forza elettromotrice essendo il circuito lineare, quindi
\begin{equation}
    I=I_0\sin(\omega t)\qquad\rightarrow\qquad\ddot{I}=-\omega^2I
\end{equation}
Possiamo ricavarci un'espressione per I attraverso la potenza, ricordando che la potenza irraggiata da un dipolo magnetico è:
\begin{equation}
    P_{el}=\frac{|\ddot{\vec{p}}_m|^2}{6\pi\epsilon_0c^5}=\frac{\ddot{I}^2(t)l^4}{6\pi\epsilon_0c^5}
\end{equation}
da cui
\begin{equation}
    \epsilon I=RI^2+P_{el}=RI^2+\frac{l^4}{\omega^4 6\pi\epsilon_0c^5}I^2
\end{equation}
Definiamo \textbf{Resistenza di irraggiamento} $R_{irr}$l'ultimo coefficiente di $I^2$ dipendente dalla frequenza. \\
Per calcolarsi le sezioni d'urto ci serve calcolare la potenza dissipata sulla resistenza $R$ e quella dissipata per irraggiamento, per prima cosa troviamo un'espressione per la corrente:
\begin{equation}
    I(t)=\frac{\epsilon(t)}{R+R_{irr}}
\end{equation}
da cui la potenza persa sulla resistenza è
\begin{equation}
    P_{abs}=RI^2=\frac{R}{(R+R_{irr})^2}\epsilon^2
\end{equation}
e quella irraggiata è
\begin{equation}
    P_{el}=R_{irr}I^2=\frac{R_{irr}}{(R+R_{irr})^2}\epsilon^2
\end{equation}
Da cui le sezioni d'urto sono:
\begin{equation}
    \sigma_{el}=\frac{<P_{el}>}{<|\vec{S}|>}=\frac{1}{\frac{E_0^2}{2Z_0}}\frac{R_{irr}}{(R+R_{irr})^2}\frac{\epsilon_0^2}{2}=\frac{Z_0}{c^2B_0^2}\frac{R_{irr}}{(R+R_{irr})^2}\omega^2l^4B_0^2=\frac{4\pi^2}{\lambda^2}l^4Z_0\frac{R_{irr}}{(R+R_{irr})^2}
\end{equation}
\begin{equation}
    \sigma_{abs}=\frac{<P_{abs}>}{<|\vec{S}|>}=\frac{1}{\frac{E_0^2}{2Z_0}}\frac{R}{(R+R_{irr})^2}\frac{\epsilon_0^2}{2}=\frac{Z_0}{c^2B_0^2}\frac{R}{(R+R_{irr})^2}\omega^2l^4B_0^2=\frac{4\pi^2}{\lambda^2}l^4Z_0\frac{R}{(R+R_{irr})^2}
\end{equation}
\paragraph{3.b.2}\textbf{Utilizzando le apposite tabelle che forniscono le masse dei nuclei, determinare il Q-valore o l'energia di soglia dei seguenti processi, valutando l'eventuale ruolo della interazione coulombiana nello stato iniziale:}
\begin{enumerate}
    \item $p+{}^{40}Ar\longrightarrow p+{}^{39}Ar+n$
    \item $p+{}^{14}N\longrightarrow X+n$
    \item $p+{}^{16}O\longrightarrow X+n$
    \item $n+{}^{14}N\longrightarrow {}^{14}C+p$
    \item ${}^4He+{}^{14}N\longrightarrow {}^{17}O+p$
    \item ${}^2H+{}^3H\longrightarrow {}^4He+n$
    \item ${}^2H+{}^2H\longrightarrow {}^4He+\gamma$
    \item $p+{}^{198}Hg\longrightarrow {}^{197}Au+p+p$
\end{enumerate}
Per un processo il Q-valore è definito come :
\begin{equation}
    Q=(\sum m_{iniziali}-\sum m_{finali})c^2
\end{equation}
Ricordiamo che la conversione da $u.m.a.$ e $MeV$ è data da 
\begin{equation}
    1\,u.m.a=931.4\,MeV/c
\end{equation}
e che le masse del $p$ e del $n$ sono:
\begin{align}
    m_p&=938.2\,MeV/c\\
    m_n&=939.5\,MeV/c
\end{align}
Per i processi abbiamo quindi che i Q-valori sono ($c=1$):
\begin{enumerate}
    \item $Q=(938.2+931.4\cdot39.96-938.2-931.4\cdot38.96-939.5)=-8.1\,MeV$\\
    \item Consideriamo X come un nucleo con 14 nucleoni e un protone in più rispetto all'azoto $\rightarrow{}^{14}O$\\ $Q=(938.2+931.4\cdot14-931.4\cdot14-939.5)=-1.3\,MeV$\\
    \item Consideriamo X come un nucleo con 16 nucleoni e un protone in più rispetto all'ossigeno $\rightarrow{}^{16}F$\\ $Q=(938.2+931.4\cdot15.99-931.4\cdot16.01-939.5)=-19.9\,MeV$\\
    \item $Q=(939.5+931.4\cdot14-938.2-931.4\cdot14)=1.3\,MeV$\\
    \item $Q=(931.4\cdot4+931.4\cdot14-938.2-931.4\cdot16.99)=2.5\,MeV$\\
    \item $Q=(931.4\cdot2.01+931.4\cdot3.01-939.5-931.4\cdot4)=10.5\,MeV$\\
    \item $Q=(931.4\cdot2.01+931.4\cdot2.01-931.4\cdot4)=18.6\,MeV$\\
    \item $Q=(938.2+931.4\cdot197.96-938.2-938.2-931.4\cdot196.96)=-6.8\,MeV$\\
\end{enumerate}
L'energia iniziale dovuta all'interazione coulombiana si può calcolare ricordando che il raggio nucleare è $R\simeq R_0A^{1/3}$ con $R_0=1\,fm$, sa cui:
\begin{equation}
    U_e\simeq\frac{q_1q_2}{R}\simeq\frac{z_1z_2}{A_1^{1/3}+A_2^{1/3}}\cdot1.4\, MeV
\end{equation}
\paragraph{3.b.3}\textbf{Dimostrare la relazione fra la definizione della sezione d’urto elastica nel caso di fotoni incidenti su un unico bersaglio e la definizione di sezione d'urto elastica per un’onda e.m. monocromatica su un unico bersaglio.}\\
\\
Con il modello particellare si ha:
\begin{equation}
    \sigma_{\gamma,el}=\frac{1}{|\vec{j}|}\frac{dN}{dt}
\end{equation}
dove $\vec{j}=n_a\vec{v_a}$ è la densità di corrente di particelle, mentre per un onda elettromagnetica si ha:
\begin{equation}
    \sigma_{e.m.,el}=\frac{R^2<|\vec{S}_{out}|>}{<|\vec{S}_{in}|>}=\frac{\hbar\omega\dot{N}}{\hbar\omega|\vec{j}|}=\sigma_{\gamma,el}
\end{equation}

\paragraph{3.b.4}\textbf{Quale calcolo si deve effettuare per determinare il numero di eventi per unità di tempo e di volume che si producono negli urti fra particelle di due specie diverse e differenti concentrazioni le cui velocità relative sono distribuite con un funzione $f(V_{rel})$, normalizzata all'unità, e la cui sezione d'urto è $\sigma(V_{rel})$?}\\
\\
Consideriamo la situazione in cui $a+b\rightarrow f$ dove $a$ e $b$ sono due fasci di particelle con relative concentrazioni $n_a$ e $n_b$ e relative velocità $\vec{V}_a$ e $\vec{V}_b$.\\
Utilizzando il sistema di riferimento in cui le particelle b sono ferme, il numero di eventi per unità di tempo in un volume $\Delta V$
\begin{equation}
    \frac{dN}{dt}=|\vec{j}_a|\sigma_f n_b\Delta V 
\end{equation}
Generalizzando:
\begin{equation}
    \frac{dN}{dt}=|\vec{j}_a|\sigma_f n_b\Delta V
\end{equation}
Definiamo la concentrazione di eventi nell'unità di tempo e di volume, utilizzando la definizione di rateo di fluenza $|\vec{j}_a|= n_a |\vec{v}_a|$ :
\begin{equation}
    \frac{dn_f}{dt}= \frac{dN}{dt}\frac{1}{\Delta V}=n_an_b|\vec{v}_a|\sigma_f
\end{equation}
Giulio buon appetito, spero di non averti rovinato il pranzo.\\

\textbf{Come farei io:}
Consideriamo la situazione in cui $a+b\rightarrow f$ dove $a$ e $b$ sono due fasci di particelle con relative concentrazioni $n_a$ e $n_b$ e relative velocità $\vec{V}_a$ e $\vec{V}_b$.\\
Conoscendo la sezione d'urto per densità di particelle su bersaglio fermo, ci si pone nel sistema di riferimento in cui le particelle b sono ferme e si ottiene:
\begin{equation}
    \frac{dN}{dt}=|\vec{j}_a|\sigma_f n_b\Delta V=n_an_b|\vec{v}_a|\sigma_f\Delta V=n_an_b|\vec{v}_{rel}|\sigma_f\Delta V
\end{equation}
Nel caso non relativistico si può generalizzare a qualunque sistema considerando $v_{rel}$ come invariante per trasformazioni galileiane.\\
Definiamo la concentrazione di eventi nell'unità di tempo e di volume:
\begin{equation}\label{3.b.4}
    \frac{dn_f}{dt}= \frac{dN}{dt}\frac{1}{\Delta V}=n_an_b|\vec{v}_{rel}|\sigma_f
\end{equation}
L'equazione da usare se $v_{rel}$ non è fissata ma è distribuita come $f(v_{rel})$ è la seguente:
\begin{equation}
    \frac{dn}{dt}=n_an_b\int v_{rel}\sigma(v_{rel})f(v_{rel})dv_{rel}
\end{equation}
\textbf{N.B:} Secondo me sopra non c'è anche $v_{rel}$ a moltiplicare, negli appunti non la ho mentre è presente sulla nota. Non capisco a cosa serva.\\
\textbf{N.B.2:} La velocità relativa dovrebbe essere invariante anche per Lorentz, il fatto che la \ref{3.b.4} non sia generalizzabile al caso relativistico (batignani dixit) non so a cosa potrebbe essere dovuto. Il numero di particelle dovrebbe essere invariante di lorentz, è possibile che i problemi vengano fuori quando se ne consoidera la derivata rispetto al tempo (in tal caso forse basta aggiungere una $\gamma$?)


\paragraph{3.b.5}\textbf{Calcolare che un fascio di particelle incidenti su un materiale omogeneo e composto da atomi di una sola specie in funzione dello spessore attraversato si attenua con una legge esponenziale. [Dati: sezione d'urto di estinzione su ogni atomo del bersaglio, densità di massa del mezzo, numero atomico del mezzo]}\\
\\
Le particelle incidono su un materiale omogeneo con rateo di fluenza $j=n\frac{dx}{dt}$.\\
In un tempo $dt$ le particelle $N$ che hanno interagito sono:
\begin{equation}
    dN=N_1j\sigma dt
\end{equation}
dove $N_1$ è il numero di bersagli nel volume in esame. Allora se n è la densità di particelle incidenti si ha
\begin{equation}
    \frac{dn}{dt}=-\frac{\rho}{Am_h}n\sigma \frac{dx}{dt} \Longrightarrow dn=-\frac{\rho}{Am_h}n\sigma dx
\end{equation}
dove A è il numero di massa degli atomi nel bersaglio. Si ottiene
\begin{equation}
    n=n_0e^{-x/l}
\end{equation}
dove $l=\frac{Am_h}{\rho\sigma}$


\paragraph{3.b.6}\textbf{Calcolare l'attenuazione di un fascio di particelle incidenti su un materiale omogeneo e composto da atomi di diverse specie in funzione dello spessore attraversato. [Dati: sezione d'urto di estinzione del processo su ogni atomo del bersaglio, densità di massa del mezzo, numeri atomici, composizione chimica del mezzo]}\\
\\
Con considerazioni analoghe al punto precedente si ha
\begin{equation}
    \frac{dn}{dx}=-n\rho\sum_i\frac{X_i\sigma_i}{A_im_h}
\end{equation}
con $X_i$ abbondanza relativa dell'elemento $i-$esimo

\paragraph{3.b.7}\textbf{Utilizzando il modello a goccia dei nuclei e ipotizzando che l'interazione "forte" avvenga per un contatto fra adroni, effettuare una stima numerica della sezione d'urto totale forte per i seguenti urti \textit{ad alte energie in modo da poter trascurare l'interazione coulombiana:}}
\begin{itemize}
    \item $p+{}^{40}Ar$
    \item ${}^4He+{}^{14}N$
    \item ${}^2H+{}^3H$
    \item $p+{}^{12}C$
    \item $p+{}^{28}Si$
\end{itemize}
Un modo per calcolare la $\sigma$ approssimata è 
\begin{equation}
    \sigma=\sigma_0 A^{2/3}
\end{equation}
con $\sigma_0$ sezione d'urto su singolo nucleone.\\
Sappiamo che l'interazione forte è un interazione a short range, per cui avviene per contato diretto fra adroni:
\begin{equation}
    \sigma=\pi (r_1^2+r_2^2)
\end{equation}
Dove si fa riferimento ai raggi dei 2 adroni. Per quanto riguardo il raggio nucleare sappiamo che vale:
\begin{equation}
    r=r_0A^{1/3}+r_{pelle} \quad A\ge 4 \qquad r_0\sim1.25 fm\qquad r_{pelle}\sim 2 fm
\end{equation}
Mentre per il protone $r_{p}=0.83 fm$, per il deuterio $r_D=2.18 fm$ ed infine per il trizio $R_t$
\paragraph{3.b.8}\textbf{Calcolare l’energia di soglia nel laboratorio per le seguenti reazioni (la seconda particella è inizialmente ferma):}
\begin{itemize}
    \item $\gamma+{}^{16}O\longrightarrow e^-+e^++{}^{16}O$
    \item $\gamma+e^-\longrightarrow e^-+e^++e^-$
    \item $p+p\longrightarrow p+p+p+\bar{p}$
    \item $p+{}^{16}O\longrightarrow p+p+\bar{p}+{}^{16}O$
    \item $e^++e^-\longrightarrow p+\bar{p}$
    \item $ e^-+p \longrightarrow n+\nu_e$
    \item $ \bar{\nu}_e +p\longrightarrow n+e^+$
\end{itemize}
Poiché sono tutti urti del tipo:
\begin{equation*}
a+b \longrightarrow c + d +...
\end{equation*}
con la particella $b$ inizialmente ferma, possiamo ricavare una formula generale da utilizzare di volta in volta per trovare l'energia che deve possedere la particella $a$ affinché avvenga la reazione. La condizione sull'energia minima necessaria è:
\begin{equation}\label{esoglia}
M_{inv} \geq \sum\limits_{i} m_i \Rightarrow (M_{inv})^2 \geq \Bigl( \sum\limits_{i} m_i \Bigr)^2 \, ,
\end{equation}
dove la somma è estesa a tutte le particelle presenti nello stato finale. Nel sistema del laboratorio il quadrimpulso totale prima dell'urto (ponendo $c=1$) è:
\begin{equation*}
P^{\mu}_{tot}=P^\mu_a+P^\mu_b=
\begin{pmatrix}
E_a \\
\mathbf{p}_a \\
\end{pmatrix}
+
\begin{pmatrix}
E_b \\
\mathbf{0} \\
\end{pmatrix}
=
\begin{pmatrix}
E_a + E_b \\
\mathbf{p}_a \\
\end{pmatrix}
=
\begin{pmatrix}
E_a+m_b \\
\mathbf{p}_a \\
\end{pmatrix} \, ,
\end{equation*}
dato che la particella $b$ è ferma. Quindi il quadrato della massa invariante è:
\begin{align*}
(M_{inv})^2&=(E_a+m_b)^2-|\mathbf{p}_a|^2 \\
&=E_a^2+2E_a m_b+m_b^2-|\mathbf{p}_a|^2 \\
&=m_a^2+2E_am_b+m_b^2 \, .
\end{align*}
A questo punto la condizione \ref{esoglia} diventa:
\begin{equation*}
m_a^2+2E_am_b+m_b^2 \geq \Bigl( \sum\limits_{i} m_i \Bigr)^2 \, ,
\end{equation*}
ovvero:
\begin{equation}
E_a \geq \frac{ \Bigl( \sum\limits_{i} m_i \Bigr)^2 - m_a^2-m_b^2}{2m_b} \, .
\end{equation}
In conclusione, per le reazioni abbiamo:
\begin{itemize}
\item $\gamma + {}^{16}O \longrightarrow e^+ + e^- + {}^{16}O $ \\
Convertiamo innanzitutto la massa di ${}^{16}O$ (15.99 u.m.a.) in MeV. Utilizzando la relazione:
\begin{equation}
1 \quad \text{MeV}=1.07 \cdot 10^{-3} \quad \text{u.m.a.}
\end{equation}
abbiamo:
\begin{equation}
1 \quad \text{u.m.a.}=\frac{1}{1.07 \cdot 10^{-3}} \quad \text{MeV}
\end{equation}
quindi:
\begin{equation*}
15.99 \quad \text{u.m.a.}=\frac{15.99}{1.07 \cdot 10^{-3}} \quad \text{MeV}=14943.92 \quad \text{MeV} \, .
\end{equation*}
Da cui:
\begin{equation*}
E_{\gamma} \geq \frac{(m_{e^+}+m_{e^-}+m_O)^2 - m_O^2}{2m_O} \simeq 1.02 \quad \text{MeV} \, .
\end{equation*}
\item $\gamma + e^- \longrightarrow e^- + e^+ +e^- $
\begin{equation*}
E_{\gamma} \geq \frac{(m_{e^-}+m_{e^+}+m_{e^-})^2-m_{e^-}^2}{2m_{e^-}} \simeq 2.04 \quad \text{MeV}
\end{equation*}
\item $p+p \longrightarrow p + p + p + \overline{p} $
\begin{equation*}
E_p \geq \frac{(m_p+m_p+m_p+m_{\overline{p}})^2-m_p^2-m_p^2}{2m_p} \simeq 6567.89 \quad \text{MeV}
\end{equation*}
\item $p+ {}^{16}O \longrightarrow p + p + \overline{p} + {}^{16}O $
\begin{equation*}
E_p \geq \frac{(m_p+m_p+m_{\overline{p}}+m_O)^2-m_p^2-m_O^2}{2m_O} \simeq 3050.45 \quad \text{MeV}
\end{equation*}
\item $e^+ +e^- \longrightarrow p+\overline{p} $
\begin{equation*}
E_{e^+} \geq \frac{(m_p+m_{\overline{p}})^2-m_{e^+}^2-m_{e^-}^2}{2m_{e^-}} \simeq 3.45 \cdot 10^6 \quad \text{MeV}	
\end{equation*}
\item $e^- + p \longrightarrow n+\nu_e $
\begin{equation*}
E_{e^-} \geq \frac{m_n^2-m_{e^-}^2-m_p^2}{2m_p} \simeq 1.29 \quad \text{MeV}
\end{equation*}
\item $\overline{\nu}_e + p \longrightarrow n + e^+ $
\begin{equation*}
E_{\overline{\nu}_e} \geq \frac{(m_n+m_{e^+})^2-m_p^2}{2m_p} \simeq 1.80 \quad \text{MeV}
\end{equation*}
\end{itemize}



\paragraph{3.b.9}\textbf{Calcolare la probabilità che un neutrino interagisca nell’attraversare la Terra lungo un diametro. Nota: sia assuma che l’energia del neutrino sia tale che la sezione d’urto totale su un singolo nucleone sia $1fb$.}\\
\\
La sezione d'urto su un singolo nucleone è $\sigma=1fb=10^{-39}cm^2$. La probabilità che il neutrino interagisca lungo un diametro è:
\begin{equation}
    P=n_s\sigma=n_{nuc}\sigma2R_T
\end{equation}
Dove $n_s$ è la densità superficiale di nucleoni, $n_{nuc}$ è la densità di nucleoni e $R_T$ è il raggio terrestre. La densità $n_{nuc}$ si può ricavare ricordando che la massa del protone è $m_p=\frac{1g}{N_A}$ e che la densità terrestre è $\rho_T=5.5g/cm^3$, quindi:
\begin{equation}
    n_{nuc}=\frac{\rho_TN_A}{1g}=3,3\cdot10^{24}cm^{-3}
\end{equation}
La probabilità di interagire è $P\simeq4,2\cdot10^{-8}$




\paragraph{3.b.10}\textbf{Dimostrare che un elettrone (non relativistico) soggetto ad una forza elastica di
richiamo, ad una forza di attrito viscoso ed alla forza di reazione radiativa, nel
campo di un’onda e.m. piana polarizzata linearmente oscilla con la legge
\begin{equation*}
    \vec{x}=\frac{e\vec{E}_0}{m_e}\frac{1}{\omega_0^2-\omega^2-i\omega\Gamma_{tot}}e^{-i\omega t} \ \ \ \ \text{con} \ \ \ \Gamma_{tot}=\Gamma'+\Gamma \frac{\omega^2}{\omega_0^2}
\end{equation*}}\\ \\
Un elettrone soggetto alle forze indicato ha un'equazione del moto:
\begin{equation}
    m_e\Ddot{\vec{x}}=-m_e \omega_0 ^2\vec{x}-\beta \dot{\vec{x}}+m_e \tau \dddot{\vec{x}} -e\vec{E}_0 e^{-i\omega t} 
\end{equation}
suppongo una soluzione oscillante $\vec{x}=\vec{x}_0 e^{-i\omega t}$.
\begin{equation}
    -\omega^2 m_e \vec{x}_0 e^{-i\omega t}=-m_e \omega_0 ^2 \vec{x}_0 e^{-i\omega t}+i\omega \beta \vec{x}_0 e^{-i\omega t} + m_e i\omega^3 \tau \vec{x}_0 e^{-i\omega t}-e\vec{E}_0 e^{-i\omega t}
\end{equation}
da cui
\begin{equation}
    \vec{x}_0(-\omega^2m_e + \omega_0 ^2 m_e -i\omega \beta -m_e i \omega^3 \tau)=-e\vec{E}_0
\end{equation}
ponendo $-\beta \dot{\vec{x}}\equiv -m_e \Gamma ' \dot{\vec{x}}$ si ha
\begin{equation}
     \vec{x}_0(-\omega^2m_e + \omega_0 ^2 m_e -i\omega m_e \Gamma ' -m_e i \omega^3\tau)=-e\vec{E}_0
\end{equation}
perciò
\begin{equation}
    \vec{x}_0=\frac{-e \vec{E}_0}{m_e}\frac{1}{\omega_0 ^2 -\omega^2 -i\omega( \Gamma' +\omega^2 \tau)}
\end{equation}
Ponendo $\Gamma_{tot}=\Gamma' + \Gamma \frac{\omega^2}{\omega_0 ^2}$ si trova infine
\begin{equation}
    \vec{x}=\frac{-e \vec{E}_0}{m_e}\frac{1}{\omega_0 ^2 -\omega^2 -i\omega\Gamma_{tot}}e^{-i\omega t}
\end{equation}




Batignani ha sbagliato il segno, perchè la mia equazione se sostituisco $-e$ con $q$ torna come quella che ha fatto lui negli appunti, quindi ha mandato $q$ in $e$ e non in $-e$.
\\
Scopro ristudiando gli appunti che l'assoluto pazzuomo considera la carica dell'elettrone $e=-|e|$. Perciò boh, è solo un discorso di segni.





\paragraph{3.b.11}\textbf{Dimostrare che la sezione d’urto differenziale \underline{elastica} per un’onda e.m. piana e
monocromatica su un elettrone legato elasticamente vale 
\begin{equation*}
    \frac{d\sigma_{el}}{d\Omega}=r_e ^2 \frac{\omega^4}{(\omega_0 ^2 -\omega^2)^2+\omega^2 \Gamma_{tot}^2}sin^2(\alpha)
\end{equation*}
con $\alpha$ angolo fra la direzione di osservazione e direzione di polarizzazione
(lineare) dell'onda.}\\ \\
Usando la definizione di sezione d'urto elastica e la seguente espressione per il vettore di Poynting elastico
\begin{equation}
    \vec{S}_{el}=\frac{e^2 |\Ddot{\vec{x}}|^2}{(4\pi)^2\varepsilon_0 c^3 R^2}sin^2\alpha \hat{r}
\end{equation}

si trova
\begin{equation}
    \frac{d\sigma_{el}}{d\Omega}=R^2\frac{<|\vec{S}_{el}|>}{<|\vec{S}_{in}|>}=R^2 \frac{1}{c \varepsilon_0 <|\vec{E}|^2>}\frac{e^2}{(4 \pi)^2 \varepsilon_0 c^3 R^2}\frac{\omega^4}{(\omega_0 ^2 -\omega^2)^2+\omega^2 \Gamma_{tot}^2}\frac{e^2 <|\vec{E}|^2>}{m_e ^2}sin^2 (\alpha)
\end{equation}
Ricordando che 
\begin{equation}
    r_e\equiv \frac{e^2}{4\pi \varepsilon_0 m_e c^2}
\end{equation}
si trova
\begin{equation}
     \frac{d\sigma_{el}}{d\Omega}=r_e^2 sin^2(\alpha) \frac{\omega^4}{(\omega_0^2-\omega^2)^2+\omega^2 \Gamma_{tot}^2}
\end{equation}




\paragraph{3.b.12}\textbf{Dimostrare che la sezione d’urto Thomson vale $\sigma_{th}=\frac{8}{3}\pi r_e ^2=0.66 \ barn$}\\ \\
La sezione d'urto Thomson si ha quando dei fotoni a bassa energia (più bassa dell'energia dell'effetto Compton), interagiscono con un elettrone libero, facendolo oscillare.
\\
Si definisce allora
\begin{equation}
    \sigma_{th}=\frac{<P_{tot}>}{<|\vec{S}|>}
\end{equation}

con
\begin{equation}
    <|\vec{S}|>=\varepsilon_0 c <|\vec{E}_0|^2>
\end{equation}
\begin{equation}
    <P_{tot}>=\frac{e^2}{6\pi \varepsilon_0 c^3}\frac{<|\vec{E}_0|^2>}{m_e^2}\frac{\omega^4}{(\omega_0-\omega^2)^2}
\end{equation}
Dal momento che l'elettrone è libero, non si ha un coefficiente di attrito viscoso e $\omega_0=0$, per cui
\begin{equation}
    \sigma_{th}=\frac{e^4}{6\pi \varepsilon_0 c^3}\frac{<|\vec{E}_0|^2>}{m_e^2}\frac{1}{\varepsilon_0 c^2 <|\vec{E}_0|^2>}
\end{equation}
Ricordando che il raggio classico dell'elettrone è 
\begin{equation}
    r_e\equiv \frac{e^2}{4\pi \varepsilon_0 m_e c^2}
\end{equation}
si trova
\begin{equation}
    \sigma_{th}=\frac{8}{3}\pi r_e^2
\end{equation}

Inserendo i dati numerici:
\begin{equation}
    \sigma_{th}=\frac{8\pi}{3}r_e^2\simeq\frac{8 \pi}{3}7.84 \times 10^{-30} m^2\simeq 66 \times 10^{-30} m^2\simeq 0.66 \ b
\end{equation}



\paragraph{3.b.13}\textbf{Dimostrare che la sezione d’urto \underline{elastica} per un’onda e.m. piana e
monocromatica su un elettrone legato elasticamente vale \begin{equation*}
    \sigma_{el}=\sigma_{th}\frac{\omega^4}{(\omega_0^2-\omega^2)^2+\omega^2\Gamma_{tot}^2}
\end{equation*}
}\\ \\
Sapendo che:
\begin{equation}
    <P_{el}>=\frac{q^2<|a|^2>}{6\pi \epsilon_0c^3}
\end{equation}
Partendo dalla definizione:
\begin{equation}\begin{split}
    \sigma_{el}=\frac{<P_{el}>}{<|\vec{S}_{in}|>}=\frac{e^2}{6\pi \varepsilon_0 c^3}\frac{\omega^4 e^2 <|\vec{E}|^2>}{m_e^2 (\omega_0^2-\omega^2)+\omega^2 \Gamma_{tot}^2}\frac{2}{\varepsilon_0 c <|\vec{E}|^2>}= \\ =r_e^2 \frac{8}{3}\pi \frac{\omega^4}{(\omega_0^2 -\omega^2)+\omega^2 \Gamma_{tot}^2}\equiv \sigma_{th}\frac{\omega^4}{(\omega_0^2 -\omega^2)+\omega^2 \Gamma_{tot}^2}
\end{split}\end{equation}





\paragraph{3.b.14}\textbf{Dimostrare che la sezione d’urto \underline{totale} per un’onda e.m. piana e monocromatica
su un elettrone legato elasticamente vale \begin{equation*}
    \sigma_{tot}=4\pi r_e c \frac{\omega^2 \Gamma_{tot}}{(\omega_0^2-\omega^2)^2+\omega^2\Gamma^2_{tot}}
\end{equation*}}\\ \\
Partendo dalla definizione
\begin{equation}
    \sigma_{tot}=\frac{<P_{tot}>}{<|\vec{S}_{in}|>}
\end{equation}
con
\begin{equation}
    <|\vec{S}_{in}|>=\frac{1}{2}\varepsilon_0 c |\vec{E}|^2
\end{equation}
\begin{equation}
    <P_{tot}>=\frac{1}{2}Re[q\dot{\vec{x}}\cdot \vec{E}]
\end{equation}
Il campo elettrico e $\vec{x}$ hanno la stessa dipendenza temporale, si sostituisce quanto trovato nei punti precedenti e risulta 
\begin{equation}
    <P_{tot}>=\frac{e^2\omega^2|\vec{E}|^2\Gamma_{tot}}{2m_e[(\omega_0^2-\omega^2)^2+\omega^2 \Gamma_{tot}^2]}
\end{equation}
Perciò
\begin{equation}
    \sigma_{tot}=\frac{e^2\omega^2\Gamma_{tot}}{m\varepsilon_0 c}\frac{1}{(\omega_0^2-\omega^2)+\omega^2\Gamma_{tot}}=4\pi r_e c \frac{\omega^2 \Gamma_{tot}}{(\omega_0^2-\omega_2)^2+\omega^2\Gamma_{tot}}
\end{equation}








\paragraph{3.b.15}\textbf{Dimostrare che la sezione d’urto elastica per un’onda e.m. piana e
monocromatica su un elettrone legato elasticamente in prossimità di una
risonanza stretta (specificare il criterio) si può approssimare con una curva
lorenziana \begin{equation*}
    \sigma_{el}\simeq \sigma_{th}\frac{\omega_0^2/4}{(\omega_0-\omega)^2+\frac{(\Gamma' +\Gamma)^2}{4}}
\end{equation*}}\\ \\

Partendo dalla sezione d'urto elastica (Formula di Breit-Wigner):
\begin{equation}
    \sigma_{el}=\sigma_{th}\frac{\omega^4}{(\omega_0^2-\omega^2)^2+\omega^2 \Gamma_{tot}^2}
\end{equation}
Per $\omega \sim \omega_0$ si ha
\begin{equation}
    (\omega_0^2-\omega^2)^2= [(\omega_0-\omega)(\omega_0+\omega)]^2\simeq[2\omega_0(\omega_0-\omega)]^2=4\omega_0^2(\omega_0-\omega)^2
\end{equation}
In più in corrispondenza della risonanza $\Gamma_{tot}\simeq \Gamma + \Gamma'$. \\
Da cui
\begin{equation}
    \sigma_{el}=\sigma_{th}\frac{\omega_0^2/4}{(\omega_0-\omega)^2+\frac{(\Gamma'+\Gamma)^2}{4}}
\end{equation}
Si ha una risonanza stretta quando $\Gamma +\Gamma' << \omega_0$, mentre la risonanza si dice larga se $\Gamma +\Gamma'\sim \omega_0$.




\paragraph{3.b.16}\textbf{Dimostrare (a partire dalle espressioni delle domande precedenti) che per un’onda
e.m. piana e monocromatica su un elettrone legato elasticamente le sezioni d’urto
al \underline{picco} valgono \begin{equation*}
  \left\{  \begin{array}{lr}
         \sigma_{el}=\frac{3\lambda_0^2}{2\pi}\left( \frac{\Gamma}{\Gamma+\Gamma'} \right)^2  \\
         \sigma_{tot}=\frac{3\lambda_0^2}{2\pi}\frac{\Gamma}{\Gamma+\Gamma'} \\
         \sigma_{abs}=\frac{3\lambda_0^2}{2\pi}\frac{\Gamma \Gamma'}{(\Gamma+\Gamma')^2} \ \ \ \ \text{dove} \ \ \ \lambda_0=\frac{2\pi c}{\omega_0}
    \end{array} \right.
\end{equation*}
}\\ \\

In base a quanto trovato ai punti precedenti risulta
\begin{equation}
    \sigma_{el}(\omega=\omega_0)=\frac{8/3 \pi r_e^2 \omega_0^4}{\omega_0^2(\Gamma +\Gamma')^2}=\frac{\frac{8}{3}\pi \frac{9}{4} \tau^2 c^2 \omega_0^4 }{\omega_0^2(\Gamma+\Gamma')^2}=\frac{3\lambda_0^2}{2\pi}\left( \frac{\Gamma}{\Gamma+\Gamma'} \right)^2
\end{equation}
\begin{equation}\begin{split}
    \sigma_{tot}(\omega=\omega_0)=\frac{4\pi r_e c \omega^2 \Gamma_{tot}(\omega_0)}{\omega^2 \Gamma_{tot}^2(\omega_0)}=\frac{4\pi r_e c}{\Gamma + \Gamma'}=\frac{4\pi(3/2\tau c)c}{\Gamma+\Gamma'}=\\
    =6\pi \frac{\Gamma}{\omega_0^2}c^2\frac{1}{\Gamma+\Gamma'}=\frac{6\pi c^2}{\omega_0^2}\frac{\Gamma}{\Gamma+\Gamma'}=\frac{6\pi }{4\pi^2}\lambda_0^2\frac{\Gamma}{\Gamma+\Gamma'}=\frac{3}{2\pi}\lambda_0^2\frac{\Gamma}{\Gamma+\Gamma'}
\end{split}\end{equation}
Dove si è sfruttato che $\tau=\frac{2}{3}\frac{e^2}{c^3}\Longrightarrow r_e=\frac{e^2}{m_ec^2}=\frac{3}{2}\tau c$.\\
Si può quindi trovare per differenza
\begin{equation}
    \sigma_{abs}(\omega=\omega_0)=\frac{3}{2\pi}\lambda_0^2\left( \frac{\Gamma}{\Gamma+\Gamma'}-\frac{\Gamma^2}{(\Gamma+\Gamma')^2} \right)=\frac{3 \lambda_0^2}{2\pi}\frac{\Gamma \Gamma'}{(\Gamma +\Gamma')^2}
\end{equation}






\paragraph{3.b.17}\textbf{Dimostrare che un elettrone (moto non relativistico) soggetto ad una forza
elastica di richiamo, ad una forza di attrito viscoso ed alla forza di reazione
radiativa, se viene lasciato libero di oscillare da una posizione iniziale perde
energia con una legge esponenziale in cui la costante tempo vale $\frac{1}{\Gamma' +\Gamma}$. Come si
chiama questa costante tempo? Quale sarebbe la costante tempo con cui, invece,
si smorza l'ampiezza delle oscillazioni?}\\ \\

La condizione (in genere verificata) necessaria per fare la seguente dimostrazione è $\Gamma'<<\omega_0<<1/\tau$.
\\
Posso cercare una soluzione del tipo:
\begin{equation}
    \vec{x}\simeq \vec{x}_0e^{-i\omega_0 t}e^{-\gamma t/2}
\end{equation}
Sostituendo nell'equazione del moto, ossia:
\begin{equation}
    -\tau \dddot{\vec{x}}+\Ddot{\vec{x}}+\Gamma'\dot{\vec{x}}+\omega_0^2\vec{x}=0
\end{equation}

si trova
\begin{equation}
    -\tau(-i(\omega_0-i\gamma/2))^3+(-i(\omega_0-i\gamma/2))^2+(-i(\omega_0-i\gamma/2))\Gamma'+\omega_0^2=0
\end{equation}
Si considerano solo i termini superiori
\begin{equation}
    -i\Gamma\omega_0-3\Gamma \gamma/2 + i\gamma \omega_0 - i\Gamma' \omega_0 - \gamma \Gamma'/2 \simeq 0
\end{equation}
Dal momento che $\Gamma ' << \omega_0$ si possono trascurare i termini reali, trovando
\begin{equation}
    \gamma=\Gamma+\Gamma'=\Gamma_{tot}(\omega=\omega_0)
\end{equation}
Perciò la costante di decadimento temporale con cui si smorzano le oscillazioni è 
\begin{equation}
    \tau_{amp}=\frac{2}{\gamma}=\frac{2}{\Gamma+\Gamma'}
\end{equation}
L'energia è invece quadratica sia nella velocità che nella posizione, perciò varia secondo la legge:
\begin{equation}
    E\simeq E_0 e^{-\gamma t}
\end{equation}

con
\begin{equation}
    \tau_{energia}=\frac{1}{\gamma}=\frac{1}{\Gamma + \Gamma'}
\end{equation}

\paragraph{3.b.18}\textbf{Calcolare la relazione tra parametro d'impatto ($b$) e angolo di scattering ($\theta$) nel
caso di scattering di Rutherford (Coulombiano) e di scattering su sfera rigida.}\\ \\
\begin{itemize}
    \item \textit{Scattering Coulombiano :}\\
    Una particella di carica $ze$ e massa $m$ urta su un nucleo (considerato fisso) di carica $Ze$.\\
    Considerando La figura (\ref{ruth1})
   \begin{figure}[ht]
    \begin{center}
    \includegraphics[width=0.6\textwidth]{ruth1.JPG}
    \end{center}
    \label{ruth1}
    \end{figure}
    
    e considerando l'urto elastico ($|\vec{v}_f|=|\vec{v}_0|$) si trova che 
    \begin{equation}
        \Delta p=\sqrt{(m\vec{v}_f-m\vec{v}_0)^2}=2mv_0sin\theta/2
    \end{equation}
    Ricercando l'analoga relazione a partire dalla variazione della forza agente sulla particella (seconda equazione cardinale), si può procedere come segue:
    \\
    In riferimento alla figura (\ref{ruth2})
    \begin{figure}[H]
    \begin{center}
    \includegraphics[scale=0.4]{ruth2.JPG}
    \end{center}
    \label{ruth2}
    \end{figure}
    si impone la conservazione del momento angolare.
    \\
    Inizialmente
    \begin{equation}
        L_z=mv_0 b
    \end{equation}
    Mentre successivamente 
    \begin{equation}
        L_z=m(\vec{r}\wedge \vec{v})_z=m[\vec{r}\wedge (\dot{r}\hat{r}+r\dot{\beta}\hat{\beta})]=mr^2\dot{\beta}=mr^2\frac{d\beta}{dt}
    \end{equation}
    Perciò
    \begin{equation}
        mv_0b=mr^2\frac{d\beta}{dt} \ \ \Rightarrow \ \ dt=\frac{r^2}{bv_0}d\beta
    \end{equation}
    A questo punto si può trovare $\Delta p$:
    \begin{equation}
        \Delta p = |\Delta \vec{p}|=\int_{-\infty}^{\infty} |\vec{F}_{\perp}|dt=\int_{-\infty}^{\infty}|\vec{F}|cos\beta dt=\int_{-\pi/2+\theta/2}^{\pi/2-\theta/2}\frac{zZe^2}{4\pi \varepsilon_0 r^2}\frac{cos\beta}{b}\frac{r^2}{v_0}d\beta=\frac{zZe^2}{2\pi\varepsilon_0 b v_0}cos\theta/2
    \end{equation}
    Angolo max e min si trovano da considerazioni geometriche: $2\beta_{max}+\theta=\pi$, $\beta_{max}=-\beta_{min}$
    \\
    Eguagliando le due espressioni si trova
    \begin{equation}
        b=\frac{zZe^2}{4\pi \varepsilon_0 m v_0^2}
cotan\theta/2\equiv \frac{d}{2}cotan\frac{\theta}{2}
\end{equation}
    \item \textit{Scattering su sfera rigida :}
    Utilizzando la leggi dell'ottica in riferimento alla figura (\ref{sfera})
    \begin{figure}[H]
    \begin{center}
    \includegraphics[width=0.6\textwidth]{sfera.JPG}
    \end{center}
    \label{sfera}
    \end{figure}
    si trova
    \begin{equation}
        b=Rsin\left(\frac{\pi-\theta}{2}\right)=Rcos\left(\frac{\theta}{2}\right)
    \end{equation}
    
    
\end{itemize}


\paragraph{3.b.19}\textbf{Ricavare la sezione d’urto differenziale dello scattering di Rutherford, sia in
funzione dell’angolo $\theta$ sia in funzione del parametro di impatto $b$}\\ \\
Per definizione 
\begin{equation}
    d\sigma=b db d\phi
\end{equation}
in questo caso c'è simmetria assiale, quindi
\begin{equation}
\frac{d\sigma}{db}=2\pi b
\end{equation}
Si può allora scrivere:
\begin{equation*}
d\sigma=2\pi b \frac{db}{d\cos\theta} d\cos\theta=2\pi b \frac{db}{\sin\theta d\theta} d\cos\theta=\frac{b}{\sin\theta} \frac{db}{d\theta} 2\pi d\cos\theta
\end{equation*}
Essendo $2\pi d\cos\theta=d\Omega$, abbiamo:
\begin{equation}
d\sigma=\frac{b}{\sin\theta} \frac{db}{d\theta} d\Omega \Rightarrow \frac{d\sigma}{d\Omega}=\frac{b}{\sin\theta} \frac{db}{d\theta} \, .
\end{equation}
In particolare abbiamo:
\begin{equation*}
\frac{db}{d\theta}=\frac{d}{d\theta} \Bigl(\frac{d}{2}\cot\frac{\theta}{2}\Bigr)=\frac{d}{2}\left( -\frac{1}{2}\right)\frac{sin^2(\theta/2)+cos^2(\theta/2)}{sin^2(\theta/2)}=-\frac{d}{4}\frac{1}{sin^2(\theta/2)}
\end{equation*}
Perciò in ultima istanza si trova:
\begin{equation}
    \frac{d\sigma}{d\Omega}=\frac{b}{\sin\theta} \frac{db}{d\theta}  =\frac{d}{2} \cot\frac{\theta}{2} \frac{1}{\sin\theta} \frac{d}{4} \frac{1}{\sin^2\frac{\theta}{2}} =\frac{d}{2} \frac{\cos\frac{\theta}{2}}{\sin\frac{\theta}{2}} \frac{1}{2\sin\frac{\theta}{2}\cos\frac{\theta}{2}} \frac{d}{4} \frac{1}{\sin^2\frac{\theta}{2}} =\frac{d^2}{16} \frac{1}{\sin^4\frac{\theta}{2}}=\frac{b^2}{2sin^2\theta}
\end{equation}
con $d$ definito come sopra:
\begin{equation}
    d=\frac{zZe^2}{2\pi \varepsilon_0 m v_0^2}
\end{equation}

\paragraph{3.b.20}\textbf{Calcolare la minima distanza fra le due particelle in uno scattering Rutherford}\\
\\
Chiamiamo $x$ la minima distanza raggiunta fra due particelle in uno scattering Rutherford in funzione dell'energia cinetica del proiettile e dell'angolo di scattering.\\
Si impone la conservazione dell'energia e del momento angolare fra lo stato iniziale e l'istante in cui raggiunge la minima distanza con velocità di modulo $V_m$.
\begin{equation}
    \left\{ \begin{array}{lr}
         \frac{1}{2} m V_0^2= \frac{1}{2}m V_m^2+ \frac{zZe^2}{4 \pi \epsilon_0 x}  \\
         mV_0b=mV_mx
    \end{array} \right.
\end{equation}
Il fatto che $x$ sia la minima distanza raggiunta e non un punto qualsiasi si impone scrivendo il momento angolare come $L_x=m V_m x$. Il momento angolare è per definizione $L=m (\vec{r}\times \vec{v})$ ma il punto di minimo avvicinamento è l'unico punto su tutta la traiettoria in cui la velocità è perpendicolare al raggio che converge la particella bersaglio alla particella proiettile.\\

Dalla seconda equazione si trova:
\begin{equation*}
    V_m=V_0b/x
\end{equation*}
Sostituendo nella prima equazione e riscrivendo l'ultimo termine esplicitando $d=\frac{zZe^2}{4 \pi \epsilon_0 \frac{1}{2}m V_0^2}$:
\begin{equation*}
    \frac{1}{2}mV_0^2=\frac{1}{2}m V_0^2 \left(\frac{b}{x}\right)^2 + \frac{1}{2} mV_0^2 \frac{d}{x}
\end{equation*}
Semplificando i termini, e sostituendo $b=\frac{d}{2}cotg\frac{\theta}{2}$ si trova:
\begin{equation*}
    x^2-dx-b^2=0 \Rightarrow x^2-dx-\frac{d^2}{4} cotg^2\frac{\theta}{2}=0
\end{equation*}
Quindi (escludendo la soluzione con il -):
\begin{equation}
    x=\frac{d}{2}\left(1+\frac{1}{sin\frac{\theta}{2}} \right)
\end{equation}
NOTA: Nei casi particolari in cui $\theta=0$ e $\theta=\pi$ si ha rispettivamente $x\rightarrow\infty$ e $x=d$.



\paragraph{3.b.21}\textbf{Discutere le differenze tra lo scattering di Rutherford (particelle $\alpha$ su nuclei) e lo
scattering di elettroni su bersaglio puntiforme.}\\ \\
Nel caso di scattering di elettroni su bersaglio puntiforme, l'interazione è di natura elettromagnetica.\\
\begin{itemize}
    \item Nostri amici: Supponendo il bersaglio carico positivamente nel primo caso l'interazione sarà puramente repulsiva.\\
Inoltre mentre per le particelle $\alpha$, molto massive, possiamo trascurare i termini relativistici, per gli elettroni, la cui massa è 2000 volte minore di quella del protone, dobbiamo tenere conto dei
termini relativistici. E’ necessario quindi utilizzare due diverse relazioni per le sezioni d’urto, in particolare per le particelle $\alpha$, essendo trascurabili i termini relativistici possiamo utilizzare la sezione d’urto di Rutherford, mentre
nel caso degli elettroni (relativistici) dobbiamo utilizzare la sezione d’urto di Mott, che tiene conto degli effetti relativistici.
\item Io aggiungerei il fatto che l'elettrone non fa interazione forte:
In questo caso è possibile ricavare la sezione d'urto differenziale in funzione di b (parametro di impatto) e di $\theta$ (angolo di scattering) come:
\begin{equation}
    \frac{d\sigma}{d\Omega}=\frac{d^2}{16sin^4\biggl(\frac{\theta}{2}\biggr)} \qquad d=\frac{zZe^2}{4\pi\epsilon_0 T}=\frac{zZ\alpha \hbar c}{T}
\end{equation}
In funzione dell'angolo di scattering la curva dovrebbe mostrare un andamento atteso:\\
\begin{figure}[H]
    \centering
    \includegraphics[scale=0.7]{3.b.21_TH.PNG}
    \caption{Sezione d'urto differenziale in funzione della frequenza per bersaglio puntiforme}
    \label{3.b.21_TH}
\end{figure}
    Tuttavia nel caso dello scattering rutherford non si possono trascurare le dimensioni dei corpi incidenti.\\
    Ad esempio, per un angolo fissato di $60^\circ$, oltre un valore critico di T=26 MeV si osserva come dati non verifichino più la previsione teorica per uno scattering con interazione EM su bersaglio puntiforme.\\
\begin{figure}[H]
    \centering
    \includegraphics[scale=0.7]{3.b.21_T.png}
    \caption{Curva sperimentale}
    \label{3.b.21_T}
\end{figure}
    \end{itemize}
    
\paragraph{3.b.22}\textbf{Quali sono le informazioni che si possono dedurre sulle caratteristiche dell'interazione nucleare osservando l’andamento delle energie di legame dei
nuclei in funzione di A? Quali sono le differenze principali con il caso dell'interazione tra gli elettroni in un atomo?}\\
\begin{figure}[H]
    \centering
    \includegraphics[scale=0.4]{3.b.22.PNG}
\end{figure}
La figura precedente mette in relazione la grandezza $\frac{B_{A,Z}}{A}$ con il numero di nucleoni A. Si nota che la maggior parte degli elementi assume valori attorno a 8 MeV, questo implica $\frac{B_{A,Z}}{A} \sim cost$. Il fatto che $ B_{A,Z} \propto A$ è un informazione sul fatto che l'interazione nucleare è "short range". Nel caso dell'interazione tra gli elettroni in un atomo si ha che $ B_{A,Z} \sim Z^2$ poiché deriva dal potenziale coulombiano. L'interazione atomica è quindi "long range".\\
Dalla figura precedente si vede anche che la curva ha un massimo in corrispondenza di A=56 (elemento più stabile), si prevedono quindi due possibilità per guadagnare energia raggiungendo una situazione più stabile: fondere due nuclei molto più leggeri ("fusione nucleare"), scindere un nucleo pesante ("fissione nucleare").

\paragraph{3.b.23}\textbf{Qual è l'andamento delle masse nucleari a parità di A in funzione di Z?}\\
\\
In riferimento alla formula semiempirica di massa in equazione (\ref{3.a.27 semiemp}) all'aumentare di Z a fisso A crescono i contributi dovuti al termine di simmetria e al termine di repulsione coulombiana (entrambi con il segno meno), inoltre varia il termine di pairing.\\
In generale a fisso A, con A dispari si ottiene una parabola in Z con massimo in:
\begin{equation}
    \frac{\partial B_{A,Z}}{\partial Z}=-2a_c\frac{Z}{A^{\frac{1}{3}}}+4a_{sym} \frac{A-2Z}{A}=0\Longrightarrow \bar{z}=\frac{2a_{sym}A}{a_cA^{\frac{2}{3}}+4a_{sym}}
\end{equation}
Diversamente a fisso A pari le masse dei nuclei si dispongono
alternativamente su due parabole traslate del doppio del termine di pairing e con punto di massimo nello stesso punto.\\
\begin{figure}[ht]
    \centering
    \includegraphics[scale=0.5]{3.b.23_Parabole.PNG}
    \label{3.b.23_Parabole}
\end{figure}
N.B: Logoteta parla di punto di massimo, poichè abbiamo definito B(A,Z) come l'energia necessaria per separare un nucleo negli A nucleoni che lo costituiscono, per cui ha sempre valore positivo e corrisponde al modulo della buca di potenziale.\\
In generale tuttavia l'energia del nucleone nel nucleo è negativa in quanto costretto a rimanere ingabbiato, questo spiega la concavità delle parabole in figura.\\

\paragraph{3.b.24}\textbf{Cercando i dati nelle apposite tabelle (reperibili sul web) si indichino gli stati finali e si calcoli il Q-valore per i decadimenti delle seguenti specie instabili: ${}^8B,{}^{39}Ar,{}^7Be, {}^{64}Cu,{}^{76}Ge$}\\
\\
Sito da cui cercare: \url{https://www.nndc.bnl.gov/wallet/wall35.pdf}
(colonna  \textbf{Decay mode} per capire lo stato finale, colonna $\Delta$ per il Q-valore.\\
\begin{itemize}

\item ${}^8B$\\
Canale di decadimento: $\epsilon$, può essere o $\beta_+$ o EC.\\
\[
\beta_+: \;\;\;\; {}^8B\rightarrow {}^8Be^-+e^++\nu_e
\]
\[
Q=\Delta_{A,Z}-\Delta_{A,Z-1}-2m_e=(22.9212)-(4.941)-1.022=16.9582
\]
Il Q-valore è positivo, viene preferito il decadimento $\beta_+$ anzi che l'EC. \warning Se fosse stato negativo allora si doveva considerare il decadimento EC.
\item ${}^{39}Ar$\\
Canale di decadimento: $\beta_-$
\[
{}^{39}Ar\rightarrow {}^{39}K^++e^-+\bar{\nu}_e
\]
\[
Q=\Delta_{A,Z}-\Delta_{A,Z+1}= (-33.242)-(-33.807)= 0.565
\]
\item ${}^7Be$\\
Canale di decadimento:$\epsilon$, può essere o $\beta_+$ o EC.\\
\[
\beta_+: \;\;\;\; {}^7Be\rightarrow {}^7Li^-+e^++\nu_e
\]
\[
Q=\Delta_{A,Z}-\Delta_{A,Z-1}-2m_e=(15.768)-(14.907)-1.022= -0.161
\]
In questo caso viene preferita la cattura elettronica.
\[
EC: \;\;\;\; {}^7Be\rightarrow {}^7Li+\nu_e
\]
\[
Q=\Delta_{A,Z}-\Delta_{A,Z-1}=(15.768)-(14.907)= 0.861
\]

\item ${}^{64}Cu$\\
Canale di decadimento: $\epsilon$ (61.5\%) $\beta_-$ (38.5\%)
\[
\beta_+: \;\;\;\; {}^{64}Cu \rightarrow {}^{64}Ni^-+e^++\nu_e
\]
\[
Q=\Delta_{A,Z}-\Delta_{A,Z-1}-2m_e=(-65.424)-(-67.098)-1.022=0.652
\]
Il Q-valore è positivo, viene preferito il decadimento $\beta_+$ anzi che l'EC.\\
\[
\beta_-:\;\;\;\; {}^{64}Cu\rightarrow {}^{64}Zn^++e^-+\bar{\nu}_e
\]
\[
Q=\Delta_{A,Z}-\Delta_{A,Z+1}=(-65.424) -(-66.003)= 0.579
\]

\item ${}^{76}Ge$: nucleo stabile (la colonna "Dacay mode" è vouta)

\end{itemize}



\paragraph{3.b.25}\textbf{Spiegare perchè i nuclei decadono alfa e fornire un range tipico delle vite medie
per i nuclei che decadono alfa. Quali sono le energie tipiche rilasciate nei decadimenti alfa? }\\ \\
I nuclei decadono $\alpha$ quando, per ragioni energetiche (in genere repulsione coulombiana) il Dio Quantistico decide di diminuire il loro numero atomico. La particella rilasciata è proprio un $^4He$ e non una particella più leggera poichè essa presenta una buona binding energy (28 MeV) e coincide con uno degli spyke che si osservano nella formula semiempirica di massa.\\
I q valori tipici dei decadimenti $\alpha$ sono dell'ordine di 5 MeV, sebbene si trovino valori superiori come nel caso del $^{218}Th$ con Q=9.85 MeV.\\
Le vite medie variano su un range molto ampio e possono essere stimate a partire dalla formula di Geiger-Nuttal:
\begin{equation}
    ln(t_{1/2})=a+\frac{b}{\sqrt{Q}} \qquad \text{a,b costanti}
\end{equation}
Si riportano 2 esempi:
\begin{align}
    &^{237}Th:\ Q=4.08\ MeV \qquad t_{1/2}\sim 1.4 \cdot 10^{10} \ years \\
    &^{218}Th:\ Q=9.85\ MeV \qquad t_{1/2}\sim  10^{7} \ s 
\end{align}
\paragraph{3.b.26}\textbf{ Cercando i dati nelle apposite tabelle (reperibili sul web) si trovino i Q-valori per le reazioni:
\[
n+{}^{154}Gd\rightarrow \gamma +{}^{155}Gd
\]
\[
n+{}^{155}Gd\rightarrow \gamma +{}^{156}Gd
\]
}
Sito da cui cercare: \url{https://www.nndc.bnl.gov/wallet/wall35.pdf}(colonna  $\Delta$ per il Q-valore).\\
\begin{itemize}
    \item $n+{}^{154}Gd\rightarrow \gamma +{}^{155}Gd$
\[
m_n+m_{{}^{154}Gd}=m_{{}^{155}Gd} +m_{\gamma} +Q \simeq m_{{}^{155}Gd} +Q
\]
\[
\Delta_m + m_u +\Delta_{{}^{154}Gd} + 154 m_u = \Delta_{{}^{155}Gd} + 155 m_u+Q
\]
\begin{equation*}
    Q= \Delta_m + \Delta_{{}^{154}Gd} - \Delta_{{}^{155}Gd}= (8.071)+(-73.705)-(-72.069)=6.436
\end{equation*}
    \item $n+{}^{155}Gd\rightarrow \gamma +{}^{156}Gd$
 \[
m_n+m_{{}^{155}Gd}=m_{{}^{156}Gd} +m_{\gamma} +Q \simeq m_{{}^{156}Gd} +Q
\]  
\[
\Delta_m + m_u +\Delta_{{}^{155}Gd} + 155 m_u = \Delta_{{}^{156}Gd} + 156 m_u+Q
\]
\begin{equation*}
    Q= \Delta_m + \Delta_{{}^{155}Gd} - \Delta_{{}^{156}Gd}= (8.071)+(-72.069)-(-72.534)= 8.536
\end{equation*}
\end{itemize}

\paragraph{3.b.27}\textbf{Dimostrare che $\frac{d^3\vec{p}}{2E}$ è un invariante relativistico effettuando esplicitamente la
trasformazione di Lorentz (si consideri il boost lungo un asse, per esempio l'asse
x)}\\\\
Suppongo che nel sistema primato la velocità sia lungo l'asse x.
\begin{align}
    \frac{d^3\vec{P}'}{2E'}&=\frac{dP_x'dP_y'dP_z'}{2E'}=\frac{\gamma d(P_x-\beta E)dP_ydP_z}{2\gamma(E-\beta P_x)}=\frac{\biggl[dP_x-\beta d\sqrt{m^2+P_x^2}\biggr]dP_ydP_z}{2(E-\beta P_x)}=\\
    &=\frac{\biggl[dP_x-\beta \frac{2P_xdP_x}{2\sqrt{m^2+P_x^2}}\biggr]dP_ydP_z}{2(E-\beta P_x)}=\frac{dP_x\biggl[1-\beta \frac{P_x}{E}\biggr]dP_ydP_z}{2(E-\beta P_x)}=\frac{dP_xdP_ydP_z}{2E}
\end{align}

\paragraph{3.b.28}\textbf{Dimostra che $d^4p\delta(p^2-m^2)\theta(p_0)=\frac{d^3\vec{p}}{2E}$ e sfruttare questo risultato per semplificare la scrittura dell'elemento infinitesimo dello spazio dei 4-impulsi di N particelle emergenti dopo la collisione di due particelle [oppure dopo il decadimento di una particella]}\\
\\
N.B. La funzione $\Theta(x)$ è la funzione gradino di Heaviside.
Dimostriamo che:
\begin{equation}
d^4P \delta(P^2-m^2)\Theta(P_0)=\frac{d^3p}{2E} \, .
\end{equation}
Abbiamo:
\begin{align*}
d^4P \delta(P^2-m^2)\Theta(P_0) &= d^4P \delta(P_0^2-|\mathbf{p}|^2-m^2)\Theta(P_0) \\
&=d^4P \delta(P_0^2-E^2)\Theta(P_0) \\
&=d^4P \biggl[\frac{\delta (P_0-E)}{|2P_0|_{P_0=E}}+\frac{\delta(P_0+E)}{|2P_0|_{P_0=-E}} \biggr]\Theta(P_0) \\
&=d^4P \frac{\delta (P_0-E)}{2E} \\
&=dP  d^3P \frac{\delta (P_0-E)}{2E} \\
&=\frac{d^3P}{2E} \, ,
\end{align*}
dove abbiamo utilizzato le proprietà della delta:
\begin{equation}
\int f(x)\delta(x-a)dx=f(a) \, , \quad \delta(f(x))=\sum\limits _i \frac{\delta(x-x_{0i})}{|f'(x_{0i})|} \, .
\end{equation}
Quindi l'elemento infinitesimo dello spazio dei 4-impulsi di $N$ particelle emergenti dopo la collisione di due particelle (oppure dopo il decadimento di una particella) può essere riscritto come:
\begin{equation}
dL_p=\frac{d^3p_1}{2E_1}\dots \frac{d^3p_N}{2E_N} \delta^4 \biggl( P_{in}-\sum\limits_{i=1}^N P_i \biggr) \, .
\end{equation}



\paragraph{3.b.29}\textbf{Dimostrare che nel centro di massa l’elemento infinitesimo dello spazio dei 4-impulsi, nel caso di 2 sole particelle nello stato finale, si scrive come $\frac{|p_{CM}|}{4\sqrt{s}}d\Omega_{CM}$.}\\ \\
Consideriamo il caso in cui nello stato finale ci sono due particelle di masse $m_1$ e $m_2$. Siano $\mathbf{p}_1$ e $\mathbf{p}_2$ gli impulsi delle due particelle ed $E_1$ e $E_2$ le loro energie nel sistema del centro di massa. Sia $\sqrt{s}$ l'energia del centro di massa, dove l'impulso totale è nullo ($\mathbf{p}_{tot}=\mathbf{p}_1+\mathbf{p}_2=\mathbf{0} \Rightarrow \mathbf{p}_1=-\mathbf{p}_2, |\mathbf{p}_1|=|\mathbf{p}_2|$). Siano $P_{in}$ e $\sum P_i$ rispettivamente i quadrimpulsi iniziale e finale nel sistema del centro di massa:
\begin{equation*}
P_{in}=
\begin{pmatrix}
\sqrt{s} \\
\mathbf{0} \\
\end{pmatrix} \, ,
\quad
\sum P_i=
\begin{pmatrix}
E_1+E_2 \\
\mathbf{p}_1+\mathbf{p}_2
\end{pmatrix}
=
\begin{pmatrix}
\sqrt{|\mathbf{p}_1|^2+m_1^2}+\sqrt{|\mathbf{p}_2|^2+m_2^2} \\
\mathbf{0}
\end{pmatrix} \,.
\end{equation*}
L'elemento infinitesimo dello spazio dei quadrimpulsi, nel caso di due sole particelle nello stato finale, è:
\begin{align}
dL_p&=\frac{d^3p_1}{2E_1}\frac{d^3p_2}{2E_2} \delta^4 \Bigl( P_{in}-\sum P_i \Bigr)=\frac{d^3p_1}{2E_1}\frac{d^3p_2}{2E_2} \delta(\sqrt{s}-E_1-E_2) \delta^3(\mathbf{0}-\mathbf{p}_1-\mathbf{p}_2) \\
&=\frac{d^3p_1}{4E_1E_2}\delta(\sqrt{s}-E_1-E_2)=\frac{|\mathbf{p}_1|^2dp_1d\Omega_1}{4E_1E_2} \delta(\sqrt{s}-E_1-E_2) \\
&=\frac{|\mathbf{p}_1|^2dp_1d\Omega_1}{4E_1E_2} \delta \Bigl(\sqrt{s}-\sqrt{|\mathbf{p}_1|^2+m_1^2}-\sqrt{|\mathbf{p}_1|^2+m_2^2} \Bigr) \, .
\end{align}
Sia ora $|\mathbf{p}_1|=|\mathbf{p}_2|=p_{cm}$. L'energia del centro di massa diventa:
\begin{equation*}
\sqrt{s}=\sqrt{p_{cm}^2+m_1^2}+\sqrt{p_{cm}^2+m_2^2} \, .
\end{equation*}
Da cui:
\begin{equation*}
p_{cm}=\sqrt{\frac{(s-(m_1+m_2)^2)(s-(m_1-m_2)^2)}{4s}} \, .
\end{equation*}
Usando la proprietà della delta, l'elemento infinitesimo si può allora scrivere come:
\begin{align*}
dL_p&=\frac{|\mathbf{p}_1|^2dp_1d\Omega_1}{4E_1E_2} \frac{\delta(|\mathbf{p}_1|-p_{cm})}{ \bigl| \frac{d}{dp_1}   (\sqrt{s}-\sqrt{|\mathbf{p}_1|^2+m_1^2}-\sqrt{|\mathbf{p}_1|^2+m_2^2}) \bigr|_{|\mathbf{p}_1|=p_{cm}}}\\
&=\frac{|\mathbf{p}_1|^2dp_1d\Omega_1}{4E_1E_2} \frac{\delta(|\mathbf{p}_1|-p_{cm})}{ \bigl| \frac{|\mathbf{p}_1|}{E_1}+\frac{|\mathbf{p}_1|}{E_2} \bigr|_{|\mathbf{p}_1|=p_{cm}}}=\frac{|\mathbf{p}_1|^2dp_1d\Omega_1}{4E_1E_2} \frac{\delta(|\mathbf{p}_1|-p_{cm})}{p_{cm} \frac{E_1+E_2}{E_1E_2}} =\frac{p_{cm}^2d\Omega_1}{4} \frac{1}{p_{cm}\sqrt{s}} =\frac{p_{cm} d\Omega_1}{4 \sqrt{s}} \, .
\end{align*}


\paragraph{3.b.30}\textbf{Dopo avere recuperato nel sito del PDG il rapporto di decadimento dei
decadimenti $\pi^0 \rightarrow \gamma \gamma$, $\pi^- \rightarrow \mu^-\nu_{\mu}$ , calcolarne l'integrale dello spazio delle fasi.}\\
\\
$\pi^0 \rightarrow \gamma \gamma$\\
\url{https://pdg.lbl.gov/2020/listings/rpp2020-list-pi-zero.pdf}\\
$\Gamma_i/\Gamma$=(98.823 $\pm$ 0.034 )\% \\
\\
$\pi^- \rightarrow \mu^-\nu_{\mu}$\\
\url{https://pdg.lbl.gov/2020/listings/rpp2020-list-pi-plus-minus.pdf}\\
$\Gamma_i/\Gamma$=(99.98770 $\pm$ 0.00004 )\% \\
Logo dice che Bati ha sbagliato a fare copia incollare per scrivere questa domanda quindi si considera non esistente.

\paragraph{3.b.31}\textbf{Nel caso di 3 particelle nello stato finale di una reazione, dimostrare che fra il
quadrato della massa invariante di due di esse e l'energia della terza (nel centro di massa) sussiste una relazione lineare.}\\ \\
Nel sistema del centro di massa:
\begin{equation}
    (P_1+P_2)^2=(P_{in}-P_{3})^2=E^2-|\vec{p}|^2+E_3^2-|\vec{p}_3|^2-2EE_3+2\vec{p}\cdot\vec{p}_3=M^2+m_3^2-2ME_3
\end{equation}

\paragraph{3.b.32} \textbf{Come si trasforma una funzione di distribuzione del 3-impulso $f(\vec{p})d^3\vec{p}$ di una particella per una trasformazione di Lorentz?}\\
\\
Ipotizziamo di avere due sistemi di riferimento inerziali, $O$ e $O'$, con $O'$ in moto rispetto ad $O$ con velocità $\mathbf{v}$. La funzione di distribuzione del 3-impulso nel sistema di riferimento $O$ è:
\begin{equation*}
f(\mathbf{p})d^3p=dn \, ,
\end{equation*}
mentre quella nel sistema di riferimento $O'$ è:
\begin{equation*}
f'(\mathbf{p'})d^3p'=dn \, ,
\end{equation*}
dato che il numero di particelle con impulso compreso tra $\mathbf{p}$ e $\mathbf{p}+d\mathbf{p}$ è lo stesso nei due sistemi di riferimento. Uguagliando le due espressioni otteniamo:
\begin{equation*}
f(\mathbf{p})d^3p=f'(\mathbf{p'})d^3p' \, .
\end{equation*}
Moltiplichiamo e dividiamo il membro a sinistra per $E$ e il membro a destra per $E'$:
\begin{equation*}
f(\mathbf{p})\frac{d^3p}{E}E=f'(\mathbf{p'})\frac{d^3p'}{E'}E' \, .
\end{equation*}
Sfruttiamo ora il fatto che $\frac{d^3p}{E}$ è un invariante relativistico (ed è quindi uguale a $\frac{d^3p'}{E'}$):
\begin{equation*}
f(\mathbf{p})E=f'(\mathbf{p'})E'
\end{equation*}
\begin{equation}
 f'(\mathbf{p'})=\frac{E}{E'}f(\mathbf{p})
\end{equation}
Abbiamo così trovato come trasforma la funzione di distribuzione.

\paragraph{3.b.33}\textbf{Come si trasforma una funzione di distribuzione nello spazio delle fasi $f(\vec{p},\vec{r})d^3pd^3r$ di una particella per una trasformazione di Lorentz?}\\ \\
Si supponga che nel sistema di riferimento S il corpo si muova lungo la x (sempre possibile farlo) con velocità v, si faccia quindi una trasformazione di Lorentz lungo l'asse x con velocità v portandosi nel sistema S'.\\
Analogamente al caso precedente:
\begin{equation}
    f(\mathbf{p},\vec{r})d^3pd^3r=dn=f'(\mathbf{p'},\vec{r}')d^3p'd^3r'
\end{equation}

\begin{equation}%3.179 
    d^3r'=dx'dy'dz'=\frac{d^3r}{\gamma}
\end{equation}

\begin{equation}
    d^3\vec{r}'=\frac{d^3\vec{r}}{\gamma}\frac{mc^2}{mc^2}=\frac{d^3rE}{E'}
\end{equation}
Da cui si trova che $d^3 r' E' = d^3 r E$, ossia il prodotto tra energia ed elemento spaziale infinitesimo è un invariante di Lorentz.
\\
$d^3\vec{p}/E$ è inoltre un invariante relativistico (si veda lo spazio delle fasi) perciò in ultima istanza si trova:

\begin{equation}
f(\mathbf{p},\vec{r})d^3pd^3r \frac{E}{E}= f(\mathbf{p},\vec{r}) \frac{d^3\vec{p}'}{E'} d^3r' E'\Rightarrow     f(\mathbf{p},\vec{r})=f'(\mathbf{p'},\vec{r}')
\end{equation}

\paragraph{3.b.34}\textbf{Dimostrare che se la probabilità di decadimento di una particella per unità di tempo non dipende dal tempo, la probabilità di trovare la particella non decaduta al tempo t segue una legge esponenziale}\\
\\
Sia $P(t)$ la probabilità che la particella esista ancora al tempo $t$, e $\frac{1}{\tau}$ la probabilità di decadimento nell'unità di tempo. Al tempo $t+dt$ abbiamo:
\begin{align*}
P(t+dt)&=P(t)-P(t)\frac{dt}{\tau} \\
P(t+dt)-P(t)&=-P(t)\frac{dt}{\tau} \\
\frac{P(t+dt)-P(t)}{dt}&=-\frac{P(t)}{\tau} \\
\frac{dP(t)}{dt}&=-\frac{P(t)}{\tau} \, . 
\end{align*}
Da cui:
\begin{equation}
P(t)=e^{-\frac{t}{\tau}} \, .
\end{equation}

\paragraph{3.b.35}\textbf{Studiando l'effetto Mossbauer si incontra uno stato risonante del 57Fe che ha una semi-vita di 97.7 ns e che decade in un fotone di energia 14.4 keV. Dimostrare che non può essere uno stato risonante descrivibile dal modello dell'elettrone legato elasticamente}\\\\
Supponiamo sia valida l'approssimazione dell'elettrone legato elasticamente.\\
Dato che lo stato è risonante possiamo stimare la frequenza di risonanza del sistema come:
\begin{equation*}
	\omega_0=\frac{E_\gamma}{\hbar}\simeq\frac{14.4\times 10^3eV}{6.58\times 10^{-16}eV}=2.18 \times 10^{19} s^{-1}
\end{equation*}
I valori caratteristici per i parametri sono:
\begin{equation}
	\Gamma'\simeq 10^{10} s^{-1} \qquad \Gamma(\omega_0)=\tau\omega_0^2\simeq 2.94 \times 10^{15} \ \ \ \text{con} \ \ \tau=6.2 \times 10^{-24} s
\end{equation}
Di conseguenza il valore della larghezza totale è:
\[ \Gamma_{tot} \simeq \Gamma(\omega_0)
\]
Se il sistema rispettasse l'approssimazione di elettrone legato elasticamente, la larghezza totale dello stato risonante corrisponderebbe all'inverso della vita media. A sua volta $t_{1/2}=ln(2)\tau$
Quindi a partire da $\tau=97.7 ns$, si trova:
\begin{equation}
	\Gamma+\Gamma'=\frac{1}{\tau}=1.02 \times 10^{7} s^{-1}
\end{equation}
Tali valori discordanti mostrano come l'ipotesi di partenza non sia verificata.

\paragraph{3.b.36}\textbf{Recuperando nel sito \url{https://pdg.lbl.gov/2020/hadronic-xsections/hadron.html} le
tabelle interattive delle sezioni d'urto, calcolare la distanza media che un $\pi^-$ o un $\pi^+$ percorrono in idrogeno liquido prima di effettuare una interazione
forte. Considerare impulsi di 1GeV/c, 10 GeV/c, 100Gev/c.}\\
\\
$\Rightarrow \pi^{\pm}p$ Total collision cross section
\begin{figure}[H]
    \centering
    \includegraphics[scale=0.3]{3.b.36.PNG}
\end{figure}
\begin{figure}[H]
    \centering
    \includegraphics[scale=0.45]{3.b.36.2.PNG}
\end{figure}
Le sezioni d'urto si possono stimare dal grafico in figura oppure si possono cercare direttamente dai dati(seconda colonna:energia della particella GeV/c, quinta colonna: valore sezione d'urto mbarn).\\
Dai dati del $\pi^-$:
\begin{itemize}
    \item $\sigma$(1 Gev/c)= 61.2 mb
    \item $\sigma$(10 Gev/c)= 26.5 mb
    \item $\sigma$(100 Gev/c)= 23.96 mb
\end{itemize}
N.B.:1 barn =$10^{-24} cm^2$
La probabilità di interagire per interazione forte su nuclei è:
\begin{equation*}
    P=\sigma_{tot} n_s= n_H \Delta x
    \sigma_{tot}=\frac{\rho_H N_{av}}{M_H} \Delta x \sigma_{tot}\equiv \frac{\Delta x}{l}
\end{equation*}
dove 
\begin{equation*}
    l=\frac{M_H}{\rho_H N_{av} \sigma_{tot}}
\end{equation*}
è la "lunghezza di estinzione" con $N_{av}$= 6.022 $\cdot 10^{23} mol^{-1}$.\\
Nel nostro caso  $\rho_H$=$70 g/l$=$0.07 g/cm^3$, $M_H0$=1.008 $g/mol$.


\paragraph{3.b.37}\textbf{Pioni neutri, di energia E nel sistema del laboratorio, decadono in due fotoni. La distribuzione è isotropa nel centro di massa. Si calcoli la distribuzione dell’energia di uno dei due fotoni nel laboratorio.}
Si consideri il decadimento:
\begin{equation}
    \pi^0\to \gamma\gamma
\end{equation}
si ponga c=1.\\
Nel sistema di riferimento del centro di massa si ha:
\begin{equation}
    \begin{cases}
    m_\pi=E_1^{CM}+E_2^{CM}\\
    E_1^{CM}=|\vec{p}_1^{CM}|\\
    E_2^{CM}=|\vec{p}_2^{CM}|\\
    |\vec{p}_1^{CM}|=|\vec{p}_2^{CM}|
    \end{cases}\Longrightarrow E_1^{CM}=\frac{m_\pi}{2}
\end{equation}
Per ottenere la distribuzione dell'energia nel Lab occorre effettuare una trasformazione di Lorentz per riportarsi nel sistema del laboratorio.\\
Si supponga quindi quindi che originariamente la velocità del pione fosse $v_{\pi}=v\hat{x}$ e sia $\theta$ l'angolo fra $|\vec{p}_{{CM}_1}|$ e l'asse x nel centro di massa.\\
Si otterrà dunque:
\begin{equation}
\begin{cases}
     E_1^l=\gamma(E^{CM}+\beta p_{1x}^{CM})\\
     p_{1x}^l=\gamma(p_{1x}^{CM}+\beta E^{CM})\\
     p_{1y}^l=p_{1y}^{CM}\\
     p_{1z}^l=p_{1z}^{CM}
\end{cases}\Longrightarrow
     E_1^l=\gamma\biggl(\frac{m_\pi}{2}+\beta |p^{CM}_1|cos(\theta^c)\biggr)=\gamma\frac{m_\pi}{2}(1+\beta cos(\theta^c))
\end{equation}
Dato che la distribuzione è isotropa nel centro di massa, l'energia risulta uniformemente distribuita tra i valori:
\begin{equation}
    \frac{m_\pi}{2(1+\beta)}\le E\le \frac{m_\pi}{2(1-\beta)}
\end{equation}
\textbf{APPENDIX:} se volessi riscrivere in funzione dell'angolo del laboratorio.
Dato che per isotropia si ha:
\begin{equation}
    p_{1x}^l=\gamma p_{1x}^{CM}(1+\beta cos(\theta^{CM}) \qquad p_{1x}^{CM}=\gamma p_{1x}^{l}(1+\beta cos(\theta^{l}) \qquad
\end{equation}
Si ricava:
\begin{equation}
    1+\beta cos(\theta^{CM})=\frac{1}{\gamma^2(1-\beta cos(\theta^l)}
\end{equation}
Da cui:
\begin{equation}
    E^l_1=\frac{m_\pi}{2(\gamma(1-\beta cos(\theta^l))}
\end{equation}


\paragraph{3.b.38}\textbf{Calcolare la funzione di distribuzione in energia nel sistema del laboratorio di un fascio di neutrini o di muoni prodotto nel decadimento di pioni carichi di energia
14 GeV.}\\
\\
Consideriamo per esempio il decadimento:
\begin{equation}
\pi^+ \rightarrow \mu^+ \nu_{\mu} \, ,
\end{equation}
dove un pione carico che si muove lungo l'asse $x$ del sistema del laboratorio decade in un antimuone e in un neutrino muonico. Nel sistema del centro di massa abbiamo:
\begin{equation}
\begin{pmatrix}
m_\pi \\
0 \\
0 \\
0 \\
\end{pmatrix}
=
\begin{pmatrix}
\sqrt{m_{\mu}^2+|\mathbf{p}_{cm}|^2} \\
|\mathbf{p}_{cm}|\cos\theta_{cm} \\
|\mathbf{p}_{cm}|\sin\theta_{cm} \\
0 \\
\end{pmatrix}
+
\begin{pmatrix}
|\mathbf{p}_{cm}| \\
-|\mathbf{p}_{cm}|\cos\theta_{cm} \\
-|\mathbf{p}_{cm}|\sin\theta_{cm} \\
0 \\
\end{pmatrix} \, ,
\end{equation}
dove abbiamo posto $|\mathbf{p}_{cm}|=|\mathbf{p}_\mu|=|\mathbf{p}_\nu|$, dato che l'antimuone e il neutrino hanno impulsi uguali ed opposti nel centro di massa, e $m_\nu=0$. Per trovare la distribuzione dell'energia dell'antimuone o del neutrino nel laboratorio basta effettuare una trasformazione di Lorentz dal sistema del centro di massa al laboratorio della componente temporale del quadrimpulso. Abbiamo così:
\begin{equation}
E_\mu^{lab}=\gamma \Bigl(\sqrt{m_{\mu}^2+|\mathbf{p}_{cm}|^2}+\beta|\mathbf{p}_{cm}|\cos\theta_{cm} \Bigr) \, ,
\end{equation}
\begin{equation}
E_\nu^{lab}=\gamma(|\mathbf{p}_{cm}|-\beta|\mathbf{p}_{cm}|\cos\theta_{cm})=\gamma|\mathbf{p}_{cm}|(1-\beta\cos\theta_{cm}) \, .
\end{equation}
Da cui:
\begin{equation}
\gamma \Bigl(\sqrt{m_{\mu}^2+|\mathbf{p}_{cm}|^2}-\beta|\mathbf{p}_{cm}| \Bigr) \leq E_\mu^{lab} \leq \gamma \Bigl(\sqrt{m_{\mu}^2+|\mathbf{p}_{cm}|^2}+\beta|\mathbf{p}_{cm}| \Bigr) \, ,
\end{equation}
\begin{equation}
\gamma|\mathbf{p}_{cm}|(1-\beta) \leq E_\nu^{lab} \leq \gamma|\mathbf{p}_{cm}|(1+\beta) \, .
\end{equation}
La funzione di distribuzione in angolo dell'antimuone nel laboratorio è data da:
\begin{align}
\frac{d\Gamma}{dE_\mu^{lab}}&=\frac{d\Gamma}{d\Omega_{cm}} \biggl \lvert \frac{d\Omega_{cm}}{dE_\mu^{lab}} \biggr \rvert \\
&=\frac{\Gamma}{4\pi} \biggl \lvert \frac{2\pi d \cos \theta_{cm}}{dE_\mu^{lab}} \biggr \rvert \\
&=\frac{\Gamma}{2} \biggl \lvert \frac{d \cos \theta_{cm}}{dE_\mu^{lab}} \biggr \rvert \\
&=\frac{\Gamma}{2} \biggl \lvert \frac{d\cos\theta_{cm}}{\gamma\beta|\mathbf{p}_{cm}| d\cos\theta_{cm}} \biggr \rvert \\
&=\frac{\Gamma}{2\gamma\beta|\mathbf{p}_{cm}|} \, .
\end{align} 
Lo stesso vale per il neutrino:
\begin{align}
\frac{d\Gamma}{dE_\nu^{lab}}&=\frac{d\Gamma}{d\Omega_{cm}} \biggl \lvert \frac{d\Omega_{cm}}{dE_\nu^{lab}} \biggr \rvert \\
&=\frac{\Gamma}{4\pi} \biggl \lvert \frac{2\pi d\cos\theta_{cm}}{dE_\nu^{lab}} \biggr \rvert \\
&=\frac{\Gamma}{2} \biggl \lvert \frac{d\cos\theta_{cm}}{dE_\nu^{lab}} \biggr \rvert \\
&=\frac{\Gamma}{2} \biggl \lvert \frac{d\cos\theta_{cm}}{-\gamma |\mathbf{p}_{cm}|\beta d\cos\theta_{cm}} \biggr \rvert \\
&=\frac{\Gamma}{2\gamma \beta |\mathbf{p}_{cm}| } \, .
\end{align} 
Dall'equazione:
\begin{equation}
m_\pi=\sqrt{m_{\mu}^2+|\mathbf{p}_{cm}|^2}+|\mathbf{p}_{cm}|
\end{equation}
ricaviamoci $|\mathbf{p}_{cm}|$:
\begin{align}
m_\pi-|\mathbf{p}_{cm}|&=\sqrt{m_{\mu}^2+|\mathbf{p}_{cm}|^2} \\
m_\pi^2-2m_\pi|\mathbf{p}_{cm}|+|\mathbf{p}_{cm}|^2&=m_{\mu}^2+|\mathbf{p}_{cm}|^2 \\
m_\pi^2-2m_\pi|\mathbf{p}_{cm}|&=m_{\mu}^2 \\
|\mathbf{p}_{cm}|&=\frac{m_\pi^2-m_{\mu}^2}{2m_\pi} \, ,
\end{align}
che assume il valore:
\begin{equation}
|\mathbf{p}_{cm}|=\frac{139.57^2-105.65^2}{2\cdot 139.57}\simeq 29.79 \quad \text{MeV} \, .
\end{equation}
Utilizzando le relazioni:
\begin{equation}
\gamma=\frac{E_\pi}{m_\pi}
\end{equation}
e
\begin{equation}
\beta=\sqrt{1-\frac{1}{\gamma^2}} \, ,
\end{equation}
per il valori richiesti abbiamo: $E_\pi=14000$ MeV $\Rightarrow \gamma\simeq 100.30 \Rightarrow \beta \simeq 0.99$
\begin{equation}
8051.83 \quad \text{MeV} \leq E_\mu^{lab} \leq 13967.9 \quad \text{MeV}
\end{equation}
\begin{equation}
29.87 \quad \text{MeV} \leq E_\nu^{lab} \leq 5945.99 \quad \text{MeV} 
\end{equation}
\begin{equation}
\frac{d\Gamma}{dE_\mu^{lab}}=\frac{d\Gamma}{dE_\nu^{lab}} \simeq \frac{\Gamma}{5916.11}
\end{equation}





%------------------------------------
\newpage
\section{Interazione fra radiazione (fotoni, particelle cariche o neutre) e la
materia stabile}

\epigraph{There was a point to this story, but it has temporarily escaped the chronicler's mind}{}
\textit{Domande a}\rule{13.7 cm}{0.4pt}
\\
\paragraph{4.a.1}\textbf{Qual è la relazione fra lo spessore di un materiale, espresso in $cm$, e lo spessore espresso in $g/cm^2$ ?}\\
\\
La relazione fra lo spessore di un materiale, espresso in cm, e lo spessore espresso in g/cm\ap{2} è:
\begin{equation}
dt=\rho dx \, ,
\end{equation}
dove $dt$ è detta "mass thickness" ed è espressa in $g/cm^{2}$, mentre $\rho$ rappresenta la densità del materiale, espressa in $g/cm^{3}$, e $dx$ il suo spessore, espresso in cm.\\
I due spessori si ottengono l’uno dall'altro moltiplicando o dividendo per la densità del mezzo. Lo spessore in $g/cm^2$ è comodo perché spesso si trovano espressioni in cui compare il termine $(1/\rho)(dE/dx)$, per essere espressioni generali che non dipendono dallo stato in cui si trova il materiale.

\paragraph{4.a.2}\textbf{Spiegare qualitativamente l’effetto fotoelettrico e lo scattering Compton, indicandone le differenti caratteristiche}\\
\\
\underline{Effetto fotoelettrico}\\
\begin{equation}
    \gamma + A \rightarrow e^- +A^+
\end{equation}
Il fotone viene assorbito completamente e ionizza l'atomo. L'energia di soglia è pari ad $I$ (energia ionizzazione dell'atomo). L'elettrone ha energia $E_{\gamma}-I$ ($E_{\gamma}$ energia del fotone); esso viene emesso preferenzialmente nella direzione di polarizzazione del fotone incidente,quindi perpendicolarmente alla direzione del fotone incidente.\\
\\
\underline{Scattering Compton}
\begin{equation}
  \gamma +A \rightarrow\gamma +e^- +A^+  
\end{equation}
Se fotone ha energia $>> I$(energia ionizzazione dell'atomo) ed urta contro un elettrone atomico, quest'ultimo può essere considerato libero. Con eccellente approssimazione la parte restante dell'atomo ha un ruolo di "spettatore".
\begin{figure}[H]
    \centering
    \includegraphics[scale=0.2]{compton.png}

\end{figure}
\paragraph{4.a.3}\textbf{Dare l'espressione ed il valore numerico della lunghezza d'onda Compton}\\
\\
La lunghezza d'onda Compton è una proprietà quanto-meccanica di una particella ed è data dall'espressione:
\begin{equation}
  \lambda_C=\frac{h}{m_0 c}  
\end{equation}
il suo valore approssimato, nel caso che la particella $m_0$ sia un elettrone, è $2.43 \cdot 10^{-12} $m. \\

\paragraph{4.a.4}\textbf{Spiegare qualitativamente lo scattering Rayleigh}\\
\begin{equation}
    \gamma + A \rightarrow \gamma +A 
\end{equation}
Urto elastico "coerente" sugli elettroni atomici (interazione elastica di fotoni sull'atomo).\\
Gli elettroni dell'atomo cominciano ad oscillare e quindi irraggiano come dipoli. Se le dimensioni atomiche sono piccole rispetto alla lunghezza d’onda, ci aspettiamo che le oscillazioni siano coerenti, dunque in questo limite si prevede $\sigma \propto Z^2$. Nel limite opposto, ciascun elettrone può essere considerato indipendente dagli
altri, quindi si prevede $\sigma \propto Z$.\\
Per le energia maggiori di 1KeV decresce come $1/E^2$.

\paragraph{4.a.5}\textbf{Spiegare qualitativamente il fenomeno della creazione di coppie $e^+$ $e^-$ da parte di un raggio gamma che incide su un atomo.}\\
\begin{equation}
    \gamma+N\rightarrow e^- +e^+ +N
\end{equation}
Se fotone ha energia $>2m_e$=1.02 MeV si può produrre una coppia elettrone-positrone nel campo elettromagnetico del nucleo, che assorbe energia (in quantità trascurabile) e impulso in modo da rendere possibile il processo.
\warning NON confondere con la creazione di coppie su elettrone fermo che avrebbe energia maggiore di $4m_e$.

\paragraph{4.a.6}\textbf{Descrivere qualitativamente l’effetto Cherenkov e dimostrare tramite il principio di Huygens che la radiazione Cherenkov è emessa ad un solo angolo.}\\
\\
L'effetto Cherenkov è l'emissione di radiazione da parte di una carica in moto superluminale in un mezzo. La radiazione Cherenkov si ha quando una particella carica si muove in un mezzo con indice di rifrazione reale $n$ tale che $v>\frac{c}{n}$, cioè la sorgente, ovvero la particella, si muove più velocemente della radiazione emessa. In un mezzo infatti la velocità della luce è $\frac{c}{n}$. L'angolo di emissione della radiazione è dato da:
\begin{equation}
\cos\theta_c=\frac{1}{\beta n}
\end{equation}
Fisicamente il passaggio di una particella carica causa la creazione di dipoli temporanei nel mezzo. Tali dipoli ritornando nella loro configurazione stabile, emettono radiazione. Tuttavia se la particella si muove in modo subluminale i dipoli si rilassano elasticamente, e questo non produce un'onda d'urto coerente. Se invece la particella si muove in modo sufficientemente veloce tale che la sua velocità sia maggiore della radiazione che essa produce, i fronti d'onda em possono sommarsi tra loro costruttivamente e creare un fronte d'onda coerente che si propaga nel mezzo.\\
\\
DIMOSTRAZIONE ANGOLO UNICO (Dagli appunti Bati 8.1.2)\\
\begin{figure}[H]
    \centering
    \includegraphics[scale=0.4]{4.a.6.1.PNG}
\end{figure}
La particella carica si muove con velocità $V=\beta c$.\\
Dalla posizione in cui si trova la particella al tempo t' si propaga una perturbazione elettromagnetica con un fronte d'onda sferico, che dopo un tempo $\Delta t$ ha un raggio $R=\frac{c}{n}\Delta t$.\\
Da tutte le posizioni occupate dalla particella fra il tempo $t'$ ed il tempo $t=t'+\Delta t$ si propagano perturbazioni sferiche analoghe, che al tempo t avranno un raggio intermedio tra $R=\frac{c}{n} \Delta t$ ed $R=0$ (per quella al tempo t).\\
Di conseguenza le perturbazioni elettromagnetica si sviluppa su un fronte che è l'inviluppo di tutte le superfici sferiche precedenti. Tale superficie è una superficie conica con un angolo di apertura il cui seno vale $1/\beta n$. La perturbazione, che a questo punto ha tutte le caratteristiche di una onda d'urto, si propaga perpendicolarmente al fronte d'onda e quindi con un angolo tale che $cos\theta=\frac{1}{\beta n}$.\\

\paragraph{4.a.7}\textbf{Descrivere la situazione in cui la legge, inerente la radiazione Cherenkov,
\[
\frac{d^2 N_{\gamma}}{dE_{\gamma}dx}= z^2 \frac{\alpha}{\hbar c}sin^2\theta_c
\]
è applicabile e spiegare il significato e l'unità di misura di ogni grandezza fisica ivi indicata}\\
\\
I termini che compaiono nella legge di Frank-Tamm sono: il numero di fotoni emessi $N_\gamma$ ($[]=$ fotoni), l'energia dei fotoni emessi $E_\gamma$ ($[]=$eV), la distanza $x$ percorsa dalla sorgente ($[]=$cm), il numero di protoni $z$ della sorgente ($[]=$ u.a.), la costante di struttura fine $\alpha$, la costante di Planck ridotta $\hbar=\frac{h}{2\pi}$, la velocità della luce nel vuoto $c$ ($\frac{\alpha}{\hbar c}=$ 370 fotoni/eV cm ), l'angolo di emissione della radiazione $\theta_c$.\\
La legge è applicabile quando una particella di carica $ze$ attraversa uno spessore infinitesimo $dx$ ed emette fotoni Cherenkov di energia $E_{\gamma}$.

\paragraph{4.a.8}\textbf{Descrivere la situazione in cui la legge, inerente la radiazione Cherenkov,
\[
N_{\gamma}=z^2\frac{\alpha}{\hbar c}L\int_{E_1}^{E_2}\left[ 1-\frac{1}{\beta^2 \epsilon_r(E)} \right] P_{det}(E) dE
\]
è applicabile e spiegare il significato e l'unità di misura di ogni grandezza fisica ivi indicata.}\\
\\
I termini che compaiono nella legge sono: il numero di protoni $z$ della sorgente ($[]=$ u.a.),, la costante di struttura fine $\alpha$, la costante di Planck ridotta $\hbar=\frac{h}{2\pi}$, la velocità della luce nel vuoto $c$, la lunghezza totale $L$ percorsa dalla sorgente ($[]=$ cm),, la $\beta=\frac{v}{c}$ di Lorentz ($[]=$ u.a.), la permittività elettrica $\epsilon_r$ del materiale attraversato, la probabilità $P_{det}(E)$ che il rivelatore veda un fotone con energia $E$.\\
La legge è applicabile se è presente un rivelatore di fotoni Cherenkov, esprime infatti il numero di fotoni osservati da esso.\\
\warning Solitamente un PD (Fotomoltiplicatore) riesce a rivelare fotoni nel visibile o nel vicino ultravioletto, con una banda passante che è dell'ordine dell'elettron-Volt, e con efficienza massima $\sim$ 30 \%.


\paragraph{4.a.9}\textbf{Descrivere qualitativamente le cause e gli effetti del fenomeno della radiazione di frenamento da parte di una particella carica nella materia.}\\
\\
Una particella carica (di massa M e carica q=ze) che attraversa un materiale subisce l'interazione coulombiana con i nuclei di carica Ze. Questa interazione coulombiana, provoca una accelerazione della particella che quindi irraggerà ("Radiazione di frenamento"-"Bremsstrahlung"). Poiché la forza di Coulomb dipende dalla carica elettrica, l'accelerazione è inversamente proporzionale alla massa M: l'energia irraggiata è inversamente proporzionale al quadrato della massa M. La radiazione di frenamento è quindi un effetto molto importante per elettroni (e positroni) purché la loro energia sia superiore ad una certa "energia critica" $E_C$ il cui valore dipende dal materiale attraversato, è invece trascurabile per le altre particelle.\\
La radiazione di frenamento avviene in modo indipendente su ogni nucleo: la radiazione viene pertanto emessa in numerosi processi incoerenti.\\
Nel sistema del LAB la radiazione sarà emessa:
\begin{itemize}
    \item proporzionale al quadrato dell'angolo fra accelerazione e direzione di osservazione nel caso non relativistico
    \item in direzione in avanti nel caso ultrarelativistico con angolo $1/\gamma$
\end{itemize}

\paragraph{4.a.10}\textbf{Spiegare perchè la perdita di energia per irraggiamento è significativa per
elettroni e non per le altre particelle cariche. 
}\\ \\

La perdita di energia per irraggiamento è una legge che deriva dalla bremsstrahlung (non relativistica) e ha una dipendenza del tipo $1/M^2$, perciò è rilevante solo per elettroni e positroni (leptoni) che hanno massa minore di protoni e antiprotoni o altri barioni e adroni.

\paragraph{4.a.11}\textbf{Descrivere la situazione in cui la legge 
\begin{equation}
    \frac{dI_{\omega}}{d\Omega}=\frac{q^2}{4\pi^2c}\left| \int \frac{\hat{n}\wedge \left[(\hat{n}-\vec{\beta})\wedge \dot{\vec{\beta}} \right]}{(1-\hat{n}\cdot \vec{\beta})^2} e^{i\omega \left( t'-\frac{\vec{r}'\cdot \hat{n}}{c} \right)}dt'\right|^2
\end{equation}
è
applicabile e spiegare il significato e l'unità di misura di ogni grandezza fisica ivi
indicata.}\\ \\

Questa legge è applicabile per ogni tipo di irraggiamento da particelle cariche in campo di radiazione.
\\

Rappresenta l'energia irraggiata per unità di angolo solido e per unità di frequenza da una particella carica $\Bigl( \frac{dI_\omega}{d\Omega}=\frac{d^2I}{d\omega d\Omega} \Bigr)$. Al suo interno troviamo la carica $q$ della particella ($[q]=\text{sC}$), la velocità della luce nel vuoto $c$ ($[c]=\text{cm/s}$), il versore $\mathbf{\hat{n}}$ che indica la direzione origine-punto di osservazione ($[\mathbf{\hat{n}}]=\text{adimensionale}$), la $\bm{\beta}=\frac{\mathbf{v}}{c}$ di Lorentz ($[\bm{\beta}]=\text{adimensionale}$), $\bm{\dot{\beta}}=\frac{\mathbf{a}}{c}$ ($[\bm{\dot{\beta}}]=\text{s}^{-1}$), la frequenza angolare $\omega$ ($[\omega]=\text{s}^{-1}$), il tempo $t'$ impiegato dal segnale che parte dalla particella per arrivare al punto di osservazione ($[t']=\text{s}$), il vettore $\mathbf{r'}$ che unisce l'origine alla particella ($[\mathbf{r'}]=\text{cm}$).


\paragraph{4.a.12}\textbf{Descrivere la situazione in cui la legge
\begin{equation}
    I_{\omega}(b)=\left\{ \begin{array}{lr}
         \frac{8z^4 Z^2 \alpha \hbar c^2}{3\pi}\left( \frac{m_e}{M} \right)^2 \frac{r_e^2}{V^2}\frac{1}{b^2} \ \ \ \ \ \text{per} \ \ \ \omega << V/b \\
         0 \ \ \ \ \ \text{per} \ \ \ \omega > V/b
    \end{array}\right.
\end{equation}
è
applicabile e spiegare il significato e l'unità di misura di ogni grandezza fisica ivi
indicata.
}\\ \\
Questa legge è applicabile nel caso non relativistico.
\\
Rappresenta l'energia irraggiata per unità di frequenza da una particella carica nella materia. Al suo interno troviamo la carica della particella $ze$ ($[ze]=\text{sC}$), la velocità della luce nel vuoto $c$ ($[c]=\text{cm/s}$), la carica del nucleo atomico $Ze$ che si trova nel materiale attraversato ($[Ze]=\text{sC}$), la massa $M$ della particella ($[M]=\text{g}$), la sua velocità $V$ ($[V]=\text{cm/s}$), il parametro di impatto $b$ ($[b]=\text{cm}$), la frequenza angolare $\omega$ della radiazione emessa ($[\omega]=\text{s}^{-1}$), $[\hbar]\simeq 6.5 \times 10^{-16} eV\cdot s$. 
\\
\'E solo un modello rozzo:
  \begin{figure}[H]
    \begin{center}
    \includegraphics[width=0.6\textwidth]{brehmstralung non rel.JPG}
    \end{center}
    \label{brem}
    \end{figure}

\paragraph{4.a.13}\textbf{Definire la sezione d'urto di irraggiamento e spiegare il suo significato fisico.}\\ \\
Si definisce la \textit{sezione d'urto di irraggiamento} come:
\begin{equation}
    \chi_{\omega}\equiv \int_{b_{min}}^{b_{max}}2\pi I_{\omega}b\ db
\end{equation}
Ma il Landau la chiama \textit{Irraggiamento efficace per intervallo di frequenza}, per cui si può anche scrivere come
\begin{equation}
    E_{irr}=n\Delta x \int \chi_\omega d\omega =n_s \int \chi_\omega d\omega \ \ \Rightarrow \ \ \frac{dE_{irr}}{d\omega}=n_s\chi_\omega
\end{equation}

da cui si vede che il suo significato fisico è che si tratta della grandezza che, moltiplicata per il numero di nuclei per unità di superficie, fornisce l'energia irraggiata per intervallo di frequenza.


\paragraph{4.a.14}\textbf{Descrivere la situazione in cui la legge
\begin{equation}
    \chi_{\omega}=\frac{16 z^4 Z^2 \alpha \hbar c^2}{3}\left( \frac{m_e}{M} \right)^2 \frac{r_e^2}{V^2}ln\left[ \frac{MV^2}{\hbar \omega} \right]
\end{equation}
è applicabile
e spiegare il significato e l'unità di misura di ogni grandezza fisica ivi indicata.
}\\ \\%ti osservo
Si applica per elettroni non relativistici (risultato derivante dalla Bhremstralung non relativistica).
\\
Rappresenta la sezione d'urto d'irraggiamento. Al suo interno troviamo la carica $ze$ della particella ($[ze]=\text{C}$), la permittività elettrica del vuoto $\epsilon_0$ ($[\epsilon_0]=\text{F/m}$), la velocità della luce nel vuoto $c$ ($[c]=\text{m/s}$), la carica del nucleo atomico $Ze$ che si trova nel materiale attraversato ($[Ze]=\text{C}$), la massa $M$ della particella ($[M]=\text{kg}$), la $\beta=\frac{v}{c}$ di Lorentz ($[\beta]=\text{adimensionale}$), la velocità $V$ della particella ($[V]=\text{m/s}$), la costante di Planck ridotta $\hbar=\frac{h}{2\pi}$ ($[\hbar]=\text{Js}$), la frequenza angolare $\omega$ della radiazione emessa ($[\omega]=\text{s}^{-1}$).

\paragraph{4.a.15}\textbf{Descrivere la situazione in cui la legge 
\begin{equation}
    \frac{dE_{irr}}{dx}=n_{nuclei}\frac{16}{3}z^4Z^2\alpha \left( \frac{m_e}{M} \right)^2 r_e^2 ln\left[ \frac{192}{Z^{1/3}}\frac{M}{m_e} \right]E
\end{equation}
è
applicabile e spiegare il significato e l'unità di misura di ogni grandezza fisica ivi
indicata.
}\\ \\
Si applica nel caso di irraggiamento da un elettrone relativistico in caso di screening attivo.\\
Rappresenta l'energia irraggiata per unità di lunghezza. Al suo interno troviamo la concentrazione dei nuclei atomici $n_{nuclei}$ del materiale ($[n_{nuclei}]=\text{1/cm}^3$), il numero di protoni $z$ della particella ($[z]=\text{adimensionale}$), il numero di protoni $Z$ del nucleo atomico che si trova nel materiale ($[Z]=\text{adimensionale}$), la costante di struttura fine $\alpha$ ($[\alpha]=\text{adimensionale}$), la massa dell'elettrone $m_e$ ($[m_e]=\text{g}$), la massa $M$ della particella ($[M]=\text{g}$), il raggio classico dell'elettrone $r_e$ ($[r_e]=\text{cm}$), l'energia $E$ della particella ($[E]=\text{erg}$).
\paragraph{4.a.16}\textbf{Descrivere la situazione in cui la legge
\begin{equation}
    \chi_{\omega}=\hbar \omega \frac{d\sigma_{\gamma}}{d\omega}
\end{equation}
è applicabile e spiegare il
significato e l'unità di misura di ogni grandezza fisica ivi indicata.
}\\ \\

Questa legge è applicabile per elettroni che attraversano un tratto di materiale di un certo spessore $\Delta x$ tale che $n_s=n \Delta x$.
\\
Rappresenta la sezione d'urto di irraggiamento in funzione della distribuzione del numero di fotoni per frequenza.
\\
$\sigma_{\gamma}=N_{\gamma}/n_s$ è il rapporto tra il numero di fotoni e la concentrazione dei nuclei per unità di superficie [$\sigma_{\gamma}=cm^{2}$], $\hbar$ è la costante di Planck ridotta [$\hbar=J\cdot s$], $\omega$ è la frequenza angolare [$\omega=s^{-1}$].
\\
Come attesto [$\chi_{\omega}=\frac{J}{Hz}cm^2$].


\paragraph{4.a.17}\textbf{Dare la definizione di lunghezza di radiazione.}\\ \\
L'energia per Bremsstrahlung decade con una legge esponenziale del tipo
\begin{equation}
    E=E_0 e^{-x/x_0}
\end{equation}
e la costante $x_0$, caratteristica del materiale, è la \textit{lunghezza di radiazione}.
\\
Nel modello semplice è definita come:

\begin{equation}
    x_0=\frac{1}{n\frac{16}{3}Z^2\alpha r_e^2 \left( ln\frac{192}{Z^{1/3}}\right)}
\end{equation}

\paragraph{4.a.18}\textbf{Dare la definizione di "Energia critica".
}\\ \\
Valore dell'energia, tipico di un materiale, al di sopra del quale la radiazione di frenamento risulta un effetto importante per elettroni e positroni.
\begin{equation}
    E_C\sim m_e c^2 \frac{192}{Z^{1/3}}\sim \frac{98 \text{MeV}}{Z^{1/3}}
\end{equation}

\paragraph{4.a.19}\textbf{Descrivere la situazione in cui la legge 
\begin{equation}
    E=E_0 e^{-x/x_0}
\end{equation}
è applicabile e spiegare il
significato e l'unità di misura di ogni grandezza ivi indicata.}\\ \\

Questa legge è applicabile quando $E>E_c$ e significa che un elettrone che penetra in un materiale con energia $E_0$, dopo un tratto $x$ ha un'energia pari a quella indicata dalla legge.
\\
L'energia è indicata in MeV e $x$ e $x_0$ in m o altre unità di sto genere.
\\ \\
\paragraph{4.a.20}\textbf{Descrivere la situazione in cui la legge}
\begin{equation*}
    \frac{1}{\rho X_0}=4\alpha r_e^2\frac{Z^2}{A(g)}\biggl[ln\biggl(\frac{184}{Z^{1/3}}\biggr)-f(Z)+\frac{L'}{Z}\biggr]\ \ \ \ \ \ \ \text{\textit{Formula di Tsai}}
\end{equation*}
\textbf{è applicabile e spiegare il significato e l'unità di misura di ogni grandezza fisica indicata}\\ \\
La formula di Tsai è un modello per il Bremsstrahlung relativistico di un elettrone basato su calcoli numerici che apporta alcune correzioni. Le grandezze fisiche indicate sono:
\begin{itemize}
    \item $X_0$ lunghezza di radiazione
    \item $\rho$ densità del materiale
    \item $\alpha$ è la costante di struttura fine
    \item $r_e$ è il raggio elettronico
    \item $Z$ è il numero atomico
    \item $A(g)$ è la massa atomica in grammi
    \item $N_A$ è il numero di Avogadro
    \item $f(Z)$ e $L'$ sono correzioni numeriche solitamente riportate in apposite tabelle.
\end{itemize}

\paragraph{4.a.21}\textbf{Descrivere qualitativamente il meccanismo della perdita di energia per collisioni da parte di una particella carica nella materia.}\\ \\
Considerando una particella di carica $ze$ e massa $m$ che si muove velocemente attraverso la materia. Essa farà interazione coulombiana con gli elettroni della materia e di conseguenza cederà energia attraverso la ionizzazione di atomi e molecole o per emissione di fotoni di scintillazione. Poiché il numero di urti anche per percorsi brevi è molto elevato l'energia persa viene trattata in modo statistico e si riassume nel \textit{valore medio dell'energia persa per collisioni} indicato tipicamente da
\begin{equation}
    \frac{dE_{coll}}{dx}
\end{equation}
\paragraph{4.a.22}\textbf{Spiegare il significato di ogni termine dell’espressione per la perdita di energia per collisioni (formula di Bethe-Bloch, non dimostrata):}
\begin{equation*}
    \frac{1}{\rho}\frac{dE_{coll}}{d(x)}=z^2\frac{Z}{A(g)}4\pi\frac{m_ec^2}{\beta^2}N_ar_e^2\biggl(\frac{1}{2}ln\biggl(\frac{2m_ec^2\beta^2\gamma^2}{I^2}T_{MAX}\biggr)-\beta^2-\frac{\delta}{2}\biggr)
\end{equation*}
\begin{itemize}
    \item $\rho$ densità del materiale
    \item $r_e$ è il raggio elettronico
    \item $Z$ è il numero atomico
    \item $z$ è la costate per cui la carica della particella è $ze$
    \item $A(g)$ è la massa atomica in grammi
    \item $N_A$ è il numero di Avogadro.
    \item $I$ è l'energia media di eccitazione dell'atomo
    \item $T_{MAX}$ è l'energia massima trasferibile a un singolo elettrone per un singolo urto
    \item $\delta/2$ è un termine di correzione di densità introdotto da fermi
\end{itemize}
in particolare $T_{MAX}$ è
\begin{equation}
    T_{MAX}=\frac{2m_ec^2\beta^2\gamma^2}{1+2\frac{m_e}{M}\gamma+\biggl(\frac{m_e}{M}\biggr)^2}
\end{equation}
dove $M$ è la massa della particella che si muove nel mezzo.

\paragraph{4.a.23}\textbf{Disegnare qualitativamente la funzione di Bethe-Bloch indicando i valori dei punti significativi.}\\ \\
Il grafico della funzione di Bethe-Bloch è
\begin{figure}[H]
    \centering
    \includegraphics[]{tempsnip.png}
    \caption{Grafico della funzione di Bethee-Bloch}
    \label{fig:bethebloch}
\end{figure}
Il punto significativo è il minimo che avviene a $\beta=3.5$ e corrisponde a particelle che si muovono con energia di ionizzazione e il valore della funzione varia tra $1.1\, \frac{Mev}{g/cm^2}$ e $2\, \frac{Mev}{g/cm^2}$ 
\paragraph{4.a.24}\textbf{Definire il "percorso residuo" ("range") per una particella carica in un materiale.}\\ \\
Considerando una particella di energia iniziale $E_0$ e impulso $p_0=\sqrt{(E_0-M)^2-M^2}$ che si muove lungo $x$ e entra in un mezzo a $x=0$. La perdita di energia per collisioni rallenta la particella e questa si arresterà dopo un percorso $R$, che viene definito come \textbf{percorso residuo} o \textbf{range}.


\paragraph{4.a.25}\textbf{Definire gli "stopping power" totale, collision, radiative e nuclear; indicare per quali particelle ognuno di essi sia o meno rilevante.}\\ \\
Lo stopping power è la funzione che descrive la perdita di energia che la particella subisce nel mezzo e si può attribuire a vari fenomeni:\\
\textit{Stopping power collision}\\

{\large$\biggl(\frac{dE}{dx}\biggr)_{collisioni}\ \ \Rightarrow$}  dovuto alla perdita di energia per collisioni (formula di Bethe-Bloch)\\

\textit{Stopping power radiative}\\

{\large$\biggl(\frac{dE}{dx}\biggr)_{radiazione}\ \ \Rightarrow$}  dovuto alla perdita di energia per radiazione (formula di Tsai), rilevante solo per\\

\underline{elettroni e positroni}\\

\textit{Stopping power nuclear}\\

{\large$\biggl(\frac{dE}{dx}\biggr)_{nucleare}\ \ \Rightarrow$}  dovuto alla perdita di energia per urti elastici attraverso interazione forte sui nuclei,\\

rilevante solo per \underline{adroni}\\

\textit{Stopping power total}\\

{\large$\biggl(\frac{dE}{dx}\biggr)_{totale}=\biggl(\frac{dE}{dx}\biggr)_{collizioni}+\biggl(\frac{dE}{dx}\biggr)_{radiazione}+\biggl(\frac{dE}{dx}\biggr)_{nucleare}$}  é la perdita di energia totale della\\

particella nel mezzo ed è dovuta alla somma di tutti i termini precedenti.


\paragraph{4.a.26}\textbf{Come si calcola il "percorso residuo" ("range"), nota la curva $\frac{dE}{dx}$, in funzione dell'energia $E_0$ della particella?}\\ \\
Il range in funzione dell'energia $E_0$ della particella nota la curva $dE/dx$ si calcola come:
\begin{equation}
    R(E_0)=\int_0^{E_0}\frac{dE}{(dE/dx)}
\end{equation}
$dE$ è la perdita di energia, quindi all'inizio la particella ha perso $0$ energia e alla fine quando si ferma ha perso $E_0$ energia, per cui gli estremi di integrazione sono a quella maniera
\paragraph{4.a.27}\textbf{Spiegare qualitativamente il cosiddetto "picco di Bragg"}\\ \\
Per una particella che viaggia in un mezzo con energia maggiore di quella di ionizzazione ($\beta\gamma>3.5$) l'energia depositata dalla particella cresce e conseguentemente l'energia della particella decresce fino al minimo. La perdita di energia raggiunge un picco, detto "picco di Bragg", in corrispondenza del punto d'arresto, questo perché l'energia persa per unità di lunghezza scala come $\beta^{-2}$. Se invece la particella non ha energia superiore a quella di ionizzazione il minimo non è visibile. 
\begin{figure}[H]
    \centering
    \includegraphics[width=0.6\linewidth]{4.a.27/Cattura.PNG}
    \caption{Picchi di Bragg a diverse energie iniziali}
    \label{fig:my_label}
\end{figure}
\paragraph{4.a.28}\textbf{Descrivere qualitativamente il fenomeno dello scattering multiplo da parte di una particella carica in moto veloce nella materia.}\\ \\
Una particella di massa $M$ e carica $ze$ in moto veloce attraverso la materia effettuerà multiple collisioni con i nuclei del materiale di carica $Ze$, e ognuna di queste interazioni produrrà una deflessione nel moto della particella. Si definisce scattering multiplo la deflessione totale dalla traiettoria iniziale della particella all'uscita dal materiale.

\paragraph{4.a.29}\textbf{Definire l'angolo di multiplo scattering (rispetto alla direzione iniziale della particella) e definire la sua proiezione su un piano (che contiene la direzione iniziale della particella). Indicare i limiti delle due variabili cosí definite.}\\ \\
\underline{Angolo di multiplo scattering}: angolo tra la velocità della particella quando entra nel mezzo e la velocità della particella quando esce dal mezzo.\\
\\
Consideriamo una particella che si muove lungo l'asse $\hat{z}$ di un sistema di riferimento e attraversa un materiale in cui effettua scattering multiplo, per cui dopo viaggia lungo un direzione $\hat{n}$. Si avrà che
\begin{equation}
    \hat{n}=
    \begin{pmatrix}
    \sin(\theta)\cos(\phi)\\\sin(\theta)\sin(\phi)\\\cos(\theta)
    \end{pmatrix}
\end{equation}
dove $\theta$ è definito come angolo di scattering multiplo e varia tra $[0, \pi]$. Proiettando lungo il piano $\hat{x}\hat{z}$ e $\hat{y}\hat{z}$ si definiscono $\theta_x$ e $\theta_y$ che sono definiti in $[-\pi,\pi]$ e sono legati a $\theta$ e $\phi$ da
\begin{align}
    \sin\theta_x&=\frac{\sin\theta\cos\phi}{\sqrt{\sin^2\theta\cos^2\phi+\cos^2\theta}},&\cos\theta_x=\frac{\cos\theta}{\sqrt{\sin^2\theta\cos^2\phi+\cos^2\theta}}\\
    \sin\theta_y&=\frac{\sin\theta\sin\phi}{\sqrt{\sin^2\theta\sin^2\phi+\cos^2\theta}},&\cos\theta_y=\frac{\cos\theta}{\sqrt{\sin^2\theta\sin^2\phi+\cos^2\theta}}
\end{align}

\paragraph{4.a.30} \textbf{Spiegare il significato di ogni termine dell'espressione per l'angolo quadratico medio di multiplo scattering (rispetto alla direzione iniziale della particella) $\sqrt{<\theta_{ms}^2>}=\theta_0 \sqrt{2}=z \frac{13.6 MeV}{P\beta c} \sqrt{\frac{L}{X_0}} \Bigl( 1+0.0038 \ln \Bigl( \frac{L}{X_0} \Bigr) \Bigr) \sqrt{2}$}\\ \\
Nella formula:
\begin{equation}
\sqrt{<\theta_{ms}^2>}=\theta_0 \sqrt{2}=z \frac{13.6 MeV}{P\beta c} \sqrt{\frac{L}{X_0}} \Bigl( 1+0.0038 \ln \Bigl( \frac{L}{X_0} \Bigr) \Bigr) \sqrt{2}
\end{equation}
troviamo il numero di protoni $z$ della particella incidente, il modulo dell'impulso $P$ della particella, la $\beta=\frac{v}{c}$ di Lorentz della particella, la velocità della luce nel vuoto $c$, la lunghezza totale $L$ percorsa dalla particella, la lunghezza di radiazione $X_0$.

\paragraph{4.a.31}\textbf{Se non fosse sufficiente la approssimazione di piccoli angoli e distribuzione
gaussiana, indicare quale fra le seguenti funzioni descriverebbe meglio il
fenomeno del multiplo scattering: i) Bethe-Bloch, ii) Moliere, iii) Breit-Wigner,
iv) Bohr.}\\ \\
Per grandi angoli occorre utilizzare la distribuzione di Molière.
\paragraph{4.a.32}\textbf{Illustrare in modo qualitativo il metodo di produzione degli antiprotoni
nell’esperimento di Segré et al.}\\ \\
Per cercare l'antiprotone si cercano i prodotti delle reazioni:
\begin{equation}
    p+p\to p+\bar{p}+p+p \qquad p+n\to n+\bar{p}+p+p
\end{equation}
Con energia di soglia di circa E$>7m_p$ (o T$>6m=5.6\ Gev$) nel sistema del laboratorio (si intende il sistema di riposo di uno dei protoni).\\
Tuttavia nel caso in cui il bersaglio abbia impulso non nullo, l’energia di soglia si riduce sino a 
4.3 GeV.......\\
Si sfrutta dunque il fascio di protoni prodotto all'interno dell'acceleratore Bevatron portandolo ad incidere su una lastra di rame.\\
Da tale fenomeno si originano diverse particelle, grazie ad un campo magnetico all'interno del bevatron e all'utilizzo di un collimatore si selezionano le particelle di carica negativa con il giusto impulso $|\vec{p}|=1.19\ Mev/c$.\\
\paragraph{4.a.33} \textbf{Spiegare qualitativamente il metodo di separazione degli antiprotoni dal fondo di pioni nell’esperimento di Segré et al. tramite contatori Cherenkov}\\ \\
\begin{figure}[H]
    \centering
    \includegraphics[scale=0.4]{BArP.PNG}
    \caption{Caption}
\end{figure}

Il fascio di protoni del Bevatron colpisce una targetta T. A quel punto le particelle prodotte uscenti hanno un impulso di circa 1.19 GeV, e tali particelle (negative) vengono piegate verso l'esterno del sincrotrone dal campo magnetico generato dall'acceleratore stesso di circa 21°. Successivamente incontrano un primo magnete M1 che le deflette ulteriormente di 32°, seguito da una serie di 3 quadrupoli magnetici Q1 utilizzati per il \emph{focusing} del fascio. Uno shield, situato prima del contatore S1 permette di pulire ulteriormente il fascio. Dopo aver attraversato il contatore S1, il fascio è nuovamente focalizzato dai quadrupoli Q2 e piegato di altri 34° dal magnete M2, che le focalizza sul contatore S2. Le particelle che arrivano a tale contatore hanno lo stesso momento all'interno del 2\%. Successivamente trovano posto due rilevatori Cherenkov C1 e C2 che permettono di misurare le particelle che hanno un $\beta$ ben preciso, incluso fra i 0.75 e i 0.78 nel caso di C2 e maggiore di 0.79 nel caso di C1. Le particelle che arrivano al contatore S2 hanno impulsi di circa 1.19 GeV: gli antiprotoni hanno velocità situate attorno a $\beta = 0.78$, quindi tramite i rilevatori Cherenkov è possibile individuare tali antiparticelle.
Infine è situato il contatore S3, che serve per una verifica aggiuntiva.\\
Vediamo ora quale è stata la strategia per separare gli antiprotoni dal fondo (molto rumoroso).

\begin{itemize}
\item Il tempo di volo, diverso per i pioni ($\beta_\pi= 0.99$) permette di selezionare gli antiprotoni, che viaggiando più lentamente possono essere così discriminati. Si stima un tempo di volo di 40 ns per gli antiprotoni e di 51 ns per i pioni (accuratezza di 1/2 ns).\\
\item Inoltre ci deve essere una coincidenza nel rilevatore C2, che identifica solo particolari velocità, ovvero un range molto ristretto in cui "viaggiano" gli antiprotoni. Tale misura è indipendente da quella del tempo di volo;
\item E' necessaria una coincidenza anche sul contatore S3, in quanto gli antiprotoni non sono deviati ad angoli molto grandi quando attraversano C2.
\end{itemize}
Si richiede quindi la coincidenza di queste risposte entro un tempo di 6 ns.\\
Esistono tuttavia delle problematiche nei rivelatori, difatti a causa di interazioni (di carattere nucleare) lungo il radiatore del contatore Cherenkov il 3$\%$ dei pioni viene conteggiato da C2. Inoltre coincidenze accidentali in S1 e S2 fanno si che alcuni pioni vengano conteggiati. Per ovviare all'unione dei 2 fenomeni si utilizza il contatore C1 che dice se la particella è un pione o meno.\\
Affinchè si osservi un antiprotone è dunque necessario che scattino i rivelatori S1S2C2S3 e non C1.\\

\paragraph{4.a.34}\textbf{Descrivere l’esperimento di Anderson sulla scoperta del positrone}.\\\\
Nell’esperimento venne usata una camera a nebbia, ossia un dispositivo isolato contenente del vapore saturo. Una particella carica e sufficientemente energetica che attraversa la camera ionizza il gas sulla sua traiettoria e gli atomi ionizzati formano dei nuclei di condensazione su cui cominciano a formarsi delle goccioline. Il numero delle bolle è proporzionale alla carica della particella.\\
Fotografando la camera è possibile ricostruire la traiettoria della particella incidente. Se poi la camera è immersa in un campo magnetico, è possibile misurare l’impulso della particella a partire dal raggio di curvatura.
Nell’esperimento di Anderson venne registrata la traccia di una particella che attraversa uno strato di piombo di 6 mm. Prima dello strato la particella aveva $p_i = 63 MeV/c$, dopo $p_f = 22.5 MeV/c$.\\
Dal verso della curvatura della traiettoria si deduce che la particella ha carica positiva. L’unica particella positiva nota al tempo era il protone, ma con quell’impulso il protone è non relativistico. La perdita di energia per ionizzazione è molto elevata e il range è di circa 5 mm, quindi un protone
con quell’impulso non può attraversare uno strato di piombo di 6 mm. Se invece assumiamo che la particella sia un positrone, deve essere ultrarelativistico sia prima che dopo aver attraversato la lastra. Dato che questa è spessa quanto una lunghezza di radiazione, il positrone perde energia sia per collisioni (7 MeV/c circa) che per irraggiamento (diminuisce di un fattore $1/e$), risultando compatibile all'energia di uscita.\\

\textit{Domande b}\rule{13.7 cm}{0.4pt}
\paragraph{4.b.1}\textbf{Dimostrare che all’interno del cono della radiazione Cherenkov vi sono due
soluzioni per $t'=t-\frac{nR}{c}$, nessuna soluzione all’esterno, ed una sola sul fronte
d’onda.}\\
\begin{figure}[H]
    \centering
    \includegraphics[scale=0.7]{4.a.6.2.PNG}
\end{figure}
Sia P il punto di osservazione: al tempo t' la particella emette il segnale ricevuto in P al tempo t e sia $\Delta t=t-t'$.\\
Si definiscone anche i vettori:
\begin{itemize}
    \item $\vec{R}$ congiungente z(t') e P;
    \item $\vec{r'}$ congiungente z(t') e z(t);
    \item $\vec{r}$ congiungente z(t) e P.
\end{itemize}
Da cui $\vec{R}=\vec{r'}+\vec{r}$, ovvero $|\vec{R}|=|\vec{r'}+\vec{r}|^2$.
\begin{equation}
    \frac{c^2}{n^2}\Delta t^2=v^2\Delta t^2+r^2+2rv\Delta cos(\alpha)
\end{equation}
Si ottiene un equazione di secondo grado in $\Delta t$:
\begin{equation*}
\biggl( v^2-\frac{c^2}{n^2} \biggr) \Delta t^2+2rv\cos\alpha \Delta t +r^2=0 \, .
\end{equation*}
Per ragioni di causalità occorre che $\cos\alpha<0 \Rightarrow \frac{\pi}{2} \leq \alpha \leq \frac{3\pi}{2}$.\\
Affinché ci siano soluzioni il discriminante dell'equazione deve essere maggiore o uguale a 0:
\begin{equation*}
\Delta=4r^2v^2\cos^2\alpha-4r^2\biggl( v^2-\frac{c^2}{n^2} \biggr)\geq 0
\end{equation*}
\begin{equation*}
\Rightarrow v^2\cos^2\alpha- \biggl( v^2-\frac{c^2}{n^2} \biggr)\geq 0
\end{equation*}
A questo punto abbiamo due possibilità:
\begin{itemize}
\item $v<\frac{c}{n}$ Questo è sempre vero per $n=1$. In tal caso c'è sempre soluzione. Tuttavia non dovrebbe presentarsi effetto cherenkov.\\
\item $v>\frac{c}{n}$ In questo caso per il discriminante abbiamo:
\begin{align*}
v^2 \biggl( \cos^2 \alpha-1 \biggr) &\geq -\frac{c^2}{n^2}  \\
-v^2 \sin^2\alpha &\geq -\frac{c^2}{n^2}  \\ 
\sin^2\alpha &\leq \frac{c^2}{n^2 v^2}=\frac{1}{n^2 \beta^2} \, 
\end{align*}
ovvero:
\begin{equation*}
|\sin\alpha|\leq \frac{1}{n\beta} \, .
\end{equation*}
\end{itemize}
Le due soluzioni del secondo caso collassano in una soluzione se il discriminante è uguale a 0, ovvero se:
\begin{equation*}
|\sin\alpha|=\frac{1}{n\beta} \Rightarrow \cos\theta_c=\frac{1}{\beta n} \, .
\end{equation*}
Ovvero sul cono Cherenkov.\\
\begin{figure}[H]
    \centering
    \includegraphics[scale=0.7]{4.a.6.3.PNG}
\end{figure}

\paragraph{4.b.2}\textbf{A partire dalla espressione $\frac{d^2 N_\gamma}{dE_\gamma dx}=z^2 \frac{\alpha}{\hbar c} \sin^2 \theta_c$, valida per la radiazione Cherenkov, dimostrare che $N_\gamma=z^2 \frac{\alpha}{\hbar c} L \int_{E_1}^{E_2} \Bigl[ 1-\frac{1}{\beta^2 \epsilon_r(E)} \Bigr] P_{det}(E)dE$.}\\ \\
Innanzitutto grazie all'indice di rifrazione riscriviamo come:
\begin{equation}
    \frac{d^2 N_\gamma}{dE_\gamma dx}=z^2 \frac{\alpha}{\hbar c}\biggl(1-\frac{1}{\beta^2n^2}\biggr)
\end{equation}
Il numero totale di fotoni emessi per unità di lunghezza per effetto Cherenkov su tutte le frequenze è:
\begin{equation}
    \frac{d N_\gamma}{dx}=\int_{E_1}^{E_2}z^2 \frac{\alpha}{\hbar c}\biggl(1-\frac{1}{\beta^2n^2}\biggr)dE_\gamma
\end{equation}
Dove $E_1=\hbar \omega_1,E_2=\hbar \omega_2$ valori estremi dell'intervallo per cui $\beta n(\omega)\ge 1$ e tali che $\beta n(\omega_{1,2})=1$.\\
Occorre tuttavia tenere conto che i fotoni vengono rivelati da fotorivelatori, dispositivi che permettono di rivelare i fotoni solo in un determinato range di frequenze (e quindi di energie) e con un efficenza minore del $100\%$ (in genere 30$\%$).\\
Si definisce dunque con P$_{det}$(E) la probabilità che un fotone generato per effetto Cherenkov sia rivelato, passando in un mezzo di lunghezza L, il numero di fotoni osservati è dato da:
\begin{equation}
    N_\gamma=z^2 \frac{\alpha}{\hbar c} L \int_{E_1}^{E_2} \Bigl[ 1-\frac{1}{\beta^2 \epsilon_r(E)} \Bigr] P_{det}(E)dE 
\end{equation}
Un esempio di $P_{det}(E)$:
\begin{figure}[H]
    \centering
    \includegraphics[scale=0.7]{4.b.2Fotorivelatore.PNG}
\end{figure}

\paragraph{4.b.3}\textbf{Calcolare il numero di fotoni Cherenkov osservati da un fotorivelatore con efficienza del 30$\%$ per luce fra 300nm e 600nm, al passaggio di un elettrone nei due casi seguenti: i) n=1.005 (gas), $\beta$=0.999, spessore attraversato = 1m; ii) n=1.5 (solido trasparente), $\beta$=0.99, spessore attraversato = 1cm.}
Si fa riferimento alla formula ricavata nella domanda precedente:
\begin{equation}
     N_\gamma=z^2 \frac{\alpha}{\hbar c} L \int_{E_1}^{E_2} \Bigl[ 1-\frac{1}{\beta^2 n^2(E)} \Bigr] P_{det}(E)dE 
\end{equation}
\begin{itemize}
    \item \begin{equation}
         N_\gamma=370\frac{fotoni}{eV \cdot cm}1 m\Bigl[1-\frac{1}{\beta^2 n^2(E)} \Bigr]12400\biggl(\frac{1}{\lambda_1[\text{\AA}]}-\frac{1}{\lambda_2[\text{\AA}]}\biggr)eV\frac{30}{100}\sim 182 fotoni
    \end{equation}
     \item \begin{equation}
         N_\gamma=370\frac{fotoni}{eV \cdot cm}1 cm\Bigl[ 1-\frac{1}{\beta^2 n^2(E)} \Bigr]12400\biggl(\frac{1}{\lambda_1[\text{\AA}]}-\frac{1}{\lambda_2[\text{\AA}]}\biggr) eV\frac{30}{100}\sim 125 fotoni
    \end{equation}
\end{itemize}
Dove si è fatto uso dell'equazione: E[eV]=$\frac{12400}{\lambda[\text{\AA}]}$
Partendo da $\hbar=6.582\cdot10^{-16}\,cms$ si ha che 
\begin{equation}
    hc=\hbar2\pi c=6.582\cdot3\cdot2\pi\cdot10^{-6}=124\cdot10^{-6}
\end{equation}
Passando in angstrom bisogna moltiplicare per un fattore $10^{10}$
\paragraph{4.b.4}\textbf{Calcolare l’angolo di emissione della radiazione Cherenkov in funzione dell’impulso (e della massa) della particella e dell’indice di rifrazione}\\ \\
Dalla definizione del coseno dell'angolo Cherenkov abbiamo:
\begin{equation*}
\cos\theta_c=\frac{1}{\beta n}=\frac{c}{vn}=\frac{m\gamma c^2}{m \gamma v n c}=\frac{E}{pnc}=\frac{\sqrt{p^2c^2+m^2c^4}}{pnc} \, ,
\end{equation*}
ovvero:
\begin{equation*}
\theta_c=\arccos \frac{\sqrt{p^2c^2+m^2c^4}}{pnc} \, .
\end{equation*}
\paragraph{4.b.5}\textbf{Descrivere il principio di funzionamento dei rivelatori Cherenkov: i) a soglia, ii) RICH, iii) DIRC}
    Innanzitutto ricordiamo che il numero di fotoni emesso per cammino unitario è:
\begin{equation}
\frac{dE}{dx} = \frac{2\pi \alpha h}{c} \int_{\beta n > 1} \left( 1- \frac{1}{\beta^2 n^2} \right) \nu d\nu
\end{equation}

Ovvero il numero di fotoni emessi fra $\lambda_1 < \lambda < \lambda_2$ sarà:
\begin{equation}
N_{pe} = 2 \pi \alpha L \int_{\lambda_1}^{\lambda_2} \sin^2 \theta \frac{d \lambda}{\lambda^2}
\end{equation}

Il numero di fotoni emessi per effetto Cherenkov è molto piccolo, in quanto tipicamente l'energia persa è dominata dalla ionizzazione (Bethe - Bloch).
\begin{itemize}
\item \textbf{A soglia.} In Figura \ref{fig:soglia} possiamo vedere lo schema di un rilevatore Cherenkov a soglia. Tali rilevatori permettono di discriminare particelle con determinati $\beta$ in quanto riescono a rilevare solo particelle con energie superiori ad una certa soglia. Infatti rileveremo particelle solo con $\beta > \beta_s = \frac{1}{n}$ ed $n$ è facilmente variabile modificando la pressione del gas, per esempio (tramite la legge di Clausius - Mossotti).

\begin{figure}[H]
    \centering
    \includegraphics{4.b.5SOGLIA.PNG}
    \caption{Rilevatore Cherenkov a soglia} \label{fig:soglia}
\end{figure}


\item \textbf{RICH - Ring Imaging CHerenkov.} In questo caso la velocità delle particelle è misurata tramite l'angolo Cherenkov. Fattivamente si osservano cerchi di fotoni che sono più o meno larghi in base alla velocità e alla massa della particella: particelle più pesanti creeranno cerchi più piccoli, mentre particelle più leggere creeranno cerchi più ampi. In Figura \ref{fig:rich} possiamo vedere lo schema di funzionamento di un rilevatore RICH.

\begin{figure}[H]
    \centering
    \includegraphics{4.b.5RICH.PNG}
    \caption{Rilevatore Cherenkov Rich} \label{fig:rich}
\end{figure}


\item \textbf{DIRC - Detection of Internally Reflected Cherenkov light.} Riesco a ricavare l'informazione relativa all'angolo di emissione in base alla riflessione totale all'interno di una fibra. In pratica una particella che viaggia all'interno di un mezzo può emettere radiazione Cherenkov. Se tale radiazione ha un angolo sufficientemente piccolo rispetto ai bordi del materiale, abbiamo riflessione totale. All'uscita del materiale trovano posto una serie di fotomoltiplicatori che riescono a discriminare l'angolo di uscita, risalendo quindi direttamente all'impulso della particella.

\begin{figure}[H]
    \centering
    \includegraphics{4.b.5DIRK.PNG}
    \caption{Rilevatore Cherenkov DIRC} \label{fig:dirc}
\end{figure}
\end{itemize}
\paragraph{4.b.6}\textbf{Partendo dalla espressione $\frac{dI_\omega}{d\Omega}=\frac{q^2}{4\pi^2 c} \Bigl| \int \frac{\mathbf{\hat{n}}\times[(\mathbf{\hat{n}}-\bm{\beta})\times\bm{\dot{\beta}} ]}{(1-\mathbf{\hat{n}}\cdot\bm{\beta})^2} e^{i\omega ( t'-\frac{\mathbf{r'}\cdot\mathbf{\hat{n}}}{c} ) } dt' \Bigr|^2$ dimostrare che l'energia persa per unità di frequenza nel caso non relativistico è approssimabile con 
\begin{equation*}
I_\omega=
\begin{cases}
\frac{2q^2}{3\pi c} |\Delta \bm{\beta}|^2\quad &\text{per} \quad \omega<1/\tau \\
0 \quad &\text{per} \quad \omega>1/\tau  \\
\end{cases}
\end{equation*}}
Partendo dall'espressione:
\begin{equation*}
\frac{dI_\omega}{d\Omega}=\frac{q^2}{4\pi^2 c} \biggl| \int_{-\infty}^\infty \frac{\mathbf{\hat{n}}\times[(\mathbf{\hat{n}}-\bm{\beta})\times\bm{\dot{\beta}} ]}{(1-\mathbf{\hat{n}}\cdot\bm{\beta})^2} e^{i\omega ( t'-\frac{\mathbf{r'}\cdot\mathbf{\hat{n}}}{c} ) } dt' \biggr|^2 \, ,
\end{equation*}
nel caso non relativistico ($\beta<<1$) abbiamo:
\begin{equation*}
\frac{dI_\omega}{d\Omega}=\frac{q^2}{4\pi^2 c} \biggl| \int \mathbf{\hat{n}}\times(\mathbf{\hat{n}}\times\bm{\dot{\beta}}) e^{i\omega t'} dt' \biggr|^2 \, .
\end{equation*}
Poichè $|\vec{r'}|\sim \beta t'$.\\
Contando la durata dell'impulso:
\begin{equation*}
\frac{dI_\omega}{d\Omega}=\frac{q^2}{4\pi^2 c} \biggl| \int_{-\tau/2}^{\tau/2} \mathbf{\hat{n}}\times(\mathbf{\hat{n}}\times\bm{\dot{\beta}}) e^{i\omega t'} dt' \biggr|^2 \, .
\end{equation*}
Per basse frequenze ($\omega\ll1/\tau\sim \frac{v}{b}$) abbiamo $e^{i\omega t'} \simeq 1$, ovvero:
\begin{equation}
\frac{dI_\omega}{d\Omega}=\frac{q^2}{4\pi^2 c} \biggl| \mathbf{\hat{n}}\times \biggl( \mathbf{\hat{n}}\times \int \bm{\dot{\beta}}dt' \biggr) \biggr|^2=\frac{q^2}{4\pi^2 c} |\mathbf{\hat{n}}\times( \mathbf{\hat{n}}\times \Delta\bm{\beta})|^2=\frac{q^2}{4\pi^2 c} |\Delta\bm{\beta}|^2 \sin^2\theta \end{equation}
dove $\theta$ è l'angolo compreso tra $\mathbf{\hat{n}}$ e $\Delta\bm{\beta}$. Integrando nell'angolo solido troviamo:
\begin{equation}
I_\omega=\frac{q^2}{4\pi^2 c} |\Delta\bm{\beta}|^2 \int_0^\pi \sin^3 \theta d\theta \int_0^{2\pi}d\phi =\frac{q^2}{4\pi^2 c} |\Delta\bm{\beta}|^2 \frac{4}{3} 2\pi=\frac{2q^2}{3\pi c} |\Delta\bm{\beta}|^2 \, .
\end{equation}
Per alte frequenze ($\omega\gg1/\tau$) abbiamo invece l'integrale va a 0 per il lemma di Riemann-Lebesgue. In particolare se $\beta$ fosse costante:
\begin{equation}
\int_{-\tau}^{\tau} e^{i\omega t'}dt'= \frac{e^{i\omega \tau}-e^{-i\omega \tau}}{i\omega}=\frac{2\sin\omega \tau}{\omega \tau} \xrightarrow{\omega\tau\gg 1} 0 \, .
\end{equation}
Quindi riassumendo:
\begin{equation*}
I_\omega=
\begin{cases}
\frac{2q^2}{3\pi c} |\Delta \bm{\beta}|^2 \quad &\text{per} \quad \omega\ll1/\tau \\
0 \quad &\text{per} \quad \omega\gg1/\tau  \\
\end{cases} \, .
\end{equation*}
Approssimando:
\begin{equation*}
I_\omega=
\begin{cases}
\frac{2q^2}{3\pi c} |\Delta \bm{\beta}|^2 \quad &\text{per} \quad \omega<1/\tau \\
0 \quad &\text{per} \quad \omega>1/\tau  \\
\end{cases} \, .
\end{equation*}
\paragraph{4.b.7}\textbf{Partendo dalla espressione $\frac{dI_\omega}{d\Omega}=\frac{q^2}{4\pi^2 c} \Bigl| \int \frac{\mathbf{\hat{n}}\times[(\mathbf{\hat{n}}-\bm{\beta})\times\bm{\dot{\beta}} ]}{(1-\mathbf{\hat{n}}\cdot\bm{\beta})^2} e^{i\omega ( t'-\frac{\mathbf{r'}\cdot\mathbf{\hat{n}}}{c} ) } dt' \Bigr|^2$ dimostrare che l'energia persa per unità di frequenza nel caso non relativistico ad un dato parametro di impatto $b$ è approssimabile con 
\begin{equation*}
I_\omega(b)=
\begin{cases}
\frac{2(ze)^2}{3\pi c^3} \Bigl( \frac{2zZe^2}{MVb} \Bigr)^2 \quad &\text{per} \quad \omega<V/b \\
0 &\text{per} \quad \omega>V/b \\ 
\end{cases} \, .
\end{equation*}}
Ricordando che in regime non relativistico vale:
\begin{equation}
    \frac{dI_\omega}{d\Omega}\sim \frac{q^2}{4\pi c} \Bigl| \int_{-\infty}^\infty\mathbf{\hat{n}}\times\mathbf{\hat{n}}\times\bm{\dot{\beta}} e^{i\omega t'} dt' \Bigr|^2
\end{equation}
La durata dell'impulso è $\tau\sim\frac{b}{v}$, da cui
\begin{equation*}
I_\omega=
\begin{cases}
\frac{2q^2}{3\pi c} |\Delta \bm{\beta}|^2 \quad &\text{per} \quad \omega<1/\tau \\
0 \quad &\text{per} \quad \omega>1/\tau  \\
\end{cases} 
\end{equation*}
\textbf{N.B.:} in realtà si dovrebbe porre $\ll$ e $\gg$, ma si fa un approssimazione.\\
Riscriviamo $\Delta \bm{\beta}$ sapendo che la traiettoria è approssimabile con una retta. La variazione dell'impulso è:
\begin{equation*}
|\Delta \mathbf{p}|=\int |\mathbf{F}| dt= \int ze |\mathbf{E_\bot}| dt= \frac{ze}{v} \int |\mathbf{E_\bot}| dx \, .
\end{equation*}
D'altra parte, per il teorema di Gauss, abbiamo:
\begin{equation*}
2\pi b \int |\mathbf{E_\bot}| dx= 4\pi Ze \Rightarrow \int |\mathbf{E_\bot}| dx= \frac{2Ze}{b}\, .
\end{equation*}
Sostituendo l'integrale sopra troviamo:
\begin{equation*}
|\Delta \mathbf{p}|=\frac{2zZe^2}{bV} \, ,
\end{equation*}
da cui, dividendo per $Mc$ entrambi i membri:
\begin{equation*}
|\Delta \bm{\beta}|=\frac{2zZe^2}{MbVc} \, .
\end{equation*}
Sostituendo $|\Delta \bm{\beta}|$ all'interno di $I_\omega$ troviamo il risultato cercato
\begin{equation*}
I_\omega=
\begin{cases}
\frac{2(ze)^2}{3\pi c^3} \Bigl( \frac{2zZe^2}{MVb} \Bigr)^2 \quad &\text{per} \quad \omega<V/b \\
0 \quad &\text{per} \quad \omega>V/b \\
\end{cases} \, .
\end{equation*}
\paragraph{4.b.8}\textbf{Dimostrare, a partire dalla espressione $\chi_\omega=\frac{16}{3} z^4 Z^2 \alpha\hbar c^2 \frac{r_e^2}{V^2} \Bigl( \frac{m_e}{M} \Bigr)^2  \ln \Bigl( \frac{MV^2}{\hbar \omega}\Bigr)$ che nel caso relativistico la sezione d’urto di irraggiamento per elettroni è approssimabile, in un modello, con
\begin{equation}
    \chi_\omega=\frac{16}{3} z^4 Z^2 \alpha\hbar r_e^2 \Bigl( \frac{m_e}{M} \Bigr)^2  \ln \Bigl( \frac{192}{Z^{1/3}}\Bigr)
\end{equation}}
Nel caso non relativistico la sezione d'urto di irraggiamento si scrive come:
\begin{equation}\label{4.b.8 nR}
    \chi(\omega)=\int_{b_{min}}^{b^{max}} I_\omega2\pi bdb=\frac{16}{3} z^4 Z^2 \alpha\hbar \frac{r_e^2}{V^2}\Bigl( \frac{m_e}{M} \Bigr)^2  \ln \Bigl( \frac{b_{max}}{b_{min}}\Bigr)
\end{equation}
Nel trattare un elettrone relativistico con velocità V incidente su un atomo si considera il sistema di riferimento $\Sigma'$ a riposo, in cui l'atomo si muove di velocità V verso l'elettrone.\\
In tale sistema $\Delta V\ll c$ e l'elettrone irraggia in modo non relativistico, per cui la sezione d'urto di irragiamento corrisponde a quella non relativistica in equazione (\ref{4.b.8 nR}).\\
La durata dell'impulso in tale sistema di riferimento risulta essere $\tau\sim \frac{b}{v\gamma}$, per cui si deve avere $\omega'<\frac{\gamma c}{b}$.\\
Inoltre deve essere $\omega'<\frac{m_e c^2}{\hbar}$ in quanto nel sistema del laboratorio l'elettrone non può irraggiare più della sua energia.\\
Quando si ha $\frac{m_ec^2}{\hbar}<\frac{\gamma c}{a}$, ovvero per $\gamma>\frac{m_ec^2a}{\hbar c}$, ovvero in regime di screening attivo si ha b$_{max}=a=\frac{1.4a_0}{z^{1/3}}$. In caso di screening non attivo si deve avere $b_{max}=\frac{\gamma c}{\omega}$, ed in generale l'effetto del processo sarà minore.\\
Per quanto riguarda b$_{min}$ si utilizza la lunghezza d'onda Compton $b_{min}=\frac{\hbar}{m_ec}\sim 400 fm$ per ragioni quanto-meccaniche

Si trova dunque:
\begin{equation}
    \chi(\omega)=\frac{16}{3} z^4 Z^2 \alpha\hbar c^2\frac{r_e^2}{V^2}\Bigl( \frac{m_e}{M} \Bigr)^2  \ln \Biggl( \frac{\frac{1.4 a_0}{Z^{1/3}}}{\frac{\hbar}{m_ec}}\Biggr)=\frac{16}{3} z^4 Z^2 \alpha\hbar r_e^2 \Bigl( \frac{m_e}{M} \Bigr)^2  \ln \Bigl( \frac{192}{Z^{1/3}}\Bigr)
\end{equation}
Dove si è posto $V\sim c$.\\
In tale equazione non è presente dipendenza dalla frequenza, per cui tornando al sistema $\Sigma$ del laboratorio si ha $\chi(\omega)=\chi'(\omega')$.\\


\paragraph{4.b.9}\textbf{Dimostrare che nel caso relativistico la perdita di energia per irraggiamento è approssimabile con quella non relativistica
\[
\frac{dE^{irr}}{dx}=n_{nuclei}\frac{16}{3}z^4Z^2\alpha \left(\frac{m_e}{M} \right)^2 r_e^2 ln\left[ \frac{192}{Z^{1/3}}\frac{M}{m_e} \right]E,
\]
da cui l'espressione approssimata per la lunghezza di radiazione per elettroni
\[
X_0=\frac{1}{\frac{16}{3}\rho \frac{N_A}{A}Z^4 \alpha r_e^2 ln\left[\frac{192_0}{Z^{1/3}} \right]}
\]
}\\
\\
Risolviamo il problema in un sistema in cui l'elettrone è inizialmente fermo ed il nucleo si muove con velocità relativistica, in questo sistema l'elettrone irraggia in modo non relativistico. Trasformeremo poi il tutto nel sistema del laboratorio.
\begin{figure}[H]
    \centering
    \includegraphics[scale=0.5]{4.b.9.1.PNG}
    \includegraphics[scale=0.5]{4.b.9.2.PNG}
\end{figure}
\underline{Calcolo in $\Sigma'$}\\
Possiamo utilizzare l'equazione non relativistica per l'irraggiamento dell'elettrone.
\begin{equation}
    \chi_{\omega'}'=\frac{16}{3}z^4Z^2\alpha\hbar c^2 \frac{r_e^2}{V^2}ln\frac{b_{max}}{b_{min}}
\end{equation}
La durata dell'impulso è $\tau'\sim b/(\gamma c)$ che implica $\omega'< \gamma c/b$.\\
Le frequenze hanno come limite superiore $\omega'<m_ec^2/\hbar$, poiché nel sistema del laboratorio l'elettrone non può irraggiare più della sua energia.\\
Il modello utilizzato come parametro di impatto minimo è la lunghezza Compton dell'elettrone:
\begin{equation*}
    b_{min}=\frac{\hbar}{m_e c}
\end{equation*}
Il limite superiore del parametro d'impatto non può eccedere il raggio atomico:
\begin{equation*}
    b_{max}=a=\frac{1.4 a_0}{Z^{1/3}}
\end{equation*}
perchè gli elettroni atomici schermano il campo elettrico del nucleo.\\
\warning Questo limite è valido solo per screening attivo in cui $m_ec^2/\hbar<\gamma c/a \Rightarrow\gamma>\frac{m_ec^2a}{\hbar c} \simeq \frac{192}{Z^{1/3}}$ (da qui trovo a).
\begin{figure}[H]
    \centering
    \includegraphics[scale=0.5]{4.b.9.3.PNG}
\end{figure}
Nel caso di screening attivo abbiamo quindi 
\begin{equation*}
    \chi_{\omega'}'=\frac{16}{3}Z^2\alpha\hbar r_e^2ln\frac{192}{Z^{1/3}}
\end{equation*}

\underline{Calcolo in $\Sigma$}\\
Utilizzando la relazione $\chi_{\omega}=\chi_{\omega'}'$ con $\omega=\gamma\omega'$
si ottiene:
\begin{equation}
    \frac{dE^{irr}}{dx}=n \int_0^{e/\hbar} \chi_w d\omega= n\frac{16}{3}z^4Z^2\alpha \left(\frac{m_e}{M} \right)^2 r_e^2 ln\left[ \frac{192}{Z^{1/3}}\frac{M}{m_e} \right]E =\frac{E}{X_0}
\end{equation}
dove E energia dell'elettrone e $X_0= \frac{1}{n \frac{16}{3} Z^2 \alpha r_e^2 \left(ln\frac{192}{Z^{1/3}}\right)}$ lunghezza di radiazione.

\paragraph{4.b.10}\textbf{Valutare la lunghezza di radiazione del Piombo e del Silicio con il modello spiegato a lezione ed effettuare un confronto con i valori sperimentali reperibili
su internet}\\
\\
Per stimare la lunghezza di radiazione di un materiale conviene usare la seguente formula in modo da essere indipendenti dallo stato (gas con differenti pressioni, liquido, solido,..) del materiale:
\begin{equation}
    \rho X_0= \frac{1}{\frac{16}{3}\frac{N_{av}}{M_A}Z^2\alpha r_e^2\left(ln\frac{192}{Z^{1/3}}\right)}
\end{equation}
$M_A$è la massa molare espressa in $g/mol$, $N_{av}=6.02 \cdot 10^{23} 1/mol$ , $\alpha=1/137$, $r_e=2.8 \cdot 10^{-13}$ cm.\\
\underline{Piombo} (Z=82, $M_A$=207.2 $g/mol$)
\begin{equation*}
\rho X_0=\frac{1}{\frac{16}{3}\frac{6.02 \cdot 10^{23}}{207.2} 6724 \frac{1}{137} 7.84 \cdot 10^{-26}ln\frac{192}{4.34}} \sim 4.4 g/cm^2
\end{equation*}

\underline{Silicio}(Z=14, $M_A$= 28.09 $g/mol$)
\begin{equation*}
\rho X_0=\frac{1}{\frac{16}{3}\frac{6.02 \cdot 10^{23}}{28.09} 196 \frac{1}{137} 7.84 \cdot 10^{-26}ln\frac{192}{2.41}}\sim 17.5 g/cm^2 \Longrightarrow
\end{equation*}
Dai dati sperimentali (\url{https://pdg.lbl.gov/}$\rightarrow$ Atomic and Nuclear Properties (\url{https://pdg.lbl.gov/2020/AtomicNuclearProperties/index.html}) $\rightarrow$ elemento che ti serve $\rightarrow$ Radiation length ):\\
$\rho X_0(Pb)=6.37 g/cm^2$ e $\rho X_0(Si)= 21.8 g/cm^2$. 

\paragraph{4.b.11}\textbf{Calcolare l'energia irraggiata da un elettrone di 60MeV che attraversi 5.6mm di Pb e calcolare il numero medio di fotoni emessi con energia fra 1eV e 1MeV}\\
\\
$X_0(Pb)=$5.6 mm\\
Un elettrone che entra con energia $E_0$ in un materiale di lunghezza di radiazione $X_0$ dopo un tratto $x$ avrà energia:
\begin{equation*}
    E=E_0 e^{-x/X_0}
\end{equation*}
\begin{equation*}
    E_{irr}=E_0-E= E_0(1-e^{-x/X_0})=E_0(1-e^{-1})\simeq 38 MeV
\end{equation*}
\begin{equation}
    \frac{dN}{dE_{\gamma}}=\frac{1}{E_\gamma}\frac{x}{X_0}
\end{equation}
\begin{equation*}
    N_{\gamma}=\int_{E_1}^{E_2} \frac{1}{E_{\gamma}}\frac{x}{X_0}dE=\frac{x}{X_0}ln\frac{E_2}{E_1}=ln\frac{1MeV}{1 eV}\simeq 11.5
\end{equation*}

\paragraph{4.b.12}\textbf{Calcolare il numero medio di fotoni emessi con energia fra 10 e 100 MeV per un elettrone di 1 GeV che attraversi 300µm di Silicio o 1 mm di Piombo. Calcolare poi la probabilità che uno di questi fotoni effettui una interazione prima di uscire dal materiale.}\\
\begin{equation*}
    N_{\gamma}=\frac{x}{X_0}ln\frac{E_2}{E_1}=\frac{x}{X_0}2.30
\end{equation*}
Il silicio ha $X_0=9.36 cm$, il piombo $X_0= 0.56cm$.
\[
N_{\gamma}(silicio)\sim 0.0073  \;\;\;\; N_{\gamma}(piombo)\sim 0.4
\]

La probabilità di interazione si può calcolare come:
\begin{equation*}
    \frac{dP}{dx}= \frac{\rho_A N_{av}}{M_A}\sigma
\end{equation*}
Per la $\sigma$: \url{https://physics.nist.gov/PhysRefData/Xcom/html/xcom1-t.html} $\rightarrow$element $\rightarrow$ numero atomico, metti barns/atom 
\\
\warning L'energia dell'elettrone serve per capire l'energia massima che possono avere i fotoni prodotti.
\\
\paragraph{4.b.13}\textbf{Utilizzando le tabelle che forniscono le sezioni d'urto di fotoni su atomi, calcolare la probabilità che un fotone da 10 MeV produca una coppia $e^+e^-$ in uno spessore di Piombo pari ad una lunghezza di radiazione.}\\
\\
\url{https://physics.nist.gov/PhysRefData/Xcom/html/xcom1-t.html} \\
$Element \rightarrow Pb $\\
Il fotone ha energia 10 MeV quindi si seleziona la riga con 1.000E+01, mentre si utilizza la colonna PAIR IN NUCLEAR FIELD, come unità di misura scegliere barns/atom e poi ricordarsi che $ 1 b= 10^{-24} cm^2$.
Dal sito internet si trova: $\sigma=$ 1.263E+01 = 12.6 barns/atom = 12.6 $\cdot 10^{-24} cm^2$\\
Si usa la seguente formula:
\begin{equation*}
    P= \frac{\rho_{Pb} N_{av}}{M_{Pb}}\sigma \Delta l
\end{equation*}
con $\rho_{Pb}= 11.3 g/cm^3 $, $M_{Pb}= 207 g/mol $ , $N_{av}=6 \cdot 10^{23} mol^{-1}$, $\Delta l=X_0(Pb)= 0.56 cm$.\\
Quindi: 
\[
P= 0.23
\]

\paragraph{4.b.14}\textbf{Calcolare l’energia minima e l'energia massima trasferibile in una singola
collisione da una particella carica, di massa molto maggiore di quella dell'elettrone, in moto veloce attraverso la materia ad un singolo elettrone atomico.}\\
\\

Per valutare l'energia massima consideriamo l'urto tra un elettrone di massa $m_e$ e una particella di massa $m_1$ e energia $E_1$. Nel CM l'energia dell'elettrone è:
\begin{equation}
    E_e^{cm}=\frac{M^2+m_e^2-m_1^2}{2M}
\end{equation}
dove
\begin{equation*}
    M^2=(m_e+E_1)^2-(E_1^2-m_1^2)=m_1^2+m_e^2+2m_eE_1
\end{equation*}
La velocità e il fattore di Lorentz del CM sono;
\begin{equation}
    v_{cm}=\frac{\sqrt{E_1^2-m_1^2}}{m_e+E_1}
\end{equation}
\begin{equation}
    \gamma_{cm}=\frac{m_e+E_1}{M}
\end{equation}
Facendo l'opportuna trasformazione di Lorentz si ottiene (trasferimento massimo per $\theta=0$):
\begin{equation}
    E_e \le \gamma_{cm} (E_e^{cm}+ v_{cm}\sqrt{(E_e^{cm})^2-m_e^2})=m_e \left( 1+ 2\frac{E_1^2-m_1^2}{m_1^2+m_e^2+2m_eE_1} \right)
\end{equation}
Usando le relazioni $E=m\gamma$ e $\gamma^2-1=v^2\gamma^2$
\begin{equation}
    E_e= m_e \left( 1+ 2\frac{E_1^2-m_1^2}{m_1^2+m_e^2+2m_eE_1} \right) = m_e \left( 1+ \frac{2m_1m_e^2v_1^2\gamma_1^2}{m_1^2+m_e^2+2m_eE_1}  \right)
\end{equation}

L'energia trasferita è ($T=E-m_e$), se la particella incidente è tale che $m_1^2>>m_e^2$, 
\begin{equation}
    T_{max}\simeq 2m_e \gamma_1^2 v_1^2
\end{equation}
L'energia minima è per motivi quantistici, l'energia corrispondente alla minima separazione tra i livelli energetici:
\begin{equation}
T_{min}=I \, , 
\end{equation}
dove $I$ è l'energia di ionizzazione.
\paragraph{4.b.15}\textbf{Indicare le ipotesi effettuate e dimostrare la seguente espressione, approssimata, per la perdita di energia per collisioni (formula di Bohr):
\[
\frac{1}{\rho}\frac{dE_{coll}}{dx}= z^2\frac{Z}{A(g)} 4\pi \frac{m_e c^2}{\beta^2}N_A r_e^2\left(ln\frac{c\beta^3\gamma^2}{z\omega_e r_e} \right)
\]
}
Consideriamo una particella di carica $q=ze$ e massa $M$ in moto attraverso la materia, essa avrà interazione coulombiana con gli elettroni presenti nella materia. Durante il suo percorso la particella trasferirà energia agli elettroni e conseguentemente ionizzerà o ecciterà gli atomi, l'energia persa per unità di percorso si chiama "energia persa per collisioni". 
Ipotesi effettuate:
\begin{itemize}
    \item energia particella molto maggiore dell'energia di legame degli elettroni con il materiale
    \item moto veloce $\Rightarrow$ si può approssimare il moto della particella con una linea retta e si può considerare ogni urto come istantaneo
    \item in un singolo urto l'energia persa è trascurabile rispetto all'energia della particella $\Rightarrow$ l'impulso (istantaneo) della forza che la carica in moto esercita sull'elettrone non ha componente parallela alla traiettoria della particella ma solo perpendicolare ($\Delta p_{\perp}\not=0, \Delta p_{//}=0 $)
\end{itemize}
\begin{figure}[H]
    \centering
    \includegraphics[scale=1]{4.b.15.PNG}
\end{figure}
Utilizzando il teorema dell'impulso e applicando la legge di Gauss per il campo elettrico ad un cilindro il cui asse sia parallelo alla traiettoria della particella e passante per la posizione dell'elettrone, si ottiene:
\begin{equation*}
    |\Delta p_{\perp}|=\int_{-\infty}^{+\infty}|F_{\perp}|dt=\int_{-\infty}^{+\infty}|-eE_{\perp}|dt=\frac{e}{v}\int_{-\infty}^{+\infty}|E_{\perp}|dx= \frac{e}{2\pi bv} \int_{-\infty}^{+\infty}2 \pi b|E_{\perp}|dx= \frac{ze^2}{2\pi \epsilon_0 b v}
\end{equation*}
L'energia trasferita ad un elettrone, per un urto con parametro d'impatto $b$ è:
\begin{equation*}
    T(b)=\frac{|\Delta p_{\perp}|^2}{2 m_e}=\frac{z^2 e^4}{8 \pi^2 m_e \epsilon_0^2 v^2 b^2}=\frac{z^2(4\pi \epsilon_0 m_e c^2 r_e)^2}{8 \pi^2 m_e \epsilon_0^2\beta^2c^2b^2} = 2z^2 \frac{m_e c^2}{\beta^2}\frac{r_e^2}{b^2}
\end{equation*}
si è usato $v=\beta c$ e $\frac{e^2}{4 \pi \epsilon_0 r_e}= m_e c^2$.\\
L'energia media trasmessa nelle collisioni in un tratto $\Delta x$ si ottiene moltiplicando $T(b)$ per la concentrazioni degli elettroni $n_e$ ed integrando sul volume $\Delta T_{coll}=\int_0^{\Delta x} \int_{b_{min}}^{b_{max}} T(b) 2 \pi b db dx$.\\
\begin{equation*}
    \left<\frac{dE_{coll}}{dx} \right>=n_e \int_{b_{min}}^{b_{max}} T(b) 2 \pi b db = 4 \pi n_e z^2 \frac{m_e c^2}{\beta^2}r_e^2 ln\frac{b_{max}}{b_{min}} 
\end{equation*}
\warning L'interazione con gli può avvenire anche se la particella passa esternamente all'atomo.\\
Modello di Bohr:\\
\begin{itemize}
    \item $b_{max}$\\
    Si impone che la durata dell'urto $\tau \sim \frac{b}{\gamma v} $ sia minore del periodo di rotazione dell'elettrone atomico $T_e\sim \frac{1}{\omega_e}$ $\Rightarrow b_{max}=\frac{\gamma v}{\omega_e}=\frac{\gamma \beta c}{\omega_e}$ 
    \item $b_{min}$\\
    L'energia cinetica massima trasferibile ad un elettrone inizialmente fermo è $T_{max}=2 m_e c^2 \beta^2 \gamma^2$, usando l'espressione trovata in precedenza $ T(b)=2z^2 \frac{m_e c^2}{\beta^2}\frac{r_e^2}{b^2}$ $\Rightarrow b_{min}=\sqrt{2z^2 \frac{m_e c^2}{\beta^2}\frac{r_e^2}{T_{max}}}=\frac{z r_e}{\beta^2 \gamma}$
\end{itemize}
Quindi sostituendo:
\begin{equation*}
      \left<\frac{dE_{coll}}{dx}\right>= z^2 4\pi n_e \frac{m_e c^2}{\beta^2}r_e^2\left(ln\frac{c\beta^3\gamma^2}{z\omega_e r_e} \right)  
\end{equation*}
Se si esprime in $\frac{MeV \cdot cm^2}{g}$:
\begin{equation}
  \left<\frac{1}{\rho}\frac{dE_{coll}}{dx}\right>= z^2\frac{Z}{A (1 g)} 4\pi \frac{m_e c^2}{\beta^2}N_A r_e^2\left(ln\frac{c\beta^3\gamma^2}{z\omega_e r_e} \right)  
\end{equation}
\paragraph{4.b.16}\textbf{Utilizzando la modellizzazione $I = (16eV)\cdot Z$ , valutare il valore minimo dell’energia persa per collisioni, in MeV/g/$cm^2$, per i seguenti materiali: i) Piombo, ii) Silicio; iii) aria a TPN.} \\ \\

Il minimo di ionizzazione si ha in corrispondenza di $\beta \gamma \sim 3.5$

\begin{figure}[H]
    \centering
    \includegraphics[scale=0.5]{4.b.16.JPG}
\end{figure}

da cui per trovare le quantità richieste si può inserire questo valore nella formula di Bethe-Bloch e porre $\delta=0$ e $z=1$:
\begin{equation*}
    <\frac{1}{\rho}\frac{dE_{coll}}{dx}>=z^2\frac{Z}{A}\frac{K}{\beta^2}\left( ln\frac{2m_ec^2(\beta\gamma)^2}{I}-\beta^2 \right) \ \ \Rightarrow \ \ <\frac{dE_{coll}}{dx}>\biggl|_{min}=\rho\frac{Z}{A}\frac{0.307 \ \frac{MeV \cdot cm^2}{g}}{0.93}\left(ln\frac{2m_ec^2(12.25)}{16 \ eV \ Z} -0.93\right)
\end{equation*}
Dove $K=0.307 \frac{MeV\cdot cm^2}{g}$.
\\
Per trovare $\beta^2$ ho invece usato tutte le mie doti di massaggiatore per flexare quest'equazione:
\begin{equation}
    \gamma^2=\frac{1}{1-\beta^2} \ \to \ 1-\beta^2=\frac{1}{\gamma^2} \ \to \ \beta^2=1-\frac{\beta^2}{\beta^2 \gamma^2} \ \to \ \beta^2 \left(1+\frac{1}{(\beta \gamma)^2} \right)=1 \ \to \ \beta^2=\frac{1}{1+\frac{1}{(\beta \gamma)^2}}=\frac{(\beta \gamma)^2}{(\beta \gamma)^2+1}
\end{equation}
che ha come valore circa $\sim 0.93 $.
\\
\\ 
Per trovare $A$ andare sulla nuclear chart e guardare quale isotopo dell'atomo non ha canali di decadimento.
\\
Ora si possono valutare per i vari materiali listati:
\begin{itemize}
    \item \textbf{Pb :} $Z=82$, $A=207.2/1.007\simeq 205$, $\rho=11340 kg/m^3= 11.34 g/cm^3$
    \begin{equation}
        <\frac{dE_{coll}}{dx}>\biggl|_{min}=1.23 \ MeV/g/cm^2
    \end{equation}
\end{itemize}

\paragraph{4.b.17}\textbf{Calcolare la differenza di energia persa per collisioni da parte di particelle di
massa diversa, ma di stessa energia, ultrarelativistiche. Spiegare come questo
possa essere utilizzato per individuare la massa di una particella.} \\ \\

Particelle con la stessa energia tracciano grafici diversi in base alla \textit{Formula di Bethe-Bloch}

\begin{figure}[H]
    \centering
    \includegraphics[scale=0.2]{discriminazione.JPG}

\end{figure}
Infatti la formula approssimata (per $\beta \sim 1$) è
\begin{equation}
    \frac{1}{\rho}\frac{dE_{1,2}}{dx}\sim\frac{Z}{A} K \left(ln\frac{2m_e  \vec{p}}{I m_{1,2}^2}-1-\frac{\delta}{2} \right)
\end{equation}
con $I$ energia media di eccitazione dell'atomo, $\delta/2$ un termine di correzione, $\rho$ la densità del materiale, $K=N_A\frac{4\pi m_e c^2 r_e^2}{1g}=0.307 \frac{MeV \cdot cm^2}{g}$.
\\
Perciò la differenza tra le due particelle è pari a
\begin{equation}
    \Delta_{1,2}=\frac{Z}{A}K\left( ln\frac{2 m_e  \vec{p}^2}{Im_{1}^2}-1-\frac{\delta}{2} \right) - \frac{Z}{A}K\left( ln\frac{2 m_e  \vec{p}^2}{Im_2^2}-1-\frac{\delta}{2} \right)=2\frac{Z}{A}Kln \frac{m_2}{m_1}
\end{equation}
Misurare la perdita di energia per ionizzazione delle due particelle equivale a misurare il logaritmo del rapporto tra le masse.




\paragraph{4.b.18}\textbf{Nell'approssimazione di piccoli angoli e distribuzione gaussiana di varianza nota, calcolare il valor medio, il valore quadratico medio e la sigma per: i) l'angolo di multiplo scattering rispetto alla direzione iniziale della particella, e ii) la sua proiezione su un piano che contenga la direzione iniziale della particella.} \\ \\
Si suppone che inizialmente la particella si muova lungo $\hat{z}$.
\\
Sia $\theta$ l'angolo di multiplo scattering e $\theta_x$ e $\theta_y$ la sua proiezione sui piani $xz$ e $yz$.
\\
Se $\theta_x$ e $\theta_y$ sono distribuiti gaussianamente;
\begin{equation}
    \frac{dP}{d\theta_x}=\frac{1}{\theta_0\sqrt{2\pi}}e^{-\frac{\theta_x^2}{2\theta_0^2}} \ \ \ \ \ \ \ \ \ \ \frac{dP}{d\theta_y}=\frac{1}{\theta_0\sqrt{2\pi}}e^{-\frac{\theta_y^2}{2\theta_0^2}}
\end{equation}

Da cui si ricava
\begin{equation}
    <\theta_x>=<\theta_y>=0 \ \ \ \ \sqrt{<\theta_x^2>}=\sqrt{<\theta_y^2>}=\theta_0
\end{equation}

Rispetto alla direzione iniziale della particella si può trovare la distribuzione gaussiana dell'angolo impostando la normalizzazione
\begin{equation}
    1=\int_{-\infty}^\infty\int_{-\infty}^\infty \frac{1}{\theta_0\sqrt{2\pi}}e^{-\frac{\theta_x^2}{2\theta_0^2}}\frac{1}{\theta_0\sqrt{2\pi}}e^{-\frac{\theta_y^2}{2\theta_0^2}}d\theta_x d\theta_y=\int_0^\infty \int_0^{2\pi} \frac{1}{2\pi \theta_0^2}e^{-\frac{\theta^2}{2\theta_0^2}}\theta d\theta d\phi=\int_0^\infty \frac{\theta}{\theta_0^2}e^{-\frac{\theta^2}{2\theta_0^2}}
\end{equation}

Da cui la distribuzione risulta:
\begin{equation}
    P(\theta)=\frac{\theta}{\theta_0^2}e^{-\frac{\theta^2}{2\theta_0^2}}
\end{equation}
A questo punto si riportano i passaggi (con sostituzioni annesse) per trovare i valori richiesti
\begin{equation*}
    <\theta>=\int_0^\infty \theta P(\theta)d\theta=\int_0^\infty \frac{\theta^2}{\theta_0^2}e^{-\frac{\theta^2}{\theta_0^2}}d\theta=\theta_0\int_0^\infty w^2e^{-\frac{w^2}{2}}dw=\frac{\theta_0}{2}\int_{-\infty}^\infty w^2 e^{-\frac{w^2}{2}}dw=\theta_0\sqrt{\frac{\pi}{2}}
\end{equation*}

\begin{equation*}
    <\theta^2>=\int_0^\infty \theta^2P(\theta)d\theta=\int_0^\infty \frac{\theta^3}{\theta_0^2}e^{-\frac{\theta^2}{2\theta_0^2}}d\theta=\int_0^\infty \frac{\theta^2}{\theta_0^2}e^{-\frac{\theta^2}{2\theta_0^2}}d\frac{\theta^2}{2}=2\theta_0^2 \int_0^\infty te^{-t}dt=2\theta_0^2
\end{equation*}
Da cui
\begin{equation}
    \sqrt{<\theta^2>}=\sqrt{<\theta_x^2>+<\theta_y^2>}=\theta_0\sqrt{2}
\end{equation}
\begin{equation}
    \sigma_{\theta}=\sqrt{<\theta^2>-<\theta>^2}=\theta_0\sqrt{2-\frac{\pi}{2}}=0.65\ \theta_0
\end{equation}
\textbf{N.B.} il fatto che sia $\theta_0\ll 1$ permette di scrivere gli estremi di integrazione come $-\infty,+\infty$. Batignani Dixit.\\
Secondo me $\theta$ è definito da 0 a $\pi$ e l'approssimazione permette di scrivere l'estremo superiore come $\infty$.\\
\textbf{N.B.2:} Si sono utilizzati i noti integrali:
\begin{equation}
     \int_0^\infty x^2e^{-x^2/b}=\frac{\pi}{4}b^{3/2} \qquad \int_0^\infty te^{-t}dt=1
\end{equation}
\textbf{N.B.3} Lorè dirà che il primo è un integrale gaussiano, non credetegli.\\
\textbf{N.B.4} Lo direi anche io, se non avessi appena controllato su wikipedia.





\paragraph{4.b.19}\textbf{Dimostrare l'espressione approssimata per l’angolo quadratico medio di multiplo scattering (proiezione su un piano):
\begin{equation}
    \theta_0=z\frac{costante}{P\beta c}\sqrt{\frac{L}{X_0}}
\end{equation}
e confrontare il valore della
costante ottenuta con la formula contenuta nel PDG.} \\ \\

Si ritiene nota la seguente espressione per l'angolo quadratico medio per il singolo urto (ricavabile con immondi contoni dalla sezione d'urto Mott - Lez 11.1, pag.4)
\begin{equation}
    <\theta^2_{urto}>\sim \frac{4\pi \ell^2}{\sigma_{ms}}ln\frac{205}{Z^{1/3}}
\end{equation}
Dove $\ell=\frac{2zZ\alpha \hbar c}{Pv}$, $\sigma_{ms}$ è la sezione d'urto di multiplo scattering e Z è il numero atomico del materiale attraversato.
\\
A questo punto l'angolo quadratico medio totale si ottiene sommando su tutti gli urti
\begin{equation}
    <\theta^2>=N_{urti}<\theta^2_{urto}>=nL\sigma_{ms} <\theta^2_{urto}>=4\pi nL\ell^2 ln \frac{205}{Z^{1/3}}
\end{equation}
Dove $L$ è la lunghezza del tratto percorso nel materiale.
\\
Si utilizza la formula approssimata per la lunghezza di radiazione derivata dalla \textit{Formula di Tsai}:
\begin{equation}
    X_0\sim \frac{1}{4Z^2n\alpha r_e^2 ln \frac{184}{Z^{1/3}}}
\end{equation}
Da cui
\begin{align}
    \sqrt{<\theta^2>}&=\sqrt{\frac{X_0}{X_0}4\pi nL\ell^2 ln \frac{205}{Z^{1/3}}}=\ell\sqrt{\frac{\ell}{X_0}}\sqrt{\frac{4\pi n ln\frac{205}{Z^{1/3}}}{4Z^2 n\alpha r_e^2 ln\frac{184}{Z^{1/3}}}}\sim \\
    &\sim \ell \sqrt{\frac{L}{X_0}}\frac{\pi}{Z^2 \alpha r_e^2}=\frac{2zZ\alpha \hbar c}{Pv}\frac{1}{r_e}\sqrt{\frac{L}{X_0}}\sqrt{\frac{\pi}{Z^2 \alpha}}=\\
    &=\frac{2zm_ec^2}{Pv}\sqrt{\frac{L}{X_0}}\sqrt{\frac{2\pi}{\alpha}}\sim z\sqrt{2}\frac{14.6 \ MeV}{Pv}\sqrt{\frac{L}{X_0}}
\end{align}
Mentre la formula del PDG (\url{https://pdg.lbl.gov/} $\rightarrow$ Search:multiple scattering $\rightarrow$ primo link:\url{https://pdg.lbl.gov/2020/reviews/rpp2020-rev-passage-particles-matter.pdf} $\rightarrow$ p.14-15)è
\begin{equation}
    \sqrt{<\theta^2>}=z\sqrt{2}\frac{13.6 \ MeV}{Pv}\sqrt{\frac{L}{X_0}}
\end{equation}



\paragraph{4.b.20}\textbf{Utilizzando le opportune tabelle, anche reperibili su web \url{https://pdg.lbl.gov/} , per
le seguenti particelle: i) elettrone 3.5MeV, ii) elettrone 100MeV, iii) pione di
1GeV, iv) muone di 45 GeV, v) protone da 7 TeV, che attraversino: a) 2
mm Pb, b) 2 mm scintillatore, iii) 0.3 mm Silicio, iv) 1 m Aria; indicare se
siano rilevanti e, in caso affermativo, calcolare le seguenti quantità: a) energia
persa per irraggiamento, b) l’energia persa per collisioni, c) probabilità di
interazione forte con i nuclei, d) angolo quadratico medio di multiplo scattering.} \\ \\

Formule usate:
\begin{center}
    \begin{tabular}{c|c}
      Energia irraggiata &  \begin{math}
      E_{irr}=E_0(1-e^{-x/X_0})
      \end{math}\\
      \hline \\
        Energia per collisione & \begin{math}
        E_{coll}=4\pi n_e z^2 \frac{m_e c^2}{\beta^2}r_e^2 ln\frac{b_{max}}{b_{min}}x
        \end{math}\\
        \hline \\
        Probabilità di int forte & 
        \begin{math}
        P_{strong}=\sigma_{strong} n_s \ \ \ \sigma=\pi (\sum R_{nucleari})^2
        \end{math}\\
        \hline \\
        Ang quad medio ms & \begin{math}
    \sqrt{<\theta^2>}=z\sqrt{2}\frac{13.6 \ MeV}{Pv}\sqrt{\frac{x}{X_0}}
\end{math}\\
    \end{tabular}
\end{center}
Dove $X_0$ è la lunghezza di radiazione. \\
L'energia irraggiata (o almeno quella formula) è rilevante solo per energia maggiori dell'energia critica $E>\frac{98\ MeV}{Z^{1/3}}$. \\
L'energia persa per collisione è stata calcolata nel modello di Bohr e si è usato:
$n_e=\rho \frac{Z}{A \cdot 1 g}N_A$, $b_{max}=\frac{\gamma \beta c}{\omega_e}$ con $\omega_e$ periodo di rotazione dell'elettrone atomico, $b_{min}=\frac{z r_e}{\beta^2 \gamma}$. Si ricorda inoltre che $E=M\gamma$, da cui $\gamma=E/M$ e $\beta=\sqrt{1-1/\gamma^2}$.
\\
Dato che non saprei manco dove ricopiarlo ricavo il periodo di rotazione (in CGS, ci finisce  un $k_0$ a numeratore altrimenti e bona)
\begin{equation}
    \frac{e^2}{R^2}=m_e \frac{v^2}{R} \ \xrightarrow{\omega_e=V/R} \ \frac{e^2}{R^2}=m_e \omega_e^2 R \ \Rightarrow \ \omega_e=\sqrt{\frac{e^2}{m_e R^3}}
\end{equation}
Si sostituisca a $R$ il raggio atomico $R\sim r_0 A^{1/3}$ con $r_0\sim 1,25 \ fm$ con un'aggiunta di $2 \ fm$ per i nuclei grandi.
\\
Per la probabilità di interazione forte mi sono arrangiato perchè non capisco dove la avremmo fatta, ma nel 4.3.2 lui cita la sezione d'urto per le interazioni forti e poco più avanti usa la sezione d'urto per calcolare la probabilità di interazione debole, per cui ho deciso che si fa così.
\\
Nella parte dell'angolo $v=\beta c$, mentre $M^2=E^2-P^2$, da cui $P=\sqrt{E^2-M^2}$.
\\
\textbf{Parte importante da sapere per tutta la domanda:} La perdita di energia per radiazione è rilevante solo per elettroni e positroni.
\\
Quella per collisione lo è per tutte.
\\
La probabilità di interazione forte è rilevante solo per adroni.
\\
Per l'angolo quadratico medio non sono sicuro, perchè si ricava dalla sezione d'urto Mott, che è la sezione d'urto per un elettrone che interagisce con un campo elettrico esterno, quindi ognuno tragga le proprie conclusioni.
\\
Ora bisogna trovare i valori necessari e prima o poi faccio anche i conti.
\\
Ricapitolando, sono necessari: raggio atomico, numero atomico, numero di nucleoni, lunghezza di radiazione, sezione d'urto forte.

\begin{itemize}

\item \textbf{Elettrone 3.5 MeV}

\begin{center}
    \begin{tabular}{|c|c|c|c|c|}
    \hline
   Materiale & Energia Irraggiata & Energia per Collisione & Probabilità di int forte & ang quad medio ms \\
    \hline
         2 mm Pb & & & &  \\
          2 mm Scintillatore & & & & \\
           0.3 mm Si & & & & \\
            1 m Aria & & & & \\
            \hline
    \end{tabular}
\end{center}

\item \textbf{Elettrone 100 MeV}
\begin{center}
    \begin{tabular}{|c|c|c|c|c|}
    \hline
   Materiale & Energia Irraggiata & Energia per Collisione & Probabilità di int forte & ang quad medio ms \\
    \hline
         2 mm Pb & & & &  \\
          2 mm Scintillatore & & & & \\
           0.3 mm Si & & & & \\
            1 m Aria & & & & \\
            \hline
    \end{tabular}
\end{center}
\item\textbf{Pione di 1 GeV}
\begin{center}
    \begin{tabular}{|c|c|c|c|c|}
    \hline
   Materiale & Energia Irraggiata & Energia per Collisione & Probabilità di int forte & ang quad medio ms \\
    \hline
         2 mm Pb & & & &  \\
          2 mm Scintillatore & & & & \\
           0.3 mm Si & & & & \\
            1 m Aria & & & & \\
            \hline
    \end{tabular}
\end{center}
\item\textbf{Muone di 45 GeV}
\begin{center}
    \begin{tabular}{|c|c|c|c|c|}
    \hline
   Materiale & Energia Irraggiata & Energia per Collisione & Probabilità di int forte & ang quad medio ms \\
    \hline
         2 mm Pb & & & &  \\
          2 mm Scintillatore & & & & \\
           0.3 mm Si & & & & \\
            1 m Aria & & & & \\
            \hline
    \end{tabular}
\end{center}
\item\textbf{Protone da 7 TeV}

\begin{center}
    \begin{tabular}{|c|c|c|c|c|}
    \hline
   Materiale & Energia Irraggiata & Energia per Collisione & Probabilità di int forte & ang quad medio ms \\
    \hline
         2 mm Pb & & & &  \\
          2 mm Scintillatore & & & & \\
           0.3 mm Si & & & & \\
            1 m Aria & & & & \\
            \hline
    \end{tabular}
\end{center}





\end{itemize}
\paragraph{4.b.21}\textbf{Per un muone che attraversi, incidendo perpendicolarmente, una lastra di Ferro di
5cm di spessore in cui è presente un campo magnetico di intensità nota, calcolare
il valore numerico del rapporto fra la deflessione angolare dovuta al campo
magnetico e la dispersione quadratica media dovuta al multiplo scattering. Come
sarà la funzione di distribuzione dell’angolo in uscita? Quale è la dipendenza
dall’energia del muone incidente?} \\ \\

Suppongo il muone abbia carica in modulo pari a $e$.
\\
Indicando con $P$ la sua quantità di moto, x lo spessore del materiale, si ricava che la deflessione magnetica, nell'ipotesi di piccoli angoli è pari a
\begin{equation}
   q\vec{v}\wedge \vec{B}=m\frac{v^2}{R} \ \Rightarrow \ qvB=m\frac{v^2}{R} \ \Rightarrow \ R=\frac{P}{eB}=numeri \ \ \Rightarrow \ \ sin\theta_B=\frac{x}{R}=numeri \ \ \Rightarrow \ \ \theta_B \sim numeri
\end{equation}

L'angolo quadratico medio di multiplo scattering si calcola invece come
\begin{equation}
    \theta_0=z\frac{13.6 \ MeV}{Pv}\sqrt{\frac{x}{x_0}}\sim numeri
\end{equation}
Risulta che nel Ferro il rapporto tra i due angoli non dipende da $P$ e vale
\begin{equation}
    \frac{\theta_B}{\theta_0}\sim \frac{3.4}{B(T)\sqrt{x(cm)}}
\end{equation}

Non so cosa cazzo sia la funzione di distribuzione dell'angolo in uscita.



\paragraph{4.b.22}\textbf{Cercando i dati nelle apposite figure o tabelle si calcolino il valore (o i limiti)
dell'energia degli elettroni emessi nello stato finale della reazione $\gamma + C$ per
energie del fotone incidente pari a: 1keV, 10keV, 100keV, 1MeV, 10MeV.} \\ \\

Si consideri la seguente figura:

\begin{figure}[H]
    \centering
    \includegraphics[scale=0.7]{gamma su C.JPG}
\end{figure}
A seconda dell'energia del fotone incidente la sezione d'urto è di vario tipo.
\\
\begin{itemize}
    \item 1 KeV: sezione d'urto fotoelettrica
    \item 10 KeV: sezione d'urto fotoelettrica
    \item 100 KeV: sezione d'urto Compton
    \item 1 MeV: sezione d'urto Compton
    \item 10 MeV: sezione d'urto Compton
\end{itemize}

Per i fotoni per cui si considera sezione d'urto fotoelettrica, l'energia dell'elettrone vale:
\begin{equation}
     E_{e^-}=E_{\gamma}- I
\end{equation}
con $I$ energia di ionizzazione dell'atomo.\\
\\
Per i fotoni per cui si considera sezione d'urto Compton, l'energia dell'elettrone vale:
\begin{equation}
    E_{e^-}=m_e\gamma c^2
\end{equation}

e si trova impostando la conservazione del quadrimpulso nella reazione
\begin{equation}
    \left(\begin{array}{lr}
         h\nu  \\
         h\nu \\
         0 \\
         0
    \end{array} \right)+\left( \begin{array}{lr}
         m_0  \\
         0 \\
         0 \\
         0
    \end{array} \right)
\end{equation}
Risulta con passaggi che ho riportato nell'esercizio dopo che
\begin{equation}
    m_e\gamma=h(\nu -\nu') +m_e
\end{equation}
e la differenza riportata in parentesi dipende dall'angolo di emissione dell'elettrone come:
\begin{equation}
    \frac{m_e}{h}\left( \frac{1}{\nu'}-\frac{1}{\nu} \right)=1-cos\theta
\end{equation}
L'unico modo per calcolare l'energia dell'elettrone dai dati noti è considerare la situazione in cui si ha massimo trasferimento di impulso, ovvero quando il fotone torna indietro sull'asse x e l'elettrone va in avanti sempre sull'asse x, in questo caso $cos\theta=1$.

\paragraph{4.b.23}\textbf{Ricavare la relazione tra angolo di scattering e cambio di frequenza nell'effetto
Compton} \\



Si consideri un fotone che urta su un elettrone. \\

\begin{figure}[H]
    \centering
    \includegraphics[scale=0.2]{compton.png}

\end{figure}
Si scrive la conservazione del quadrimpulso:
\begin{equation}
    \left( 
    \begin{array}{lr}
         E_\gamma  \\
         E_\gamma \\
         0 \\
         0
    \end{array}
    \right) + \left(     \begin{array}{lr}
         m_0  \\
         0 \\
         0 \\
         0
    \end{array}
    \right) = \left( 
        \begin{array}{lr}
         E_{\gamma'}  \\
         E_{\gamma'}cos \theta \\
        - E_{\gamma'} sin\theta\\
         0 
    \end{array}
    \right) + \left(
        \begin{array}{lr}
         m  \\
         mvcos\phi \\
        - mvsin \phi\\
         0
    \end{array}
    \right)
\end{equation}
Dove $E_\gamma=h\nu$ e $m=m_0 \gamma$.
\begin{equation}
    \left\{ 
    \begin{array}{lr}
         h\nu + m_0=h\nu'+m  \\
         h\nu=h\nu'cos\theta+mvcos\phi \\
         0=h\nu'sin\theta+mvsin\phi
    \end{array}
    \right.
\end{equation}

Sommo in quadratura le due espressioni per eliminare la dipendenza da $\phi$
\begin{equation}
    \left(mv\right)^2=h^2[(\nu - \nu'cos\theta)^2+(\nu'sin\theta)^2]     \ \ \Rightarrow \ \  \left(\frac{mv}{h}\right)^2=\nu^2 + \nu'^2 - 2\nu \nu' cos\theta
\end{equation}

Ora elevo al quadrato la relazione
\begin{equation}
    m=h\nu - h\nu' + m_0 \ \ \Rightarrow \ \ m^2=h^2(\nu - \nu')^2+m_0^2+2m_0 h(\nu - \nu') 
\end{equation}
E uso che 
\begin{equation}
    m^2=m_0^2 + m^2v^2 \ \ \Rightarrow \ \ m-m_0^2 =m^2 v^2
\end{equation}
Da cui ricavo (massaggiando propriamente le equazioni)
\begin{equation}
    m_0(\nu - \nu')=\nu \nu' h(1-cos\theta) \ \ \Rightarrow \ \ \frac{m_0}{h}\left( \frac{\nu - \nu'}{\nu \nu'} \right)=1-cos\theta
\end{equation}
E in ultima istanza si ricava
\begin{equation}
    \frac{m_0}{h}\left( \frac{1}{\nu'}-\frac{1}{\nu} \right)=1-cos\theta
\end{equation}

\paragraph{4.b.24}\textbf{Cercando i dati delle sezioni d’urto totali nelle apposite figure o tabelle (reperibili anche nella compilazione Particle Data Group \url{https://pdg.lbl.gov} ) si calcoli la probabilità di interazione di:
\begin{enumerate}
    \item un fotone da $100 eV$ che incida su $1 \mu m$ di grafite
    \item un fotone da $1 MeV$ che incida su $1 mm$ di grafite
    \item un fotone da $10 MeV$ che incida su $1 mm$ di piombo
    \item un fotone da $50 KeV$ che incida su $1 \mu m$ di piombo
    \item un neutrino da $100 GeV$ che incida su $1 km$ di grafite
    \item un protone da $100 GeV$ che incida su $1 cm$ di grafite
\end{enumerate}
}
La probabilità di interazione dopo che la particella ha percorso una distanza $x$ nel materiale è data dalla formula:
\begin{equation}
P(x)=1-e^{-\frac{x}{\lambda}} \, ,
\end{equation}
dove
\begin{equation}
\lambda=\frac{1}{N \sigma}
\end{equation}
è il cammino libero medio, con
\begin{equation}
N=\frac{\rho N_A}{A_r}
\end{equation}
numero di centri di scattering per unità di volume, $\rho$ densità del materiale, $N_A$ numero di Avogadro, $A_r$ peso atomico e $\sigma$ sezione d'urto. Da cui:
\begin{equation}
P(x)=1-e^{-x\frac{\rho N_A}{A_r} \sigma} \, .
\end{equation}
Applicando tale formula abbiamo come risultato:
\begin{enumerate}
\item $\sigma$ non pervenuta
\item 0.0135 = 1.35\%
\item 0.0546 = 5.46\%
\item 0.0853 = 8.53\%
\item $1.13\cdot 10^{-11}$
\item 0.0113 = 1.13\%
\end{enumerate}

\paragraph{4.b.25}\textbf{Esprimere la sezione d'urto Rayleigh in funzione della sezione d'urto differenziale Thomson e del fattore di forma atomico $F(\theta)$.}\\
\\
UGUALE ALLA 2.b.23\\
\\
La sezione d'urto Rayleigh è definita in maniera differenziale attraverso
\begin{equation}
    \frac{d\sigma}{d\Omega}=\frac{d\sigma_{el}}{d\Omega}Z^2|F(\theta)|
\end{equation}
per ottenere la sezione d'urto integriamo utilizzando l'espressione per la sezione d'urto Thomson:
\begin{align}
    \sigma&=Z^2\int\frac{d\sigma_{el}}{d\Omega}|F(\theta)|d\Omega\\
          &=2\pi Z^2\int_0^\pi\frac{d\sigma_{el}}{d\Omega}|F(\theta)|\sin\theta d\theta\\
          &=2\pi Z^2r_e^2\int_0^\pi<\sin^2\alpha>|F(\theta)|\sin\theta d\theta\\
          &=\frac{3}{4}\sigma_{th}Z^2\int_0^\pi\frac{1+\cos^2\theta}{2}|F(\theta)|\sin\theta d\theta
\end{align}
\paragraph{4.b.26}\textbf{Dimostrare, utilizzando il materiale distribuito, perché nell'esperimento di Anderson sulla scoperta del positrone alcune tracce positive osservate non possono essere nessuna delle particelle positive conosciute nel 1932.}\\
\\
Anderson utilizzò una camera a nebbia con all'interno una lastra di piombo, come si può vedere in Figura \ref{fig:positron}. Il campo magnetico è entrante, quindi le particelle cariche positivamente e provenienti dalla parte bassa della camera sono piegate verso sinistra.

Dalla traccia innanzitutto possiamo capire se la particella arriva dal basso o arriva dall'alto: la traccia alta ha un raggio di curvatura più piccolo, ergo la particella ha perso energia rispetto alla traccia bassa. Quindi la particella è prima entrata in basso, poi a superato la lastra di piombo ed infine è entrata nella parte alta. Questa traccia implica una particella carica positivamente.

Inoltre la traccia è molto sottile, il che implica una particella di massa molto piccola. Non possiamo escludere a priori il protone, che avendo carica positiva può avere un comportamento simile. Tuttavia il protone non può attraversare una lastra di piombo di tale spessore (circa 6 mm, ovvero circa $1X_0$).

Dalla differenza dei raggi delle due tracce possiamo ricavare la perdita di energia, valutata in circa 7 MeV per ionizzazione ed inoltre si riduce di un fattore $1/e$ per irraggiamento. Tutto ciò è compatibile con l'energia di uscita pari a 22.5 MeV. 

\begin{figure}[H]
    \centering
    \includegraphics[width=0.4\linewidth]{PositronDiscovery.png}
    \caption{Prima immagine della traccia del positrone scattata da Anderons}
    \label{fig:positron}
\end{figure}


\paragraph{4.b.27}\textbf{Calcolare la velocità media dei pioni e antiprotoni nell'esperimento di Segrè (quantità di moto $1.19\ Gev/c$) dopo che hanno attraversato il contatore Cherenkov a quarzo (spessore $2.5"$, densità relativa $2.2$).}\\
\\
L'energia persa da pioni ed antiprotoni dopo aver attraversato un contatore Cherenkov è dovuta principalmente alla perdita di energia per collisioni.
Per prima cosa si calcola la "mass thickness" del mezzo:
\begin{equation}
    \Delta x= \rho \Delta l= 2.2 (\frac{g}{cm^3}) \cdot 2.5 * 2.54 (cm)= 13.9 (\frac{g}{cm^2})
\end{equation}
Si considera il minimo della Bethe-Bloch $\frac{dE}{dx}\biggr|_{MIN}\sim 2 \; MeV g^{-1} cm^2$, quindi si può calcolare l'energia persa come:
\begin{equation}
    \Delta E \sim \Delta x\cdot \frac{dE}{dX}\Biggr|_{MIN}=13.9\cdot 2=27.8\, Mev
\end{equation}
Per calcolare la quantità di moto finale si utilizza la seguente relazione:
\begin{equation}
   \Delta E= \sqrt{p_{in}^2+m_p^2}-\sqrt{p_{fin}^2+m_p^2}
\end{equation}
Da cui,consideriamo la quantità di moto iniziale come $p=1.19\; GeV/c$:
\begin{equation}
    p_{fin}^2=\Delta E^2-2\Delta E\sqrt{p_{in}^2+m_p^2} + p_{in}^2=  27.8^2-2\cdot27.8\sqrt{1190^2+938^2}+1190^2
\end{equation}
\begin{equation}
    p_{fin}=1173\, Gev/c
\end{equation}
Considerando $p=m\gamma \beta$, si trova che il $\beta$ finale risulta essere:
\begin{equation}
    \beta=\frac{p}{\sqrt{m^2+p^2}} =\frac{1173}{\sqrt{938^2+1173^2}}=0.78
\end{equation}



%-----------------------------------------------
\newpage
\section{Domande a scelta}
\epigraph{This is what you call sarcasm, isn't it?.}{}
\subsection*{Gruppo i)}
\paragraph{5.c.10}\textbf{Scrivere le equazioni di Maxwell per i quadri-potenziali in "gauge di Lorentz" e in "gauge di Coulomb", presentando alcune situazioni pratiche in cui sia conveniente utilizzare l'uno o l'altro gauge.}\\
\\
Le equazioni non omogenee di Maxwell per i quadri-potenziali in formalismo quadridimensionale sono:
\begin{equation}
   \partial_{\mu}(\partial^{\mu}A^{\nu}-\partial^{\nu}A^{\mu}) =\frac{4 \pi}{c}j^{\nu} 
\end{equation}
\begin{equation}
 \partial_\mu F^{\mu \nu}=\frac{4\pi}{c}j^\nu    
\end{equation}

Mentre le equazioni omogenee si esprimono in termini del tensore di campo duale  $\Tilde{F^{\mu \nu}}$:
\begin{equation}
    \partial_{\mu}\Tilde{F^{\mu \nu}}=0
\end{equation}

Risulta che una trasformazione del tipo
\begin{equation}
    A^\mu \ \to \ A^\mu - \partial^\mu f
\end{equation}
con $f$ funzione generica della posizione, lascia invariato il tensore dei campi. Questa invarianza è nota come \textit{Invarianza di Gauge dell'elettromagnetismo}.
\\
In notazione tridimensionale diviene:
\begin{equation}
    \phi \ \to \ \phi - \frac{1}{c}\frac{\partial f}{\partial t} \ \ \ ; \ \ \ \ \vec{A} \ \to \ \vec{A}+\vec{\nabla}f
\end{equation}
Si può usare questa invarianza per imporre dei vincoli sulle equazioni di Maxwell per i potenziali e renderle più semplici in alcune applicazioni.
\\
Un esempio è la \textbf{Gauge di Lorenz}:
\begin{equation}
    \partial_\mu A^\mu =0
\end{equation}
In questa Gauge risulta
\begin{equation}
    \partial_\mu F^{\mu \nu}=\partial_\mu \partial^\mu A^\nu -\partial^\nu \partial_\mu A^\mu =\partial_\mu \partial^\mu A^\nu
\end{equation}

Perciò le equazioni di Maxwell non omogenee divengono:
\begin{equation}
    \partial_\mu \partial^\mu A^\nu=\frac{4\pi}{c}j^\nu
\end{equation}
Che sono delle equazioni d'onda in cui $c$ si intravede a occhio essere la velocità di propagazione.
\\
\\
Una scelta alternativa è la \textbf{Gauge di Coulomb}:
\begin{equation}
    \vec{\nabla}\cdot \vec{A}=0
\end{equation}
Con tale scelte si ottengono le seguenti scritture per i potenziali:
\begin{equation}
    \nabla^2\phi=-4\pi\phi
\end{equation}
\begin{equation}
    \frac{1}{c^2}\frac{\partial^2\vec{A}}{\partial t^2}-\nabla^2\vec{A}=\frac{4\pi}{c}\vec{j}-\frac{1}{c}\vec{\nabla}\frac{\partial \phi}{\partial t}
\end{equation}
Da cui si intravede che il potenziale scalare soddisfa l'equazione di Poisson nel caso elettrostatico, mentre il potenziale vettore soddisfa a un'equazione delle onde con un termine di corrente modificato.
\\
Nella gauge di Lorenz è evidente che il campo elettromagnetico propaga con velocità c, visto che le equazioni del moto sono delle equazioni d’onda; nella gauge di Coulomb, invece, questo è molto meno chiaro: il potenziale scalare reagisce infatti in ogni punto istantaneamente ad un cambio di $\rho$, mentre il calcolo del
potenziale vettoriale è complicato dalla presenza della componente longitudinale della corrente. La gauge di Coulomb è invece vantaggiosa nel caso elettrostatico.\\
Di conseguenza si privilegia la gauge di Lorentz quando si tratta la propagazione di onde elettromagnetiche. Diversamente se si vuole risolvere problemi riguardanti distribuzioni di carica (e di corrente?) costanti nel tempo si privilegia la gauge di Coulomb.\\

\paragraph{5.c.13}\textbf{Dimostrare il teorema ottico nel caso particolare di diffrazione da un ostacolo opaco} NON COMPLETAAAAAAAAAAAAAAAAAAAAAAAAAAAAAAAAAAAAAAA\\
\\

Si considera un'onda elettromagnetica piana che incide ortogonalmente su un bersaglio (piano) opaco.
\\
Il campo totale nello spazio risulta essere una sovrapposizione del campo dell'onda e di quello derivante dalla polarizzazione del bersaglio.
\\
Per scrivere il campo generato dal bersaglio bisogna tenere conto della polarizzazione.
In questi casi si considera il campo generato da un dipolo con momento di dipolo proporzionale al campo elettrico incidente
\begin{equation}
    \vec{E}_{dip}(\vec{r},t)=k^2e^{-i\omega t}\int_\Sigma \vec{p}_{0\perp} \frac{e^{ik|\vec{r}-\vec{r}'|}}{|\vec{r}-\vec{r}'|}d\Sigma 
\end{equation}
In cui il momento di dipolo indotto è pari a
\begin{equation}
    \vec{p}_0=-\frac{\vec{E}_0}{2\pi i k}
\end{equation}
che è pari a $\vec{p}_{0\perp}$ dal momento che l'onda vi incide perpendicolarmente.
\\
Il campo totale risulta quindi
\begin{equation}
    \vec{E}_{tot}(\vec{r},t)=\vec{E}_0 e^{-i(\omega t -kz)}+
\end{equation}

\paragraph{5.c.14}\textbf{Dimostrare che la sezione d’urto differenziale elastica per un’onda e.m. piana,
non polarizzata e monocromatica su un elettrone legato elasticamente vale:$\frac{d\sigma_{el}}{d\Omega}=r_e^2\frac{\omega^4}{(\omega_0^2-\omega^2)^2+\omega^2\Gamma_{tot}^2}\frac{1+cos^2\theta}{2}$ con $\theta$ angolo (di scattering) fra la direzione di osservazione e direzione dell'onda incidente}\\ \\
La sezione d'urto elastica per un'onda e.m. piana che incide su un elettrone legato elasticamente si può ricavare dallo scattering Thompson. Se però l'onda non è polarizzata è necessario successivamente mediare la sezione d'urto differenziale su tutte le possibili polarizzazioni dell'onda incidente. \\ 
Si suppone che l'onda incidente si propaghi lungo l'asse z, da cui $\hat{k}=(0,0,1)$ è la direzione della luce incidente.\\
Si parametrizza la direzione di osservazione tramite $\theta$, angolo fra la direzione della luce incidente e la direzione di osservazione, e $\phi$: $\hat{k}'=(sin\theta cos\phi, sin \theta sin\phi, cos\theta)$.\\
La generica polarizzazione dell'onda sarà dunque $\hat{x}_0=(cos\psi,sin\psi,0)$, con $\psi$ angolo tra la direzione del campo elettrico dell'onda e l'asse $\hat{x}$, poiché deve essere perpendicolare all'asse z.\\
Chiamiamo $\alpha$ l'angolo tra la direzione di osservazione $\hat{k}'$ e la direzione di polarizzazione $\hat{x}_0$:
\begin{equation}
    cos \alpha= \vec{x}_0 \cdot \vec{k}'= sin\theta cos\phi cos\psi+ sin\theta sin\phi sin\psi= sin\theta cos(\psi-\phi) 
\end{equation}
La sezione d'urto differenziale derivante dallo scattering Thompson è la seguente:
\begin{equation}\label{urto}
    \frac{d\sigma_{el}}{d\Omega}=\frac{r_e^2\omega^4}{(\omega_0^2-\omega^2)^2+\omega^2\Gamma^2_{tot}(\omega)}sin^2\alpha
\end{equation}
Si deve ora sostituire $sin^2\alpha=1-cos^2\alpha=1-sin^2\theta cos^2(\psi-\phi) $ e mediare sulle direzioni della polarizzazione, sarà quindi necessario considerare la media rispetto a $\psi$.
\begin{equation}\label{media}
    <sin^2\alpha>=\frac{1}{2\pi}\int_0^{2\pi} sin^2\alpha d\psi=\frac{1}{2\pi}\int_0^{2\pi}1-sin^2\theta cos^2(\psi-\phi) d\psi=1-\frac{1}{2}sin^2\theta=\frac{1+cos^2\theta}{2}
\end{equation}
Sostituendo la \ref{media} nella \ref{urto} si ottiene:
\begin{equation}
    \frac{d\sigma_{el}}{d\Omega}\biggl|^{NP}=r_e^2\frac{\omega^4}{(\omega_0^2-\omega^2)^2+\omega^2\Gamma_{tot}^2}\frac{1+cos^2\theta}{2}
\end{equation}
La media sulle polarizzazioni non cambia la sezione d'urto integrata, in quanto essa risulta indipendente dalla specifica polarizzazione.\\
Difatti integrando sull'angolo solido (considerando l'asse polare diretto lungo $\vec{k}$):
\begin{equation}
    \int_0^\pi d\theta\int_0^{2\pi}d\phi \frac{1+cos^2\theta}{2}sin\theta=\pi\int_{-1}^{1}(1+x^2)dx=\frac{8}{3}\pi
\end{equation}








\subsection*{Gruppo ii)}
\paragraph{5.c.21}\textbf{Spiegare in modo quantitativo, e nelle varie ipotesi citate nell'articolo, il valore dell'energia di soglia di produzione degli antiprotoni osservata nell’esperimento di Segrè et al. Calcolare esplicitamente l'impulso trasferito al bersaglio nelle varie ipotesi fatte.}\\
\\
Per prima cosa si studia il caso di un protone che incide su un nucleone fermo
\begin{equation}
    p+p \to p+p+p+\bar{p} \ \ \ \ \ \ (p+n \to p+n+p+\bar{p})
\end{equation}
L'energia di soglia in questo caso risulta (supponendo l'impulso lungo una sola direzione spaziale, ma è solo notazione)
\begin{equation}
   \left( \begin{array}{lr}
         E  \\
         p 
    \end{array}\right)
    +
    \left( \begin{array}{lr}
         m  \\
         0 
    \end{array}\right)=\left(\begin{array}{lr}
         E+m  \\
         p 
    \end{array}\right)
\end{equation}
Da cui, trovandone la massa invariante si può imporre la disequazione per trovare il valore di soglia. Ossia:
\begin{equation}
    s=(E+m)^2-p^2=(E+m)^2-(E^2-m^2)\geq (4m)^2
\end{equation}
Dove con $m$ si indica sempre la massa del protone ($m\sim 938.28 \ MeV/c$).
\\
Svolgendo i passaggi si ricava
\begin{equation}
    E\geq 7m \ \ \Rightarrow \ \ T=E-m\geq 6m \simeq 5.63 \ GeV
\end{equation}
Ora si effettua invece il calcolo per un urto su un nucleo $A$ di massa $M_A$ fermo.
\begin{equation}
    p+A\to p+A+p+\bar{p}
\end{equation}
In questo caso le stesse equazioni di conservazione forniscono:
\begin{equation}
     \left( \begin{array}{lr}
         E  \\
         p 
    \end{array}\right)
    +
    \left( \begin{array}{lr}
         M_A  \\
         0 
    \end{array}\right)=\left(\begin{array}{lr}
         E+M_A  \\
         p 
    \end{array}\right)
\end{equation}
Il che fornisce come soglia per la massa invariante
\begin{align}
    s&=(E+M_A)^2-p^2\geq (3m+M_A)^2 \\
    &E^2+M_A^2+2EM_A-E^2+m^2 \geq 9m^2+M_A^2+6mM_A \\
    &2EM_A \geq 8m^2+6mM_A \\
    &E\geq 3m + 4 \frac{m^2}{M_A}\label{5.29}
\end{align}
Da cui per la soglia dell'energia cinetica si trova
\begin{equation}
    T=E-m\geq 2m + 4 \frac{m^2}{M_A}
\end{equation}
Che è chiaramente una situazione molto più favore dal punto di vista energetico.
\\
Prendendo l'esempio del ${}^{63}_{29}Cu$ con $M_A \simeq 41465,5 \ MeV/c $ (si ricorda che $m_n\sim 939,57 \ MeV/c$) risulta
\begin{align}
    T\geq & \left[2(938.28)+4\frac{(938,28)^2}{41465.5}\right] \ MeV = \\
    & = (1876,56 + 84,92) \ MeV = 1961,48 \ MeV \simeq 1,96 \ GeV
\end{align}
Questo processo non è tuttavia possibile. Si osservi, infatti, l'impulso trasferito al nucleo:
\begin{equation*}
q\equiv P^\mu _{fin}-P^\mu_{in} \ \ \Rightarrow \ \ q= \left( \begin{array}{lr}
     M_a \gamma  \\
     p 
\end{array}\right) -\left( \begin{array}{lr}
     M_A  \\
     0
\end{array}\right)
\end{equation*}
Con 
\begin{equation*}
    p=\sqrt{(M_A\gamma)^2-M_A^2}=M_A\sqrt{\gamma^2-1}=M_A\sqrt{\frac{1}{1-\beta^2}-1}=M_A\sqrt{\frac{1-1+\beta^2}{1-\beta^2}}=M_A\beta \gamma
\end{equation*}
Risulta, facendo i conti riportati di seguito:
\begin{align}
    q^2&= (M_A\gamma-M_A)^2-M_A^2\beta^2\gamma^2 = M_A^2[(1-\gamma)^2-\beta^2\gamma^2]=\\
    &=M_A^2[1+\gamma^2-2\gamma -\beta^2 \gamma^2]=M_A^2\left[ \frac{1-\beta^2+1}{1-\beta^2}-2\gamma -\frac{\beta^2}{1-\beta^2}\right]= \\
    &=M_A^2\left[ \frac{2-2\beta^2}{1-\beta^2}-2\gamma \right]=2 M_A^2(1-\gamma)
\end{align}
Dove $\beta$ e $\gamma$ si riferiscono al cambiamento di sistema da quello del laboratorio a quello del centro di massa. All'energia di soglia (i prodotti sono fermi) abbiamo che 
\begin{equation}
    \gamma=\frac{E_{tot,lab}}{E_{CM}}=\frac{E+M_A}{3m+M_A}=\frac{3m+\frac{4m^2}{M_A}+M_A}{3m+M_A}=\frac{M_A(3m+M_A)+4m^2}{M_A(3m+M_A)}=1+\frac{4m^2}{M_A(3m+M_A)}
\end{equation}
Dove si è utilizzato la \ref{5.29} per l'energia di soglia. L'impulso trasferito sarà quindi:
\begin{equation}
    q^2=-\frac{8M_am^2}{3m+M_A}\ \ \Rightarrow\ \ \sqrt{|q^2|}\simeq2.57\ Gev/c
\end{equation}
molto maggiore dell'impulso di Fermi per il quale il nucleo di rame si romperebbe.\\
Nell'articolo viene infatti stimato $T_f\sim 25 \ MeV$, da cui $p_f=\sqrt{2mT_f}\sim 216.6 \ MeV/c$.

\noindent\rule{\textwidth}{0.7pt}

\textbf{Energia di Fermi:} \textit{Energia del più alto livello occupato in un sistema di fermioni alla temperatura dello zero assoluto.}\\
Un fermione è una particella che obbedisce alla statistica di Fermi-Dirac, i.e. è una particella con spin semintero (protoni, neutroni e elettroni sono fermioni).
\\ \\
\textbf{Importante precisazione:} Il modello a goccia dei nuclei e il modello in cui si applica la statistica di Fermi-Dirac sono due modelli differenti!
\\
Il modello a cui ci si riferisce è noto come \textit{modello a gas di Fermi}.
\\
Si tratta di un modello quantistico che considera un potenziale medio come una buca di potenziale (quella stessa che aveva introdotto Logoteta la prima lezione sui nuclei). 
\\
La buca ha un certo numero di stati legati e, per via della sopracitata statistica di Fermi-Dirac, in un ipotetico gas di nucleoni allo zero assoluto, questi sono attesi occupare i livelli energetici del mare di Fermi (nome figo per chiamare sta roba).
\\
Il succo è che le fluttuazioni date dall'energia cinetica termodinamica sono dell'ordine $\sim kT$, che a temperatura ambiente sono totalmente trascurabili, perciò i nuclei, in condizioni standard, sono approssimabili come gas di Fermi allo zero assoluto.
\\
Anche i nuclei considerati da Segrè allora. Perciò i nucleoni occupano i vari stati (a quanto pare ogni livello è composto da 2 protoni e 2 neutroni, divisi a coppie e con spin antiparallelo) e in generale li riempiono, ossia lo stato che ha l'energia più alta (quella di Fermi) è occupato.
\\
Se si fornisce energia oltre il livello di Fermi, si esce dalla buca. Si passa da uno stato legato a uno stato del continuo.
\\
\textit{Da qui me la invento}
\\
Ad un'energia si può associare un impulso $E=p^2/2m$ (classico e non relativistico perchè non si sta usando una teoria quantistica relativistica ma un modello classico).
\\
Se una particella con impulso di Fermi urta un nucleone fermo nel mare di Fermi, per le leggi di conservazione trasferirà un impulso e (se il nucleone è già in uno stato non troppo basso (ma abbiamo detto che in generale tutti gli stati sono occupati)) l'energia del nucleone andrà ad eccedere quella di Fermi, quindi diverrebbe uno stato del continuo.
\\
Batignani in particolare nella nota 3 dell'articolo considera che il nucleone incidente abbia proprio l'energia di Fermi.
\\
\textit{Cose che non mi tornano:} Boh sì, ci sta, nel caso potrei supercazzolarla, ma come faccio a dire che il nucleo si rompe e non ho solo l'emissione di un protone o di un neutrone?
\\
Bati suppone che l'urto sia frontale e per avere il caso più favorevole possibile suppone che il nucleone nel nucleo si stia muovendo con l'impulso massimo possibile, ossia quello relativo all'energia di Fermi. Il Nucleo non si rompe perché noi stiamo considerando l'energia di soglia e quindi le 4 particelle prodotte dalla reazione sono ferme nel sistema del centro di massa.


\noindent\rule{\textwidth}{0.7pt}
\\
La situazione risulta allora più favorevole se si suppone che il nucleone si stia muovendo contro il protone incidente, da cui il 4-impulso totale è:
\begin{equation}
    P_{tot}=\left(\begin{array}{c}
          E\\
         \sqrt{E^2-m^2}\\
         0\\
         0
    \end{array}\right)
    +
    \left(\begin{array}{c}
          m+T_f\\
        - p_f\\
         0\\
         0
    \end{array}\right)
    =
    \left(\begin{array}{c}
          E+m+T_f\\
         \sqrt{E^2-m^2}-p_f\\
         0\\
         0
    \end{array}\right)\,.
\end{equation}
Allora 
\begin{align}
    s=P_{tot}^2 &=(E+m+T_f)^2-(\sqrt{E^2-m^2}-p_f)^2=\\
                &=E^2+m^2+T_f^2+2mE+2mT_f+2ET_f-E^2+m^2-p_f^2+2p_f\sqrt{E^2-m^2}=\\
                &=2mE+2ET_f+2m^2+T_f^2+2p_f\sqrt{E^2-m^2}\simeq\\
                &\simeq 2mE+2ET_f+2m^2+2p_fE
\end{align}
Dove abbiamo considerato che $T_f=\frac{p_f^2}{2m}$ e che $T_f^2$ e $m^2$ fossero trascurabili nell'ultima approssimazione.\\
Infatti $p_f\sim 200 MeV/c$, $T_f \sim 25 \ MeV$ mentre $E^2 \sim 16-25 GeV^2 \ \gg \ m^2 \sim 1 \ GeV^2 $.\\
Adesso l'energia di soglia verifica che :
\begin{equation}
    s=2mE+2ET_f+2m^2+2p_fE\geq(4m_p)^2\quad\Rightarrow\quad E\geq\frac{7m^2}{m+T_f+p_f}\geq\frac{7m^2}{m+p_f} \ \sim \ 5.336 \ GeV
\end{equation}
Da cui si ricava l'energia cinetica necessaria per la produzione di coppie.
\begin{equation}
    T=E-m\geq\frac{7m^2}{m+p_f}-m \ \simeq \ 4.398 \ GeV
\end{equation}
% bruh <3  goodnight

\paragraph{5.c.22}\textbf{Spiegare, con le valutazioni numeriche (quantitative) opportune, i metodi di separazione degli antiprotoni dal fondo di pioni nell’esperimento di Segrè et al. ed in particolare perchè sia necessaria sia una identificazione tramite contatori Cherenkov che con il tempo di volo. Calcolare in particolare la variazione di velocità degli antiprotoni nell'attraversare i contatori (Cherenkov o a scintillazione).}\\
\\
Una delle maggiori difficoltà affrontate durante il corso dell'esperimento è stata selezionare gli antiprotoni da un fondo molto più grande di pioni ($\sigma(p+Cu\to\pi^{-}+al)\sim 10^5\sigma(p+Cu\to\bar{p}+al))$.\\

\begin{figure}[H]
    \centering
    \includegraphics[scale=0.4]{BArP.PNG}
    \label{fig:BarP}
\end{figure}

Si fa incidere un fascio di protoni su un bersaglio di rame, i prodotti vengono poi separati attraverso un campo magnetico, magneti e focalizzatori fino ad avere un fascio incidente contenete solo pioni e antiprotoni (e mesoni K) di impulso p=1.19 Gev/c.\\
Per individuare gli antiprotoni sono stati utilizzati tre contatori a scintillazione (S1-S2-S3) e due contatori Chérenkov (C1-C2) con lo scopo di misurare la velocità delle particelle attraverso diverse modalità di misura.\\
(PER NOI: so impulso, misuro velocità-> massa, se trovo una particella negativa con massa uguale al protone ho dimostrato l'esistenza dell'antiprotone)\\
I contatori a scintillazione S1 e S2, posti a 40' (12 m) , emettono un segnale quando la particella vi passa attraverso.\\
Si può stimare il tempo di volo a partire dalla formula $\Delta t= l/\beta$ dove l è la distanza tra S1 e S2, nell'esperimento 40'(12 m).\\
Nel caso di pioni ed antiprotoni, per particelle di impulso p=1.19 Gev/c, si suppone di osservare dei picchi in corrispondenza:\\
\begin{equation}
    \beta=\begin{cases}
    0.99 &  m=m_{\pi} \\
    0.78 &  m=m_{p}
    \end{cases}\Longrightarrow
    \Delta t=  \begin{cases}
    40 ns &  m=m_{\pi} \\
    51 ns &  m=m_{p}
    \end{cases}
\end{equation}
Sono presenti tuttavia incertezze di misura sia sulla misura del tempo di volo (1/2 ns) sia su quella dell'impulso, su cui viene affermato un 2$\%$ di errore per le particelle in arrivo a S2 (p=$1.190 \pm 0.024$ Gev/c.\\
Inoltre viene segnalata la presenza di problematiche per misure spurie legate ad accidentali coincidenze di alcuni mesoni che portano al loro conteggio(?).\\
Inoltre utilizzando solo il tempo di volo si osserva una distribuzione in cui il numero di antiprotoni non si riesce ad evidenziare dal fondo di pioni, per risolvere questo problema vengono inseriti i contatori Cherenkov.\\
\begin{figure}[H]
    \centering
    \includegraphics[scale=0.4]{5.c.22.PNG}
    \label{fig:5.c.22}
\end{figure}
Il contatore C2 ha la funzione di selettore di velocità, esso emette un segnale per particelle con $0.75<\beta<0.78$. Tuttavia a causa dello scattering nucleare talvolta (per circa il 3$\%$ dei pioni) C2 emetteva un segnale anche quando passava un mesone, per ovviare al problema viene aggiunto un contatore C1 che emette un segnale solo se $\beta>0.79$. Un ulteriore contatore a scintillazione S3 viene aggiunto come controprova che le particelle siano effettivamente passate.\\
A questo punto l' unica sequenza permessa che ci attesta il passaggio di un antiprotone è se viene rilevato un segnale da S1-S2-S3-C2 e contemporaneamente C1 non rileva nulla. Se anche C1 si accendesse si tratterrebbe di un pione che ha fatto scattering nucleare, mentre se uno dei contatori a scintillazione non rilevasse segnali si tratterebbe di eventi accidentali.\\
"C2 ha la funzione di trigger, quando esso emetteva un segnale allora si andavano a prelevare i segnali degli altri contatori per capire che particella fosse passata."
\begin{figure}[H]
    \centering
    \includegraphics[scale=0.4]{5.c.22.1.PNG}
    \includegraphics[scale=0.4]{5.c.22.2.PNG}
    \label{fig:5.c.22.1}
\end{figure}
\begin{figure}[H]
    \centering
    \includegraphics[scale=0.6]{5.c.22.3.PNG}
    \label{fig:5.c.22.3}
\end{figure}

\underline{Variazione velocità antiprotoni}\\
\\
Come già anticipato si ipotizza che il fascio incidente su S2 abbia impulso $p=1190\pm 24$ MeV, si vuole ora quantificare la perdita di energia attraverso i contatori S2,S3,C1,C2.\\
Particelle con massa m=m$_p$ dovrebbero avere velocità $\beta=0.78$, .\\
Nell'articolo viene riportato che la loro velocità si riduce a $\beta=0.765$ a causa dell'energia perse per collisioni. 
\begin{equation}
    |\vec{p}|=m\gamma \beta=m\frac{\beta}{\sqrt{1-\beta^2}}=1.115 GeV/c
\end{equation}
Dalle informazioni sperimentali si può stimare l'energia totale persa:
\begin{equation}
     \Delta E= \sqrt{p_{in}^2+m_p^2}-\sqrt{p_{fin}^2+m_p^2}=\sqrt{1190^2+938^2}-\sqrt{1115^2+938^2}=58 MeV
\end{equation}

La perdita di energia si può ricondurre a vari fenomeni. Come affermato anche nell'articolo in questo caso prevale il fenomeno di energia persa per collisioni rispetto ai contributi dovuti all'effetto Chèrenkov e alla Bremsstrahlung (non trascurabili solo per elettoni o positroni).\\
Nel caso di una particella di massa m=$m_p$ come l'antiprotone risulta valida l'approssimazione $\gamma\ll \frac{M}{m_e}$
per cui la formula di Bethe-Bloch si riduce a:
\begin{equation}
    <\frac{1}{\rho}\frac{dE}{dx}>=z^2\frac{Z}{A}\frac{K}{\beta^2}\left(ln\frac{2m_ec^2\beta^2\gamma^2}{I}-\beta^2-\frac{\delta}{2}\right)
\end{equation}
Dove:
\begin{equation}
    K=N_A\frac{4 \pi c^2r_e^2}{1 g}=0.307 MeV\frac{g}{cm^2}
\end{equation}
Le caratteristiche dei materiali con cui sono costruiti i contatori sono riportate nella Tabella 1 dell'articolo che riportiamo in seguito:
\begin{figure}[H]
    \centering
    \includegraphics[scale=0.9]{tabellac.PNG}
    \label{fig:tabellac}
\end{figure}
Utilizzando i dati della Tabella si può stimare lo spessore dei contatori pesato attraverso la densità del mezzo ($\Delta X= \Delta l \rho $, con $\Delta l$ spessore del contatore e $\rho$ densità del materiale).
\begin{equation*}
    \Delta X_{S_2}=0.62''\cdot 1.69 \; g/cm^3(\text{plastica}) = 2.3 \;\; g/cm^2
\end{equation*}
\begin{equation*}
    \Delta X_{C_1}=2''\cdot 1.76 \; g/cm^3(C_8F_16) =8.9\;\; g/cm^2
\end{equation*}
\begin{equation*}
    \Delta X_{C_2}=2.5''\cdot 2.2 \; g/cm^3 (\text{quarzo}) = 13.9\;\; g/cm^2
\end{equation*}
RICORDA 1'' (pollici)$\sim$ 2.54 cm\\
Usando la formula di Bethe-Bloch:
\begin{equation}
    <\frac{dE_{coll}}{\rho d(x)}>= z^2 \frac{Z}{M_A}  \frac{K}{\beta^2} \left(\ln\left(\sqrt{\frac{2m_ec^2\beta^2\gamma^2}{I^2} T_{MAX}}\right)-\beta^2-\frac{\delta}{2}\right)
\end{equation}
In prima approssimazione $ \frac{dE}{dx}\bigg|_{MIN}\sim 2$ MeV $g^{-1} cm^2$ si ottiene:
\begin{equation}
    \Delta E \sim (\Delta X_{S_2}+\Delta X_{C_1}+\Delta X_{C_2})\;g/cm^2\cdot 2 \; MeV g^{-1} cm^2 \sim 53\; \text{MeV}
\end{equation}
Il risultato calcolato risulta compatibile con quello trovato sperimentalmente.\\
Se si è provato a migliorare la precisione di questo calcolo si può prendere il valore di $|dE/dx|$ relativo all'impulso della particella per i diversi materiali dei contatori dal sito del NIST nella sezione PSTAR.
Si è trovata l'energia della particella iniziale $E=1.51$ GeV, dai dati reperibili sul sito del NIST:
\begin{itemize}
    \item S2 (Plastic Scintillator): $\frac{dE}{\rho dx}=2.032$, da cui $\Delta E_{S_2}=4.67$ MeV
    \item C1 (Polymethyl Methaclarate-Lucite- Perspex):$\frac{dE}{\rho dx}=2.004$, da cui $\Delta E_{C_1}=17.83$ MeV
    \item C2 (Silicon Dioxide):$\frac{dE}{\rho dx}=1.745$, da cui $\Delta E_{C_2}=24.25$ MeV
\end{itemize}
Da cui si stima dai dati ottenuti
\begin{equation}
    \Delta E_{tot}\sim 47 \: MeV
\end{equation}
L'energia finale è quindi
\begin{equation}
    E_f=E_i-\Delta E\sim 1.463 GeV
\end{equation}
Da cui
\begin{equation}
    p_f=\sqrt{E_f^2-m_p^2}=1.118 Gev/c
\end{equation}

Considerando $p=m\gamma \beta$, si trova che il $\beta$ finale risulta essere:
\begin{equation}
    \beta=\frac{p}{\sqrt{m^2+p^2}} =\frac{1118}{\sqrt{938^2+1118^2}}=0.765
\end{equation}
Il risultato risulta compatibile con quanto riportato nell'articolo.


\underline{Approfondimento: Contatore Cherenkov differenziale C2}\\
Il contatore C2 è un particolare tipo di contatore Cerenkov in grado di rilevare solo particelle in una finestra di velocità. Riportiamo il suo schema nella seguente figura:
\begin{figure}[H]
    \centering
    \includegraphics[scale=0.5]{5.c.22ceren2.PNG}
    \label{fig:5.c.22ceren2}
\end{figure}
Nello schema è riportato a sinistra il radiatore Cerenkov, esso è formato da un cilindro trasparente contenente quarzo fuso. Il fascio di particelle attraversa il radiatore e continua in linea retta mentre i fotoni vengono fermati dal \textit{blackened baffle}.
Le particelle in moto con una data velocità emettono luce ad un angolo limite definito; la luce dopo aver attraversato il mezzo ottico fuoriesce dal cilindro subendo rifrazione per il passaggio in aria secondo la legge di Snell.
La geometria del rivelatore  definisce un intervallo angolare preciso per accettare la luce Cherenkov emessa.
Se i fotoni vengono emessi all'angolo cercato essi raggiungono uno specchio cilindrico, per poi essere focalizzati sul fotocatodo di un fotomoltiplicatore tramite un ulteriore specchio piano. Se i fotoni vengono invece emesse ad angoli più grandi o più piccoli non incontrano mai gli specchi e non arrivano al fotomoltiplicatore.
L’uso dei due diversi specchi fa in modo che l’asse dello strumento non incontri mai la finestra del fotomoltiplicatore, in modo da evitare falsi conteggi.


\paragraph{5.c.22}\textbf{Spiegare, con le valutazioni numeriche (quantitative) opportune, i metodi di separazione degli antiprotoni dal fondo di pioni nell’esperimento di Segrè et al. ed in particolare perchè sia necessaria sia una identificazione tramite contatori Cherenkov che con il tempo di volo. Calcolare in particolare la variazione di velocità degli antiprotoni nell'attraversare i contatori (Cherenkov o a scintillazione).}\\
\\
\begin{figure}[H]
    \centering
    \includegraphics[scale=0.4]{BArP.PNG}
    \label{fig:BarP}
\end{figure}
L'idea di fondo dell'esperimento consiste nel misurare in modo indipendente quantità di moto e velocità delle particelle, in modo da poter risalire alla loro massa.\\
Si fa incidere un fascio di protoni con impulso $p_{max}=6.3\ GeV/c$ su un bersaglio di rame, i prodotti subiscono una deviazione dal fascio, andando incontro al magnete M1 con lo scopo di selezionare particelle negative aventi momento p=1.19 Gev/c. Le particelle passano quindi attraverso collimatori per ridurre le dimensioni trasversali (quadrupolo magnetico Q1 e \textit{shielding}) per arrivare al primo contatore a scintillazione S1. Infine un' ulteriore selezione viene attuata dal magnete deflettente M2, in modo da avere al contatore S2 particelle con impulso 1.19 Gev/c con un incertezza del 2$\%$.\\
Una delle maggiori difficoltà affrontate durante il corso dell'esperimento è stata selezionare gli antiprotoni da un fondo molto più grande di pioni ($\sigma(p+Cu\to\pi^{-}+al)\sim 10^5\sigma(p+Cu\to\bar{p}+al))$.\\
Per individuare gli antiprotoni sono stati utilizzati tre contatori a scintillazione (S1-S2-S3) e due contatori Chérenkov (C1-C2) con lo scopo di misurare la velocità delle particelle attraverso diverse modalità di misura.\\
(PER NOI: so impulso, misuro velocità-> massa, se trovo una particella negativa con massa uguale al protone ho dimostrato l'esistenza dell'antiprotone)\\
I contatori a scintillazione S1 e S2, posti a 40' (12 m), emettono un segnale quando la particella vi passa attraverso: la particella carica cede energia agli elettroni nell'attraversare il mezzo, questi tornando al loro stato fondamentale emettono fotoni e tramite l'uso dei fotomoltiplicatori è possibile visualizzare il segnale all'oscilloscopio.\\
A partire dalla differenza temporale fra i segnali si può stimare il tempo di volo a partire dalla formula $\Delta t= l/\beta$ dove l è la distanza tra S1 e S2, nell'esperimento 40'(12 m).\\
Nel caso di pioni ed antiprotoni, per particelle di impulso p=1.19 Gev/c, si suppone di osservare dei picchi in corrispondenza:\\
\begin{equation}
    \beta=\begin{cases}
    0.99 &  m=m_{\pi} \\
    0.78 &  m=m_{p}
    \end{cases}\Longrightarrow
    \Delta t=  \begin{cases}
    40 ns &  m=m_{\pi} \\
    51 ns &  m=m_{p}
    \end{cases}
\end{equation}
Sono presenti tuttavia incertezze di misura sia sulla misura del tempo di volo (1/2 ns) (un' altra fonte dice 5 ns) sia su quella dell'impulso, su cui viene affermato un 2$\%$ di errore per le particelle in arrivo a S2 (p=$1.190 \pm 0.024$ Gev/c).\\
Inoltre viene segnalata la presenza di problematiche misure spurie quando 2 mesoni arrivano ai contatori S1 con una distanza temporale di 51 ns.\\
In sintesi, data anche la grandissima percentuale di pioni rispetto agli antiprotoni, utilizzando solo il tempo di volo si osserva una distribuzione in cui il numero di antiprotoni non si riesce ad evidenziare dal fondo di pioni.\\ Al fine di risolvere questo problema vengono inseriti i contatori Cherenkov.\\
\begin{figure}[H]
    \centering
    \includegraphics[scale=0.4]{5.c.22.PNG}
    \label{fig:5.c.22}
\end{figure}
Il contatore C2 ha la funzione di selettore di velocità, esso emette un segnale per particelle con $0.75<\beta<0.78$. Tuttavia a causa dello scattering nucleare talvolta (per circa il 3$\%$ dei pioni) C2 emette un segnale anche quando passava un mesone, per ovviare al problema viene aggiunto un contatore C1 che emette un segnale solo se $\beta>0.79$. La presenza del contatore C1 risolve inoltr la problematica riguarda alle misure spurie di cui si è trattato prima.\\
Un ulteriore contatore a scintillazione S3 viene aggiunto come controprova che le particelle siano effettivamente passate.\\
A questo punto l' unica sequenza permessa che ci attesta il passaggio di un antiprotone è se viene rilevato un segnale da S1-S2-S3-C2 e contemporaneamente C1 non rileva nulla. Se anche C1 si accendesse si tratterrebbe di un pione che ha fatto scattering nucleare, mentre se uno dei contatori a scintillazione non rilevasse segnali si tratterebbe di eventi accidentali.\\
"C2 ha la funzione di trigger, quando esso emetteva un segnale allora si andavano a prelevare i segnali degli altri contatori per capire che particella fosse passata."
\begin{figure}[H]
    \centering
    \includegraphics[scale=0.4]{5.c.22.1.PNG}
    \includegraphics[scale=0.4]{5.c.22.2.PNG}
    \label{fig:5.c.22.1}
\end{figure}
\begin{figure}[H]
    \centering
    \includegraphics[scale=0.6]{5.c.22.3.PNG}
    \label{fig:5.c.22.3}
\end{figure}


\newpage
\subsection*{Gruppo iii)}
\paragraph{5.c.34}\textbf{Calcolare nel caso relativistico l’energia in cui la perdita di energia per irraggiamento è paragonabile a quella per collisioni in funzione della massa della particella. Determinare il valore per elettroni e protoni nel Piombo o in aria.}\\
\\
Innanzitutto si generalizza la perdita di energia per Bremsstrahlung a qualunque particella incidente.\\
Sappiamo che la sezione d'urto d'irraggiamento nel caso non relativistico è data da:
\begin{equation}
    \chi_\omega=\frac{16}{3}z^4Z^2\alpha \hbar c^2\frac{r_e^2}{v^2}\left(\frac{m_e}{M}\right)^2\ln{\frac{b_{max}}{b_{min}}}
\end{equation}
Come noto nel caso relativistico occorre porsi nel sistema di riferimento $\Sigma'$ della particella incidente, dove sarà possibile utilizzare l'equazione precedente.\\
In tale sistema di riferimento:
\begin{itemize}
    \item la velocità dello ione sarà ultrarelativistica, per cui $v\sim c$;
    \item $\omega'<\frac{\gamma c}{b}$, poichè l'impulso sarà $\tau\sim\frac{b}{\gamma v}\sim \frac{b}{\gamma c}$;
    \item $\omega'<\frac{Mc^2}{\hbar}$, poichè nel sistema del laboratorio la particella non potrà irraggiare più della sua energia;
    \item $b_{min}=\frac{\hbar}{Mc}$, ovvero la lunghezza d'onda compton, per ragioni quanto-meccaniche;
    \item $b_{max}=a=1.4\frac{a_0}{Z^{1/3}}$ nel caso di screening attivo, ovvero $\gamma>\frac{Mc^2 a}{\hbar c}\sim \frac{192.5}{Z^{1/3}}\frac{M}{m_e} \Longrightarrow E_c=M\gamma c=\frac{M^2}{m_e}c\frac{192.5}{Z^{1/3}}$.\\
    Qual'ora non fossimo nel regime di screening attivo dovremmo porre $b_{max}=\frac{\gamma c}{\omega'}$.\\
\end{itemize}
Per cui, nel regime di screening attivo:
\begin{equation}
    \chi_\omega'=\frac{16}{3}z^4Z^2\alpha \hbar c^2\frac{r_e^2}{v^2}\left(\frac{m_e}{M}\right)^2\ln{\frac{192}{Z^{1/3}}\frac{m_e}{M}}=\chi_\omega
\end{equation}
Per la perdita di energia per irraggiamento si ricava dunque:
\begin{equation}
    <\frac{dE_{irr}}{dx}>=n\int_0^{E/\hbar}\chi_\omega d\omega=\frac{E}{ X_0} \qquad  X_0=\left(n\frac{16}{3}z^4Z^2\alpha \hbar
    c^2\frac{r_e^2}{v^2}\left(\frac{m_e}{M}\right)^2\ln{\frac{192}{Z^{1/3}}\frac{m_e}{M}}\right)^{-1}
\end{equation}
Si può fare uso di un modello più accurato detta formula di Tsai:
\begin{equation}
    <\frac{dE_{irr}}{\rho dx}>=\frac{E}{\rho X_0} \qquad \rho X_0=\left(4\frac{N_A}{M_A}Z^2\alpha r_e^2\left(\frac{m_e}{M}\right)^2z^4\left(\ln(\frac{M}{m_e}\frac{184}{Z^{1/3}})+\frac{L}{Z}-f(Z)\right)\right)^{-1}
\end{equation}
Dove si è sfruttato il fatto che $n=\frac{\rho N_A}{M_A}$.\\
La perdita di energia per collisioni si stima invece attraverso la formula di Bethe-Bloch:
\begin{equation}
    <\frac{dE_{coll}}{\rho d(x)}>= z^2 \frac{Z}{M_A}  \frac{K}{\beta^2} \left(\ln\left(\sqrt{\frac{2m_ec^2\beta^2\gamma^2}{I^2} T_{MAX}}\right)-\beta^2-\frac{\delta}{2}\right)
\end{equation}
Dove 
\begin{equation}
     K=N_A4\pi m_ec^2r_e^2\qquad T_{MAX}=\frac{2m_ec^2\beta^2\gamma^2}{1+2\frac{m_e}{M}\gamma+\biggl(\frac{m_e}{M}\biggr)^2}
\end{equation}
Riscriviamola per esteso:
\begin{align}
    <\frac{dE_{coll}}{\rho d(x)}>= z^2 \frac{Z}{M_A}  \frac{K}{\beta^2} \left(\ln\left(\frac{2m_ec^2\beta^2\gamma^2}{I\sqrt{1+2\frac{m_e}{M}\gamma+\biggl(\frac{m_e}{M}\biggr)^2}} \right)-\beta^2-\frac{\delta}{2}\right)
\end{align}
Per confrontare le due equazioni occorre fare una serie di approssimazioni:
\begin{itemize}
    \item La formula di Tsai è valida solo in regime di screening attivo:
    \begin{equation}
        \gamma\geq\frac{M}{m_e}\frac{192}{Z^{1/3}}\,;
    \end{equation}
    \item Trascuriamo le correzioni numeriche $\delta$, $L$ e $f(Z)$;
    \item Approssimiamo $\beta$ a 1 mettendoci nel caso relativistico;
    \item Nel sito del PDG si è potuto trovare $I_{Pb}=823\ eV$ (coerente con $I\simeq I_0=10 \, eV\cdot Z$ per $Z>20$).Per quanto riguarda l'aria si considera in prima approssimazione un gas di azoto (composta da 78,09$\%$ di azoto, 20,9$\%$ di ossigeno, 0,93$\%$ di Argon e per lo 0,04$\%$ di (CO2)) $I_{N}=82 \ eV$. Inoltre viene riportato anche il valore $I_{air}=85.7 \ eV$ per aria secca ad 1 atm, valore maggiore coerente con la presenza di atomi con più nucleoni;
    \item Trascuriamo in prima battuta nella Bethe-Bloch i termini $-\beta^2$ e $-\frac{1}{2}\ln\left(1+2\frac{m_e}{M}\gamma+\biggl(\frac{m_e}{M}\biggr)^2\right)$
\end{itemize}
A questo punto possiamo imporre la compatibilità della perdita di energia come:
\begin{equation}
    <\frac{dE_{irr}}{\rho dx}>\simeq<\frac{dE_{coll}}{\rho dx}>\quad\Rightarrow\quad\frac{E}{\rho X_0}\simeq z^2 \frac{Z}{M_A} K \ln\left(\frac{2m_ec^2\gamma^2}{I} \right)
\end{equation}
\begin{equation}
    M\gamma c^2\simeq\left(4\frac{N_A}{M_A}Z^2\alpha r_e^2\left(\frac{m_e}{M}\right)^2z^4\ln(\frac{M}{m_e}\frac{184}{Z^{1/3}})\right)^{-1}\left( z^2 \frac{Z}{M_A} K \ln\left(\frac{2m_ec^2\gamma^2}{I} \right)\right)
\end{equation}
\begin{equation}
    M\gamma c^2\simeq\left(4N_A Z\alpha r_e^2\left(\frac{m_e}{M}\right)^2z^2\ln(\frac{M}{m_e}\frac{184}{Z^{1/3}})\right)^{-1}\left(  K \ln\left(\frac{2m_ec^2\gamma^2}{I} \right)\right)
\end{equation}
Sostituendo l'espressione per $K$ si semplifica anche $4N_Ar_e^2$ e un $m_e$:
\begin{equation}
    M\gamma c^2\simeq\frac{1}{Zz^2\alpha}\frac{M^2}{m_e}  \pi c^2\frac{\ln\left(\frac{2m_ec^2\gamma^2}{I} \right)}{\ln(\frac{M}{m_e}\frac{184}{Z^{1/3}})}
\end{equation}
\begin{equation}
    \gamma\simeq\frac{M}{m_e}\frac{400}{Zz^2}\frac{\ln\left(\frac{2m_ec^2\gamma^2}{I} \right)}{\ln\left(\frac{184}{Z^{1/3}}\frac{M}{m_e}\right)}
\end{equation}
Possiamo risolvere questa equazione graficamente tracciano i termini a destra e a sinistra in funzione di $\gamma$ e osservando i punti di intersezione. Applicando questo metodo a tutti i casi in esame si ottengono i seguenti plots:
\begin{figure}[H]
    \centering
    \includegraphics[width=0.8\linewidth]{5.c.34/Figure_1.pdf}
    \label{elettrone_pb}
\end{figure}
\begin{figure}[H]
    \centering
    \includegraphics[width=0.8\linewidth]{5.c.34/Figure_2.pdf}
    \label{protone_pb}
\end{figure}
\begin{figure}[H]
    \centering
    \includegraphics[width=0.8\linewidth]{5.c.34/Figure_3.pdf}
    \label{elettrone_N}
\end{figure}
\begin{figure}[H]
    \centering
    \includegraphics[width=0.8\linewidth]{5.c.34/Figure_4.pdf}
    \label{protone_N}
\end{figure}
da cui abbiamo ricavato delle stime per $\gamma$ che riportiamo nella seguente tabella:
\begin{center}
    \begin{tabular}{|>{\centering}m{0.25\linewidth}|>{\centering}m{0.25\linewidth}|   >{\centering\arraybackslash}m{0.25\linewidth}|@{}m{0pt}@{}}
        \toprule
        & $e^-$ & $p$ \\[4pt]
        \hline \vspace{5pt}
        Pb  ($Z=82$, $I=836\,eV$) &  $\gamma\simeq16.5$ & $\gamma\simeq2.15\cdot10^4$\\[5pt]
        \hline \vspace{5pt}
        N  ($Z=7$, $I=82\,eV$) &  $\gamma\simeq256$ & $\gamma\simeq3.00\cdot10^5$\\[4pt]
        \bottomrule
    \end{tabular}
\end{center}

Giustifichiamo adesso le nostre approssimazioni nella formula di Bethe-Bloch, infatti abbiamo detto che i termini:
\begin{equation}
    -\beta^2\qquad \frac{1}{2}\ln(1+2\frac{m_e}{M}\gamma+\frac{m_e^2}{M^2}) 
\end{equation}
sono trascurabili rispetto al termine
\begin{equation}
    \ln(\frac{2m_ec^2\beta^2\gamma^2}{I})
\end{equation}
La nostra approssimazione risulta ragionevole in quanto\\
\begin{align}
    &\beta^2\sim1\\
    &\frac{1}{2}\ln(1+2\frac{m_e}{M}\gamma+\frac{m_e^2}{M^2})\sim\frac{1}{2}\ln(2\frac{m_e}{M}\gamma)\sim\frac{1}{2}\left(\ln(\gamma)-\ln(\frac{M}{m_e}) \right)\\
    &\ln(\frac{2m_ec^2\beta^2\gamma^2}{I})\sim\ln(10^6eV\frac{\gamma^2}{I})\sim2\ln(1000\gamma)-\ln(I)
\end{align}
\begin{itemize}
    \item e$^-$ su Pb:
    \begin{align}
        &\frac{1}{2}\left(\ln(\gamma)-\ln(\frac{M}{m_e}) \right)\sim 1.4\\
        &\ln(\frac{2m_ec^2\beta^2\gamma^2}{I})\sim\ln(10^6eV\frac{\gamma^2}{I})\sim 15.1
    \end{align}
    \item e$^-$ su N:
    \begin{align}
        &\frac{1}{2}\left(\ln(\gamma)-\ln(\frac{M}{m_e}) \right)\sim 2.5\\
        &\ln(\frac{2m_ec^2\beta^2\gamma^2}{I})\sim\ln(10^6eV\frac{\gamma^2}{I})\sim 20.4
    \end{align}
    \item p su Pb:
    \begin{align}
        &\frac{1}{2}\left(\ln(\gamma)-\ln(\frac{M}{m_e}) \right)\sim 2.3\\
        &\ln(\frac{2m_ec^2\beta^2\gamma^2}{I})\sim\ln(10^6eV\frac{\gamma^2}{I})\sim 27.05
    \end{align}
    \item p su N:
    \begin{align}
        &\frac{1}{2}\left(\ln(\gamma)-\ln(\frac{M}{m_e}) \right)\sim 2.5\\
        &\ln(\frac{2m_ec^2\beta^2\gamma^2}{I})\sim\ln(10^6eV\frac{\gamma^2}{I})\sim 34.6
    \end{align}
\end{itemize}
\newpage
Abbiamo infine controllato per l'elettrone, su cui ci sono dati per la bremmstrahlung,
se i nostri dati coincidevano con quelli riportati dal NIST con esito positivo per una stima così brutale. Riportiamo i grafici dei dati sperimentali:
\begin{figure}[H]
    \centering
    \includegraphics[width=0.8\linewidth]{5.c.34/Piombo.PNG}
    \label{fig:nist_pb}
\end{figure}
\begin{figure}[H]
    \centering
    \includegraphics[width=0.8\linewidth]{5.c.34/Aria.PNG}
    \label{fig:nist_air}
\end{figure}
Per quanto riguarda il Pb si osserva una buon accordo nonostante il regime non attivo (?). Per l'aria il dato sembrerebbe discostarsi maggiormente (dal grafico $\gamma\sim 200$), la motivazione è dovuta al fatto che si è considerato solo l'azoto trascurando gli altri elementi chimici che la compongono.
\paragraph{5.c.36}\textbf{Si consideri una particella che attraversa, incidendo perpendicolarmente, una regione di lunghezza $x$ di materiale omogeneo, di cui sono note tutte le sue caratteristiche ed in particolare la lunghezza di radiazione $(X_0)$. Nel materiale è presente un campo magnetico, di intensità nota B, perpendicolare alla direzione della particella. Calcolare la deflessione angolare dovuta sia al campo magnetico che al multiplo scattering nel caso particolare: materiale $1\, m$ di aria, $B=1\, T$, per pioni di impulso di $1\, GeV/c$ oppure $100\, GeV/c$.}
\\ \\
DISEGNINO\\
Nei calcoli seguenti non consideriamo la perdita di energia per irraggiamento (rilevante solo per elettroni e positroni) e la perdita di energia dovuta all'interazione forte su nuclei. \\
Per calcolare la probabilità di collisione nucleare si utilizza la formula approssimata:
\begin{equation}
 P=n\sigma=\rho \frac{N_a x}{A}\sigma_0 A^{2/3} 
\end{equation}
Con $A$ massa della grammo-molecola, nonchè numero di nucleoni e $x$ distanza percorsa nel materiale.
\\
Il valore di $\sigma_0$ è stato preso dal sito del PDG dove sono riportate le sezioni d'urto su singolo protone per gli adroni, per muoni da 1 GeV $\sigma_0= 24.58$ mb, per quelli da 100, $\sigma_0=23.13$ mb. Come densità si è utilizzata quella dell'aria $\rho=1.205 \times 10^{-3}\ g/cm^3$,  come numero atomico si è considerata l'aria come se fosse composta solo da molecole $N_2$, quindi A=28. \\
Quindi sostituendo i valori numerici si ottiene:
\begin{itemize}
    \item per P=1GeV $\rightarrow$ $\mathcal{P}= $ $5.9 \times 10^{-4}$
    \item per P=100 GeV $\rightarrow$ $\mathcal{P}= 5.5 \times 10^{-4} $
\end{itemize}
Anche per quanto riguarda la perdita di energia per collisioni possiamo ritenerla trascurabile. Per affermare ciò abbiamo osservato dal grafico che mette in relazione l'energia persa per collisioni per unità di lunghezza con $\beta \gamma$, per $\beta \gamma> 3.5$, ovvero $p>3.5 m_{\pi}= 490 \; MeV/c$, il valore non supera in maniera rilevante $|dE/dx|\sim 2 MeV g^{-1} cm^2$.
\begin{equation}
    \Delta E= |dE/dx|\rho(\text{aria})x\sim 2 (MeVg^{-1} cm^2) 1.2 \cdot 10^{-3}(g cm^{-3}) 1\cdot 10^2 cm= 0.24 MeV
\end{equation}
Inoltre, anche se $|dE/dx|$ fosse più alta, a causa della bassa densità dell'aria avremmo comunque una perdita di energia trascurabile rispetto all'impulso iniziale delle particelle in esame .
\\
Calcoliamo ora la deflessione della traiettoria del fascio di pioni dovuto al campo magnetico e al multiplo scattering.\\
\\
\underline{Deflessione dovuta al campo magnetico}\\
Consideriamo le particelle ultrarelativistiche poiché $\beta=\frac{p}{E}=\frac{p}{\sqrt{m_{\pi}^2c^4+p^2c^2}}\sim 1$.\\
Nel caso in esame si può porre la forza di Lorentz percepita dalla particella uguale alla forza centripeta, trovando il raggio di curvatura della sua traiettoria:
\begin{equation}
    qv B=m\gamma \frac{v^2}{R}  \Rightarrow \ P=qBR
\end{equation}
Sostituendo la carica del pione (uguale in modulo alla carica elementare) e cambiando le unità di misura si ottiene:
\begin{equation}
    P (MeV/c)=300 B(Tesla)R(m) \ \Rightarrow \ R(m)=\frac{P(MeV)}{300 B(Tesla)}
\end{equation}

Da questo è possibile ricavare l'angolo di deflessione dovuto al campo magnetico come indicato in figura \ref{mag}

\begin{figure}[H]
    \centering
    \includegraphics[scale=0.5]{5.c.36.JPG}
    \label{mag}
\end{figure}
\begin{equation}
    sin(\theta_B)=\frac{x}{R} \ \Rightarrow \ \theta_B=arcsin\left(x (m)\frac{300 B (Tesla) }{P (MeV)}\right)
\end{equation}
Per angoli piccoli si può comunque approssimare $\theta_B \sim sin(\theta_B)$.
\begin{itemize}
    \item per P= 1 GeV/c = $10^{3}$ MeV/c $\rightarrow \theta_B= 0.3$ rad
    \item per P= 100 GeV/c= $10^{5}$ MeV/c $\rightarrow \theta_B= 3 \times 10^{-3}$ rad 
\end{itemize}

\underline{Deflessione dovuta al Multiplo Scattering}\\
\\
Per calcolare l'angolo quadratico medio di multiplo scattering si fa l'ipotesi di piccoli angoli che verrà poi verificata a posteriori.\\

Si applica la formula per l'angolo quadratico medio trovata sul PDG:
\begin{equation}
 \theta_0= z\frac{13.6 \;MeV}{PV}\sqrt{\frac{x}{X_0}}
\end{equation}
dove $X_0$ è per definizione la lunghezza di radiazione per un elettrone ad alta energia, sempre sul PDG si trova che nel caso dell'aria vale: $X_0=36.62\;g/cm^2 =3.039 \cdot 10^4 \;cm$. \\
Nel nostro caso ($z=1$, $\beta \sim 1$): 
\begin{equation}
    \theta_0\sim \frac{13.6 \;MeV}{P c}\sqrt{\frac{10^2}{3.039 \cdot 10^4}}=\frac{0.78 MeV}{P(MeV/c) c} 
\end{equation}

\begin{itemize}
    \item per P= 1 GeV/c = $10^{3}$ MeV/c $\rightarrow \theta_0= 0.78 \cdot 10^{-3}$ rad
    \item per P= 100 GeV/c= $10^{5}$ MeV/c $\rightarrow \theta_0= 0.78 \times 10^{-5}$ rad
\end{itemize}
Gli angoli trovati si possono considerare piccoli.

\newpage
\section{Costanti e valori numerici}
\epigraph{A tremendous feeling of peace came over him. He knew that at last, for once and for ever, it was now all, finally, over.}{}
\subsection{Masse}
Tutte le masse sono misurate in $MeV/c$ dove $c$ è stata posta a $1$
\begin{itemize}
    \item massa dell'elettrone: $m_e \sim 0.5 \; MeV$
    \item massa del protone: $m_p \sim 938 \; MeV$
    \item massa del neutrone: $m_n \sim 939.7 \; MeV$
    \item massa del muone: $m_\mu \sim 105 \; MeV$
    \item massa del tauone: $m_\tau \sim 1767 \; MeV$
    \item massa del fotone: $m_\gamma = 0$
    \item massa del pione: $m_\pi=139.6 MeV$
\end{itemize}

\textbf{N.B.} L'elettrone e il protone sono stabili: il tauone, il muone e il neutrone no, il quale decade in circa 15 minuti (tramite decadimento $\beta$: $n \rightarrow p + e^- + \bar{\nu}_e$).
\subsection{Dimensioni}
\begin{itemize}
    \item fisica atomica $\sim \angstrom$
    \item fisica nucleare $ \sim fm$
    \item atomo idrogeno $\simeq 0.5\, \angstrom$ 
    \item atomi superiori  $\simeq 1\, \angstrom$
    \item raggio classico elettrone $\simeq 2.8\,fm $
    \item sezione d'urto thompson $\simeq 0.66\, barn$
\end{itemize}

\subsection{Unità di misura}
\begin{itemize}
    \item barn $= 10^{-24}\, cm^2$
    \item fermi $= 10^{-15}\, m$
    \item eV $= 1.602\cdot10^{-19}\, J $
    \item 1 u.m.a $=0.93\cdot10^{3}=\frac{1}{1.07}\cdot10^3\, MeV$
    \item 1\angstrom $=10^{-10}\, m$
\end{itemize}

\subsection{Ordini di tempo \textbf{TIPICI} per interazioni}
Molto approssimativamente, dipende molto dall'energia sopratutto per la Forte:
\begin{itemize}
    \item Elettromagnetica $\sim10^{-5}-10^{8}\, s$
    \item Forte $\sim 10^{-23}\, s$
    \item Debole $\sim 10^{-5}-10^{8}\, s$
\end{itemize} 
\subsection{Costanti}
\begin{itemize}
    \item$\alpha=\frac{e^2}{4\pi \epsilon_0 \hbar c}=7.29 \cdot 10^{-3} \simeq \frac{1}{137}$
    \item pippo
    \item lorè che si lamenta
    \item giulio e bla che litigano
\end{itemize}

\newpage

\section{Conclusione}
''I always thought something was fundamentally wrong with the universe'' \citep{adams1995hitchhiker}
\vfill
\bibliographystyle{plain}
\bibliography{references}
\end{document}
